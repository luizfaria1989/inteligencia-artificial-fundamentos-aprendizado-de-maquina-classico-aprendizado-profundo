\chapter{Métricas de Avaliação Para Outros Algoritmos}
\label{cap:metricas-de-avaliacao-para-especificos}

% ----------------------------------------------------------------------
\section{Métricas Para Classificação de Imagens}
\label{sec:metricas-classificacao-imagens}
% (Esta seção geralmente reutiliza métricas do capítulo anterior, 
% como Acurácia, Precisão, Recall, F1 e AUC-ROC)

% ----------------------------------------------------------------------
\section{Métricas Para Detecção de Objetos}
\label{sec:metricas-deteccao-objetos}

% --- IoU é fundamental para esta seção ---
\begin{equacaodestaque}{Interseção sobre União (IoU) / Índice Jaccard}
    IoU(A, B) = \frac{|A \cap B|}{|A \cup B|} = \frac{\text{Área da Interseção}}{\text{Área da União}}
    \label{eq:equacao-iou}
\end{equacaodestaque}

\medskip
\begin{center}
 * * *
\end{center}
\medskip

\textbf{Algumas Aplicações do xxx}
\vspace{1em}

\begin{itemize}
    \item \textbf{Aplicação 1 (Área):}
    \item \textbf{Aplicação 2 (Área):}
    \item \textbf{Aplicação 3 (Área):}
\end{itemize}

\medskip
\begin{center}
 * * *
\end{center}
\medskip

\subsection{Average Precision (AP)}
\label{sec:average-precision-ap}
% (Média da precisão em vários níveis de recall)
\begin{equacaodestaque}{Average Precision (AP)}
    AP = \sum_{k=1}^{N} (r_k - r_{k-1}) p_k
    \label{eq:equacao-ap}
\end{equacaodestaque}
% Onde p_k e r_k são a precisão e o recall no k-ésimo limiar.
% Uma definição alternativa comum é a integral: AP = \int_{0}^{1} p(r) dr

\medskip
\begin{center}
 * * *
\end{center}
\medskip

\textbf{Algumas Aplicações do xxx}
\vspace{1em}

\begin{itemize}
    \item \textbf{Aplicação 1 (Área):}
    \item \textbf{Aplicação 2 (Área):}
    \item \textbf{Aplicação 3 (Área):}
\end{itemize}

\medskip
\begin{center}
 * * *
\end{center}
\medskip

\subsection{Average Recall (AR)}
\label{sec:average-recall-ar}
% (Média do recall em vários limiares de IoU ou contagens de detecção)

\medskip
\begin{center}
 * * *
\end{center}
\medskip

\textbf{Algumas Aplicações do xxx}
\vspace{1em}

\begin{itemize}
    \item \textbf{Aplicação 1 (Área):}
    \item \textbf{Aplicação 2 (Área):}
    \item \textbf{Aplicação 3 (Área):}
\end{itemize}

\medskip
\begin{center}
 * * *
\end{center}
\medskip

% ----------------------------------------------------------------------
\section{Métricas Para Segmentação de Imagens}
\label{sec:metricas-segmentacao-imagens}
% (Muitas métricas aqui também dependem da IoU, calculada em nível de pixel)

% Corrigido o typo "Accuarcy" e traduzido
\subsection{Acurácia de Pixel (Pixel Accuracy)}
\label{sec:pixel-accuracy}
\begin{equacaodestaque}{Acurácia de Pixel}
    PA = \frac{\sum_{i} \text{Pixels Corretos}}{\sum_{i} \text{Total de Pixels}} = \frac{VP + VN}{VP + VN + FP + FN}
    \label{eq:equacao-pixel-accuracy}
\end{equacaodestaque}

\medskip
\begin{center}
 * * *
\end{center}
\medskip

\textbf{Algumas Aplicações do xxx}
\vspace{1em}

\begin{itemize}
    \item \textbf{Aplicação 1 (Área):}
    \item \textbf{Aplicação 2 (Área):}
    \item \textbf{Aplicação 3 (Área):}
\end{itemize}

\medskip
\begin{center}
 * * *
\end{center}
\medskip

\subsection{Boundary F1-Score (BF)}
\label{sec:boundary-f1}
% (É o F1-Score padrão, mas calculado apenas em pixels próximos às bordas)

\medskip
\begin{center}
 * * *
\end{center}
\medskip

\textbf{Algumas Aplicações do xxx}
\vspace{1em}

\begin{itemize}
    \item \textbf{Aplicação 1 (Área):}
    \item \textbf{Aplicação 2 (Área):}
    \item \textbf{Aplicação 3 (Área):}
\end{itemize}

\medskip
\begin{center}
 * * *
\end{center}
\medskip

\subsection{Masked Average Precision (Mask AP)}
\label{sec:mask-ap}
% (É o mesmo conceito do AP, mas usando a IoU das máscaras de segmentação)

\subsection{Panoptic Quality (PQ)}
\label{sec:panoptic-quality-pq}
\begin{equacaodestaque}{Panoptic Quality (PQ)}
    PQ = \underbrace{\frac{\sum_{(p,g) \in TP} \text{IoU}(p,g)}{|TP|}}_{\text{Qualidade de Segmentação (SQ)}} \times \underbrace{\frac{|TP|}{|TP| + \frac{1}{2}|FP| + \frac{1}{2}|FN|}}_{\text{Qualidade de Reconhecimento (RQ)}}
    \label{eq:equacao-pq}
\end{equacaodestaque}

\medskip
\begin{center}
 * * *
\end{center}
\medskip

\textbf{Algumas Aplicações do xxx}
\vspace{1em}

\begin{itemize}
    \item \textbf{Aplicação 1 (Área):}
    \item \textbf{Aplicação 2 (Área):}
    \item \textbf{Aplicação 3 (Área):}
\end{itemize}

\medskip
\begin{center}
 * * *
\end{center}
\medskip

% ----------------------------------------------------------------------
\section{Métricas Para Geração de Imagens}
\label{sec:metricas-geracao-imagens}

\subsection{Peak Signal-to-Noise Ratio (PSNR)}
\label{sec:psnr}
\begin{equacaodestaque}{Peak Signal-to-Noise Ratio (PSNR)}
    PSNR = 10 \cdot \log_{10} \left( \frac{MAX_I^2}{MSE} \right)
    \label{eq:equacao-psnr}
\end{equacaodestaque}
% Onde MAX_I é o valor máximo possível do pixel (ex: 255)
% e MSE é o Erro Quadrático Médio entre as imagens.

\medskip
\begin{center}
 * * *
\end{center}
\medskip

\textbf{Algumas Aplicações do xxx}
\vspace{1em}

\begin{itemize}
    \item \textbf{Aplicação 1 (Área):}
    \item \textbf{Aplicação 2 (Área):}
    \item \textbf{Aplicação 3 (Área):}
\end{itemize}

\medskip
\begin{center}
 * * *
\end{center}
\medskip

\subsection{Structural Similarity Index (SSIM)}
\label{sec:ssim}
\begin{equacaodestaque}{Structural Similarity Index (SSIM)}
    SSIM(x,y) = \frac{(2\mu_x\mu_y + C_1)(2\sigma_{xy} + C_2)}{(\mu_x^2 + \mu_y^2 + C_1)(\sigma_x^2 + \sigma_y^2 + C_2)}
    \label{eq:equacao-ssim}
\end{equacaodestaque}
% \mu é a média, \sigma^2 é a variância, \sigma_{xy} é a covariância.

\medskip
\begin{center}
 * * *
\end{center}
\medskip

\textbf{Algumas Aplicações do xxx}
\vspace{1em}

\begin{itemize}
    \item \textbf{Aplicação 1 (Área):}
    \item \textbf{Aplicação 2 (Área):}
    \item \textbf{Aplicação 3 (Área):}
\end{itemize}

\medskip
\begin{center}
 * * *
\end{center}
\medskip

\subsection{Inception Score (IS)}
\label{sec:inception-score-is}
\begin{equacaodestaque}{Inception Score (IS)}
    IS(G) = \exp\left( \mathbb{E}_{x \sim p_g} D_{KL}(p(y|x) || p(y)) \right)
    \label{eq:equacao-is}
\end{equacaodestaque}
% Mede a divergência KL entre a distribuição de classes (p(y|x)) e a marginal (p(y)).

\medskip
\begin{center}
 * * *
\end{center}
\medskip

\textbf{Algumas Aplicações do xxx}
\vspace{1em}

\begin{itemize}
    \item \textbf{Aplicação 1 (Área):}
    \item \textbf{Aplicação 2 (Área):}
    \item \textbf{Aplicação 3 (Área):}
\end{itemize}

\medskip
\begin{center}
 * * *
\end{center}
\medskip

\subsection{Fréchet Inception Distance (FID)}
\label{sec:fid}
\begin{equacaodestaque}{Fréchet Inception Distance (FID)}
    FID = ||\mu_r - \mu_g||_2^2 + \text{Tr}\left(\Sigma_r + \Sigma_g - 2(\Sigma_r \Sigma_g)^{1/2}\right)
    \label{eq:equacao-fid}
\end{equacaodestaque}
% \mu_r, \mu_g são as médias e \Sigma_r, \Sigma_g são as matrizes de covariância
% das ativações do Inception para imagens reais (r) e geradas (g).

\medskip
\begin{center}
 * * *
\end{center}
\medskip

\textbf{Algumas Aplicações do xxx}
\vspace{1em}

\begin{itemize}
    \item \textbf{Aplicação 1 (Área):}
    \item \textbf{Aplicação 2 (Área):}
    \item \textbf{Aplicação 3 (Área):}
\end{itemize}

\medskip
\begin{center}
 * * *
\end{center}
\medskip

% ----------------------------------------------------------------------
\section{Métricas Para Processamento de Linguagem Natural}
\label{sec:metricas-pln}

\subsection{Accuracy}
% (Reutilização)

\medskip
\begin{center}
 * * *
\end{center}
\medskip

\textbf{Algumas Aplicações do xxx}
\vspace{1em}

\begin{itemize}
    \item \textbf{Aplicação 1 (Área):}
    \item \textbf{Aplicação 2 (Área):}
    \item \textbf{Aplicação 3 (Área):}
\end{itemize}

\medskip
\begin{center}
 * * *
\end{center}
\medskip

\subsection{Precision, Recall and F1-Score}
% (Reutilização)

\medskip
\begin{center}
 * * *
\end{center}
\medskip

\textbf{Algumas Aplicações do xxx}
\vspace{1em}

\begin{itemize}
    \item \textbf{Aplicação 1 (Área):}
    \item \textbf{Aplicação 2 (Área):}
    \item \textbf{Aplicação 3 (Área):}
\end{itemize}

\medskip
\begin{center}
 * * *
\end{center}
\medskip

\subsection{AUC-ROC}
% (Reutilização)

\medskip
\begin{center}
 * * *
\end{center}
\medskip

\textbf{Algumas Aplicações do xxx}
\vspace{1em}

\begin{itemize}
    \item \textbf{Aplicação 1 (Área):}
    \item \textbf{Aplicação 2 (Área):}
    \item \textbf{Aplicação 3 (Área):}
\end{itemize}

\medskip
\begin{center}
 * * *
\end{center}
\medskip

\subsection{BLEU Score}
\label{sec:bleu}
\begin{equacaodestaque}{BLEU (Bilingual Evaluation Understudy)}
    BLEU = \text{BP} \cdot \exp\left(\sum_{n=1}^{N} w_n \log p_n\right)
    \label{eq:equacao-bleu}
\end{equacaodestaque}
% BP é a "Brevity Penalty" (penalidade de brevidade).
% p_n é a precisão dos n-gramas modificados.

\medskip
\begin{center}
 * * *
\end{center}
\medskip

\textbf{Algumas Aplicações do xxx}
\vspace{1em}

\begin{itemize}
    \item \textbf{Aplicação 1 (Área):}
    \item \textbf{Aplicação 2 (Área):}
    \item \textbf{Aplicação 3 (Área):}
\end{itemize}

\medskip
\begin{center}
 * * *
\end{center}
\medskip

\subsection{METEOR}
\label{sec:meteor}
% (Métrica complexa baseada em alinhamento e F-score; sem fórmula única simples)

\medskip
\begin{center}
 * * *
\end{center}
\medskip

\textbf{Algumas Aplicações do xxx}
\vspace{1em}

\begin{itemize}
    \item \textbf{Aplicação 1 (Área):}
    \item \textbf{Aplicação 2 (Área):}
    \item \textbf{Aplicação 3 (Área):}
\end{itemize}

\medskip
\begin{center}
 * * *
\end{center}
\medskip

\subsection{ROUGE Score}
\label{sec:rouge}
% (É um conjunto de métricas; ROUGE-N é o mais comum)
\begin{equacaodestaque}{ROUGE-N (Recall-Oriented Understudy for Gisting Evaluation)}
    \text{ROUGE-N} = \frac{\sum_{S \in \text{Ref}} \sum_{g_n \in S} \text{Count}_{\text{match}}(g_n)}{\sum_{S \in \text{Ref}} \sum_{g_n \in S} \text{Count}(g_n)}
    \label{eq:equacao-rouge-n}
\end{equacaodestaque}

\medskip
\begin{center}
 * * *
\end{center}
\medskip

\textbf{Algumas Aplicações do xxx}
\vspace{1em}

\begin{itemize}
    \item \textbf{Aplicação 1 (Área):}
    \item \textbf{Aplicação 2 (Área):}
    \item \textbf{Aplicação 3 (Área):}
\end{itemize}

\medskip
\begin{center}
 * * *
\end{center}
\medskip

\subsection{Perplexidade (Perplexity)}
\label{sec:perplexity}
\begin{equacaodestaque}{Perplexidade (Perplexity)}
    PPL(W) = \exp\left( -\frac{1}{N} \sum_{i=1}^{N} \log P(w_i | w_1, ..., w_{i-1}) \right)
    \label{eq:equacao-perplexity}
\end{equacaodestaque}
% É o exponencial da entropia cruzada (Cross-Entropy).

\medskip
\begin{center}
 * * *
\end{center}
\medskip

\textbf{Algumas Aplicações do xxx}
\vspace{1em}

\begin{itemize}
    \item \textbf{Aplicação 1 (Área):}
    \item \textbf{Aplicação 2 (Área):}
    \item \textbf{Aplicação 3 (Área):}
\end{itemize}

\medskip
\begin{center}
 * * *
\end{center}
\medskip

\subsection{Correspondência Exata (Exact Match - EM)}
\label{sec:exact-match}
\begin{equacaodestaque}{Correspondência Exata (Exact Match - EM)}
    EM = \frac{\text{Número de Predições Idênticas à Referência}}{\text{Número Total de Predições}}
    \label{eq:equacao-em}
\end{equacaodestaque}

\medskip
\begin{center}
 * * *
\end{center}
\medskip

\textbf{Algumas Aplicações do xxx}
\vspace{1em}

\begin{itemize}
    \item \textbf{Aplicação 1 (Área):}
    \item \textbf{Aplicação 2 (Área):}
    \item \textbf{Aplicação 3 (Área):}
\end{itemize}

\medskip
\begin{center}
 * * *
\end{center}
\medskip

\subsection{Taxa de Erro de Palavra (Word Error Rate - WER)}
\label{sec:wer}
\begin{equacaodestaque}{Taxa de Erro de Palavra (WER)}
    WER = \frac{S + D + I}{N}
    \label{eq:equacao-wer}
\end{equacaodestaque}
% S = Substituições, D = Deleções, I = Inserções
% N = Número de palavras no texto de referência

\medskip
\begin{center}
 * * *
\end{center}
\medskip

\textbf{Algumas Aplicações do xxx}
\vspace{1em}

\begin{itemize}
    \item \textbf{Aplicação 1 (Área):}
    \item \textbf{Aplicação 2 (Área):}
    \item \textbf{Aplicação 3 (Área):}
\end{itemize}

\medskip
\begin{center}
 * * *
\end{center}
\medskip

\subsection{Taxa de Erro de Caractere (Character Error Rate - CER)}
\label{sec:cer}
\begin{equacaodestaque}{Taxa de Erro de Caractere (CER)}
    CER = \frac{S + D + I}{C}
    \label{eq:equacao-cer}
\end{equacaodestaque}
% S = Substituições, D = Deleções, I = Inserções
% C = Número de caracteres no texto de referência

\medskip
\begin{center}
 * * *
\end{center}
\medskip

\textbf{Algumas Aplicações do xxx}
\vspace{1em}

\begin{itemize}
    \item \textbf{Aplicação 1 (Área):}
    \item \textbf{Aplicação 2 (Área):}
    \item \textbf{Aplicação 3 (Área):}
\end{itemize}

\medskip
\begin{center}
 * * *
\end{center}
\medskip