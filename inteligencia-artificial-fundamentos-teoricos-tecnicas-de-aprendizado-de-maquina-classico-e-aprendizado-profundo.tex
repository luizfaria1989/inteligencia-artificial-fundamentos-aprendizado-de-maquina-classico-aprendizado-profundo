% ===================================================================
% ARQUIVO MESTRE DO LIVRO
% Autor: Seu Nome
% Projeto: Um Mergulho Profundo no Aprendizado de Máquina
% ===================================================================

% --- CLASSE DO DOCUMENTO ---
% Usamos a classe 'abntex2' para seguir as normas da ABNT.
% As opções configuram o formato para um livro acadêmico padrão.
\documentclass[
    12pt,            % Tamanho da fonte do corpo do texto
    a4paper,         % Tamanho do papel
    book,            % Formato de livro
    openright,       % Força capítulos a começarem em páginas ímpares (da direita)
    twoside,         % Layout para impressão frente e verso (margens diferentes)
    brazil,          % Configurações para o idioma português do Brasil
    citacao=authoryear
]{abntex2}


% --- PREÂMBULO ---
% Importa todas as configurações, pacotes e comandos customizados
% do arquivo 'preambulo.tex'. Isso mantém este arquivo principal limpo
% e focado apenas na estrutura do conteúdo do livro.
% ===================================================================
% ARQUIVO DE PREÂMBULO
% Contém todas as configurações de pacotes e layout para o livro.
% ===================================================================


% --- CONFIGURAÇÕES ESSENCIAIS DO DOCUMENTO ---

% Codificacao da fonte. Essencial para a correta exibição de caracteres acentuados no PDF.
\usepackage[T1]{fontenc}

% Codificacao de entrada do arquivo .tex. Permite que você escreva com acentos diretamente.
\usepackage[utf8]{inputenc}

% Suporte ao idioma Português do Brasil.
% A classe abntex2 já cuida disso, mas é uma boa prática garantir.
% A opção 'brazil' ajusta hifenizacao, nomes de seções ("Sumário"), etc.
\usepackage[brazil]{babel}

% Para usar a fonte Palatino
\usepackage{mathpazo}

% --- PACOTES PARA CONTEÚDO TÉCNICO E MATEMÁTICO ---

% Pacotes padrao e essenciais para matemática avançada.
\usepackage{amsmath, amsfonts, amssymb}

% Pacote para inclusão de imagens.
\usepackage{graphicx}
% Define um ou mais diretórios padrão para suas imagens.
% Isso evita que você precise digitar "imagens/" toda vez.
\graphicspath{ {./imagens/} }

% Pacote para formatar e incluir snippets de código-fonte.
\usepackage{listings}
\usepackage{xcolor} % Necessário para definir cores no listings

% Define um estilo customizado para código Python, para ficar mais bonito.
\lstdefinestyle{PythonStyle}{
    language=Python,
    backgroundcolor=\color{gray!10},   % Cor de fundo leve
    basicstyle=\ttfamily\small,       % Fonte monoespaçada e pequena
    keywordstyle=\color{blue},        % Palavras-chave em azul
    commentstyle=\color{green!50!black},% Comentários em verde escuro
    stringstyle=\color{purple},       % Strings em roxo
    breaklines=true,                  % Quebra de linhas longas
    showstringspaces=false,           % Não mostra espaços em strings com um símbolo
    frame=single,                     % Adiciona uma moldura
    rulecolor=\color{black!30},       % Cor da moldura
    numbers=left,                     % Numeração de linhas à esquerda
    numberstyle=\tiny\color{gray},    % Estilo dos números de linha
}


% --- REFERÊNCIAS, CITAÇÕES E LINKS (PADRÃO ABNT) ---

% O sistema moderno e mais poderoso para bibliografia em LaTeX, já configurado para ABNT.
\usepackage[
    backend=biber,     % Ferramenta que processa a bibliografia (mais moderna que bibtex)
    style=abnt,        % Estilo de citação e referência da ABNT
    ittitles,          % Títulos de artigos, livros, etc., em itálico
    backref=true,      % Nas referências, mostra em quais páginas a citação aparece
]{biblatex}

\renewcommand*{\mkbibnamefamily}[1]{\MakeUppercase{#1}}
\renewcommand*{\mkbibnamelast}[1]{\MakeUppercase{#1}}

% Aponta para o arquivo que contém seu banco de dados de referências.
\addbibresource{bibliografia.bib}

% Melhora a formatação de URLs nas referências.
\usepackage{url}
\urlstyle{same} % Usa a mesma fonte do texto para as URLs.

% --- LAYOUT E ESTRUTURA ---

% Pacote para ajustar as margens e geometria da página.
% A classe abntex2 já define margens padrão ABNT, use para sobrescrever se necessário.
\usepackage{geometry}
\geometry{
    a4paper,
    left=3cm,
    right=2cm,
    top=3cm,
    bottom=2cm
}

% --- CONFIGURAÇÃO DO HYPERREF ---
% A classe abntex2 já carrega o pacote hyperref.
% Apenas usamos \hypersetup para definir as opções que queremos.

\hypersetup{
    pdftitle={O Título do Seu Livro},
    pdfauthor={Seu Nome},
    pdfsubject={Assunto do Livro},
    pdfkeywords={Palavra-chave1, Palavra-chave2},
    colorlinks=true,                % Habilita links coloridos
    linkcolor=blue,                 % Cor dos links internos (sumário, etc.)
    citecolor=red,                  % Cor das citações
    urlcolor=cyan,                  % Cor das URLs
    hidelinks=false,                % Se 'true', esconde as caixas e cores (bom para impressão)
    pdfstartview={FitH},            % Abre o PDF ajustado à largura da página
    bookmarksopen=true              % Abre o painel de bookmarks (índices)
}

% Pacote para customização avançada de cabeçalhos e rodapés.
% Descomente as linhas abaixo apenas se precisar de um layout diferente
% do padrão oferecido pelo abntex2.
%
% \usepackage{fancyhdr}
% \pagestyle{fancy}
% \fancyhf{} % Limpa todos os campos
% \fancyhead[LE,RO]{\nouppercase{\leftmark}} % Nome do capítulo no cabeçalho
% \fancyfoot[LE,RO]{\thepage} % Número da página no rodapé


% --- COMANDOS CUSTOMIZADOS (OPCIONAL) ---
% Aqui você pode definir seus próprios comandos para agilizar a escrita.
% Exemplo:
% \newcommand{\keras}{\texttt{Keras}}

\usepackage{tikz}
\usetikzlibrary{patterns, positioning, fit, shapes.geometric, arrows.meta, fpu, decorations.pathmorphing, shadows, shapes.symbols}
\usepackage{pgfplots}
\pgfplotsset{compat=1.18}

% --- Definindo as cores (hues) ---
\definecolor{darkblue}{rgb}{0.16, 0.32, 0.75}
\definecolor{darkorange}{rgb}{0.9, 0.35, 0.0}
\definecolor{pointred}{rgb}{0.85, 0.2, 0.2}

\usepackage[most]{tcolorbox}
\tcbuselibrary{skins} % Necessário para o estilo enhanced

\newtcolorbox{definicaomoderna}{
    enhanced,
    frame hidden, % Remove a moldura da caixa
    borderline west={2pt}{0pt}{gray!80}, % Cria uma linha de 2pt à esquerda (west)
    colback=gray!5, % Um fundo cinza muito sutil para diferenciar
    fonttitle=\bfseries,
    coltitle=black,
    sharp corners, % Cantos retos
    boxsep=5pt,
    left=10pt, % Espaço interno à esquerda
    attach boxed title to top left={yshift=-2mm, xshift=5mm}, % Posição do título
    boxed title style={
        frame hidden,
        colback=gray!5 % Fundo do título igual ao da caixa
    },
}

% --- Comando especial para equação destacada COM NOME ---
\newtcolorbox{equacaodestaque}[1]{ % O [1] indica que o comando aceita um argumento
    enhanced, ams equation,
    frame hidden,
    borderline west={2pt}{0pt}{gray!80},
    colback=gray!5,
    sharp corners,
    % --- Opções para o título ---
    fonttitle=\bfseries,
    coltitle=black,
    attach boxed title to top left={yshift=-2mm, xshift=5mm}, % Posiciona o título
    boxed title style={
        frame hidden,
        colback=gray!5 % Faz o fundo do título ser igual ao da caixa
    },
    title={#1} % Usa o argumento como título da caixa
}

\usepackage{xcolor} % Para definir cores customizadas
\usepackage{listings} % Para formatar o código
\usepackage[most]{tcolorbox} % Para criar a caixa estilizada
\usepackage{witharrows}

% --- 1. Definição das Cores para o Código ---
\definecolor{codegray}{rgb}{0.5,0.5,0.5}
\definecolor{codepurple}{rgb}{0.58,0,0.82}
\definecolor{codeblue}{rgb}{0,0,0.8}
\definecolor{codered}{rgb}{0.8,0,0}

% --- 2. Configuração Global do Pacote listings ---
\lstset{
    language=Python,
    backgroundcolor=\color{white},   
    commentstyle=\color{codegray}\itshape,
    keywordstyle=\color{codeblue}\bfseries,
    numberstyle=\tiny\color{codegray},
    stringstyle=\color{codered},
    basicstyle=\ttfamily\small,
    breakatwhitespace=false,         
    breaklines=true,                 
    captionpos=b,                    
    keepspaces=true,                 
    numbers=left,                    
    numbersep=5pt,                  
    showspaces=false,                
    showstringspaces=false,
    showtabs=false,                  
    tabsize=4
}

% --- Criação do Ambiente de Listagem de Código Estilizado (Fundo Branco) ---
\newtcblisting{codelisting}[2]{
  listing only,
  enhanced,
  colback=white,  % Fundo branco, igual ao da página
  colframe=gray!80,
  frame hidden,
  borderline west={2pt}{0pt}{gray!80},
  sharp corners,
  top=5pt,
  bottom=5pt,
  % --- Título e numeração (Listagem X: Título) ---
  fonttitle=\bfseries,
  coltitle=black,
  title={Bloco de Código~\thetcbcounter: #1},
  label={lst:#2},
  % --- Opções passadas para o pacote listings ---
  listing options={
    language=Python,
    numbers=left,
  }
}

\usepackage{algorithm} % Para o ambiente algorithm
\usepackage{algpseudocode} % Para os comandos do pseudocódigo

\algrenewcommand\algorithmicrequire{\textbf{Requer:}}
\algrenewcommand\algorithmicreturn{\textbf{Retorne:}}

\usepackage{threeparttable}

\usepackage{longtable}     % Para tabelas que ocupam várias páginas
\usepackage{booktabs}      % Para as linhas \toprule, \midrule, \bottomrule
\usepackage{amsmath}       % Para os ambientes matemáticos como 'cases'
\usepackage[labelfont=bf, singlelinecheck=false, labelsep=endash]{caption} % Para formatar o título

\usepackage{makecell}

\numberwithin{figure}{chapter}
\numberwithin{table}{chapter}
\numberwithin{equation}{chapter}

\usepackage{nomencl}
\usepackage{etoolbox} % Para customizações avançadas

% --- Comandos Customizados para Funções ---
\newcommand{\fativ}[1]{\ensuremath{\mathcal{A}_{#1}}}
\newcommand{\fperd}[1]{\ensuremath{\mathcal{L}_{#1}}}

\usepackage{ragged2e}   % Para um melhor alinhamento do texto justificado

\usepackage{titling}

\usepackage{subcaption}

\usepackage{mathrsfs}
\newsavebox\foobox
\newlength{\foodim}
\newcommand{\slantbox}[2][0]{\mbox{%
        \sbox{\foobox}{#2}%
        \foodim=#1\wd\foobox
        \hskip \wd\foobox
        \hskip -0.5\foodim
        \pdfsave
        \pdfsetmatrix{1 0 #1 1}%
        \llap{\usebox{\foobox}}%
        \pdfrestore
        \hskip 0.5\foodim
}}
% \def\Loss{\slantbox[-.45]{$\mathscr{L}$}} % <-- Comment out or remove the old def
\DeclareRobustCommand{\Loss}{\slantbox[-.45]{$\mathscr{L}$}} % <-- Use this instead

% \def\Activation{\slantbox[-.45]{$\mathscr{A}$}} % <-- Comment out or remove the old def
\DeclareRobustCommand{\Activation}{\slantbox[-.45]{$\mathscr{A}$}} % <-- Use this instead

\usepackage{pdflscape}

\usepackage{imakeidx}
\makeindex[title=Índice Remissivo, intoc] % Cria o índice com título customizado e o adiciona ao sumário (intoc)

\usepackage[printonlyused, withpage]{acronym} % Opções recomendadas

\usepackage[toc, style=long]{glossaries} % 'toc' adds the glossary to your table of contents, which is usually nice.
\makeglossaries

% ===================================================================
% FIM DO PREÂMBULO
% ===================================================================


% --- INFORMAÇÕES DO DOCUMENTO (para a folha de rosto) ---
\title{INTELIGÊNCIA ARTIFICIAL \\ 
    \large Fundamentos Teóricos, Técnicas de Aprendizado de Máquina Clássico e Aprendizado Profundo}
\author{Luiz Guilherme Morais da Costa Faria}
\date{\today} % Usa a data atual no momento da compilação

% Informações adicionais que o abntex2 usa (opcional)
\instituicao{Universidade de Brasília}
\local{Brasília, DF}
\orientador{Nome do Orientador/Revisor (se aplicável)}


% ===================================================================
% INÍCIO DO DOCUMENTO
% ===================================================================
\begin{document}

\begin{acronym}
    % --- Siglas Gerais ---
    \acro{IA}{Inteligência Artificial}
    \acro{ML}{Aprendizado de Máquina (Machine Learning)} % Embora não explícito, é o tema central
    \acro{RNA}{Rede Neural Artificial} % Implícito no contexto de redes neurais
    \acro{DNN}{Rede Neural Profunda (Deep Neural Network)}
    \acro{GPU}{Unidade de Processamento Gráfico (Graphics Processing Unit)} % Mencionada no contexto de processamento

    % --- Otimizadores (Cap. 6 e Apêndice A) ---
    \acro{GD}{Gradiente Descendente (Gradient Descent)}
    \acro{SGD}{Gradiente Descendente Estocástico (Stochastic Gradient Descent)}
    \acro{NAG}{Gradiente Acelerado de Nesterov (Nesterov Accelerated Gradient)}
    \acro{AdaGrad}{Adaptive Gradient Algorithm}
    \acro{RMSProp}{Root Mean Square Propagation}
    \acro{Adam}{Adaptive Moment Estimation}
    \acro{AdaMax}{Adaptive Moment Estimation based on the infinity norm}
    \acro{Nadam}{Nesterov-accelerated Adaptive Moment Estimation}
    \acro{AdamW}{Adam with Decoupled Weight Decay}
    \acro{RAdam}{Rectified Adam}

    % --- Funções de Ativação (Cap. 7, 8, 9 e Apêndice B) ---
    \acro{ReLU}{Unidade Linear Retificada (Rectified Linear Unit)}
    \acro{LReLU}{Leaky Rectified Linear Unit}
    \acro{PReLU}{Parametric Rectified Linear Unit}
    \acro{RReLU}{Randomized Leaky Rectified Linear Unit}
    \acro{ELU}{Unidade Linear Exponencial (Exponential Linear Unit)}
    \acro{SELU}{Unidade Linear Exponencial Escalonada (Scaled Exponential Linear Unit)}
    \acro{GELU}{Unidade Linear de Erro Gaussiano (Gaussian Error Linear Unit)}
    \acro{SiLU}{Sigmoid Linear Unit}
    \acro{h-swish}{Hard-Swish}
    \acro{h-mish}{Hard-Mish}

    % --- Funções de Perda (Cap. 10, 11, 12 e Apêndice C) ---
    \acro{MSE}{Erro Quadrático Médio (Mean Squared Error)}
    \acro{MAE}{Erro Absoluto Médio (Mean Absolute Error)}
    \acro{MSLE}{Erro Quadrático Médio Logarítmico (Mean Squared Logarithmic Error)}
    \acro{MAPE}{Erro Percentual Absoluto Médio (Mean Absolute Percentage Error)}
    \acro{sMAPE}{Erro Percentual Absoluto Médio Simétrico (Symmetric Mean Absolute Percentage Error)}
    \acro{BCE}{Entropia Cruzada Binária (Binary Cross-Entropy)}
    \acro{WCE}{Entropia Cruzada Ponderada Binária (Binary Weighted Cross-Entropy)}
    \acro{CCE}{Entropia Cruzada Categórica (Categorical Cross-Entropy)}
    \acro{WCCE}{Weighted Categorical Cross-Entropy}
    \acro{KL}{Divergência Kullback-Leibler (Kullback-Leibler Divergence)}
    \acro{FL}{Focal Loss}

    % --- Métricas (Cap. 13 e Apêndice D) ---
    \acro{VP}{Verdadeiro Positivo (True Positive)}
    \acro{VN}{Verdadeiro Negativo (True Negative)}
    \acro{FP}{Falso Positivo (False Positive)}
    \acro{FN}{Falso Negativo (False Negative)}
    \acro{ROC}{Característica de Operação do Receptor (Receiver Operating Characteristic)}
    \acro{AUC}{Área Sob a Curva (Area Under the Curve)}
    \acro{RMSE}{Raiz do Erro Quadrático Médio (Root Mean Square Error)}
    \acro{R2}{Coeficiente de Determinação (Coefficient of Determination)}

    % --- Redes Neurais e Componentes (Parte V) ---
    \acro{MLP}{Perceptron Multicamadas (Multi-Layer Perceptron)}
    \acro{FFN}{Rede FeedForward (FeedForward Network)}
    \acro{DBN}{Rede de Crença Profunda (Deep Belief Network)}
    \acro{RBM}{Máquina de Boltzmann Restrita (Restricted Boltzmann Machine)}
    \acro{CNN}{Rede Neural Convolucional (Convolutional Neural Network)}
    \acro{FCN}{Rede Totalmente Convolucional (Fully Convolutional Network)}
    \acro{YOLO}{You Only Look Once}
    \acro{ResNet}{Rede Residual (Residual Network)}
    \acro{SENet}{Squeeze-and-Excitation Network}
    \acro{RNN}{Rede Neural Recorrente (Recurrent Neural Network)}
    \acro{LSTM}{Memória Longa de Curto Prazo (Long Short-Term Memory)}
    \acro{GRU}{Unidade Recorrente Gated (Gated Recurrent Unit)}
    \acro{ViT}{Vision Transformer}
    \acro{GAN}{Rede Adversária Generativa (Generative Adversarial Network)}
    \acro{MoE}{Mistura de Especialistas (Mixture of Experts)}
    \acro{GNN}{Rede Neural de Grafos (Graph Neural Network)}

    % --- Outras ---
    \acro{PCA}{Análise de Componentes Principais (Principal Component Analysis)}
    \acro{SVD}{Decomposição em Valores Singulares (Singular Value Decomposition)}
    \acro{t-SNE}{t-Distributed Stochastic Neighbor Embedding}
    \acro{UMAP}{Uniform Manifold Approximation and Projection}
    \acro{DBSCAN}{Density-Based Spatial Clustering of Applications with Noise}
    \acro{SVM}{Máquina de Vetores de Suporte (Support Vector Machine)}
    \acro{ILSVRC}{Desafio de Reconhecimento Visual em Larga Escala ImageNet (ImageNet Large Scale Visual Recognition Challenge)}
    \acro{SIFT}{Scale-Invariant Feature Transform}
    \acro{FVs}{Fisher Vectors}
    \acro{SNN}{Rede Neural Auto-Normalizadora (Self-Normalizing Neural Network)}
    \acro{WER}{Taxa de Erro de Palavra (Word Error Rate)}
    \acro{SWBD}{Switchboard (dataset)}
    \acro{CH}{CallHome (dataset)}
    \acro{EV}{Conjunto de Avaliação (Evaluation set)}
    \acro{LAD}{Mínimo Desvio Absoluto (Least Absolute Deviation)}
    \acro{EW-RSM}{Erro Quadrático Médio Exponencialmente Ponderado (Exponentially Weighted Mean Squared Error)}
    \acro{fMAE}{Erro Absoluto Médio Ponderado pela Frequência (Frequency-weighted Mean Absolute Error)}
    \acro{RDA}{Regularized Dual Averaging}
    \acro{FB}{Forward-Backward splitting} 
    \acro{PA}{Passive-Aggressive algorithms} 
    \acro{AROW}{Adaptive Regularization of Weight Vectors} 
\end{acronym}

\newglossaryentry{labelUnica}{
    name={Nome do Termo},
    description={A explicação clara e concisa do que diabos esse termo significa. Pode ser longa, pode ter \textit{itálico}, o que você precisar.},
    symbol={\ensuremath{\eta}} % Opcional: Se tiver um símbolo associado, use \ensuremath{} para modo matemático
}

% Exemplo Concreto do seu livro:
\newglossaryentry{retropapagacao}{
    name={Retropropagação},
    description={Um dos principais algoritmos para o treinamento de redes neurais. Permite o aprendizado atráves do ajuste sucessivo dos parâmetros da rede com auxílio do cálculo do gradiente da perda propagado das camadas finais para as camadas iniciais.} \index{Glossário!Retropropagação} % Indexar no índice remissivo também? Boa ideia!
}

\newglossaryentry{gradiente-descendente}{ 
    name={Gradiente descendente},
    description={Método de otimização iterativo que tem como objetivo encontrar pontos de mínimo de uma função (geralmente funções de perda) dando pequenos "passos" na direção contrária do vetor gradiente.} \index{Glossário!Gradiente Descendente}
}

\newglossaryentry{sigmoide}{
    name={Sigmoide logística},
    description={Função de ativação do tipo sigmoidal que foi comumente empregada em redes neurais antes da popularização das funções retificadoras. Pode ser utilizada na camada de saída de um modelo para que a sua saída fique limitada em um intervalo [0,1], sendo útil para problemas de classificação binárias.}
    symbol={\Loss_{\text{sigmoid}}}
}

\newglossaryentry{tangente-hiperbolica}{
    name={Tangente hiperbólica},
    description={Função de ativação do tipo sigmoidal que, assim como a sigmoide, foi amplamente utilizada em redes neurais antes do surgimento das funções retificadoras. Ela é uma função limitada em um intervalo de [-1,1], que tem como característica empurrar os valores de sua entrada para esses extremos.}
    symbol={\Loss_{\text{tanh}}}
}

% --- ELEMENTOS PRÉ-TEXTUAIS ---
% O comando \frontmatter inicia a contagem de páginas em algarismos romanos (i, ii, ...)
% e desativa a numeração de capítulos para os elementos iniciais.
\frontmatter

% Gera a capa e a folha de rosto com base nas informações acima,
% seguindo o padrão ABNT.
\imprimircapa
\imprimirfolhaderosto

% Páginas opcionais como Dedicatória, Agradecimentos, Epígrafe...
% \begin{dedicatoria}
%    \vspace*{\fill} % Centraliza verticalmente
%    \noindent
%    \textit{Para ...}
%    \vspace*{\fill}
% \end{dedicatoria}

% Gera o Sumário automaticamente com base nos comandos \part, \chapter, \section, etc.
\tableofcontents

% --- IMPRIMIR A LISTA DE SIGLAS ---
\cleardoublepage
\phantomsection
\chapter*{Lista de Siglas}
\addcontentsline{toc}{chapter}{Lista de Siglas}

% --- ELEMENTOS TEXTUAIS (O CONTEÚDO PRINCIPAL DO LIVRO) ---
% O comando \mainmatter reinicia a contagem de páginas em algarismos arábicos (1, 2, ...)
% e reativa a numeração de capítulos.
\mainmatter

% =======================================================
% PARTE I: HISTÓRIA DA IA E DO COMPUTADOR
% =======================================================
\part{História da IA e do Computador}

% ===================================================================
% Arquivo: capitulos/parte-01-historia/cap-01-historia-do-computador.tex
% ===================================================================

\chapter{Uma Breve História do Computador}
\label{cap:historia-computador}

O texto do seu capítulo começa aqui...
% ===================================================================
% Arquivo: capitulos/parte-01-historia/cap-01-historia-da-ia.tex
% ===================================================================

\chapter{Uma Breve História da Inteligência Artificial}
\label{cap:historia-ia}

O texto do seu capítulo começa aqui...

% =======================================================
% PARTE II: Conceitos Matemáticos
% =======================================================
\part{Conceitos Matemáticos}

% ===================================================================
% Arquivo: capitulos/parte-II-matematica/cap-03-calculo.tex
% ===================================================================

\chapter{Cálculo para Aprendizado de Máquina}
\label{cap:calculo-ia}

% ===================================================================
% Resumo do capítulo
% ===================================================================

% ===================================================================
% Funções
% ===================================================================

\section{Funções: A Base do Cálculo}

\section{Derivadas Ordinárias}

\section{Integrais Simples}

\section{Derivadas Parciais}
% ===================================================================
% Arquivo: capitulos/parte-II-matematica/cap-04-algebra-linear.tex
% ===================================================================

\chapter{Álgebra Linear para Aprendizado de Máquina}
\label{cap:algebra-linear-ia}

\section{A Unidade Fundamental: Vetores e Espaços Vetoriais}

\section{Organizando Dados: Matrizes e Suas Operações}

\section{Tensores: A Estrutura de Dados do Deep Learning} 

\section{Resolvendo Sistemas e Encontrando Propriedades: Autovalores e Autovetores}

\section{Decomposição de Matrizes (SVD e PCA)}
% ===================================================================
% Arquivo: capitulos/parte-II-matematica/cap-05-probabilidade-e-estatistica.tex
% ===================================================================

\chapter{Probabilidade e Estatística para Aprendizado de Máquina}
\label{cap:probabilidade-e-estatistica-ia}

O texto do seu capítulo começa aqui...

% =======================================================
% PARTE III: Pilares das Redes Neurais
% =======================================================
\part{Pilares das Redes Neurais}

% ===================================================================
% Arquivo: capitulos/parte-III-pilares/cap-06-retropropagacao-e-gradiente.tex
% ===================================================================

\chapter{O Algoritmo da Repropropagação e Os Otimizadores Baseados em Gradiente}
\label{cap:retropropagacao-gradiente}

% ===================================================================
% Resumo do capítulo
% ===================================================================

% ===================================================================
% Método do Gradiente Descendente
% ===================================================================

\section{O Método do Gradiente Descendente}

\subsection{Exemplo Ilustrativo}

\subsection{O Método em Si}

\subsection{Implentação em Python}

% ===================================================================
% A Retropropagação
% ===================================================================

\section{A Retropropagação: Aprendendo com os Erros}

% ===================================================================
% Otimizadores Baseados em Gradiente
% ===================================================================

\section{Otimizadores Baseados em Gradiente}

\subsection{Método do Gradiente Estocástico}

\subsection{Método do Gradiente com Momentum}

\subsection{Nesterov}

\subsection{AdaGrad}

\subsection{RMSProp}

\subsection{Adam}

\subsection{Nadam}

% ===================================================================
% Método de Newton
% ===================================================================

\section{O Método de Newton: Indo Além do Gradiente}


% ===================================================================
% Arquivo: capitulos/parte-III-pilares/cap-07-sigmoidais.tex
% ===================================================================

\chapter{Funções de Ativação Sigmoidais}
\label{cap:ativacao-sigmoidais}

% ===================================================================
% Resumo do capítulo
% ===================================================================

% ===================================================================
% Teoremas da Aproximação Universal
% ===================================================================

\section{Teoremas da Aproximação Universal}

\section{Exemplos Ilustrativo}

\section{A Sigmoide Logística}

\subsection{Implementação em Python}

\begin{codelisting}{Classe completa do função de ativação Sigmoid}{gd_class}
import numpy as np

class Sigmoid(Layer):
    def __init__(self):
        super().__init__()
        self.input = None
        self.sigmoid = None

    def forward(self, input_data):
        self.input = input_data
        self.sigmoid_output = 1/ (1 + np.exp(-input_data))
        return self.sigmoid_output

    def backward(self, grad_output):
        sigmoid_grad = self.sigmoid_output * (1 - self.sigmoid_output)
        return grad_output * sigmoid_grad, None
\end{codelisting}

\section{Tangente Hiperbólica}

\subsection{Implementação em Python}

\begin{codelisting}{Classe completa do função de ativação Tangente Hiperbólica}{gd_class}
import numpy as np
from layers.base import Layer  # Assuming your base class is here

class Tanh(Layer):
    def __init__(self):
        super().__init__()
        self.input = None
        self.tanh_output = None

    def forward(self, input_data):
        self.input = input_data
        self.tanh_output = np.tanh(self.input)
        return self.tanh_output

    def backward(self, grad_output):
        tanh_grad = 1 - self.tanh_output**2
        return grad_output * tanh_grad, None
\end{codelisting}

\section{Softsign: Uma Sigmoidal Mais Barata}

\subsection{Implementação em Python}

\begin{codelisting}{Classe completa do função de ativação Softsign}{gd_class}
from layers.base import Layer
import numpy as np

class Softsign(Layer):
    def __init__(self):
        super().__init__()
        self.input = None

    def forward(self, input_data):
        self.input = input_data
        return self.input / (1 + np.abs(self.input))

    def backward(self, grad_output):
        grad =  (1 / (1 + np.abs(self.input))**2)
        return grad_output * softsign_grad, None
\end{codelisting}

\section{Hard Sigmoid e Hard Tanh: O Sacrifício da Suavidade em Prol do Desempenho}

\subsection{Implementação em Python}

\begin{codelisting}{Classe completa do função de ativação Hard Sigmoid}{gd_class}
from layers.base import Layer
import numpy as np

class HardSigmoid(Layer):

    def __init__(self):
        super().__init__()
        self.input = None

    def forward(self, input_data):
        self.input = input_data

        output = self.input / 6 + 0.5
        output = np.clip(output, 0, 1)  # A more concise way to handle the bounds

        return output

    def backward(self, grad_output):
        hard_sigmoid_grad = np.full_like(self.input, 1 / 6)

        hard_sigmoid_grad[self.input < -3] = 0
        hard_sigmoid_grad[self.input > 3] = 0

        return grad_output * hard_sigmoid_grad, None
\end{codelisting}

\begin{codelisting}{Classe completa do função de ativação Hard Sigmoid}{gd_class}
from layers.base import Layer
import numpy as np


class HardTanh(Layer):
    def __init__(self):
        super().__init__()
        self.input = None

    def forward(self, input_data):
        self.input = input_data
        return np.clip(self.input, -1, 1)

    def backward(self, grad_output):

        hard_tanh_grad = np.where((self.input > -1) & (self.input < 1), 1, 0)

        return grad_output * hard_tanh_grad, None
\end{codelisting}

\section{O Desaparecimento de Gradientes}

\section{Comparativo de Desempenho das Sigmoidais}
% ===================================================================
% Arquivo: capitulos/parte-III-pilares/cap-08-retificadoras.tex
% ===================================================================

\chapter{Funções de Ativação Retificadoras}
\label{cap:ativacao-retificadoras}

% ===================================================================
% Resumo do capítulo
% ===================================================================

\section{Exemplo Ilustrativo: Vendendo Pipoca}

Imagine que você está querendo ganhar dinheiro e decidiu vender pipoca em uma praça da sua cidade. Você comprou milho, óleo, sal e manteiga, um carrinho para poder levar e fazer as pipocas, além disso, você também comprou vários pacotes para poder colocar as pipocas para vender.

Nisso, você teve que estipular um valor para vender essas pipocas, após pensar um pouco e analisar todos os seus gastos, você estimou que um valor de R\$ 5,00 seria ideal, pois conseguiria pagar os seus gastos mas você ainda ia obter lucro dos seus clientes.

Agora você está pronto para vender, começou a fritar o milho e colocou uma plaquinha com o preço ao lado do seu carrinho. Então chega uma pessoa com R\$ 6,00 e decide comprar um pacote, você vende e entrega um real de troco. Logo em seguida aparece uma segunda pessoa com R\$ 4,99 e decide negociar com você, ela afirma que é quase R\$ 5,00, e por isso, você deveria vender a pipoca para ela, mas você explica que só vende pelo valor de R\$ 5,00.

Com base nisso, nós podemos chegar em uma situação em que um pacote de pipoca será vendido somente se uma pessoa possuir R\$ 5,00 no bolso, ou mais. Podemos então escrever algo como o da equação \ref{eq: VendaPipoca}. Em casos em que uma venda ocorre, você poderá vender mais um pacote, para isso, o seu comprador deverá possuir pelo menos R\$ 10,00, assim, $x$ que indica a quantidade de pacotes vendido seguirá a lei de formação $x = 5 \mod d$, em que $d$ é o dinheiro que a pessoa possui.

\begin{equation}
    \text{Número de Pacotes} = \begin{cases} 0 & \text{quando } R\$ \leq  4,99 \\ x & \text{quando } R\$ > 4,99 \end{cases}
    \label{eq: VendaPipoca}
\end{equation}

Saindo do assunto da pipoca e voltando para o tema deste texto, existe uma família de funções de ativação que funciona de forma semelhante a lógica de venda dos pacotes de pipoca, elas são as unidades lineares retificadoras. A ReLU, que dá nome a essa família, funciona de forma semelhante a essa venda, ela tem um comportamento de "tudo ou nada", em que irá comandar quando um neurônio de uma rede neural irá disparar seu resultado.

As funções retificadoras, como a ReLU, são o tópico principal deste texto, para isso, antes de conhecê-las vamos entender um pouco do cenário que elas surgiram, com uso das funções sigmoides e o problema do gradiente em fuga. Em seguida veremos a ReLU os problemas que ela pode causar em uma rede neural. Seguindo adiante, conheceremos as suas variantes com vazamento, como a Leaky ReLU, depois as variantes suaves, como a ELU. Para conhecermos todas essas funções, serão utilizados gráficos, equações, comparativos de pesquisas e citações que explicam sobre suas propriedades. No final, veremos um comparativo com essas funções utilizando a construção de uma rede neural convolucional com o dataset CIFAR-10.

% ===================================================================
% ReLU
% ===================================================================

\section{Rectified Linear Unit e Revolução Retificadora}

Como vimos anteriormente, as funções sigmoides surgiram com inspiração nos neurônios humanos e como eles se comportam com determinados estímulos. Mas essas não foram as únicas funções que tiveram essa origem. Na década de 40, o pesquisador Alton Householder estava estudando um cenário parecido em seu trabalho \textit{A theory of steady-state activity in nerve fiber network: I. Definition of mathematical biofysics}, nele o autor analisou o comportamento de fibras nervosas e quando elas irão assumir caráter excitatório ou inibitório, para isso ele apresentou a equação \ref{eq: EquacaoFibraNervosa} \parencite{Householder1941}.

\begin{equation}
    a_{ij} = \begin{cases} 0 & \text{quando } \eta_i \le h_{ij} \\ a_{ij}, & \text{quando } \eta_i > h_{ij} \end{cases}
    \label{eq: EquacaoFibraNervosa}
\end{equation}

Essa equação nos mostra quando uma fibra nervosa irá disparar, para isso, devemos olhar o limiar da fibra $h_{ij}$ e o estímulo total $\eta_i$, com base nesses valores e no que a fórmula apresenta, uma fibra irá disparar quando o estímulo total for maior que o seu limiar, quando isso não ocorrer, ela não irá disparar \parencite{Householder1941}. Além disso, \textcite{Householder1941} explica também sobre o termo $a_{ij}$, a saída dessa função, segundo o autor ele é utilizado para representar o parâmetro de atividade, sendo um valor diferente de zero, podendo ser positivo (quando a fibra possui ação excitatória), ou negativo (apresentando caráter inibitório).

Essa equação criada por Householder, nos lembra bastante a expressão da função ReLU, a qual é denotada pela fórmula \ref{eq: EquacaoReLU}.

\begin{equation}
    \text{ReLU}(z_i) = \begin{cases}z_i, & \text{se } z_i > 0 \\0, & \text{se } z_i \leq 0\end{cases}
    \label{eq: EquacaoReLU}
\end{equation}

A ReLU é o tópico principal desse texto e será a primeira função de ativação que iremos estudar. Dito isso, mesmo com ela existindo a mais de 80 anos, ela só passou a ser amplamente utilizada nos anos 2010, antes disso, as sigmoides eram a grande maioria quando o assunto era função de ativação. Contudo, as sigmoides eram funções saturantes, e isso fazia com que sua derivada retornasse muitos valores pequenos ao longo da função. Ao multiplicar vários valores pequenos na retropropagação do gradiente, o vetor gradiente ia diminuindo até chegar um ponto em que ele não conseguia atualizar os pesos e vieses das redes neurais de forma eficiente, assim, tínhamos o problema do gradiente em fuga. As funções retificadoras, sendo a principal delas a ReLU, surgem para corrigir esse problema crônico. 

Dessa forma antes de conhecermos de fato a ReLU e suas propriedades, vamos antes entender o cenário que ela se popularizou, com os cientistas buscando novos tipos de funções de ativação que substituísse as sigmoides, funções saturantes, por outro tipo de função que resolvesse o problema do gradiente em fuga.

Nesse cenário, artigos como \textit{Rectified Linear Units Improve Restricted Boltzmann Machines} foram essenciais para popularizar a ReLU como uma função de ativação interessante para se utilizar em redes neurais. No trabalho, \textcite{Nair2010} foram responsáveis por demonstrar propriedades úteis das funções retificadoras, como a capacidade da NReLU de auxiliar em reconhecimentos de objetos por possuir equivariância de intensidade(\textit{intensity equivarience}), o que significa que se a intensidade da entrada de uma função for alterada por um determinado fator a intensidade de sua saída será alterada pelo mesmo fator. Essa propriedade se torna bastante útil em casos que queremos preservar informações, como ao comparar imagens, garantindo melhor precisão por exemplo em situações de baixa luz quando comparados com cenários em que possuem muita luz nas imagens.

Além disso, no texto \textit{Deep Sparse Rectifier Neural Networks} dos autores \textcite{Bordes}, o uso de unidades retificadoras não lineares são propostos como alternativas para a tangente hiperbólica e sigmoide em redes neurais profundas, mas também os pesquisadores são capazes de demonstrar que as unidades retificadoras se aproximam melhor do comportamento de neurônios biológicos. Um ponto chave desse texto é que os autores destacam características importantes que a esparsidade traz para uma rede neural possibilitada pelo uso de funções retificadoras \parencite{Bordes}. Entre elas estão:

\begin{itemize}
    \item \textbf{Desembaraçamento de Informações:} Um dos principais objetivos dos algoritmos de aprendizado profundo é desembaraçar os fatores que explicam as variações nos dados, assim, existem diferentes tipos de representações, uma representação densa é altamente emaranhada porque quase qualquer mudança na entrada modifica a maior parte as entradas no vetor de representação, contudo, se tivermos uma representação esparsa e robusta a pequenas mudanças na entrada, o conjunto de características diferentes de zero é quase sempre aproximadamente conservado por pequenas mudanças na entrada \parencite{Bordes};
    \item \textbf{Representação eficiente de tamanho variável}. Diferentes entradas podem conter diferentes quantidades de informação e seriam mais convenientemente representadas usando uma estrutura de dados de tamanho variável, o que é comum em representações computacionais de informação, assim é interessante poder variar o número de neurônios ativos permitindo que um modelo controle a dimensionalidade efetiva da representação para uma determinada entrada e a precisão necessária \parencite{Bordes};
    \item \textbf{Separabilidade linear}. Representações esparsas também são mais propensas a serem linearmente separáveis, ou mais facilmente separáveis com menos maquinário não linear, simplesmente porque a informação é representada em um espaço de alta dimensão, além disso, isso pode refletir o formato original dos dados \parencite{Bordes};
    \item \textbf{Distribuídas, mas esparsas}. Representações densamente distribuídas são as representações mais ricas, sendo potencialmente exponencialmente mais eficientes do que as puramente locais, além disso a eficiência das representações esparsas ainda é exponencialmente maior, com a potência do expoente sendo o número de características diferentes de zero, elas podem representar uma boa compensação em relação aos critérios acima \parencite{Bordes}.
\end{itemize}

Por fim, um último trabalho que colaborou para a popularização da ReLU foi a \textit{AlexNet}, de \textcite{AlexNet}, essa rede neural convolucional (CNN) foi capaz ganhar o Desafio de Reconhecimento Visual em Larga Escala ImageNet (ILSVRC) sendo treinada para classificar 1.2 milhões de imagens de alta resolução e classificá-las em 1000 diferentes classes. Para isso, a \textit{AlexNet} foi construída utilizando 8 camadas com pesos, sendo as primeiras 5 camadas convolucionais, enquanto as três últimas são camadas totalmente conectadas, a última camada de neurônios faz uso da função de ativação \textit{softmax} para fazer a distribuição em 1000 diferentes classes, além disso a \textit{AlexNet} fez uso da ReLU em sua arquitetura \parencite{AlexNet}. Podemos ver um esquema de sua arquitetura na figura \ref{fig: AlexNetArquietura}.

Assim, como podemos ver na tabela \ref{tab:alexnet_apa}, a \textit{AlexNet} foi capaz de alcançar uma taxa de erro de 15.3\% na fase de testes, podemos notar com base na variação de camadas convolucionais, que essa é uma rede que se beneficia da sua profundidade, algo que provavelmente só foi capaz de ocorrer devido ao uso da ReLU como função de ativação, por não gerar o problema do gradiente em fuga como nas sigmoides. Além disso, a rede SIFT + FVs (\textit{Scale-Invariant Feature Transform + Fisher Vectors}) é mostrada na tabela como base de comparativo, podemos notar que a AlexNet foi capaz de diminuir com mais de 10\% dos erros que essa rede gerava.

Além disso, o modelo SIFT + FVs (\textit{Scale-Invariant Feature Transform + Fisher Vectors},) o qual é apresentado como base de comparação, apresenta uma taxa de erro de 26.2\%, um aumento de 10 pontos percentuais quando comparado com o melhor modelo da AlexNet de 15.3\%.

\begin{table}[ht]
    \caption{Comparação das Taxas de Erro no AlexNet}
    \label{tab:alexnet_apa}
    \centering
    \begin{tabular}{lccc}
        \toprule
        \textbf{Modelo} & \textbf{Top-1 (validação)} & \textbf{Top-5 (validação)} & \textbf{Top-5 (teste)} \\
        \midrule
        SIFT + FVs & -    & -      & 26.2\% \\
        1 CNN      & 40.7\% & 18.2\% & -      \\
        5 CNNs     & 38.1\% & 16.4\% & 16.4\% \\
        1 CNN\textsuperscript{a}      & 39.0\% & 16.6\% & -      \\
        7 CNNs\textsuperscript{a}     & 36.7\% & 15.4\% & \textbf{15.3\%} \\
        \bottomrule
    \end{tabular}
    \vspace{2mm}
    
    \parbox{\linewidth}{\small
        \textit{Nota.} A tabela compara as taxas de erro de diferentes modelos nos conjuntos de validação e teste do ILSVRC-2012. Os valores em negrito indicam o melhor resultado. 
        \textsuperscript{a}Modelos que foram pré-treinados para classificar todo o conjunto de dados ImageNet 2011 Fall. 
        Adaptado de "ImageNet Classification with Deep Convolutional Neural Networks", por A. Krizhevsky, I. Sutskever, \& G. E. Hinton, 2012, \textit{Advances in Neural Information Processing Systems, 25}.
    }
\end{table}

Agora que conhecemos um pouco de como foi o contexto em que a ReLU surgiu, podemos de fato conhecê-la. Para isso, primeiro conhecemos sua fórmula representada pela equação \ref{eq: EquacaoReLU}, mas também podemos representá-la com a expressão reduzida $\max(0, z_i)$. Nós podemos interpretar essa definição como uma pergunta em que a função ReLU recebe um número como entrada e faz uma pergunta "esse número é menor que zero?", se a resposta for sim, ela retorna como resultado o número zero, se a resposta for não, ela irá retornar o próprio número como sua saída. Neste caso estamos falando de números, mas a analogia utilizada no início do texto em que o pacote de pipoca só é vendido caso a pessoa tenha mais de R\$ 5,00 também pode ser utilizada, em que o resultado seria um valor booleano, indicando se a pessoa vende ou não a pipoca.

Além de sua fórmula, temos também o seu gráfico, que está presente na figura \ref{fig: GraficoReLU}, podemos ver que ele é bem mais simples quando comparamos com a sigmoide, por exemplo, ele é a apenas a junção de duas retas, sendo uma delas uma função constante que irá retornar sempre zero e a outra a função identidade. Essa simplicidade da ReLU é algo muito atrativo para os desenvolvedores, pois, ao utilizá-la ao invés de uma função mais complexa como a sigmoide ou a tangente hiperbólica, estamos diminuindo a complexidade da rede neural, se essa rede se torna mais simples, a tendencia é de que ela possua um custo de poder de processamento menor permitindo que um volume maior de dados seja processado em menos tempo e com isso seu tempo de treinamento seja menor. Note que, antes da ReLU surgir, muitos das funções de ativação faziam uso de exponenciais, a ReLU não só resolvia o problema do gradiente em fuga mas também era muito mais barata.

\begin{equacaodestaque}{Rectified Linear Unit (ReLU)}
    \text{ReLU}(z_i) = \begin{cases}z_i, & \text{se } z_i > 0 \\0, & \text{se } z_i \leq 0\end{cases}
    \label{eq:relu}
\end{equacaodestaque}

\begin{figure}{h!}
        \centering
        \caption{Gráfico da função ReLU.}
        \begin{tikzpicture}
            \begin{axis}[
                xlabel={$z_i$},
                ylabel={$\text{ReLU}(z_i)$},
                xmin=-2.3, xmax=2.3,
                ymin=-0.8, ymax=2.3,
                axis lines=middle,
                grid=major,
            ]
            \addplot[blue, thick, domain=-2:2] {max(0, x)};
            \end{axis}
        \end{tikzpicture}
        \label{fig:relu}
\end{figure}

Como estamos trabalhando com redes neurais, uma das maiores vantagens dessas redes é o fato delas “aprenderem” com base na retropropagação do gradiente nas camadas da rede. Assim, ao calcularmos o gradiente para fazer a retropropagação do erro e ajustar os pesos e vieses das camadas, é necessário ter em mente também a derivada daquela função de ativação que vamos aplicar em uma camada de neurônios da rede, dado que ela entrará no backward pass do modelo.

Para achar a derivada da ReLU, podemos simplesmente derivar as duas condicionais dela, assim, quando $x$ for maior que zero, a saída será 1, já quando $x$ for menor que zero, a saída será zero. Mas encontramos um problema nisso, a derivada dessa função não existe quando $x$ é 0, pois o limite lateral à esquerda dessa função é zero, enquanto o limite lateral a direita dela é um. Isso passa a ser um problema quando queremos calcular o valor de saída justamente quando aquele valor de entrada é zero. Na prática, esse problema é fácil de resolver quando estamos trabalhando com o código dessa função, basta escolhermos qual será o resultado da ReLU quando esse valor de entrada for zero. Podemos dizer que ele será 1 ou zero, isso irá depender somente da nossa implementação da derivada da ReLU. Note que, essa descontinuidade da ReLU, se torna um problema quando estamos trabalhando com sua derivada.

Assim, temos a equação \ref{eq:DerivadaReLU}.

\begin{equacaodestaque}{Derivada Rectified Linear Unit (ReLU)}
    \frac{d}{dz_i} [ReLU](z_i) = \begin{cases}1, & \text{se } z_i > 0 \\0, & \text{se } z_i \leqslant 0 \end{cases}
    \label{eq:relu-derivada}
\end{equacaodestaque}

Esse detalhe da descontinuidade da ReLU no ponto zero foi algo que acabou mudando em funções futuras, que buscam corrigir erros da ReLU e melhorá-la,assim, com o passar do tempo foram surgindo outras alternativas que também trabalhassem com os atributos da ReLU, mas que fossem contínuas em toda a reta, permitindo a sua derivação também em todos os pontos. Uma dessas funções a ELU, ela será explicada mais em frente.

Além de termos a sua derivada em equação, podemos fazer também o seu gráfico na figura \ref{eq:DerivadaReLU}, note que ela é ainda mais simples que a própria função de ativação, são só duas retas constantes que irão retornar zero quando o número for menor que zero, ou irão retornar 1 quando a entrada for um número maior que zero.

\begin{figure}[h!] % Use [htbp] para dar flexibilidade ao LaTeX
    \centering % Centraliza o gráfico na página
    \caption{Gráfico da Derivada da Função ReLU}
    \begin{tikzpicture}
        \begin{axis}[
            xlabel={$z_i$},
            ylabel={$\text{ReLU}'(z_i)$},
            xmin=-2.3, xmax=2.3,
            ymin=-0.8, ymax=1.8,
            axis lines=middle,
            grid=major
        ]
        \addplot[red, thick, domain=-2:0] {0};
        \addplot[red, thick, domain=0:2] {1};
        \addplot[red, only marks, mark=o, mark size=1.5pt] coordinates {(0,0)};
        \addplot[red, only marks, mark=*, mark size=1.5pt] coordinates {(0,1)};
        \end{axis}
    \end{tikzpicture}
    \label{fig:relu-derivada}
\end{figure}

\subsection{Implementação em Python}

Agora que sabemos qual é o comportamento da ReLU, quais são suas fórmulas e quais são seus gráficos, podemos trabalhar em uma implementação dessa função utilizando Python. Para fazer isso, vamos criar uma classe, com duas funções: a \textit{forward} e a \textit{backward}. A função \textit{forward} é responsável por implementar a ReLU, já a \textit{backward} implementa a sua derivada. Para criar a \textit{forward}, podemos utilizar uma função já pronta da biblioteca numpy, a maximum, essa função é responsável por comparar dois valores e escolher o maior deles. Neste caso, um dos valores será sempre zero, enquanto o outro será a entrada da função. Já a \textit{backward}, irá receber o gradiente retropropagado, e irá aplicar na fórmula da derivada da ReLU, comparando se aquele valor é negativo e retornando zero, caso for, ou um caso seja positivo.

Assim, conhecemos a ReLU, uma função que é considerada padrão para a maioria das redes neurais \textit{feedforward} \parencite{Goodfellow2016}. Vimos que ela veio para ser uma alternativa para as funções sigmoides, sendo mais simples e resolvendo o problema do gradiente em fuga, contudo, a ReLU também apresenta problemas, como o do neurônios agonizantes a o explosão de gradientes, uma versão diferente do problema do gradiente em fuga. Dito isso, eles serão explicados em seguida para entendermos também as desvantagens de se utilizar essa função em uma rede neural.

\textbf{Implementação em Python}

\begin{codelisting}{Classe completa do função de ativação Rectified Linear Unit}{gd_class}
import numpy as np
from layers.base import Layer

class ReLU(Layer):
    def __init__(self):
        super().__init__()
        self.input = None

    def forward(self, input_data):
        self.input = input_data
        return np.maximum(0, self.input)

    def backward(self, grad_output):
        relu_grad = (self.input > 0)

        # Apply the chain rule
        return grad_output * relu_grad, None
\end{codelisting}

\section{Dying ReLUs Problem}

\section{Corrigindo o Dying ReLUs Problem: As Variantes com Vazamento}

Diferente da ReLU tradicional que retorna zero para os casos em que sua entrada é negativa e por isso na sua derivada irá também retornar zero nestes casos, as variantes com vazamento atuam de outra forma, elas retornam um valor muito pequeno como 0.1, multiplicado pela entrada da função quando ela é negativa. Por isso, a sua derivada será algo também 0.1 (ou valores muito pequenos), isso permite um "vazamento" do gradiente em cenários nos quais a entrada do neurônio será negativa.

Nós vimos, que a causa do neurônios agonizantes era justamente isso: muitas situações em que a entrada era negativa, que gerava um gradiente nulo e consequentemente impedia os neurônios de terem seus pesos e vieses ajustados, e futuramente morrendo, retornando zero independente de qual fosse a sua entrada.

Assim, essas variantes, como a Leaky ReLU e a PReLU buscam tentar corrigir um amenizar esse problema da ReLU mas mantendo algumas de suas principais propriedades, como a não linearidade, a capacidade de ser escrita compondo duas retas permitindo a criação de uma função simples e rápida de ser computada em uma rede neural.

\subsection{Leaky ReLU (LReLU)}

Seguindo adiante, a próxima função que iremos ver é a Leaky ReLU, ela é uma variante da ReLU que foi criada com intuito de corrigir o problema do neurônios agonizantes. Assim como a ReLU, que foi explicada com a analogia do vendedor de pipoca, podemos extender essa explicação para essa nova função, antes o limiar para comprar um pacote de pipoca era de R\$ 5,00, quem tivesse menos que isso não comprava nada. Mas agora, para garantir que todos possam comprar pipoca, você como vendedor definiu que quando uma pessoa tiver menos que R\$ 5,00 ela também será capaz de comprar pipoca, só que neste caso ela comprará um punhado de pipoca que será proporcional ao dinheiro que ela tem multiplicado por uma constante $\alpha$. Assim, uma pessoa com um valor próximo de R\$ 5,00 pode sair com um punhado de pipoca quase igual ao do pacote original se essa constante $\alpha$ for um valor muito proximo de um. Com isso, você como vendedor consegue obter lucro com uma nova clientela além de não perder clientes por não possuírem o valor total do pacote de pipoca. A Leaky ReLU traz uma proposta parecida para resolver com o problema do neurônios agonizantes.

Ela foi apresentada no artigo \textit{Recfier Nonlinearites Improve Neural Networks Acustic Models}, em que os autores exploram o uso de redes retificadoras profundas como modelos acústicos para a tarefa de reconhecimento de fala conversacional \textit{switchboard} \parencite{Mass2013}. Além disso, a sua principal diferença, como explicam \textcite{Mass2013}, está no fato dela permitir que um pequeno gradiente diferente de zero flua quando a unidade está saturada e não ativa. Esse gradiente diferente de zero que flui quando a unidade está saturada e não ativa são os seus compradores de pipoca que não possuem o valor total mas são capazes de comprar um punhado dela, neste caso a unidade estará não ativa pois o valor de entrada é negativo mas irá retornar um valor diferente de zero, algo que não acontecia na ReLU.

Em testes de desempenho realizados por \textcite{Mass2013}, eles foram capazes de analisar como uma rede neural que faz uso dessa função pode performar quando comparada com a ReLU tradicional e também com redes que fazem uso da Tangente hiperbólica, esse comparativo pode ser visto na tabela \ref{tab:lrelu_desempenho_apa}. Nós podemos nessa tabela que as redes neurais que fizeram uso da Leaky ReLU como função de ativação obtiveram melhores resultandos quando comparadas com as redes que utilizaram a ReLU tradicional ou  a Tangente Hiperbólica, vemos que a rede que foi construída com 3 camadas utilizando a LReLU foi capaz de obter a taxa de erro de palavra (WER) mais baixa no conjunto SWBD, com 17.8\%, esse resultado 0.3 pontos percentuais menor quando comparado com uma mesma rede de três camadas que utilizou a ReLU tradicional.

Além disso, ainda na \ref{Tab:lrelu_desempenho_apa}, nas redes compostas por três camadas, a rede que fez uso da LReLU também foi melhor que suas outras concorrentes que fizeram uso da ReLU e da tanh, sendo capaz de ter a menor taxa de erro de palavra no conjunto de avaliação (EV), com 24.3\%, uma diferença de 0.1 pontos percentuais quando comparada com a ReLU de 24.4\%, já quando essa rede é comparada com a tangente hiperbólica, a diferença é ainda maior, sendo de 2.1 pontos percentuais, indicando que a LReLU trás resultados melhores quando comparada com essas duas funções de ativação.

\begin{table}[ht]
    \caption{Comparativo de Desempenho de Redes Neurais para Reconhecimento de Fala}
    \label{tab:lrelu_desempenho_apa}
    \centering
    \begin{tabular}{lccccc}
        \toprule
        \textbf{Modelo} & \textbf{Dev CrossEnt} & \textbf{Dev Acc (\%)} & \textbf{SWBD WER} & \textbf{CH WER} & \textbf{EV WER} \\
        \midrule
        
        GMM Baseline & N/A & N/A & 25.1 & 40.6 & 32.6 \\ 
        \addlinespace % Adiciona um espaço para separar o baseline dos outros
        2 Camadas Tanh  & 2.09 & 48.0 & 21.0 & 34.3 & 27.7 \\
        2 Camadas ReLU  & 1.91 & 51.7 & 19.1 & 32.3 & 25.7 \\
        2 Camadas LReLU & 1.90 & 51.8 & 19.1 & 32.1 & 25.6 \\ 
        \addlinespace
        3 Camadas Tanh  & 2.02 & 49.8 & 20.0 & 32.7 & 26.4 \\
        3 Camadas ReLU  & 1.83 & 53.3 & 18.1 & 30.6 & 24.4 \\
        3 Camadas LReLU & 1.83 & 53.4 & \textbf{17.8} & 30.7 & \textbf{24.3} \\ 
        \addlinespace
        4 Camadas Tanh  & 1.98 & 49.8 & 19.5 & 32.3 & 25.9 \\
        4 Camadas ReLU  & 1.79 & 53.9 & 17.3 & 29.9 & 23.6 \\
        4 Camadas LReLU & \textbf{1.78} & 53.9 & 17.3 & 29.9 & 23.7 \\
        
        \bottomrule
    \end{tabular}
    \vspace{2mm}
    
    \parbox{\linewidth}{\small
        \textit{Nota.} Comparação de métricas de erro para sistemas de redes neurais profundas (DNN) em reconhecimento de fala. As métricas de quadro a quadro (frame-wise) foram avaliadas em um conjunto de desenvolvimento, e as taxas de erro de palavra (WER) no conjunto de avaliação Hub5 2000 e seus subconjuntos. Abreviações: Dev CrossEnt = Entropia Cruzada no conjunto de desenvolvimento; Dev Acc = Acurácia no conjunto de desenvolvimento; WER = Taxa de Erro de Palavra (Word Error Rate); SWBD = Switchboard; CH = CallHome; EV = Evaluation set. Valores em negrito indicam os melhores resultados para modelos de 3 e 4 camadas.
        Adaptado de "Rectifier Nonlinearities Improve Neural Network Acoustic Models", por A. L. Maas, A. Y. Hannun, \& A. Y. Ng, 2013, \textit{In Proceedings of the 30th International Conference on Machine Learning, Workshop on Deep Learning for Audio, Speech and Language Processing}.
    }
\end{table}

Agora comparando as redes que fazem uso de quatro camadas, cabe destacar os resultados da entropia cruzada no conjunto de desenvolvimento (Dev CrossEnt), que é uma métrica responsável por medir a diferença entre duas distribuições de probabilidade, neste cenário: a distrubição de probabilidade prevista pelo modelo e a distribuição de probabilidade real, com base nesses dois valores, a entropia cruzada consegue medir o quão bem o modelo de rede neural criado pelos pesquisadores está prevendo a transcrição correta da fala durante a fase de treinamento e ajuste, para isso, é utilizado o conjunto de dados de desenvolvimento (Dev Set). Tendo isso em mente, o modelo de quatro camadas que fez uso da Leaky ReLU em sua arquitetura obteve o melhor resultado dos seus outros dois concorrentes, para isso, sendo assim, ele teve como resultado uma entropia cruzada de 1.78, 0.01 menor que o modelo que fez uso da ReLU tradicional (que obteve 1.79) e 0.2 menor que o modelo que fez uso da tangente hiperbólica (que obteve 1.98).

Ainda no grupo de redes que possuem quatro camadas, podemos ver um empate quando analisamos a acurácia no conjunto de desenvolvimento (Dev Acc), que é uma métrica responsável por medir a proporção das previsões corretas feitas pelo modelo quando comparadas com o total de previsões feitas. Assim, quando vemos a tabela \ref{tab:lrelu_desempenho_apa}, as redes que fizeram uso tanto da ReLU quanto da Leaky ReLU obtiveram a mesma acurácia de 53.9\%, já quando comparamos com a tangente hiperbólica, vemos uma diferença de 4.1 pontos percentuais, indicando as funções retificadoras acabam sendo mais precisas para essa análise. Não somente elas são mais precisas, mas quando comparamos com a Leaky ReLU, vemos outros ganhos também, como menores taxas de erro de palavra (WER) tanto no conjunto Switchboard (SWBD) quanto no conjunto de avaliação (EV).

Seguindo adiante, podemos discutir a expressão matemática da Leaky ReLU, a qual é dada pela equação \ref{eq:EquacaoLeakyReLU}, que é bem parecida com a ReLU, porém, ela também irá retornar valores negativos quando a sua entrada for um valor negativo, diferente da ReLU, que iria retornar como saída zero. A constante $\alpha$, no texto original é dada por 0.1 fazendo com que os valores negativos sejam pequenos mas ainda sim, diferentes de zero quando passam pela entrada \parencite{Mass2013}. Cabe destacar que, essa constante $\alpha$ pode ser ajustada para diferentes cenários, podendo ser valores diferentes de 0.1 como foram propostos no texto original, é possível ver isso acontecendo em comparativos ao longo desse texto, em que diferentes autores optam por valores diferentes de $\alpha$ para melhor ajustar ao problema que está sendo analisado.

\begin{equacaodestaque}{Leaky ReLU (LReLU)}
    \text{LReLU}(z_i) = \begin{cases}z_i, & \text{se } z_i \ge 0 \\ \alpha \cdot z_i, & \text{se } z_i < 0\end{cases}
    \label{eq:leaky-relu}
\end{equacaodestaque}

Cabe destacar, que assim como a ReLU tradicional, a Leaky ReLU também possuí uma expressão compacta, dada por $\max(\alpha z_i, z_i)$, sendo uma forma mais simples de escrever a expressão \ref{eq:EquacaoLeakyReLU} com condicionais mas ainda sim manter o seu sentido. Já para a sua representação gráfica, podemos plotar o seu gráfico na figura \ref{fig:GraficoLeakyReLU}, analisando ele, vemos que a Leaky ReLU possui características muito semelhantes com a ReLU, como o fato dela assumir o comportamento de uma função identidade para para valores positivos em sua entrada, mas, quando analisamos seus valores negativos vemos uma diferença, agora eles são dados por um gráfico de uma função do primeiro grau, diferente da ReLU que era uma função constante em zero. Além disso, a LReLU, também é uma função assimétrica e não linear, bem como apresenta um ponto de descontinuidade em zero, pois ao traçarmos os seus limites laterais, eles apresentam valores diferentes, por isso ela não pode ser derivada nesse ponto, assim como a ReLU que vimos anteriormente.

\begin{figure}[h!]
    \centering
    \caption{Gráfico da Função Leaky ReLU (LReLU) com $\alpha = 0.1$}
    \begin{tikzpicture}
        \begin{axis}[
            xlabel={$z_i$},
            ylabel={$\text{LReLU}(z_i)$},
            xmin=-2.3, xmax=2.3,
            ymin=-0.8, ymax=2.3,
            axis lines=middle,
            grid=major,
        ]
        % \leakyalpha é o comando que você definiu no preâmbulo (0.1)
        \addplot[blue, thick, domain=-2:2] {x > 0 ? x : 0.1*x};
        \end{axis}
    \end{tikzpicture}
    \label{fig:leaky-relu}
\end{figure}

Sabendo de sua expressão e seu gráfico, podemos nos preocupar agora a calcular sua derivada, para isso, devemos derivar as duas condicionais que estão na função da Leaky ReLU. Assim, quando a entrada dessa função for maior que zero, essa função será $x$ que derivada é 1, já quando a entrada for menor que zero, a função será $\alpha x$, que quando derivada tem como resultado a própria constante $\alpha$. Contudo, como dito anteriormente, a derivada da Leaky ReLU não existe quando a entrada é exatamente zero, mas na prática, quando estamos trabalhando com a sua definição na retropropagação, podemos definir um valor para a derivada nesse ponto, assim como fizemos com a ReLU tradicional. Com isso em mente, temos a equação \ref{eq:DerivadaLeakyReLU} que representa a derivada da Leaky ReLU, note que para a sua derivada nós não temos uma expressão reduzida quando comparada com a expressão original.

\begin{equacaodestaque}{Derivada Leaky ReLU (LReLU)}
    \frac{d}{dz_i} [LReLU](z_i) = \begin{cases}1, & \text{se } z_i > 0 \\ \alpha, & \text{se } z_i \leqslant  0 \end{cases}
    \label{eq:leaky-relu-derivada}
\end{equacaodestaque}

Já que agora conhecemos a sua derivada, podemos também plotar o seu gráfico, o qual é dado pela figura \ref{fig:GraficoLeakyReLUDerivada}. Ele também é parecido com o gráfico da ReLU que vimos anteriormente, sendo composto por duas retas constantes, para valores positivos ele retorna 1 (assim como a ReLU), e para valores negativos ou nulos ele irá sempre retornar a constante $\alpha$, diferente da ReLU, que iria retornar zero, indicando que neste caso o neurônio não está passando nenhuma informação na na retropropagação do gradiente. Por esse motivo que e a Leaky ReLU tem esse nome, pois leaky em inglês significa "vazamento", e neste caso, como a derivada dela é diferente de zero, mesmo quando a entrada for negativa, ela irá passar informações durante a retropropagação, isso permite que o neurônio não morra, como acontecia em algumas redes que faziam uso da ReLU tradicional, e por esse motivo continue aprendendo pelo fato do gradiente continuar fluindo pela rede e consequentemente atualizando os pesos e vieses dos neurônios.

\begin{figure}[h!]
    \centering
    \caption{Gráfico da Derivada da Função Leaky ReLU (LReLU) com $\alpha = 0.1$}
    \begin{tikzpicture}
        \begin{axis}[
            xlabel={$z_i$},
            ylabel={$\text{LReLU}'(z_i)$},
            xmin=-2.3, xmax=2.3,
            ymin=-0.2, ymax=1.2,
            axis lines=middle,
            grid=major
        ]
        \def\alphaVal{0.1} % Define alpha for the derivative graph

        \addplot[red, thick, domain=-2:0] {\alphaVal};
        \addplot[red, thick, domain=0:2] {1};
        \addplot[red, only marks, mark=o, mark size=1.5pt] coordinates {(0,\alphaVal)};
        \addplot[red, only marks, mark=*, mark size=1.5pt] coordinates {(0,1)};
        \end{axis}
    \end{tikzpicture}
    \label{fig:leaky-relu-derivada}
\end{figure}

Voltando para \textit{Recfier Nonlinearites Improve Neural Networks Acustic Models}, \textcite{Mass2013} explicam sobre outras propriedades da LReLu, como o fato dela sacrificar a esparsidade de zero duros (\textit{hard-zero sparsity}) para ter um gradiente  que é potencialmente mais robusto durante a otimização da rede neural. Além disso, os autores também explicam que ambas as funções retificadoras, tanto a ReLU quando a Leaky ReLU se saem bem nos testes de performance quando comparadas com a tangente hiperbólica além de se beneficiarem melhor da profundidade de rede quando comparadas com as sigmoides \parencite{Mass2013}.

A Leaky ReLU já é uma evolução quando comparamos com a ReLU tradicional, mas, podemos ir além e encontrar funções ainda mais complexas que também buscam assim como a LReLU resolver o problema do neurônios agonizantes com um vazamento de gradiente nos casos negativos, uma dessas funções é a PReLU, a qual veremos em seguida.

\textbf{Implementação em Python}

\begin{codelisting}{Classe completa do função de ativação Leaky ReLU}{gd_class}
import numpy as np
from layers.base import Layer


class LeakyReLU(Layer):
    def __init__(self, alpha=0.01):
        super().__init__()
        self.input = None
        self.alpha = alpha

    def forward(self, input_data):
        self.input = input_data
        return np.maximum(self.input * self.alpha, self.input)

    def backward(self, grad_output):
        leaky_relu_grad = np.where(self.input > 0, 1, self.alpha)
        return grad_output * leaky_relu_grad, None
\end{codelisting}

\subsection{Parametric ReLU}

Continuando nas analogias do vendedor de pipoca para explicar as funções retificadoras, podemos também pensar uma para a Parametric ReLU. Na LReLU nós tínhamos uma constante fixa, que valia para todos os valores de entrada e não mudava, era como se o vendedor de pipoca definisse um valor para a proporção de pipoca que a pessoa irá receber quando tiver com uma quantia menor de dinheiro que o limiar da venda. Mas agora, este vendedor está mais experiente, e sabe que pode ajustar essa constante sempre que quiser, assim, quando estiverem muitas pessoas na praça em que está vendendo pipoca, ele poderá colocar uma constante que será capaz de dar uma quantidade ainda maior de pipoca para aqueles que não possuem o valor total de um pacote, o que incentivaria a venda para as pessoas. Já quando estivesse em um lugar mais vazio, colocaria uma constante que daria menos pipoca, para maximizar o seu lucro. A Diferença da PReLU para a LReLU está justamente nessa constante e como ela irá se comportar.

Proposta por \textcite{He2015} no artigo \textit{Delving Deep into Rectifiers: Surpassing Human Level Performance on Image Net Classification}, a PReLU surgiu como uma variação não somente da ReLU, mas também uma evolução da Leaky ReLU que vimos anteriormente, isso ocorre, pois diferente da LReLU que possuía uma constante $\alpha$ fixa que multiplicava o valor da entrada nos casos negativos, a PReLU trás essa mesma constante, mas neste caso ela é adaptável, se ajustando as particularidades de cada problema que uma rede neural está tentando resolver. Assim, a PReLU é como o pipoqueiro mais experiente, que ajusta como vai vender o seu punhado de pipoca em cada uma das situações para poder maximizar os seus lucros mas ao mesmo tempo garantir mais clientes para si.

Ainda em \textit{Delving Deep into Rectifiers: Surpassing Human Level Performance on Image Net Classification}, os autores realizam testes comparando a ReLU tradicional com a PReLU utilizando como base o dataset de 1000 classes do ImageNet 2012, o qual contêm cerca de 1.2 milhões de imagens de treino, 50.000 imagens de validação e 100.000 imagens de teste sem rótulos publicados \parencite{He2015}. Para isso, \textcite{He2015} criaram três modelos diferentes (A, B e C), baseados na arquitetura VGG-16 mas com variações entre si como diferentes números de camadas convolucionais e consequentemente complexidades distintas para cada algoritmo. Esses modelos podem ser vistos na tabela \ref{tab:arquitetura_prelu}. 

Conhecendo cada um dos modelos apresentados por \textcite{He2015}, podemos analisar como eles se comportaram nos testes de performance utilizando o dataset do ImageNet 2012, para isso, os autores realizaram testes tanto utilizando a nova função de ativação PReLU, mas também a ReLU tradicional, podendo ter uma base melhor para suas comparações. Além disso, como podemos ver na tabala \ref{tab:prelu_desempenho1}, é medido o erro Top-1 e o erro Top-5, o erro Top-1 nos mostra quão preciso é o modelo em seu melhor chute, já o erro Top-5 nos mostra se a resposta correta estava entre os top 5 melhores chutes feito pelo modelo. Assim, conhecendo esses parâmetros de medida, podemos chegar na conclusão de que quanto menor esses valores, melhor o modelo está performando, seguindo essa lógica, podemos ver que todos os modelos que fazem uso de funções retificadoras, seja a PReLU ou mesmo a ReLU tradicional, são capazes de performar melhor que os modelos VGG-16 e GoogleLet que fazem uso de outras funções de ativação.

Quando comparamos o modelo C, que faz uso da PReLU e possui mais camadas convolucionais, com o modelo A que faz uso da ReLU, vemos uma diferença de 2,21 pontos percentuais no erro Top-1, já quando analisamos o erro Top-5, essa diferença é de 1,21 pontos percentuais, o que indica que a ReLU é capaz de trazer melhores resultados quando comparada com redes que fazem uso da ReLU tradicional. Já quando comparamos com o VGG-16, essa diferença de desempenho é ainda maior, sendo de 3,8 pontos percentuais no erro Top-1 e 1,95 pontos percentuais no erro Top-5, note que a VGG-16, a qual é indicada na tabela \ref{tab:arquitetura_prelu} possui bem menos camadas convolucionais que o modelo C, podemos notar também a sua complexidade computacional, que é menos da metade da do modelo C, isso indica que a PReLU, por ser uma função não saturante e consequentemente corrigir o problema do gradiente em fuga, é capaz de criar redes neurais que se beneficiam melhor com uma maior profundidade, sendo capazes de extrair mais informações e com isso performar melhor.

Agora que conhecemos a Parametric ReLU e sabemos de seu potencial, podemos estudar a sua fórmula matemática, a qual é dada pela expressão \ref{eq:EquacaoPReLU}. Como explicam \textcite{He2015}, a PReLU generaliza a tradicional ReLU, além de melhorar o \textit{model fitting} apresentando quase nenhum custo computacional extra e com um baixo risco de sobreajuste (\textit{overfitting}). Este coeficiente $\alpha$, que ela apresenta assim como a Leaky ReLU que vimos anteriormente, é aprendível, e não uma constante fixa, isso indica ser otimizado utilizando a retropropagação do gradiente de forma simultânea com as outras camadas da rede neural criada \parencite{He2015}. Assim, por esse fato temos uma melhor eficiência no aprendizado e no tempo da rede, dado que não precisamos criar uma nova etapa só para ajustar os valores de alpha das camadas densas que fazem o uso da Parametric ReLU.

\begin{equacaodestaque}{Parametric ReLU (PReLU)}
    \text{PReLU}(z_i) = \begin{cases}z_i, & \text{se } z_i \ge 0 \\ \alpha_i \cdot z_i, & \text{se } z_i < 0\end{cases}
    \label{eq:prelu}
\end{equacaodestaque}


Assim como a LReLU e a ReLU, também podemos expressar a PReLU como uma função reduzida ao invés das condicionais vistas na expressão \ref{eq:EquacaoPReLU}, para isso, ela utiliza a mesma forma $\max{\alpha x, x}$, que vimos na Leaky ReLU, mas agora, a constante possui outro significado, indicando que ela é um parâmetro aprendível, e não mais fixo como na Leaky ReLU.

Com base em sua equação, podemos plotar o gráfico da PReLU na figura \ref{eq:EquacaoPReLU}, note que caso este coeficiente for igual a zero, nos temos então a ReLU tradicional, já quando ele for igual a 0.1, teremos a LReLU, no gráfico $\alpha$ está com valor de 0.2, mas ele irá variar conforme a rede aprende e para cada problema, podendo apresentar diferentes valores em diferentes situações. Com isso, nos notamos que a PReLU é composta por duas funções do primeiro grau, sendo assimétrica, e também apresentando um ponto de descontinuidade em zero, o que impede de ser derivada neste ponto.

\begin{figure}[h!]
    \centering
    \caption{Gráfico da Função Parametric ReLU (PReLU) com $\alpha=0.2$}
    \begin{tikzpicture}
        \begin{axis}[
            xlabel={$z_i$},
            ylabel={$\text{PReLU}(z_i)$},
            xmin=-2.3, xmax=2.3,
            ymin=-0.5, ymax=2.3,
            axis lines=middle,
            grid=major,
        ]
        % Define um valor exemplo de alpha para o gráfico
        \def\alphaVal{0.2} 
        \addplot[blue, thick, domain=-2:2] {x >= 0 ? x : \alphaVal*x};
        \end{axis}
    \end{tikzpicture}
    \label{fig:prelu}
\end{figure}

Se conhecemos a sua fórmula e como ela se comporta, podemos também derivar a PReLU, para isso, derivamos cada uma das expressões dos condicionais de forma separada, assim como fizemos com a Leaky ReLU e a ReLU anteriormente. Como a fórmula da PReLu é igual a LReLU, podemos apenas nos lembrar dela e citar a expressão \ref{eq:DerivadaPReLU} como sua derivada. Note que, essa derivada também não existe quando a entrada é exatamente zero, mas assim como na Leaky ReLU, "corrigimos" isso dizendo que ela será igual a $\alpha$ nesse ponto, para que o gradiente continue fluindo durante o backward pass. Vale lembrar, que essa reta em $\alpha$ irá variar conforme a rede neural aprende as características e se ajusta ao problema que está tentando resolver.

\begin{equacaodestaque}{Derivada Parametric ReLU (PReLU)}
    \frac{d}{dz_i} [PReLU](z_i) = \begin{cases}1, & \text{se } z_i > 0 \\ \alpha_i, & \text{se } z_i < 0 \\ \nexists, & \text{se } z_i = 0\end{cases}
    \label{eq:prelu-derivada}
\end{equacaodestaque}

Sabendo a fórmula da derivada da PReLU, podemos também plotar o seu gráfico, o qual é dado pela figura \ref{fig:GraficoPReLUDerivada}, ele é semelhante ao gráfico da Leaky ReLU visto na seção anterior, mas agora, a constante $\alpha$ está com valor em 0.2. Note que, o gráfico da derivada é também muito simples, sendo apenas duas funções constantes, uma que vale 1 para os valores de entrada positivos e outra que irá valer 0,2 para os outros valores. Assim, a PReLU consegue manter a simplicidade da ReLU, mas ao mesmo tempo fazendo pequenos ajustes garantindo melhorias de desempenho em redes mais profundas, como as que vimos apresentas por \textcite{He2015}.

\begin{figure}[h!]
    \centering
    \caption{Gráfico da Derivada da Função Parametric ReLU (PReLU) com ($\alpha=0.2$)}
    \begin{tikzpicture}
        \begin{axis}[
            xlabel={$z_i$},
            ylabel={$\text{PReLU}'(z_i)$},
            xmin=-2.3, xmax=2.3,
            ymin=-0.2, ymax=1.2,
            axis lines=middle,
            grid=major
        ]
        % Define um valor exemplo de alpha para o gráfico
        \def\alphaVal{0.2}

        \addplot[red, thick, domain=-2:0] {\alphaVal};
        \addplot[red, thick, domain=0:2] {1};
        \addplot[red, only marks, mark=o, mark size=1.5pt] coordinates {(0,\alphaVal)};
        \addplot[red, only marks, mark=*, mark size=1.5pt] coordinates {(0,1)};
        \end{axis}
    \end{tikzpicture}
    \label{fig:prelu-derivada}
\end{figure}

Um feito importante que deve ser destacado sobre a PReLU, que inclusive é o nome do artigo que foi responsável por introduzir essa função para a comunidade científica, é de que durante a pesquisa do texto \textit{Delving Deep into Rectifiers: Surpassing Human Level Performance on Image Net Classification}, \textcite{He2015} foram capazes de criar uma rede neural capaz de superar a capacidade humana de reconhecer diferentes conjuntos de imagens no ImagneNet Classification, isso ocorreu porque a um humano ao analisar o ImageNet apresenta uma taxa de erro Top-5 de 5.1\% e em um dos testes realizados pelos autores, uma rede neural alcançou uma taxa de erro Top-5 de 4.94\%, superando assim a capacidade humana de reconhecimento de imagens. 

Assim, é nítido destacar que não somente a PReLU, mas as funções retificadoras de forma geral, foram capazes de trazer melhorias significativas para os modelos de aprendizado profundo quando comparadas com as funções sigmoides, as quais foram o padrão da indústria por muitos anos. De fato, a resolução do problema do gradiente em fuga, possibilitou a criação de redes neurais ainda mais profundas, e com isso, sendo capazes de extrair mais informações e consequente melhores métricas, sendo capazes até de superar a capacidade humanas em algumas tarefas como nós vimos no texto.

\textbf{Implementação em Python}

\subsection{Randomized Leaky ReLU}

Anteriormente, na Parametric ReLU, nós tínhamos uma padrão aprendível que era atualizado ao longo da retropropagação da rede, nós o comparamos com o caso do vendedor de pipoca ficando mais experiente para as vendas. No caso da RReLU temos um vendedor um tanto quanto instável, ele se baseia na sorte/aleatoriedade para definir qual será o punhado de pipoca que cada pessoa irá receber ao comprar com um valor abaixo do limiar de venda. Para isso, não existe mais um padrão, algo como o horário ou a quantidade de pessoas na praça para fazer aumentar o diminuir a quantidade de pipoca que tera em um punhado. Essa estratégia parece caótica, mas pode ser interessante caso voce queira instigar as vendas e deixa-las divertidas, você pode comprar um punhado de pipoca, mas nao ira saber quanto irá receber, é uma grande aposta.

Segundo \textcite{XuRReLU}, em \textit{Empirical Evaluation of Rectified Activations in Convolutional Network}, a Randomized Leaky ReLU foi proposta pela primeira vez em uma competição do Kaggle NDSB, ela era uma função semelhante a leaky ReLU mas que o seu coeficiente $\alpha$ é um número aleatório dado pela distribuição normal da forma $U(l, u)$, nessa mesma competição os valores escolhidos para essa distribuição foram de $U(3, 8)$.

Ainda no artigo, \textcite{XuRReLU} investigam a performance de diferentes funções retificadoras em uma rede neural convolucional, cuja arquitetura pode ser vista na tabela \ref{tab:nin_arquitetura}, para a classificação de imagens, os autores analisaram a ReLU tradicional, a Leaky ReLU, a Parametric ReLU e a Randomized Leaky ReLU. Para fazer a análise dos modelos, foram escolhidos os datasets CIFAR-10 e CIFAR-100, e o desempenho dessas funções nos respectivos datasets pode ser visto nas tabelas \ref{tab:ativacoes_cifar10} e \ref{tab:erro_ativacoes_cifar100}.

Analisando a tabela \ref{tab:ativacoes_cifar10}, que mostra a taxa de erro das funções retificadoras na rede NIN para o dataset CIFAR-10, nós podemos notar que a Randomized Leaky ReLU foi a função que performou melhor, com um total de 11.19\% de erro nos casos de teste, enquanto a leaky ReLU com $a = 100$ obteve o pior resultado. Contudo, essa diferença de resultado é pequena, indicando que caso essas funções sejam muito mais complexas quando comparadas com a ReLU na hora de treinar o modelo, pode ser melhor optar por uma função mais "barata" mas com uma taxa de erro um pouco maior.

Já ao analisarmos a tabela \ref{tab:erro_ativacoes_cifar100}, que nos mostra a taxa de erro dessas funções no dataset CIFAR-10, vemos que assim como no CIFAR-10, a RReLU foi a função que obteve melhor resultado, neste caso nós vemos uma diferença de 2.65 pontos percentuais, quando comparada com a ReLU tradicional. Outro ponto interessante a ser destacado ao analisar essa tabela é de que provavelmente a rede que utilizou a Parametric ReLU sofreu um sobreajuste (\textit{overfitting}) fazendo com que ela decorasse o dados de treino e com isso conseguisse uma taxa de erro consideravelmente menor, já quando ela foi apresentada para o conjunto de testes houve uma grande disparidade das taxas de erro.

Agora que entendemos um pouco de como a Randomized Leaky ReLU funciona, podemos entender melhor a sua fórmula, a qual é dada pela expressão \ref{eq:EquacaoRReLU}, note que é a mesma expressão da Leaky ReLU e da PReLU, mas o que muda é o significado do termo $\alpha$ em cada uma delas. Neste caso temos que $\alpha \sim U (l, u)$ em que $l < u$  e $l, u \in [0, 1)$ 

\begin{equacaodestaque}{Randomized Leaky ReLU (RReLU)}
    \text{RReLU}(z_i) = \begin{cases} z_i, & \text{se } z_i > 0 \\ \alpha_i z_i, & \text{se } z_i \leq 0 \end{cases}
    \label{eq:rrelu}
\end{equacaodestaque}

Na fase de testes, devemos calcular a média de todos os valores de $\alpha$ durante o treino, e com isso $\alpha$ se torna uma constante fixa do tipo $(l+u)/2$ de forma que com isso seja possível obter um resultado determinístico, no artigo, os autores utilizam a fórmula \ref{eq:EquacaoRReLU2} para calcular a RReLU durante o teste do modelo \parencite{XuRReLU}.

Conhecendo como a Randomized Leaky ReLU se comporta, podemos plotar o seu gráfico, o qual está presente na figura \ref{fig:GraficoRReLU}. Na representação, é possível ver várias retas, isso ocorre pois elas irão variar de caso a caso, e como a RReLU é uma função que utiliza de conceitos probabilísticos, não podemos garantir um gráfico exato de como ela seria pois não temos os valores de $\alpha$ até que a distribuição seja feita. Note que mesmo com essa particularidade, ela ainda continua sendo uma função bem simples, sendo a construção de duas retas originarias de equações do primeiro grau, a primeira delas sendo a própria função identidade, para os casos em que a entrada é positiva, e a outra é dada pela variável da entrada multiplicada pela constante $\alpha$, para os casos em que a saída é negativa. Além disso, devemos nos atentar também para a sua descontinuidade no ponto zero, é o mesmo problema que acontece com outras variantes, como a ReLU e a Leaky ReLU.

\begin{figure}[h!]
    \centering
    \caption{Gráfico da Função Randomized Leaky ReLU (RReLU) com Diferentes Inclinações Aleatórias para a Parte Negativa}
    \begin{tikzpicture}
        \begin{axis}[
            xlabel={$z_i$},
            ylabel={$\text{RReLU}(z_i)$},
            xmin=-2.3, xmax=2.3,
            ymin=-0.8, ymax=2.3,
            axis lines=middle,
            grid=major,
            legend pos=north west,
            legend style={font=\tiny}
        ]
        \addplot[blue, thick, domain=0:2.3] {x};
        \addplot[red, dashed, domain=-2.3:0, samples=2] {0.1*x};
        \addplot[red, dashed, domain=-2.3:0, samples=2] {0.25*x};
        \addplot[red, dashed, domain=-2.3:0, samples=2] {0.4*x};
        \end{axis}
    \end{tikzpicture}
    \label{fig:rrelu}
\end{figure}

Se conhecemos com a RReLU se comporta podemos também calcular a sua derivada, a qual será de extrema utilidade durante a retropropagação, fazendo que os pesos e vieses do modelo sejam ajustados e com base nisso ele consiga aprender melhor o problema que está sendo analisado. Como a RReLU utiliza a mesma fórmula que funções como a Leaky ReLU e a PReLU, nós podemos apenas repetir a expressão de sua derivada novamente, a qual será dada pela equação \ref{eq:DerivadaRReLU}. Note que mesmo compartilhando a mesma fórmula, o termo $\alpha$ possui significado distintos em cada uma dessas funções, neste caso, ele é um valor aleatório dado pela distribuição $U(l, u)$.

Com relação ao problema da descontinuidade no ponto zero, nós podemos apenas escolher para qual valor essa função irá retornar neste caso, assim, vamos considerar que quando a sua entrada for zero, ela irá retornar o segundo caso, em que é a própria constante $\alpha$

\begin{equacaodestaque}{Derivada Randomized Leaky ReLU (RReLU)}
    \frac{d}{dz_i} [RReLU](z_i) = \begin{cases}1, & \text{se } z_i > 0 \\ \alpha_i, & \text{se } z_i \leqslant  0 \end{cases}
    \label{eq:rrelu-derivada}
\end{equacaodestaque}

Esse detalhe da aleatoriedade da constante $\alpha$ afetou o desenho do gráfico da RReLU, e com isso, ele também afeta a plotagem de sua derivada. Como nós podemos ver na figura \ref{fig:GraficoRReLUDerivada}, existem um conjunto de retas em um intervalo, neste caso, estamos considerando a distribuição como sendo de $U(0.1, 0.3)$, mas para cada uma dessas distribuições, teremos um conjunto de retas diferentes e com isso gráficos distintos para cada um dos problemas.

\begin{figure}[h!]
    \centering
    \caption{Gráfico da Derivada da Função Randomized Leaky ReLU (RReLU) com $l=0.1, u=0.3$}
    \begin{tikzpicture}
        \begin{axis}[
            xlabel={$z_i$},
            ylabel={$\text{RReLU}'(z_i)$},
            xmin=-2.3, xmax=2.3,
            ymin=-0.2, ymax=1.2,
            axis lines=middle,
            grid=major,
            legend pos=north west,
            legend style={font=\scriptsize}
        ]
        % Define l e u para a distribuição uniforme U(l,u)
        \def\lVal{0.1}
        \def\uVal{0.3}

        % Plota a derivada para z > 0
        \addplot[red, thick, domain=0:2.1] {1};
        \addlegendentry{$f'(z_i) = 1$}

        % Plota a região hachurada para z < 0
        \addplot[
            pattern=north east lines, 
            pattern color=blue!50,
            draw=none
        ] coordinates {(-2.1, \lVal) (0, \lVal) (0, \uVal) (-2.1, \uVal)} -- cycle;
        \addlegendentry{$\alpha_i \sim U(l,u)$}

        % Marcadores na descontinuidade
        \addplot[blue, only marks, mark=o, mark size=1.5pt, forget plot] coordinates {(0,\lVal)};
        \addplot[blue, only marks, mark=o, mark size=1.5pt, forget plot] coordinates {(0,\uVal)};
        \addplot[red, only marks, mark=*, mark size=1.5pt, forget plot] coordinates {(0,1)};
        
        \end{axis}
    \end{tikzpicture}
    \label{fig:rrelu-derivada}
\end{figure}

Ainda em \textit{Empirical Evaluation of Rectified Activations in Convolutional Network} \textcite{XuRReLU} explicam que a RReLU é uma função que ajuda a combater o sobreajuste (overfitting) do modelo, mas ainda devem ser feitos mais testes para descobrir como a aleatoriedade afeta os processos de treino e teste \parencite{XuRReLU}. Essa característica de ajudar a combater o sobreajuste é uma vantagem que a RReLU possui, permitindo com que modelos maiores e mais profundos possam ser criados e mesmo assim obtenham resultados significativos. Provavelmente, um dos motivos dela possuir essa função está no fato de que ela introduz uma maior aleatoriedade para o modelo, ajudando a impedir que ele decore os padrões, como em imagens dos conjuntos CIFAR-10 e CIFAR-100.

Considerando as fórmulas da RReLU, podemos construir o seu bloco de código \ref{lst:codigo-rrelu} que é capaz de implementar uma classe em python para atuar como a função Randomized Leaky ReLU.

\textbf{Implementação em Python}

\section{Em Busca da Suavidade: As Variantes Não Lineares}

Seguindo adiante, agora veremos um novo conjunto de variantes da ReLU tradicional, elas incluem funções que apresentam gráficos com curvas mais suaves, como é o caso da ELU, que faz uso de funções exponenciais para a sua composição e com isso consegue não só resolver o problema do neurônios agonizantes, mas também sendo uma função contínua em na origem e portanto derivável em todo o seu domínio.

Além disso, veremos uma variante da ELU, a Scaled Exponetial Linear Unit, uma função que é utilizada para construir redes capazes de se autonormalizarem, ademais veremos a Noisy ReLU, outra variante da ReLU, mas que dessa vez adiciona ruído em sua saída a fim de garantir uma melhor performance quando comparada com a sua função original.

\subsection{Exponential Linear Unit (ELU)}

Continuando com as alogias do vendedor de pipoca, o vendedor de pipoca que faz uso da ELU para as suas vendas trabalha de forma diferente. Ao invés de vender um punhado de pipoca de forma linear com base no dinheiro que o cliente tem quando ele não quer comprar o pacote inteiro por não possuir o valor total, ele adota uma curva exponencial como base, assim, clientes com valores muito próximos de R\$ 5,00 recebem uma quantia muito grande de pipoca, quase equivalente ao pacote total, enquanto aqueles que possuem valores pequenos irão receber um punhado pequeno de pipoca. Talvés seja uma forma de incentivar aqueles que quase tem o valor total para comprar uma pipoca, mas ainda sim garantir a clientela dos que possuem pouco dinheiro e aumentando o seu lucro como vendendor.

A ELU ou Exponential Linear Unit foi introduzida no artigo \textit{Fast and Accurate Deep Networks Leaning By Exponential Linear Units (ELUs)}, sendo uma variação que acelera o aprendizado de uma rede neural densa e apresentando uma maior acurácia em problemas de classificação \parencite{Djork}. Um dos testes realizados pelos autores para analisar o desempenho na ELU, foi na criação de uma CNN com 18 camadas convolucionais para fazer a classifição dos datasets CIFAR-10 e CIFAR-100, para isso, outras técnicas foram utilizas em conjunto como o decaimento do peso L2 e reduções das taxas de aprendizado \parencite{Djork}.

Com base nessa CNN e seus experimentos, é possível ver os resultados na tabela \ref{tab:elu_cifar_comparativo}, em que \textcite{Djork} comparam a ELU com outras redes no mesmo problema, como a AlexNet que vimos anterioremente na explicação do surgimento da ReLU. Ao analisarmos esses resultados, vemos que a ELU obteve um desempenho excelente no dataset CIFAR-100, com uma diferença de 21.52 pontos percentuais quando comparada com a AlexNet, que ficou em último lugar. Já quando analisamos o seu desempenho para um problema de classificação mais simples, como o CIFAR-10, ela ficou em segundo lugar, estando atrás apenas da Fract. Max-POoling, mas ainda sim, apresentando uma diferença considerável de 2.05 pontos percentuais a mais de erro. 

Isso nos indica que a ELU é uma excelente opção para problemas de classificação, especialmente se tivermos um grande número de classes a ser analisados. Contudo, uma rede neural convolucional que apresenta 18 camadas de convolução pode ser um tanto quanto custosa para ser processada por um computador, assim, faz se necessário o uso de unidades de GPUs para o processamento de uma rede como essa, para que mesmo sendo pesada para ser processada, os resultados possam sair um pouco mais rápidos. Em uma das seções a frente, será possível comparar o desempenho das funções retificadoras as quais estamos vendo nesse texto, e a partir desses comparativos, veremos que talvez não seja uma opção interessante utilizar funções que fazem uso de exponenciais em suas fórmulas para resolver problemas de classificação mais simples, como o caso do dataset CIFAR-10.

Agora que entendendo que a ELU pode ser uma boa alternativa para problemas de classifacação de múltiplas classes, podemos conhecer como ela é escrita e como se comporta graficamente. A Exponential Linear Unit pode é descrita utilizando a expressão \ref{EquacaoELU}. Note que temos uma grande diferença dela quando comparamos com a ReLU, a Exponential Linear Unit faz uso de funções exponenciais, algo que é computacionalmente mais pesado para um computador quando comparado com apenas cálculos simples como uma função identidade, podemos esperar que ela será mais lenta quando comparamos com a ReLU, mas para ter certeza precisaremos fazer comprativos, os quais serão vistos futuramente.

\begin{equacaodestaque}{Exponential Linear Unit (ELU)}
    \text{ELU}(z_i) = \begin{cases}z_i, & \text{se } z_i \ge 0 \\ \alpha \cdot (e^{z_i} - 1), & \text{se } z_i < 0\end{cases}
    \label{eq:elu}
\end{equacaodestaque}

Conhecendo como é a fórmula da ELU, podemos também plotar o seu gráfico, o qual está presente na figura \ref{fig:GraficoELU}. Ao analisarmos, vemos uma diferença notável quando o comparamos com as funções vistas anterioemente, a ELU não apresenta um bico no ponto de origem, ela é uma função bem mais suave. Além disso, vemos que ela segue o mesmo padrão das outras funções: ela retorna a função identidade nos casos em que a entrada é maior que zero, assim como as outras variantes, mas quando analisamos a os casos em que a entrada é negativa, vemos que ela utiliza uma funcão exponencial, o que garante a suavidade vista no gráfico. Podemos notar também que ela possui valores negativos, assim como as variantes com vazamento, indicando que ela também pode ser capaz de lidar com o problema do neurônios agonizantes causado pela ReLU tradicional e que vem sendo mitigado com outras variantes como a Leaky ReLU.

\begin{figure}[h!]
    \centering
    \caption{Gráfico da Função Exponential Linear Unit (ELU) com $\alpha=1$}
    \begin{tikzpicture}
        \begin{axis}[
            xlabel={$z_i$},
            ylabel={$\text{ELU}(z_i)$},
            xmin=-2.3, xmax=2.3,
            ymin=-1.2, ymax=2.3,
            axis lines=middle,
            grid=major,
        ]
        % Define alpha para o gráfico. O valor comum para ELU é 1.
        \def\alphaVal{1} 
        \addplot[blue, thick, domain=-2:2] {x >= 0 ? x : \alphaVal*(exp(x) - 1)};
        \end{axis}
    \end{tikzpicture}
    \label{fig:elu}
\end{figure}

Já que sabemos a sua fórmula e o seu gráfico, podemos também calcular a sua derivada, a qual será útil na retropropagação do modelo. Para isso, seguimos a mesma estratégia vista até agora, derivamos a função em cada um dos casos, gerando assim a sua derivada. Nos cenários em que a entrada é positiva, a derivada será sempre 1, pois quando derivamos a expressão $x$, ela nos irá retornar 1. Já quando temos o cenário negativo, teremos como resultado da derivação da expressão $alpha \cdot (e^{z_i} - 1)$ o termo $alpha \cdot e^{z_i}$. Um ponto a ser destacado é de que a ELU é contínua na origem, assim, não temos que nos preocupar em escolher um valor da derivada quando o seu valor de entrada for zero. Assim, temos como resultado final a expressão \ref{eq:DerivadaELU}

Para calcularmos sua derivada, basta derivamos as duas expressões da equação \ref{eq:EquacaoELU} encontrando então a função \ref{eq:DerivadaELU}.

\begin{equacaodestaque}{Derivada Exponential Linear Unit (ELU)}
    \frac{d}{dz_i} [ELU](z_i) = \begin{cases}1, & \text{se } z_i > 0 \\ \alpha \cdot e^{z_i}, & \text{se } z_i \le 0 \end{cases}
    \label{eq:elu-derivada}
\end{equacaodestaque}

Sabendo a sua derivada, podemos também plotar o seu gráfico, para isso, temos a figura \ref{fig:GraficoELUDerivada}. Note que ele é composto de duas partes diferentes, sendo a primeira delas, para os casos em que a entrada é negativa, a função constante em um, e para os casos em que a entrada é negativa, nós temos uma curva exponencial. Podemos notar também que a sua derivada irá sempre retornar valores positivos quando é calcula para qualquer ponto do seu domínio.

\begin{figure}[h!]
    \centering
    \caption{Gráfico da Derivada da Função Exponential Linear Unit (ELU) com $\alpha=1$}
    \begin{tikzpicture}
        \begin{axis}[
            xlabel={$z_i$},
            ylabel={$\text{ELU}'(z_i)$},
            xmin=-2.3, xmax=2.3,
            ymin=-0.2, ymax=1.2,
            axis lines=middle,
            grid=major
        ]
        % Define alpha para o gráfico
        \def\alphaVal{1} 

        % Plota a derivada usando uma única expressão condicional
        \addplot[red, thick, domain=-2.3:2.3, samples=100] {x > 0 ? 1 : \alphaVal*exp(x)};
        
        \end{axis}
    \end{tikzpicture}
    \label{fig:elu-derivada}
\end{figure}

Por fim, cabe destacar algumas afirmações realizadas pelos autores ainda em \textit{Fast and Accurate Deep Networks Leaning By Exponential Linear Units (ELUs)}, segundo \textcite{Djork}, quando nós comparamos a ELU com funções como a ReLU tradicional e Leaky ReLU, nós podemos notar um melhor e mais rápido aprendizado, além de que a Exponetial Linear Unit é capaz de garantir uma melhor generalização quando passa a ser utilizada em redes com mais de cinco camadas. Outro ponto destacado pelos autores, está no fato da ELU garantir noise rebust deactivation states, algo que mesmo com a Leaky ReLU e Parametric ReLU possuindo valores negativos, não são capazes de garantir ao serem utilizadas para construir uma rede neural.

Mesmo apresentando um grande salto, quando comparada com a ReLU tradicional, a ELU também pode ser modificada para atender outros casos. Para isso, ela também tem variações, sendo uma delas a SELU, a qual adiciona a ELU um termo $\lambda$ para garantir uma autonormalização da rede que está sendo criada. Nós veremos essa função em seguida.

\textbf{Implementação em Python}

\subsection{Scaled Exponential Linear Unit (SELU)}

A próxima função que iremos conhecer é uma variante da ELU, a Scaled Exponential Linear Unit, ou SELU. Ela se distingue da ELU tradicional pelo fato de que ela é capaz de implemtar propriedades auto-normalizadoras em uma rede que faz uso dessa função, como explicam os autores no aritofo de sua introdução \textit{Self-Normalizing Neural Networks} \parencite{SELU}.

No texto, \textcite{SELU}, comparam as redes neurais criadas por eles, as quais são chamadas de Self-Normalizing Neural Networks (SNN), com outras redes feedforward, como MSRAinit (uma FNN que não possui técnicas de normalização, com funções de ativação ReLU, e que faz uso do Microsoft weight initialization), a BatchNorm (uma FNN com normalização em lote), a LayerNorm (uma FNN com normalização nas camadas), a WightNorm (uma FNN com normalização nos pesos), a Highway e também com redes residuais ResNet. Para comparar essas redes, os autores escolhem 121 datasets do UCI, em que são apresentadas áreas de aplicação diversas como física e biologia, nesses dadatasets o seus tamanhos podem variar de 10 até 130.000 pontos de dados com o número de features variando de 4 a 250, na tabela \ref{tab:comparativo-selu}, é possível ver o ranking médio entre as SNNs e as outras diferentes arquiteturas em 75 tarefas de classificação.

Para analisar essa tabela, podemos primeiro olhar o ranking médio de cada uma desses modelos, que é dado pela média de como esses modelos performaram nos diferentes datasets, considerando isso, nós notamos que as redes que fazem uso da SELU em sua composição, as SNNs, são as melhores, por uma diferença de 1.4 pontos quando comparadas com o segundo colocado, isso nos indica que a SELU pode ser uma ótima alternativa quando ainda não se sabe exatamente qual será o conjunto de dados que será trabalhado, se ele será de conceitos como física ou geologia, assim, elas garantem uma maior versatilidade para encarar diversos problemas. Além disso, se olharmos também o seu p-value, que vem de um teste de Wilcoxon pareado para verificar se a diferença em relação ao melhor colocado é signifitiva, vemos que as SNNs continuam se destacando, com o p-value mais alto, indicando que existe uma diferença que não é estatisticamente significativa quando ela é comparada com o modelo SVM, nos mostrando que podemos considerar as SNNs como se tivessem empatadas com o campeão.

O interessante dessa comparação é olharmos ela considerando aquelas redes que fazem uso de técnicas de normalização para conseguir uma maior desempenho, como a BatchNorm e a LayerNorm, cada uma delas utiliza uma técnica de normalização diferente de forma a garantir que que problemas como os gradientes explosivos e fuga do gradiente não ocorra com tanta frequência e com isso permitindo um melhor convergência do modelo que está sendo treinado. Se olharmos por essa ótica, podemos chegar a conclusão que criar uma rede neural utilizando a SELU não só ura garantir uma maior versatilidade para a resolução de problemas, como também não teremos que nos preocupar em aplicar técnicas de normalização ao construir essa rede.

Agora que temos uma noção maior de como a SELU pode ser uma opção interessante para criar uma RNA, podemos conhecer como ela funciona de fato. Para isso, os autores apresentam a fórmula \ref{eq:EquacaoSELU} para calcular essa função, em que o termo $\lambda$ é uma constante que será maior que 1 \parencite{SELU}. Note que essa fórmula é bem parecida com a da Exponential Linear Unit, a única diferença é que ela estará sendo multiplicada pela constante $\lambda$, assim podemos escrever também que $\text{SELU}(z_i) = \lambda \text{ELU}(z_i)$.

\begin{equacaodestaque}{Scaled Exponential Linear Unit (ELU)}
    \text{SELU}(z_i) = \lambda \begin{cases}z_i, & \text{se } z_i > 0 \\ \alpha \cdot (e^{z_i} - 1), & \text{se } z_i \le 0\end{cases}
    \label{eq:selu}
\end{equacaodestaque}

Conhendo sua equação, podemos também plotar o seu gráfico, o qual está presente na figura \ref{fig:GraficoSELU}, neste caso, estamos considerando que as constantes $\alpha$ e $\lambda$ são dados por 1.7 e 1.05 respectivamente. Podemos notar que é um gráfico que lembra bastante a Leaky ReLU, mas que neste caso, quando a função recebe valores negativos, ela não estará mais assumindo o comportamento de uma reta, e sim o de uma curva exponencial, já para os cenários em que a entrada é postiva os resultados serão próximos os de uma função identidade, mas com uma reta um pouco mais inclinada. 

Pelo fato da SELU ter um comportamento que também retorna valores para a saída quando a sua entrada é negativa, ela consegue combater o problema dos Dying ReLUs, causado pela ReLU, além de que também não é uma função saturante, como a sigmoide, o que também ajuda a resolver o problema da fuga dos gradientes. Mas, por não ser uma função saturante, e pelo fato de que sua saída vai para valores infinitos conforme os valores de sua entrada aumentam, ela está sujeita ao problema da explosão de gradientes.

\begin{figure}[h!]
    \centering
    \caption{Gráfico da Função Scaled Exponential Linear Unit (SELU) com $\alpha \approx 1.67 e \lambda \approx 1.05$}
    \begin{tikzpicture}
        \begin{axis}[
            xlabel={$z_i$},
            ylabel={$\text{SELU}(z_i)$},
            xmin=-2.3, xmax=2.3,
            ymin=-2, ymax=2.5,
            axis lines=middle,
            grid=major,
        ]
        % Define as constantes da SELU para o gráfico
        \def\alphaVal{1.67326}
        \def\lambdaVal{1.0507}
        \addplot[blue, thick, domain=-2:2, samples=100] {x > 0 ? \lambdaVal*x : \lambdaVal*\alphaVal*(exp(x) - 1)};
        \end{axis}
    \end{tikzpicture}
    \label{fig:selu}
\end{figure}

Considerando agora que sabemos como é a equação da SELU e qual é o seu comportamento no gráfico, podemos também calcular a sua derivada para utilizarmos na retroproapagação do gradiente, para isso, devemos derivar a expressão \ref{eq:EquacaoSELU}, considerando os dois cenários, em que a sua entrada será positiva e quando sua entrada for negativa ou zero. Um ponto que nos ajuda bastante ao calcular a derivada da SELU está no fato dela ser composta pela função ELU multiplicada por uma constante, se utilizarmos regras de derivação para esse cenário, precisaremos apenas derivar a ELU e depois adicionar a constante $\lambda$ multiplicando-a. Como nós já calculamos a derivada da ELU, podemos ver então a derivada da Scaled Exponential Linear Unit na equação \ref{eq:DerivadaSELU}.

\begin{equacaodestaque}{Derivada Scaled Exponential Linear Unit (ELU)}
    \frac{d}{dz_i} [SELU](z_i) = \lambda \begin{cases}1, & \text{se } z_i > 0 \\ \alpha \cdot e^{z_i}, & \text{se } z_i \le 0\end{cases}
    \label{eq:selu-derivada}
\end{equacaodestaque}

Se sabemos a sua derivada, podemos também plotar o seu gráfico, para isso, temos a representação na figura \ref{fig:GraficoSELUDerivada}. Note, que o gráfico da derivada da SELU também possui grandes similaridades com a ELU original mas também com as outras retificadoras, pois também pode ser dividido em duas partes principais. A primeira parte, para os casos em que a entrada é negativa segue o comportamento de uma curva exponencial, enquanto a segunda parte é semelhante a uma reta constante com inclinação zero. 

Um ponto interessante dessa reta da segunda parte é que ela retorna justamente um valor bem próximo de um, assim como nas retificadoras, isso trás como benefício uma menor chance para ocorrer casos de fuga de gradiente, pois ele não estará sendo constantemente sendo multiplicado por valores pequenos e com isso reduzindo o seu valor. Mas, por outro lado, isso também acaba colaborando para que gradientes explosivos possam ocorrer.

\begin{figure}[h!]
    \centering
    \caption{Gráfico da Derivada da Função Scaled Exponential Linear Unit (SELU) com $\alpha \approx 1.67, \lambda \approx 1.05$}
    \begin{tikzpicture}
        \begin{axis}[
            xlabel={$z_i$},
            ylabel={$\text{SELU}'(z_i)$},
            xmin=-2.3, xmax=2.3,
            ymin=-0.2, ymax=2.0, % Ajustado para lambda > 1
            axis lines=middle,
            grid=major
        ]
        % Define as constantes da SELU para o gráfico
        \def\alphaVal{1.67326}
        \def\lambdaVal{1.0507}

        \addplot[red, thick, domain=-2:2, samples=100] {x > 0 ? \lambdaVal*1 : \lambdaVal*\alphaVal*exp(x)};
        \end{axis}
    \end{tikzpicture}
    \label{fig:selu-derivada}
\end{figure}

Voltando para o seu texto de introdução, os autores destacam propriedades importantes dessa nova função de atiavção criada, como o fato de que de que elas possbilitam a criação de redes neurais mais profundas além de serem capazes de aplicar fortes esquemas de regularização \parencite{SELU}. Por favorecer a criação de RNAs mais profundas, como consequência, a SELU se torna uma excelente alternativa para ser utilizada em problemas complexos, que possuem muitas características e por essa razação necessitam de que mais camadas sejam construídas a fim de garantir um melhor processamento e aprendizado dos dados e com base nisso, alcançar métricas maiores, como uma maior acurácia indicando uma gerenalização maior e também uma perda menor, indicando que o gradiente conseguiu uma convergência melhor.

\textbf{Implementação em Python}

\subsection{Noisy ReLU}

Seguindo adiante, a última variante da ReLU que iremos ver nessa seção das variantes suaves, é a Noisy ReLU, também conhecida como NReLU. Um dos trabalhos que explora as característica dessa função e como ela pode ser aplicada em uma RNA é o \text{Rectified Linear Units Improve Restricted Boltzmann Machines} dos autores \textcite{Nair2010}. Nesse texto, os autores comparam o desenpenho dessa função com a função binária, dada pela equação \ref{eq: equacao-binary}, que era a opção mais comum para ser utilizada na construção de máquinas restritas de Boltzmann.

\begin{equation}
    f(z_i) = \begin{cases} 
    1 & \text{se } z_i \ge \theta \\ 
    0 & \text{se } z_i < \theta 
    \end{cases}
    \label{eq: equacao-binary}
\end{equation}

Para entendermos como a comparação dessas funções ocorreu, é interessante primeiro conhecer o conceito das máquinas restritas de boltzmann.

\begin{quote}
    \textit{As Máquinas de Boltzmann Restritas (RBMs) têm sido usadas como modelos generativos de muitos tipos diferentes de dados, incluindo imagens rotuladas ou não rotuladas, sequências de coeficientes mel-cepstrais que representam a fala, sacos de palavras que representam documentos e classificações de usuários de filmes. Em sua forma condicional, elas podem ser usadas para modelar sequências temporais de alta dimensão, como dados de vídeo ou captura de movimento. Seu uso mais importante é como módulos de aprendizagem que são compostos para formar redes de crenças profundas.}

\hfill \textemdash (Nair \& Hinton, 2010, p. 1, tradução nossa)
\end{quote}

No trabalho, \textcite{Nair2010} exploram o desempenho dessas duas funções utilizando o dataset NORB, que é um dataset para o reconhecimento de objetos 3D sintéticos que contém cinco classes: humanos, animais, carros, aviões e caminhões. No texto, os autores utilização a vrsão Jittered-Cluttered NORB, uma variante que tem imagens estereoscópicas em tons de cinza com fundo desorganizado e um objeto central que é aleatoriamente instável em posição, tamanho, intensidade de pixels \parencite{Nair2010}. O desempenho dessas funções nesse dataset pode ser visto pelas tabelas \ref{tab:norb_error_rate} e \ref{tab:nrelu_norb_comparativo}.

A tabela \ref{tab:norb_error_rate}, nos mostra a taxa de erro dos diferentes modelos no dataset, para isso, são construídos modelos com 4000 unidades ocultas treinados com imagens de dimensões 32x32x2. Podemos notar ao analisar a tabela que é nítido que os modelos que são pré-treinados utilizando máquinas de Boltzmann restritas apresentam um melhor desempenho de forma geral, da mesma forma que a NReLU oferece uma perda menor quando comparada com a função binária em ambos os casos: pré-treinado ou não. 

Mas, podemos fazer uma comparação ainda mais interessante, o modelo que faz uso da Noisy ReLU que não foi pré-treinado apresenta um resultado com uma diferença de 0.9 pontos quando comparado com o modelo treinado que faz uso da função binária. Isso é interessante porque nos indica que podemos conseguir um resultado muito melhor utilizando a Noisy ReLU mediante a função binária mesmo quando não tivermos condições de pré-treinar uma rede.

\begin{table}[ht]
    \caption{Taxa de Erro de Classificadores no Dataset NORB Jittered-Cluttered}
    \label{tab:norb_error_rate}
    \centering
    \begin{tabular}{lcc}
        \toprule
        \textbf{Pré-treinado?} & \textbf{NReLU (\%)} & \textbf{Binary (\%)} \\
        \midrule
        
        Não & 17.8 & 23.0 \\
        Sim & \textbf{16.5} & \textbf{18.7} \\
        
        \bottomrule
    \end{tabular}
    \vspace{2mm}
    
    \parbox{\linewidth}{\small
        \textit{Nota.} Taxas de erro no conjunto de teste para classificadores com 4000 unidades ocultas. Os valores em negrito indicam a menor taxa de erro (melhor resultado) em cada coluna. Os modelos foram treinados com imagens de 32x32x2 do dataset NORB Jittered-Cluttered. A coluna "NReLU" refere-se a unidades de ativação ReLU com ruído (Noisy ReLU), enquanto "Binary" refere-se a unidades binárias tradicionais. A coluna "Pré-treinado?" indica se o modelo utilizou uma Máquina de Boltzmann Restrita para pré-treinamento.
        Adaptado de "Rectified Linear Units Improve Restricted Boltzmann Machines", por V. Nair \& G. E. Hinton, 2010, \textit{Proceedings of the 27th International Conference on Machine Learning (ICML-10)}.
    }
\end{table}

Seguindo adiante, na tabela \ref{tab:nrelu_norb_comparativo} tembém é possível analisar as taxas de erro dos classicadores, mas, neste caso, eles possuem duas camadas ao invés de somente uma como mostrado da comparação da tabela \ref{tab:norb_error_rate}, assim a primeira camada é composta de 4000 unidades ocultas (assim como no primeiro caso), enquanto a segunda camada possui 2000 unidades ocultas. 

Com base essa tabela nós podemos notar que o desempenho dos modelos que fazem uso da Noisy ReLU melhorou tanto nos casos em que as camadas não foram pré-treinadas, quanto nos casos em que uma ou ambas foram, quando se compara com os resultados da tabela \ref{tab:norb_error_rate}. Um ponto interessante disso é que no caso em que somente uma das camadas foi pré-treinada no modelo que usa a NReLU o seu resultado foi igual ao do primeiro caso onde existia somente uma camada, o que pode nos indicar que talvez não seja vantajoso adicionar mais camadas em uma rede caso não estejamos dispostos a pré-treiná-las. Note também que esse cenário com a unidade binária foi ainda pior, nos mostrando que não temos um grande ganho em adicionar mais camadas em uma rede que faz uso dessa função.

Esse resultado prová-se ainda mais nítido quando vemos o caso em que ambas as camadas foram pré-treinadas, na unidade binária não houve nenhuma diminuição na sua perda, ela inclusive é pior que a do modelo que faz uso de apenas uma camada. Já quando observamos a Noisy ReLU e seus resultados, vemos um cenário bem diferente, os modelos que fazem uso dela apresentam um desempenho melhor quando são adicionadas mais camadas na rede, fazendo com que a perda da rede diminua, indicando que essa função permite a criação de redes mais profundas e com isso, desempenhos melhores possam ser alcançados.

Esse tópico de permitir a criação de redes mais profundas, que aconteceu justamente ao optar por funções retificadoras como a ReLU e a Noisy ReLU ao invés de funções sigmoide, está intrinsicamente relacionado ao problema da fuga do gradiente. Em redes que faziam uso de funções sigmoide, os programadores e pesquisadores estavam constantemente correndo o risco de que ao adicionar mais camadas a fim de que essa rede pudesse alcançar melhores métricas, o problema da fuga do gradiente viesse a tona. Isso acontece, porque ao adicionar mais camadas, existe uma maior chance de que esse vetor seja mais uma vez multiplicado por valores pequenos e com isso diminuísse o seu valor.

Sabendo de como a NReLU surgiu e sua importância para permitir a criação de redes mais profundas, podemos conhecer a sua fórmula e entender como ela funciona na prática. No texto, \textcite{Nair2010}, apresentam a NReLU como sendo dada pela equação $\max{0, z_i + \mathcal{N}(0, \sigma(z_i))}$, em que $\mathcal{N}(0, V)$ representa o ruído Gaussiano (Gaussian noise em inglês), com méida zero e vairância dada por $V$. Como neste texto estamos dando maior enfoque em uma notação com condicionais para as funções retificadoras, também podemos expressar a Noisy ReLU com a equação \ref{eq: EquacaoNReLU}.

\begin{equation}
    \text{NReLU}(z_i) = \begin{cases} 
    0 & \text{se } z_i \le 0 \\ 
    z_i + \mathcal{N} (0, \sigma(z_i)) & \text{se } z_i > 0 
    \end{cases}
    \label{eq: EquacaoNReLU}
\end{equation}

Antes de seguir em frente, é útil entende primeiro o que é ruído gaussiano, e para isso, precisamos entender antes a distribuição gaussiana. Segundo \textcite{Goodfellow2016}, a distribuição gaussiana é a distribuição mais utilizada para números reais, ela também é conhecida também por ser chamada de distribuição normal, ela é dada pela equação \ref{eq: DistribuicaoGaussiana}.

\begin{equation}
    \mathcal{N}(x; \mu, \sigma^2) = \sqrt{\frac{1}{2\pi\sigma^2}} \exp\left( -\frac{1}{2\sigma^2}(x - \mu)^2 \right)
    \label{eq: DistribuicaoGaussiana}
\end{equation}

Nessa equação, os dois parâmetros $\mu \in \mathbb{R}$ e $\sigma (0, \infty)$ controlam como a distribuição normal funciona; o termo $\mu$ é responsável por dar as coordenadas para o pico do centro, que é também a média da distribuição $\mathbb{E}[x] = \mu$, já o desvio padrão é dado por $\sigma$, equanto a variância é denotada por $\sigma^2$ \parencite{Goodfellow2016}. Para chegarmos no ruído gaussiano, precisamos então adicionar como parâmetros da equação da distribução gaussiana (equação \ref{eq: DistribuicaoGaussiana}) os termos que são dados pela equação \ref{eq: EquacaoNReLU}. 

A distribuição normal, com os parâmetros $\mu = 0$ e $\sigma=1$, é responsável por nos dar um gráfico em formato de sino, como é mostrado na figura \ref{fig:GraficoDistNormalPadrao}. Esse gráfico indica quais são os casos que possuem uma maior probabilidade de acontecer, os casos que estão no centro, onde, $p(x)$ possuem uma maior probabilidade de acontecer, já conforme eles se distânciam desse centro essa propabilidade diminui.

\begin{figure}[htbp]
    \centering
    \caption{Gráfico da Distribuição Gaussiana (ou Normal) para o caso padrão, com média 0 ($\mu = 0$) e desvio padrão 1 ($\sigma = 1$).}
    \begin{tikzpicture}
        \begin{axis}[
            xlabel={$x$},
            ylabel={$p(x)$}, % p(x) é a densidade de probabilidade
            xmin=-4, xmax=4,
            ymin=0, ymax=0.5,
            axis lines=middle,
            grid=major,
            samples=200, % Aumenta o número de pontos para uma curva mais suave
            domain=-4:4,
        ]
        
        % Declara a função da distribuição normal para facilitar o uso
        \def\normaldist#1#2{1/(#2*sqrt(2*pi))*exp(-((x-#1)^2)/(2*#2^2))}
        
        % Adiciona a área sombreada para +/- 1 desvio padrão
        \addplot[fill=blue!20, draw=none, domain=-1:1] {\normaldist{0}{1}} \closedcycle;

        % Plota a curva da distribuição normal padrão (mu=0, sigma=1)
        \addplot[blue, thick] {\normaldist{0}{1}};

        % Adiciona linhas verticais para marcar a média e os desvios padrão
        \draw[dashed, gray] (axis cs:0, 0) -- (axis cs:0, 0.45);
        \draw[dashed, gray] (axis cs:1, 0) -- (axis cs:1, 0.24);
        \draw[dashed, gray] (axis cs:-1, 0) -- (axis cs:-1, 0.24);

        % Adiciona os labels para a média e desvio padrão
        \node[below] at (axis cs:0, 0) {$\mu=0$};
        \node[below] at (axis cs:1, 0) {$\sigma$};
        \node[below] at (axis cs:-1, 0) {$-\sigma$};
        
        \end{axis}
    \end{tikzpicture}
    \label{fig:GraficoDistNormalPadrao}
\end{figure}

Entendendo a estrutura e como funciona a Noisy ReLU, podemos também plotar o sey gráfico, o qual é dado pela figura \ref{fig:GraficoNReLU}. Vemos que ela compartilha a suavidade das funções ELU e SELU, apresentando uma curva para os casos em que a entrada é negativa, o que é bom, pois ela permite um vazamento de gradiente, evitando assim o Dying ReLU problem. Já para os cenários que a entrada é positiva ela assume o comportamento de uma função identidade, lembrando bastante as outras variantes da ReLU. Contudo, como os autores destacam no texto, ela não é capaz de resolver o problema da descontinuidade em zero, para isso, em sua derivada que é utilizada na retropropagação, os casos em que sua entrada é zero irão retornar zero como saída \parencite{Nair2010}.

\begin{equacaodestaque}{Noisy ReLU (NReLU)}
    \text{NReLU}(z_i) = \begin{cases} 
    0 & \text{se } z_i \le 0 \\ 
    z_i + \mathcal{N} (0, \sigma(z_i)) & \text{se } z_i > 0 
    \end{cases}
    \label{eq:nrelu}
\end{equacaodestaque}

Entendendo a estrutura e como funciona a Noisy ReLU, podemos também plotar o sey gráfico, o qual é dado pela figura \ref{fig:GraficoNReLU}. Vemos que ela compartilha a suavidade das funções ELU e SELU, apresentando uma curva para os casos em que a entrada é negativa, o que é bom, pois ela permite um vazamento de gradiente, evitando assim o Dying ReLU problem. Já para os cenários que a entrada é positiva ela assume o comportamento de uma função identidade, lembrando bastante as outras variantes da ReLU. Contudo, como os autores destacam no texto, ela não é capaz de resolver o problema da descontinuidade em zero, para isso, em sua derivada que é utilizada na retropropagação, os casos em que sua entrada é zero irão retornar zero como saída \parencite{Nair2010}.

\begin{figure}[h!]
    \centering
    \caption{Gráfico da Função Noisy ReLU (NReLU)}
    \begin{tikzpicture}
        \begin{axis}[
            xlabel={$z_i$},
            ylabel={$\text{NReLU}(z_i)$},
            xmin=-1.3, xmax=2.3,
            ymin=-2.3, ymax=2.3,
            axis lines=middle,
            grid=major,
            domain=-0.999:2, % Domínio para evitar o log de zero ou negativo
        ]
        % A função ln(1+x) está definida para x > -1
        \addplot[blue, thick] {x >= 0 ? x : ln(1+x)};
        % Linha vertical para mostrar a assíntota em z = -1
        \draw[dashed, gray] (axis cs:-1, -2.3) -- (axis cs:-1, 2.3);
        \end{axis}
    \end{tikzpicture}
    \label{fig:nrelu}
\end{figure}

Quando vamos calcular a sua derivada, encontramos um problema que não havia aparecido nas outras funções. O termo de ruído $\mathcal{N}$ é um termo não determísitico, o que significa que mesmo que tivéssemos a mesma entrada para a função Noisy ReLU duas ou mais vezes, não poderíamos afirmar com certeza de que essas saídas seriam iguais. Para resolver esses problemas, os autores consideram para a NReLU que sua função para o backward pass será irá retornar zero quando o valor de entrada for negativo ou nulo, e irá retornar um, quando o valor de entrada for positivo \parencite{Nair2010}. Então a expressão que representa a NReLU para a retroproagação é dada pela equação \ref{eq: DerivadaNReLU}, note que ela é a mesma expressão da derivada da ReLU tradicional.

\begin{equacaodestaque}{Derivada Noisy ReLU (NReLU)}
    \frac{d}{dz_i}[\text{NReLU}](z_i) = \begin{cases} 
    0 & \text{se } z_i \le 0 \\ 
    1 & \text{se } z_i > 0 
    \end{cases}
    \label{eq:nrelu-derivada}
\end{equacaodestaque}

Se a função da Noisy ReLU no backward pass será a mesma que a derivada da ReLU, podemos então utilizar como base o gráfico da derivada da ReLU. Para isso, temos então a figura \ref{fig:GraficoNReLUDerivada}.

\begin{figure}[htbp] % Use [htbp] para dar flexibilidade ao LaTeX
    \centering % Centraliza o gráfico na página
    \caption{Gráfico da função Backward Pass da Noisy ReLU}
    \begin{tikzpicture}
        \begin{axis}[
            xlabel={$z_i$},
            ylabel={$\text{NReLU}'(z_i)$},
            xmin=-2.3, xmax=2.3,
            ymin=-0.8, ymax=1.8,
            axis lines=middle,
            grid=major
        ]
        \addplot[red, thick, domain=-2:0] {0};
        \addplot[red, thick, domain=0:2] {1};
        \addplot[red, only marks, mark=o, mark size=1.5pt] coordinates {(0,0)};
        \addplot[red, only marks, mark=*, mark size=1.5pt] coordinates {(0,1)};
        \end{axis}
    \end{tikzpicture}
    \label{fig:nrelu-derivada}
\end{figure}

Ainda em \textit{Rectified Linear Units Improve Restricted Boltzmann Machines}, \textcite{Nair2010} citam que uma das propriedades interessantes da NReLU é a intensity equivarience (), a qual é bem útil para o reconhecimento de objetos. No texto, os autores destacam que um dos principais objetivos ao contruir um sistema que faça o reconhecimento de objetos, é garantir que a saída seja invariante às propriedades da sua entrada, como localização, escala, iluminação e orientação, e a NReLU é uma das fubções que quando adicionada em uma rede neural, garante que isso possa ser atingido \parencite{Nair2010}.

\textbf{Implementação em Python}

\section{O Problema dos Gradientes Explosivos}

Anteriormente, nós conhecemos as funções sigmoides, vimos que elas eram comumente utilizadas como as funções de ativação padrão de uma rede neural antes das funções retificadoras. Mas elas possuíam um problema, o do gradiente em fuga. Esse problema acontecia porque essas funções retornavam sempre números muito pequenos em suas derivadas, que consequentemente eram multiplicadas no \textit{backward pass} com o gradiente retropropagado gerando como produto um número pequeno, esse número era então novamente multiplicado por outra constante de baixo valor e por aí vai, como resultado, o gradiente retropropagado que chegava nas primeiras camadas para atualizar os pesos e vieses da rede possuía um valor tão pequeno que muitas vezes não resultava em uma atualização capaz de gerar impacto no aprendizado da rede. Assim, tínhamos o problema do gradiente em fuga.

Já nessa seção nós conhecemos a ReLU, e vimos que ela veio como uma alternativa para as funções sigmoides e que era capaz de corrigir esse problema. Contudo, ela também possui problemas, pelo fato de ser uma função que não é suturante como as sigmoides, isso faz com que ela possa gerar o problema da explosão do gradiente, para entendê-lo primeiro precisamos nos lembrar da fórmula do gradiente retropropagado.

\[
        \frac{\partial E}{\partial w_{ik}} = 
            \left( \sum_j \frac{\partial E}{\partial x_j} \cdot w_{ji} \right)
        \cdot 
            \sigma'(z_i)
        \cdot 
            a_k^{\text{ant}}
\]

O primeiro termo, o que possui o somatório, é responsável por calcular o erro total que chega para um neurônio $i$ da camada atual, para isso temos a equação \ref{eq: ErroRetropropagado}.

\begin{equation}
    \text{Erro}_i = \left( \sum_j \frac{\partial E}{\partial x_j} \cdot w_{ji} \right)
    \label{eq: ErroRetropropagado}
\end{equation}

$\partial E / \partial x_j$ nos indica o gradiente do erro em relação à saída de um neurônio $j$ que está na camada seguinte da rede. Para nós calcularmos ele, devemos aplicar a regra da cadeia, assim $\partial E / \partial x_j$ será dado pelo cálculo do erro vindo depois da camada $j$, uma camada $k$ por exemplo, assim, o erro será dado pela equação \ref{eq: ErroRetropropagadoK}.

\begin{equation}
    \text{Erro}_j = \left( \sum_k \frac{\partial E}{\partial x_k} \cdot w_{kj} \right)
    \label{eq: ErroRetropropagadoK}
\end{equation}

Note que temos um padrão nos cálculos dos erros.

\begin{enumerate}
    \item Para que nós possamos calcular o erro de uma camada $w_ba$ devemos encontrar o $\text{Erro}_b$;
    \item Para encontrar o $\text{Erro}_b$, precisamos calcular o gradiente da camada seguinte: $\partial E / \partial x_c$
    \item O gradiente ($\partial E / \partial x_c$) é calculado a partir do $\text{Erro}_c$
    \item Para calcular o $\text{Erro}_c$, precisamos da camada seguinte e seu gradiente $\partial E / \partial x_d$ que envolve os pesos $w_de$
    \item E assim por diante até a última camada da rede.
\end{enumerate}

Com base nessa análise podemos encontrar uma expressão que será capaz de nos mostrar qual será o valor do gradiente retropropagado agora considerando as atualizações de pesos que é feita na rede, ela é dada pela equação \ref{eq: GradienteRetropropagado}.

\begin{equation}
    \delta^l \propto (W^{l+1})^T \cdot (W^{l+2})^T \cdots (W^L)^T \cdot \delta^L 
    \label{eq: GradienteRetropropagado}
\end{equation} 

Note que, com base nessa expressão, se um cenário em que sempre houver valores muito grandes nos pesos das camadas da rede, ou ate mesmo muitas camadas, o gradiente que chegará na última camada poderá ter um valor muito alto, podendo atrapalhar o desempenho da rede neural e como ela irá aprender, assim temos um sério problema ao construir uma rede.

Esse tipo de problema, é chamado de problema da explosão do gradiente e ele pode ser definido da seguinte forma:

\begin{quote}
    Quando o erro é retropropagado por uma rede neural, ele pode aumentar exponencialmente de camada para camada. Nesses casos, o gradiente em relação aos parâmetros em camadas inferiores pode ser exponencialmente maior do que o gradiente em relação aos parâmetros em
    camadas superiores. Isso torna a rede difícil de treinar se ela for suficientemente profunda.
\end{quote}

\parencite[p.~2]{explodingGradient}

Assim, mesmo a ReLU corrigindo o problema do gradiente em fuga, ela acabou por introduzir uma nova categoria de problemas para uma rede neural. Acontecimentos assim são comuns, muitas vezes queremos concertar algo mas acabamos por atrapalhar outra parte de um projeto de rede neural, por isso, devemos escolher com calma quais funções serão utilizadas além de realizar testes para garantir uma melhor performance do modelo que está sendo criado.

\section{Comparativo de Desempenho das Funções Retificadoras}

Além do problema do explosão de gradientes que pode ocorrer ao utilizar a ReLU em uma rede neural, essa função também sofre de outro problema crônico: o do neurônios agonizantes (ou \textit{Dying ReLU Problem} em inglês). Esse problema pode ser definido como uma variação do problema do gradiente em fuga, em que uma parte dos neurônios da rede morrem e passam a retornar somente zero independente de sua entrada \parencite{dyingReLU}.

Para entendermos melhor devemos nos lembrar da estrutura básica de uma camada densa de neurônios de uma rede neural, de forma que ela pode ser vista na equação \ref{eq: EquacaoNeuronio2}, em que $W$ representa um vetor de pesos para os neurônios dessa camada, $X$ é o vetor de elementos de entrada e $b$ é o viés.

\begin{equation}
    y = W^T X + b
    \label{eq: EquacaoNeuronio2}
\end{equation}

Se estivemos utilizando a ReLU como função de ativação dessa camada, essa variável $y$ irá passar para uma função $\max(0, y)$, podem existir conjuntos de dados que quando passados por essa camada, a condição mais comum de ocorrer será a de $y < 0$ para os diferentes padrões de treinamento, isso tem efeito no \textit{backward pass}, em que vamos utilizar a derivada da ReLU, que também será zero para casos em que $y < 0$, e nós sabemos que com a derivada dessa função nós calculamos o gradiente que irá calcular o como o peso da rede irá mudar de uma época para outra, se essa saída for zero, quando ela for multiplicada ao calcular o gradiente ele também será zero, e consequentemente os pesos e vieses da rede neural não serão atualizados, se eles não são atualizados, o neurônio não aprende, tendo então o problema do neurônios agonizantes \parencite{douglasDyingRelu}.

Essa condição de vários neurônios morrendo causada pela ReLU acabou por gerar um novo conjunto de funções, as quais possuem propriedades comuns da ReLU, como a não linearidade e a simplicidade nos cálculos mas que buscam resolver ou amenizar esse problema em uma rede neural. Uma das funções que busca resolver esse problema é a Leaky ReLU \parencite{douglasDyingRelu}.

Agora que conhecemos a ReLU, vimos como ela surgiu e se popularizou, além de suas propriedades e problemas, podemos conhecer as suas variações, as quais buscam corrigir alguns problemas que a função original possui mas mesmo assim mantendo as sua essência como inspiração para a criação de uma função melhor. Primeiro, conheceremos as variações com vazamentos, começando pela Leaky ReLU ou LReLU.
% ===================================================================
% Arquivo: capitulos/parte-III-pilares/cap-09-modernas.tex
% ===================================================================

\chapter{Funções de Ativação Modernas e Outras Funções de Ativação}
\label{cap:ativacao-modernas-outras}

O texto do seu capítulo começa aqui...
% ===================================================================
% Arquivo: capitulos/parte-III-pilares/cap-10-perda-binaria.tex
% ===================================================================

\chapter{Funções de Perda para Regressão}
\label{cap:perda-regressao}

Agora agora foi visto o funcionamento da retropropagação, e como ela faz uso dos otimizadores, os quais funcionam como um barco, percorendo as ondas em busca dos pontos de mínimo. Além disso, foram vistas em seguida diversas funções de ativação, começando pelas sigmoidais, depois pelas retificadoras, e por fim uma coletânea de diferentes funções. Contudo, está na hora de entender um outro lado da retropropagação: as funções de perda, as quais são justamente as ondas que os otimizadores percorrem.

Para isso, esse capítulo busca explicar algumas dos diferentes tipos de funções de perda, mais precisamente as funções de perda para tarefas de regressão. Assim, o capítulo pode ser dividido em quatro grandes partes: funções de perda para prorósitos gerais, funções de perda para medir o erro relativo, funções que vão além do cálculo da média dos erros, por último funções que são utilizadas para problemas que seguem outros tipos de distribuição (como as distribuições de Poisson e de Tweedie).

Para explicar cada uma das funções é apresentado as suas equações, os gráficos (contendo as vistas em duas e três dimensões), as derivadas parciais junto com os seus respectivos gráficos. Além disso, no final de cada explicação dessas funções, e selecionado uma série de artigos que exploram o uso dessas funções para resolver problemas variados de regressão. Já no final do capítulo, pode ser visto uma tabela resumo, explicando as principais características das funções e seus usos, também é possível ver uma seção que apresenta um diagrama, que serve de guia para escolher a função de perda ideal para um problema de regressão.

\section{Exemplo Ilustrativo: Jogando Dardos}

Pense que você está jogando dardos com seus amigos e quer decidir quem está com mais pontos. Mas você não está satisfeito em considerar as marcações que estão no jogo, e decidiu inovar. Para isso, você pegou uma régua e passou a medir a distância que os dardos que você e seus amigos haviam jogado no centro. Quem chegasse mais próximo do centro, ganhava o jogo.

Essa ideia de medir o quão próximo você está do resultado desejado utilizando a disância entre esses dois pontos como parâmetro, é justamente o motivador pela criação das funções de perda para regressão. Para isso, elas utilizam diferentes fórmulas, com todas com o mesmo intuito, medir a distância em que o "chute" dado pelo modelo está do ponto real (desejado).

\section{Características das Funções de Perda}

Assim, antes de discutir as funções de perda para tarefas de regressão, é importante citar as principais propriedades que uma função de perda, seja ela para tarefas de regressão ou para outros tipos de tarefa pode ter. Dito isso, em \textit{Loss Functions and Metrics in Deep Learning}, \textcite{LossesArticle} explicam algumas características desse grupo de funções. Para isso, os autores argumentam que algumas propriedades das funções de perda são: convexidade, diferenciabilidade, robustez, suavidade, esparsidade, monotocidade \parencite{LossesArticle}. Todas essas propriedades devem ser consideradas quando estiver escolhendo a função de perda ideal para resolver um determinado tipo de problema.

Para isso, cabe agora discutir essas propriedades.

\medskip
\textbf{Convexidade}
\medskip

Uma função convexa é uma função na qual qual ponto de mínimo local é também o ponto de mínimo global \parencite{LossesArticle}. Uma forma fácil de identificar se uma função que está sendo analisada é convexa ou não é verificar se ela possui o formato de um funil ou um formato da letra "V". Funções convexas são ideais para ser utilizadas como funções de perda porque facilitam a otimização utilizando métodos baseados em gradiente. Como foi visto no Capítulo \ref{cap:retropropagacao-gradiente}, é bem mais fácil para o modelo encontrar pontos de mínimo em uma função que apresenta o formato de um funil do que em uma função não-convexa, cheia de ondas e com vários pontos de mínimo locais.

\medskip
\textbf{Diferenciabilidade}
\medskip

A questão da diferenciabilidade está realacionada também com a utilização de otimizadores baseados em gradiente. Ao utilizar a retropropagação para fazer o ajuste de parâmetros, o primeiro cálculo do gradiente será pelas derivadas parciais da função de perda, as quais juntas formam o vetor gradiente que será propagado por toda a rede, das camadas mais próximas da saída até as camadas mais próximas da entrada, ajustando os valores dos pesos e vieses do modelo. Utilizar uma função que não seja diferenciável ou que possuam muitos pontos de descontinuidae certamente irá afetar negativamente o algoritmo da retropagação, atrapalhando o aprendizado com o uso do gradiente \footnote{Vale dizer também que mesmo que uma função apresente um ou outro ponto de descontinuidade, pode ser que isso não atrapalhe a otimização por gradiente, um exemplo disso é a função erro absoluto médio, que a apresenta um ponto de descontinuidade, mas, é possível contornar esse "problema" utilizando o cálculo do subgradiente.}.

\medskip
\textbf{Robustez}
\medskip

\medskip
\textbf{Suavidade}
\medskip

Falando agora da suavidade, ela se relaciona com a questão da continuidade da função. Uma função que não é suave, que apresenta pontos de descontinuidade, como "bicos" ou transições bruscas entre um pedaço e e outro da função afeta diretamene o cálculo da derivada e consequêntemente do gradiente. Esses pontos de descontinuidade, são um empecilho para o cálculo da derivada, pois uma função não pode ser derivada em um ponto no qual não é contínua. Considerar se uma função é suave ou não é ideal pois também irá refletir no cálculo do gradiente dessa função de perda.

\medskip
\textbf{Esparsidade}
\medskip

\medskip
\textbf{Monotocidade}
\medskip

\section{Funções de Perda para Regressão para Propósitos Gerais}

A primeira seção desse capítulo que explora as funções de perda busca focar nas funções de perda mais comuns, que geralmente são umas das primeiras alternativas para serem utilizadas ao contruir um modelo de regressão. Entre elas, vale destacar a dupla de funções erro quadrático médio (também conhecido como perda L2) e o erro absoluto médio (que também recebe o nome de perda L1). Contudo, essa seção também apresenta outras alternativas, como a perda de Huber, uma função que tem como objetivo unir as principais características do \textit{MSE} e do \textit{MAE}. Além disso, também é apresentada uma alternativa para a perda de Huber, a Log-Cosh \textit{Loss}, que apresenta propriedades parecidas com essa outra função, mas que resolve os problemas de descontinuidade.

\subsection{Erro Quadrático Médio (MSE)} \index{Funções de Perda!Erro Quadrático Médio (MSE)}
\label{sec:mse-loss}

A primeira função de perda a ser vista nesse capítulo é a erro quadrático médio (\textit{mean squared error}) também conhecida com \textit{MSE} ou perda L2. Suas origens são voltadas para o século XIX, com o avanço dos estudos da astronomica, em que os estudiosos buscavam entender o comportamento das estrelas e dos outros planetas. A \textit{MSE} surge naturalmente no trabalho \textit{Nouvelles méthodes pour la détermination des orbites des comètes} (Novos métodos para determinar as órbitas dos cometas), nele \textcite{Legendre1805} introduz o método conhecido como método dos mínimos quadrados, o qual tem como objetivo minimizar a soma dos quadrados dos erros.

Contudo, o trabalho de Legendre não foi o único resposável por popularizar o método dos mínimos quadrados. Em \textit{Theoria Motus Corporum Coelestium in Sectionibus Conicis Solem Ambientium}, \textcite{Gauss1809}, em uma série de artigos, discute um problema que se inicia com um sistema de equações linerares com mais equações que icognitas derivadas de observações atronômicas que possuem erros, seu objetivo é então encontrar o valor mais provável para essas icôgnitas.

Para isso, Gauss define que o valor mais provável de uma quantidade de medidas de igual precisão é dado pela média aritmética dessas medidas \parencite{Gauss1809}. Com base nisso, \textcite{Gauss1809} se pergunta qual deve ser a lei de probabilidade dos erros para que a média aritmética seja sempre a estimativa mais provável, para resolver esse problema ele utiliza o princípio da máxima-verossimilhança das probabilidades de todos os erros, e com ele, Gauss consegue ser capaz de demonstrar matematicamente que a única função que satisfaz o seu postulado da média aritmética é a própria distribuição Normal.

Sabendo disso, Gauss inverte a lógica: se a probabilidade de um conjunto de erros é máximizada quando a soma dos seus quadrados é minimizada, então o método dos mínimos quadrados é o método que dá a solução mais provável para a suposição de que os erros de medição são normalmente distribuídos \parencite{Gauss1809}. Dessa forma, o matemático foi capaz de adicionar mais embasamento matemático na técnica de Legedre, e com isso aumentando a popularização do \textit{MSE}.

Passado quase 200 anos, o erro quadrático médio se torna uma das principais funções a ser utilizada para calcular o erro dos modelos. Caso você tenha lido o Capítulo \ref{cap:retropropagacao-gradiente}, pode ter notado que ela foi uma das funções, junto com a sigmoide logística e a equação do neurônio, a ser utilizada para deduzir os cálculos das atualizações de pesos para o algoritmo da retropropagação \parencite{BackpropagationArticle}.

Voltando para analogia do jogo dos dardos, é possível extendê-la para explicar o erro quadrático médio. Imagine que no jogo de dardos todos os jogadores começam com 1.000 pontos, e que vão perdendo conforme vão errando os lançamentos. Além disso, você definiu uma regra que diz que o erro (ou débito dos pontos totais) será dado pelo quadrado da distância entre o centro até o dardo jogado. Assim, jogadores que são muito precisos, e com isso acertam dardos mais próximos do centro perdem poucos pontos, contudo, aqueles jogadores que são mais desleixados e acertam longe do centro acabam perdendo muitos pontos, porque seu além de já perderem muitos pontos por estarem longe do centro, essa distância ainda será elevada ao quadrado.

Essa ideia de elevar o erro ao quadrado é o grande motivador para entender o cálculo do \textit{MSE}. Dito isso, agora cabe finalmente analisar essa função de fato, em primeiro lugar será analisada a sua fórmula, a qual está apresentada na Equação \ref{eq:mse}.

\begin{equacaodestaque}{Erro Quadrático Médio (\textit{MSE})}
    \Loss_{\text{MSE}} (y_j, \hat{y}_j) = \frac{1}{N} \sum_{j=1}^{N} (y_j - \hat{y}_j)^2
    \label{eq:mse}
\end{equacaodestaque}

Em que:

\begin{itemize}
    \item $y_j$ representa o valor real para a saída;
    \item $\hat{y}_j$ representa o valor previsto pelo modelo;
    \item $N$ representa o número de predições feitas.
\end{itemize}

Neste caso, o \textit{MSE} calcula o erro individualmente para cada uma das predições, soma esses valores e em seguida calcula a média.

Tendo a sua equação, é possível também plotar o seu gráfico, para isso, ele pode ser visto na Figura \ref{fig:mse}. Note que existem duas figuras, a primeira a esquerda, Figura \ref{fig:mse-2d} mostra uma visão em duas dimensões dessa função de perda. Contudo, pelo fato do cálculo do erro ser uma função de duas variáveis, que apresenta como entradas o valor real para a saída ($y_j$) e o valor previsto pelo modelo ($\hat{y}_j$), a sua representação real deve ser feita com o uso de três dimensões, como mostrado na Figura \ref{fig:mse-3d}. Dessa forma, é possível ver uma superfície para essa função de perda.

\begin{figure}[h!]
    \centering % Centraliza a figura na página

    % --- SUBFIGURA (a): SEU GRÁFICO 2D ORIGINAL (MODIFICADO) ---
    \begin{subfigure}[b]{0.48\textwidth}
        \centering
        \begin{tikzpicture}
            \begin{axis}[
                % Dimensões ajustadas para caber lado a lado
                width=\linewidth,  
                height=7cm,
                xlabel={Erro},
                ylabel={Perda},
                axis lines=middle,
                grid=major,
                grid style={dashed, gray!40},
                xmin=-3.5, xmax=3.5,
                ymin=-0.5, ymax=9.5,
                legend pos=north west,
                title style={font=\bfseries\small},
                label style={font=\small},
                tick label style={font=\scriptsize}
            ]
                % Gráfico da função x^2
                \addplot[
                    domain=-3:3, 
                    samples=100, 
                    color=blue, 
                    very thick
                ] {x^2};

            \end{axis}
        \end{tikzpicture}
        \caption{Representação gráfica em duas dimensões da função criada.} % Legenda da subfigura
        \label{fig:mse-2d}
    \end{subfigure}
    \hfill % Adiciona espaço horizontal flexível entre as subfiguras
    % --- SUBFIGURA (b): NOVO GRÁFICO 3D ---
    \begin{subfigure}[b]{0.48\textwidth}
        \centering
        \begin{tikzpicture}
            \begin{axis}[
                % Dimensões consistentes com o gráfico (a)
                width=\linewidth,
                height=7cm,
                xlabel={$y$ (Real)},
                ylabel={$\hat{y}$ (Previsto)},
                zlabel={Perda},
                grid=major,
                view={150}{45}, % Ângulo de visão (azimute, elevação)
                zmin=0, zmax=35, % Ajuste o zmax conforme o domínio
                title style={font=\bfseries\small},
                label style={font=\small},
                tick label style={font=\scriptsize}
            ]
                % Gráfico da superfície (y - y_hat)^2, que é (x - y)^2 no pgfplots
                \addplot3[
                    mesh,           % Tipo de gráfico: superfície
                    color=blue,
                    shader=interp,  % Suaviza as cores
                    domain=-5:5,    % Domínio de x (y_real)
                    domain y=-5:5,  % Domínio de y (y_previsto)
                    samples=15      % Resolução da malha
                ] { (x - y)^2 }; % A função de perda MSE
            \end{axis}
        \end{tikzpicture}
        \caption{Representação gráfica em três dimensões da superfície criada.} % Legenda da subfigura
        \label{fig:mse-3d}
    \end{subfigure}

    % --- Legenda e Fonte da Figura Principal ---
    \caption{Visualizações da função de perda erro quadrático médio (\textit{MSE}) em duas e em três dimensões.}
    \label{fig:mse} % Rótulo principal da figura
    \fonte{O autor (2025).}
\end{figure}

\medskip
\begin{center}
 * * *
\end{center}
\medskip

\textbf{Características do Erro Quadrático Médio}
\vspace{1em}

Tendo os gráficos do erro quadrático médio além de sua equação, cabe agora discutir algumas das principais característica dessa função de perda.

\begin{itemize}
    \item \textbf{Não-negatividade:} Como é visto nos gráficos da Figura \ref{fig:mse}, o \textit{MSE} é uma função que retorna apenas valores positivos, isso se dá devido a diferença das entradas estar sendo calculada e logo em seguida elavada ao quadrado, impendido que valores negativos ocorram na saída;
    \item \textbf{Sensibilidade para \textit{outliers}:} Como \textcite{LossesArticle} discutem, a função erro quadrático médio possui a tendência de punir em maior força os erros que são originarios de uma distância muito grande. Isso acontece por conta do jeito que é calculada essa função, lembre que ela calcula a distância entre os pontos e depois eleva ela ao quadrado, se essa distância for muito grande, o erro será maior ainda. Como consequência, se existe um conjunto grande de pontos que ficam fora da regressão, os \textit{outliers}, a função de perda irá constamente retornar valores altos. Isso pode atrapalhar o aprendizado, porque será mais difícil otimizar o modelo, mas também pode ser uma vantagem em cenários em que os erros devem ser fortemente penalizados;
    \item \textbf{Convexa (nas predições):} Voltando para a Figura \ref{fig:mse} é possível notar que o \textit{MSE} é uma função convexa, o seu gráfico tem a típica forma de um funíl, apresentando um único ponto de mínimo global, isso ajuda na otimização dessa função caso se esteja usando um otimizador baseado no gradiente descendente. Contudo, \textcite{LossesArticle} destacam que em redes neurais profundas essa função pode se tornar não-convexa, devido as transformações não-lineares que são feitas ao construir o modelo;
    \item \textbf{Baixa intepretabilidade:} Note que a função \textit{MSE} eleva o erro ao quadrado, de forma que ao mostrar a perda não é visto diretamente a distância entre os pontos reais e os pontos preditos pelo modelo. Isso um gera um gargalo pois não é possível de forma instânea saber exatamente quão bem ou mal o modelo está performando. Funções como a \textit{MAE}, em que o erro é calculado como o módulo da distância entre os dois pontos são mais diretas em mostrar a performance do modelo;
    \item \textbf{Continuidade:} Ainda nos gráficos da Figura \ref{fig:mse} é possível notar que o \textit{MSE} é uma função contínua em todo o seu domínimo, isso é uma vantagem muito boa, pois indica que é possível derivar essa função sem encontrar grandes problemas.
\end{itemize}

\medskip
\begin{center}
 * * *
\end{center}
\medskip

Além disso, cabe também destacar a derivada da função \textit{MSE}, a qual está presente na Equação \ref{eq:mse-derivada}. A derivada de uma função de perda é de extrema utilizada para uma rede neural, pois é a partir dela que é calculado o primeiro vetor gradiente, e passa a ser retropropagado por toda a rede no sentido inverso, indo primeiro das últimas camadas para as camadas de entrada.

\begin{equacaodestaque}{Derivada Parcial do Erro Quadrático Médio (\textit{MSE}) em Relação a $\hat{y}_j$}
    \frac{\partial \Loss_{\text{MSE}}}{\partial \hat{y}_j} = \frac{2}{N}(\hat{y}_j - y_j)
    \label{eq:mse-derivada}
\end{equacaodestaque}

Note que diferente das funções de ativação em que era calculada a sua derivada ordinária, aqui é calculado as derivadas parciais das funções de perda para cada um dos seus componentes, a saída do modelo $\hat{y}_j$ e a saída esperada $y_j$. Com base nessas duas derivadas, é possível construir o vetor gradiente, como mostrado na Equação \ref{eq:vetor-gradiente-mse} \footnote{Perceba então que para ter o vetor gradiente completo, deve-se calcular a derivada parcial do \textit{MSE} também em relação os valor real do rótulo ($y_j$). Neste livro será dado um foco maior em apresentar apenas uma das derivas, pois elas muitas vezes apresentarão grandes semelhanças.}.

\begin{equation}
    \nabla (y_j, \hat{y}_j) = \left( \frac{\partial \Loss_{\text{MSE}}}{\partial y_j}, \frac{\partial \Loss_{\text{MSE}}}{\partial \hat{y}_j} \right)
    \label{eq:vetor-gradiente-mse}
\end{equation}

Considerando a derivada parcial da Equação \ref{mse-derivada}, é possível contruir gráficos como os da Figura \ref{eq:mse-derivada} para a derivada parcial do \textit{MSE} em relação a $(\hat{y}_j)$. Note também que semelhante as representações do erro quadrático médio, a sua derivada também é apresentada em vistas em duas e três dimensões, isso será algo recorrente nas representações das funções desse capítulo.

\begin{figure}[h!]
    \centering % Centraliza a figura na página

    % --- SUBFIGURA (a): Gráfico 2D da Derivada do MSE ---
    \begin{subfigure}[b]{0.48\textwidth}
        \centering
        \begin{tikzpicture}
            \begin{axis}[
                % Dimensões ajustadas para caber lado a lado
                width=\linewidth,  
                height=7cm,
                xlabel={Erro ($e = \hat{y} - y$)},
                ylabel={Gradiente ($\frac{\partial L}{\partial \hat{y}}$)},
                axis lines=middle,
                grid=major,
                grid style={dashed, gray!40},
                xmin=-3.5, xmax=3.5,        % Limites do seu gráfico
                ymin=-6.5, ymax=6.5,         % Limites do seu gráfico
                legend pos=north west,
                title style={font=\bfseries\small},
                label style={font=\small},
                tick label style={font=\scriptsize}
            ]
                % Gráfico da função 2*x
                \addplot[
                    domain=-3:3, 
                    samples=100, 
                    color=red, 
                    very thick
                ] {2*x};
                
                % Mantendo sua legenda (com o fator 1/N)
                \addlegendentry{$\frac{\partial L}{\partial \hat{y}} = \frac{2}{N} \cdot e$}
            \end{axis}
        \end{tikzpicture}
        \caption{Visão 2D (Gradiente vs. Erro).} % Legenda da subfigura
        \label{fig:mse-derivada-2d}
    \end{subfigure}
    \hfill % Adiciona espaço horizontal flexível entre as subfiguras
    % --- SUBFIGURA (b): Gráfico 3D da Derivada do MSE ---
    \begin{subfigure}[b]{0.48\textwidth}
        \centering
        \begin{tikzpicture}
            \begin{axis}[
                % Dimensões consistentes com o gráfico (a)
                width=\linewidth,
                height=7cm,
                xlabel={$y$ (Real)},
                ylabel={$\hat{y}$ (Previsto)},
                zlabel={Gradiente ($\frac{\partial L}{\partial \hat{y}}$)},
                grid=major,
                view={150}{45}, % Mesmo ângulo de visão do seu template
                zmin=-20.5, zmax=20.5, % Ajustado para 2 * (erro máx 10)
                title style={font=\bfseries\small},
                label style={font=\small},
                tick label style={font=\scriptsize}
            ]
                % Gráfico da superfície do gradiente: 2 * (y_hat - y)
                \addplot3[
                    mesh,           
                    color=red,      % Cor consistente com o gráfico 2D
                    shader=interp,  
                    domain=-5:5,    % Mesmo domínio do seu template
                    domain y=-5:5,  % Mesmo domínio do seu template
                    samples=15      % Mesma resolução da malha
                ] { 2 * (y - x) }; % A função da derivada: 2 * (y_previsto - y_real)
            \end{axis}
        \end{tikzpicture}
        \caption{Superfície 3D completa.} % Legenda da subfigura
        \label{fig:mse-derivada-3d}
    \end{subfigure}

    % --- Legenda e Fonte da Figura Principal ---
    \caption{Visualizações da derivada (gradiente) da função de perda MSE.}
    \label{fig:mse-derivada} % Rótulo principal do seu gráfico
    \fonte{O autor (2025).}
\end{figure}

É possível perceber pelo gráfico da derivada da função de perda erro quadrático médio que o gradiente da perda é proporcional ao erro. Seguindo essa lógica, se o erro está alto, o gradiente também estará alto, como consequência as atualizações de pesos e vieses, as quais seguem o método do gradiente (visto nas Equações \ref{eq:gradiente-do-erro-em-relacao-a-um-peso-de-um-neuronio-regressao}) serão mais bruscas, dando maiores saltos no gráfico da função de perda. Enquanto em situações que o erro está pequeno em magnitudade, o gradiente também será pequeno e o modelo fará pequenas atualizações nos seus parâmetros, garantindo um ajuste fino.

\begin{equation}
    \frac{\partial E}{\partial w_{ji}} = \frac{\partial E}{\partial x_j} \cdot y_i \quad \text{ou} \quad \frac{\partial E}{\partial w_{ji}} = \frac{\partial E}{\partial y_j} \cdot \sigma'(x_j) \cdot y_i
    \label{eq:gradiente-do-erro-em-relacao-a-um-peso-de-um-neuronio-perda-regressao}
\end{equation}

\medskip
\begin{center}
 * * *
\end{center}
\medskip

\textbf{Algumas Aplicações do Erro Quadrático Médio em Problemas de Regressão} \index{Aplicações práticas! Erro quadrático médio (MSE)}
\vspace{1em}

Além de estar presente nas deduções do algoritmo da retropropagação como uma função de perda, o erro quadrático médio acaba sendo uma função bem versátil para ser aplicada em problemas de regressão. Não somente isso, mas seu uso não se restringe a somente uma função de perda que será utilizada como "mapa" para a otimização do modelo, o \textit{MSE} também é uma ótima métrica para avaliar o desempenho de um modelo de regressão. E junto do \textit{MSE} está o \textit{RMSE}, cujo o cálculo é dado pela raíz quadrada do erro quadrático médio. Juntas, essas duas funções se tornam ótimas escolhas para medir como um modelo está performando.

Dito, isso, vale apenas destacar alguns desses cenários em que o erro quadrático médio é utilizado tanto como função de perda, quanto como métrica avaliativa. Além disso, é possível notar que muitas vezes ele irá aparecer indiretamente em formato de raíz do erro quadrático médio.

Assim, algumas aplicações do \textit{MSE} são:

\begin{itemize}
    \item \textbf{Estimação de custos médicos (Saúde):} Em \textit{Medical Costs Estimation Using Linear Regression Method}, \textcite{MedicalCostsEstimationUsingLR} utilizam técnicas de regressão linear como a regressão linear múltipla para fazer previsões de custos médicos. Para avaliar os modelos de regressão criados, os autores fazem uso do erro quadrático médio mas também aplicam outras métricas, como o erro absoluto médio (tópico principal da Seção xx), e também a métrica $R^2$ \parencite{MedicalCostsEstimationUsingLR};
    \item \textbf{Estimação de Preços de Imóveis (Mercado Imobiliário):} No artigo \textit{An Optimal House Price Prediction Algorithm: XGBoost}, \textcite{OptimalHousePricePrediction} aplicam diferentes modelos de regressão (como regressão linear, florestas aleatórias e XGBoost) com intuito de criar um modelo ideal para a prever valores de casas. Com esses diferentes modelos criados, os autores precisaram de diversas métricas para encontrar o modelo ideal, para isso, uma das técnicas foi o uso do \textit{MSE}, além disso, eles utilizam também o \textit{RMSE}, que é dado pelo cálculo da raíz quadrada do próprio erro quadrático médio \parencite{OptimalHousePricePrediction};
    \item \textbf{Previsão da Produção Agrícola (Agronomia):} No trabalho \textit{Coupling Machine Learning and Crop Modeling Improves Crop Yield Prediction in the US Corn Belt}, \textcite{CouplingMachineLearningAndCropModeling} estavam estudando formas de combinar técnicas com modelamento de culturas para prever a produção das plantações na regiaão do cinturação do milho nos Estados Unidos. No artigo, os pesquisadores não utilizam diretamente o \textit{MSE}, ao invés disso, utilizam a raíz do erro quadrático médio como uma das diferentes métricas para avaliação dos modelos de previsão desenvolvidos ao longo do projeto \parencite{CouplingMachineLearningAndCropModeling};
    \item \textbf{Previsão de Demanda de Energia (Gestão Energética):} Já no texto \textit{Optimizing Federated Learning for Scalable Power-demand Forecasting in Microgrids} ocorre uma situação diferente ao aplicar o erro quadrático médio, \textcite{OptimizingFL} fazem uma adptação nessa função transformando-a no \textit{Exponentially Weighted Mean Squared Error} (\textit{EW-RSM}). Essa adptação é justificada pelos autores como uma forma de enfatizar a acurácia em previsões de longo prazo atribuindo pesos exponencialmente crescentes aos erros em etapas de tempo posteriores \parencite{OptimizingFL}.
\end{itemize}

\medskip
\begin{center}
 * * *
\end{center}
\medskip

Conhecendo a função de perda \textit{MSE}, é possível agora discutir uma outra abordagem para resolver problemas de regressão. Para isso, a próxima seção busca apresentar o erro absoluto médio, que é uma alternativa para o erro quadrático médio que tem como principal diferença o jeito que lida com \textit{outliers}.

\subsection{Erro Absoluto Médio (MAE)} \index{Funções de Perda!Erro Absoluto Médio (MAE)}
\label{sec:mae-loss}

O erro absoluto médio, também chamado de perda L1, é uma função que tem o mesmo propósito do erro quadrático médio, ser utilizada para tarefas de regressão. Neste caso o \textit{MAE} não possui uma origem definida que como o \textit{MSE}, esse conceito de minimizar a diferença de um resultado pelo seu valor real já havia sendo utilizado a bastante tempo. Contudo, trabalhos como \textit{Greedy function approximation: A gradient boosting machine} de \textcite{GreedyFunctionApproximation} fazem uso do erro absoluto médio para resolver problemas de aprendizado de máquina. No texto, o autor desenvolve um algoritmo de \textit{boosting} específico para o \textit{MAE}, neste caso, essa função é apresentada com outro nome, \textit{Least-Absolute-Deviation} (\textit{LAD}), sendo responsável por dar nome ao algoritmo criado, o \texttt{LAD\_TreeBoost} \parencite{GreedyFunctionApproximation}. Neste caso, \textcite{GreedyFunctionApproximation} explica que esse algoritmo faz uso de uma árvore de regressão com a perda, de forma que para prever a pseudo-resposta que é o sinal dos resíduos atuais é utilizado o método dos mínimos quadrados. Assim, o modelo é atualizado adicionando em cada nó terminal da nova árvore criada a mediana dos resíduos daquela região específica.

Um trabalho mais recente que explora o uso dessa função é o \textit{Image-to-Image Translation with Conditional Adversarial Networks} \parencite{ImageToImage}. Nele, \textcite{ImageToImage} argumentam que preferiram trabalhar com a função de perda \textit{L1 distance} (um dos diferentes nomes utilizados para se referir ao \textit{MAE}) devido a essa função gerar imagens menos borradas. É possível ver essa comparação nas imagens da Figura \ref{fig:comparativo-perdas-image-to-image}. Perceba que a perda que apresenta os resultados mais consistentes com a realidade é justamente a \textit{L1 + cGAN}, a qual possuí o \textit{MAE} em sua composição.

\begin{figure}[h]
    \centering
    \includegraphics[width=0.65\linewidth]{../imagens/perda-regressao/image-to-image-perdas-comparativo.png}
    
    \caption[Perdas diferentes induzem qualidades de resultados diferentes. Cada coluna mostra resultados treinados sob uma perda diferenteas de aprendizado no dataset MNIST]{%
        \newline
        \small Fonte: \parencite{ImageToImage}.
    }
    \label{fig:comparativo-perdas-image-to-image}
\end{figure}

Além disso, vale a pena destacar como o erro absoluto médio se enquadra no exemplo ilustrativo do jogo dos dardos. Para isso, diferente do erro quatrático médio, em que quanto mais longe do centro, o número de pontos perdidos aumentava de forma quadrática, no \textit{MAE} a penalização com relação aos jogadores ruins é mais suave, sendo dada de forma linear. 

Além disso, é interessante fazer uma outra analogia considerando o cenário em que o \textit{MAE} se encontra. Para isso, é preciso considerar que será feito um novo jogo de dardos com dez participantes, em que cada um tem direito de jogar apenas um dardo, e que no final os dez dardos serão somados e com base neles será dado o resultado da equipe. Se tivéssemos uma equipe muito ruim utilizando como métrica o \textit{MSE} para avaliar os pontos, o resultado seria péssimo, porque esses dardos estariam muito longes do centro, e como o erro é elevado ao quadrado, isso geraria uma pontuação muito baixa. 

Contudo, se a métrica escolhida para avaliar o jogo fosse o \textit{MAE} o resultado seria bem melhor, visto que o erro cresce de forma linear. Saindo da alogia e voltando para o cenário de aprendizado de máquina, é possível associar os jogadores ruins com os \textit{outliers}, eles estão presentes nos dados e vão atrapalhar o aprendizado do seu modelo. Contudo, escolher uma função que não seja tão agressiva na medição desses \textit{outliers} pode ser uma excelente alternativa para trabalhar em cenários em que essa configuração será muito comum.

Visto esses diferentes cenários em que o erro absoluto médio foi utilizado para resolver problemas de regressão, além de como ele pode ser associado com o jogo dos dardos, é possível ver agora a sua definição matemática. Para isso, o \textit{MAE} está definido na Equação \ref{eq:mae}. Perceba que ele é responsável por calcular a diferença entre os dois pontos, o valor real $\hat{y}_j$ e o valor predito $y_j$, para todos os $N$ casos analisados e a partir disso calcular a média dos resultados.

\begin{equacaodestaque}{Erro Absoluto Médio (\textit{MAE})}
    \Loss_{\text{MAE}} = \frac{1}{N} \sum_{j=1}^{N} |y_j - \hat{y}_j|
    \label{eq:mae}
\end{equacaodestaque}


Também vale a pena analisar o erro absoluto médio de forma gráfica. Para isso, suas representações em duas e três dimensões estão presentes na Figura \ref{fig:mae} de forma semelhante ao que foi feito ao apresentar o \textit{MSE}.

\begin{figure}[h!]
    \centering % Centraliza a figura na página

    % --- SUBFIGURA (a): Gráfico 2D do MAE ---
    \begin{subfigure}[b]{0.48\textwidth}
        \centering
        \begin{tikzpicture}
            \begin{axis}[
                % Dimensões ajustadas para caber lado a lado
                width=\linewidth,  
                height=7cm,
                xlabel={Erro ($e = y - \hat{y}$)},
                ylabel={Perda Calculada},
                axis lines=middle,
                grid=major,
                grid style={dashed, gray!40},
                xmin=-4.5, xmax=4.5,        % Limites do seu gráfico MAE
                ymin=-0.5, ymax=4.5,         % Limites do seu gráfico MAE
                legend pos=north west,
                title style={font=\bfseries\small},
                label style={font=\small},
                tick label style={font=\scriptsize}
            ]
                % Gráfico da função abs(x)
                \addplot[
                    domain=-4:4, 
                    samples=100, 
                    color=red, 
                    very thick
                ] {abs(x)};
                
                \addlegendentry{$L = |e|$}
            \end{axis}
        \end{tikzpicture}
        \caption{Representação gráfica em duas dimensões.} % Legenda da subfigura
        \label{fig:mae-2d}
    \end{subfigure}
    \hfill % Adiciona espaço horizontal flexível entre as subfiguras
    % --- SUBFIGURA (b): Gráfico 3D do MAE ---
    \begin{subfigure}[b]{0.48\textwidth}
        \centering
        \begin{tikzpicture}
            \begin{axis}[
                % Dimensões consistentes com o gráfico (a)
                width=\linewidth,
                height=7cm,
                title={(b) Superfície de Perda (Visão 3D)},
                xlabel={$y$ (Real)},
                ylabel={$\hat{y}$ (Previsto)},
                zlabel={Perda},
                grid=major,
                view={150}{45}, % Mesmo ângulo de visão do seu template
                zmin=0, zmax=10.5, % Limite Z ajustado para o MAE (abs(-5 - 5) = 10)
                title style={font=\bfseries\small},
                label style={font=\small},
                tick label style={font=\scriptsize}
            ]
                % Gráfico da superfície abs(y - y_hat)
                \addplot3[
                    mesh,           
                    color=red,      % Cor consistente com o gráfico 2D
                    shader=interp,  
                    domain=-5:5,    % Mesmo domínio do seu template
                    domain y=-5:5,  % Mesmo domínio do seu template
                    samples=15      % Mesma resolução da malha
                ] { abs(x - y) }; % A função de perda MAE
            \end{axis}
        \end{tikzpicture}
        \caption{Representação gráfica em três dimensões.} % Legenda da subfigura
        \label{fig:mae-3d}
    \end{subfigure}

    % --- Legenda e Fonte da Figura Principal ---
    \caption{Visualizações da função de perda erro absoluto médio (\textit{MAE}) em duas e em três dimensões.}
    \label{fig:mae} % Rótulo principal do seu gráfico MAE
    \fonte{O autor (2025).}
\end{figure}

\medskip
\begin{center}
 * * *
\end{center}
\medskip

\textbf{Características do Erro Absoluto Médio}
\vspace{1em}

A partir do seu gráfico e de sua equação é possível retirar diferentes informações dessa função de perda, as quais estão discutidas a seguir:

\begin{itemize}
    \item \textbf{Não-negatividade:} Como é possível ver em seu gráfico da Figura \ref{fig:mae}, o erro absoluto médio compartilha dessa mesma propriedade com o \textit{MSE}, isso significa que sua saída será sempre positiva ou zero, independente do valor de entrada. Isso se dá, devido a propriedade do módulo, que não admite números negativos para sua saída;
    \item \textbf{Robustez para \textit{outliers}}: Como \textcite{LossesArticle} explicam, o \textit{MAE} não penaliza os \textit{outliers} de forma tão severa como o erro quadrático médio. Isso acontece pois diferente do \textit{MSE}, em que o erro cresce de forma quadrática, no \textit{MAE}, ele cresce de forma linear, significando que diferente do \textit{MSE} que era sensível aos \textit{outliers}, o \textit{MAE} também reage à eles, mas de forma menos severa;
    \item \textbf{Convexa (nas predições):} Voltando para o seu gráfico, é possível ver que o erro absoluto médio segue a mesma forma de funíl do erro quadrático médio, podendo ser considerado uma função convexa e com um só ponto de mínimo global. Contudo, assim como no \textit{MSE}, \textcite{LossesArticle} advertem que para modelos de apredenziado profundo, o \textit{MAE} pode deixar de ser uma função convexa devido às muitas camadas e funções não-lineares;
    \item \textbf{Não-derivável em zero:} Esse ponto acontece justamente devido a forma que a função apresenta, ela faz uma especíe de "bico" em zero, além disso, ao calcular os seus limites laterais para verificar a continuidade da função é possível notar que eles apresentam valores diferentes;
    \item \textbf{Boa intepretabilidade:} Diferente do erro quadrático médio em que o erro total é elevado ao quadrado, no \textit{MAE} isso não ocorre, o erro é simplemente a média das diferenças dos pontos. Isso significa que é mais fácil interpretar os resultados que essa função de perda retorna, pois não é preciso fazer nenhum cálculo adicional para ter uma ideia precisa se as previsões feitas pelo modelo estão longe ou não dos valores reais;
\end{itemize}

\medskip
\begin{center}
 * * *
\end{center}
\medskip

Como foi destacado anteriormente, o \textit{MAE} não é derivável em zero. Inicialmente pode ter-se a ideia de que devido a isso, essa não seja uma função boa para ser utilizada em conjunto com otimizadores baseados em gradiente, porque pontos de descontinuidade como este podem acabar atrapalhando a forma como o molode é otimizado. Contudo, \textcite{LossesArticle} argumentam, é possível resolver esse problema utilizado técnicas de subgradiente, sendo possível então escrever a derivada do \textit{MAE} com a Equação \ref{eq:mae-derivada}.

\begin{equacaodestaque}{Derivada Parcial do Erro Absoluto Médio (\textit{MAE}) em Relação a $\hat{y}_j$}
    \frac{\partial \Loss_{\text{MAE}}}{\partial \hat{y}_j} = 
    \begin{cases} 
      -1 & \text{se } \hat{y}_j > y_j \\
      +1 & \text{se } \hat{y}_j < y_j \\
      [-1, +1] & \text{se } \hat{y}_j = y_j
    \end{cases}
    \label{eq:mae-derivada}
\end{equacaodestaque}

Contudo, mesmo possuindo esse detalhe de descontinuidade em zero, isso não atrapalha a plotagem do gráfico do \textit{MAE}, o qual está representado na Figura \ref{fig:mae-derivada}. De forma semelhante ao que foi feito até agora, na figura da esquerda, a Figura \ref{fig:mae-derivada-2d}, está a representação em duas dimensões dessa função, enquanto na figura da direta, a Figura \ref{fig:mae-derivada-3d}.

\begin{figure}[h!]
    \centering % Centraliza a figura na página

    % --- SUBFIGURA (a): Gráfico 2D da Derivada do MAE ---
    \begin{subfigure}[b]{0.48\textwidth}
        \centering
        \begin{tikzpicture}
            \begin{axis}[
                % Dimensões ajustadas para caber lado a lado
                width=\linewidth,  
                height=7cm,
                xlabel={Erro ($e = \hat{y} - y$)},
                ylabel={Gradiente ($\frac{\partial L}{\partial \hat{y}}$)},
                axis lines=middle,
                grid=major,
                grid style={dashed, gray!40},
                xmin=-3.5, xmax=3.5,        % Limites do seu gráfico
                ymin=-1.5, ymax=1.5,         % Limites do seu gráfico
                ytick={-1, 0, 1},           % Pontos no eixo y
                legend pos=north west,
                title style={font=\bfseries\small},
                label style={font=\small},
                tick label style={font=\scriptsize}
            ]
                % Parte negativa da derivada (-1)
                \addplot[
                    domain=-3:0, 
                    samples=100, 
                    color=red, 
                    very thick
                ] {-1};

                % Parte positiva da derivada (+1)
                \addplot[
                    domain=0:3, 
                    samples=100, 
                    color=red, 
                    very thick
                ] {1};
                
                % Círculos abertos para a descontinuidade em x=0
                \addplot[only marks, mark=o, color=red, mark size=2pt] coordinates {(0,-1) (0,1)};
            \end{axis}
        \end{tikzpicture}
        \caption{Visão 2D (Gradiente vs. Erro).} % Legenda da subfigura
        \label{fig:mae-derivada-2d}
    \end{subfigure}
    \hfill % Adiciona espaço horizontal flexível entre as subfiguras
    % --- SUBFIGURA (b): Gráfico 3D da Derivada do MAE ---
    \begin{subfigure}[b]{0.48\textwidth}
        \centering
        \begin{tikzpicture}
            \begin{axis}[
                % Dimensões consistentes com o gráfico (a)
                width=\linewidth,
                height=7cm,
                xlabel={$y$ (Real)},
                ylabel={$\hat{y}$ (Previsto)},
                zlabel={Gradiente ($\frac{\partial L}{\partial \hat{y}}$)},
                grid=major,
                view={150}{45}, % Mesmo ângulo de visão do seu template
                zmin=-1.5, zmax=1.5, % Espelha o eixo Y do gráfico 2D
                ztick={-1, 0, 1}, % Consistente com o 2D
                title style={font=\bfseries\small},
                label style={font=\small},
                tick label style={font=\scriptsize}
            ]
                % Gráfico da superfície do gradiente: sign(y_hat - y)
                \addplot3[
                    mesh,           
                    color=red,      % Cor consistente com o gráfico 2D
                    shader=interp,  
                    domain=-5:5,    % Mesmo domínio do seu template
                    domain y=-5:5,  % Mesmo domínio do seu template
                    samples=15      % Mesma resolução da malha
                ] { sign(y - x) }; % A função da derivada: sign(y_previsto - y_real)
            \end{axis}
        \end{tikzpicture}
        \caption{Superfície 3D completa.} % Legenda da subfigura
        \label{fig:mae-derivada-3d}
    \end{subfigure}

    % --- Legenda e Fonte da Figura Principal ---
    \caption{Visualizações da derivada (gradiente) da função de perda MAE.}
    \label{fig:mae-derivada} % Rótulo principal do seu gráfico
    \fonte{O autor (2025).}
\end{figure}

Perceba que a derivada do erro absoluto médio é representado em forma de uma função por partes, neste caso, ela é divida em duas diferentes retas, parecida com a função de ativação degrau unitário. Note também que existe um "bico" na união dessas duas retas, isso acontece devido a descontinuidade dessa função em zero, o que impede de ser derivada nesse ponto. Assim, a derivada do \textit{MAE} pode ser interpretada como a composição de duas retas constantes, a primeira constante em -1, e a segunda constante em 1.

\medskip
\begin{center}
 * * *
\end{center}
\medskip

\textbf{Algumas Aplicações do Erro Absoluto Médio em Problemas de Regressão} \index{Aplicações práticas! Erro absoluto médio (MAE)}
\vspace{1em}

Além dos casos discutidos no início da seção: o \texttt{LAD\_TreeBoost} e o artigo \textit{Image-to-Image Translation with Conditional Adversarial Networks}, o \textit{MAE} também está presente em uma série de trabalhos, atuando tanto como função de perda, quanto servindo como uma métrica de avaliação, indicando se o modelo desenvolvido está performando bem ou não. Dito isso, essa seção busca explorar algumas dessas aplicações do erro absoluto médio, semelhante ao que foi feito ao analisar o erro quadrático médio. 

Dito isso, vale destacar os trabalhos:

\begin{itemize}
    \item \textbf{Avaliação de idade óssea e estimação do escore de cálcio na artéria coronária (Saúde):} Em \textit{Regression Metric Loss: Learning a Semantic Representation Space for Medical Images}, \textcite{chao2022regressionmetriclosslearning} desenvolvem algortimos de regressão para estimar escore de cálcio da artéria coronária e também um segundo algoritmo para avaliação da idade óssea. Além disso, os autores apresentam uma nova função de perda, a \textit{RM-Loss} que demonstra ser mais apta para resolver os problemas propostos de regressão \parencite{chao2022regressionmetriclosslearning}. Como forma de avaliar essa nova função criada e também as diferentes outras funções comparadas no artigo, \textcite{chao2022regressionmetriclosslearning} utilizam o erro absoluto médio como uma das métricas;
    \item \textbf{Restauração de imagens (Engenharia):} No artigo \textit{Noise2Noise: Learning Image Restoration without Clean Data}, \textcite{Noise2Noise} estavam estudando formas de restaurar imagens corrompidas sem a utilização de dados limpos. Em um dos experimentos os autores estavam buscando uma forma ideal de remover textos de imagens, de forma que a perda L1, por ser uma função de perda robusta, conseguiu atingir bons resultados nessa tarefa \parencite{Noise2Noise};
    \item \textbf{Previsão da produção de energia eólica (Setor energético):} Já em \textit{Minimum Open Data Subset for Wind Power Prediction}, \textcite{MinimumOpenDataSubsetForWindPowerPrediction} utilizam um modelo de florestas aleatórias com o objetivo de prever a produção de energia eólica. Para avaliar o modelo de regressão desenvolvido, os autores utilizam como métricas o \textit{MAE} além do \textit{RMSE}. Além disso, vale comentar que em testes realizados pelos pesquisadores foi possível criar um modelo com erro absoluto médio de 0,071, indicando um excelente resultado para para o algoritmo criado \parencite{MinimumOpenDataSubsetForWindPowerPrediction}.
    \item \textbf{Previsão de poluição do ar (Setor ambiental):} Por fim, vale a pena destacar o trabaho de \textcite{nedungadi2025aircastimprovingairpollution}, \textit{AirCast: Improving Air Pollution Forecasting Through Multi-Variable Data Alignment}, em que os autores buscam formas de melhorar a previsão da poluição do ar. Como forma de ajudar a resolver esse problema, os autores utilizam uma função inspirada pelo erro absoluto médio, o \textit{Frequency-weighted Mean Absolute Error} (\textit{fMAE}), que tem como principal vantagem lidar com variáveis que apresentam uma distribuição de cauda pesada, como as variáveis PM1, PM2.5 e PM10, que indicam a qualidade do ar \parencite{nedungadi2025aircastimprovingairpollution}.
\end{itemize}

\medskip
\begin{center}
 * * *
\end{center}
\medskip

Visto o erro quadrático médio, que penaliza fortamente ou \textit{outliers}, e o erro absoluto médio, que não penaliza de forma agressiva os \textit{outliers}, surge uma pergunta: Existe alguma forma de ter uma função que penalize os erros gravemente até um certo ponto e depois desse, ela não se preocupe tanto com os \textit{outliers}? Essa é a proposta da perda de Huber, a qual busca unir os principais benefícios dessas duas funções de regressão. Ela é o tópico principal da próxima seção.

\subsection{Perda de Huber (Huber Loss)} \index{Funções de Perda!Perda de Huber (Huber Loss)}
\label{sec:huber-loss}

A Huber \textit{Loss} recebe o seu nome devido ao seu criador, Peter J. Huber, que apresentou para a comunidade científica no trabalho \textit{Robust Estimation of a Location Parameter} \parencite{HuberLoss}. No artigo, \textcite{HuberLoss}, estava estudando maneiras de fazer uma estimação robusta de um parâmetro de localização (como a média o mediana de um conjunto de dados) quando a distribuição dos dados é aproximadamente conhecida. Além disso, no trabalho, o autor define um estimador robusto $p$ que segue a Equação \ref{eq:huber-loss-do-huber}, a qual prova ser uma solução ideal para o problema estudado \parencite{HuberLoss}.

\begin{equation}
    \rho(t) = 
    \begin{cases}
        \frac{1}{2} t^2 & \text{se} |t| < k \\
        k |t| - \frac{1}{2} k^2 & \text{se} t \ge k
    \end{cases}
    \label{eq:huber-loss-do-huber}
\end{equation}

O que Huber estava querendo basicamente com a Equação \ref{eq:huber-loss-do-huber} era uma função que se comportasse de forma quadrática para os casos em que $|t| < k$ e que se comportasse de forma linear para os casos em que $t \ge k$. Essa função criada pelo pesquisador é a função que está sendo estudada, a perda de Huber, a qual pode ser representada, agora com notações voltadas para o cenário de aprendizado de máquina, com a Equação \ref{eq:huber-loss}.

\begin{equacaodestaque}{Perda de Huber (Huber \textit{Loss})}
    \Loss_{\text{Huber}}(y, \hat{y}) = 
    \begin{cases} 
      \frac{1}{2}(y - \hat{y})^2 & \text{para } |y - \hat{y}| \le \delta \\
      \delta (|y - \hat{y}| - \frac{1}{2}\delta) & \text{caso contrário}
    \end{cases}
    \label{eq:huber-loss}
\end{equacaodestaque}

É possível também representar a perda de Huber utilizando gráficos, para isso, ela pode ser vista na Figura \ref{fig:huber-loss}. Para isso, na Figura \ref{fig:huber-2d} está representado a visualização em duas dimensões dessa função, enquanto na Figura \ref{fig:huber-3d} está a representação no espaço da superfície, assim como o que vem sendo feito até agora. Perceba que é como se ela fosse duas funções em uma, até um certo ponto do gráfico ela age parecido a uma função quadrática, contudo, após passar do limite de $\delta$ ela passa a ser uma função linear.

\begin{figure}[h!]
    \centering % Centraliza a figura na página

    % --- SUBFIGURA (a): Gráfico 2D da Huber Loss ---
    \begin{subfigure}[b]{0.48\textwidth}
        \centering
        \begin{tikzpicture}
            % Define o valor de delta para este gráfico
            \def\delta{1.0} 
            
            \begin{axis}[
                % Dimensões ajustadas para caber lado a lado
                width=\linewidth,  
                height=7cm,
                xlabel={Erro ($e = y - \hat{y}$)},
                ylabel={Perda Calculada},
                axis lines=middle,
                grid=major,
                grid style={dashed, gray!40},
                xmin=-4.5, xmax=4.5,        % Limites do seu gráfico
                ymin=-0.5, ymax=4.5,         % Limites do seu gráfico
                legend pos=north west,
                title style={font=\bfseries\small},
                label style={font=\small},
                tick label style={font=\scriptsize}
            ]
                % Gráfico da função Huber Loss
                \addplot[
                    domain=-4:4, 
                    samples=201, 
                    color=orange, 
                    very thick
                ] { abs(x) <= \delta ? 0.5*x^2 : \delta*(abs(x) - 0.5*\delta) };
                
                \addlegendentry{$L_{\delta=1}(e)$}

                % Linhas de transição em delta
                \draw[dashed, gray] (axis cs:-\delta, 0) -- (axis cs:-\delta, {\delta*(\delta-0.5*\delta)});
                \draw[dashed, gray] (axis cs:\delta, 0) -- (axis cs:\delta, {\delta*(\delta-0.5*\delta)});
            \end{axis}
        \end{tikzpicture}
        \caption{Representação gráfica em duas dimensões.} % Legenda da subfigura
        \label{fig:huber-2d}
    \end{subfigure}
    \hfill % Adiciona espaço horizontal flexível entre as subfiguras
    % --- SUBFIGURA (b): Gráfico 3D da Huber Loss ---
    \begin{subfigure}[b]{0.48\textwidth}
        \centering
        \begin{tikzpicture}
            % Define o valor de delta para este gráfico
            \def\delta{1.0}
            
            \begin{axis}[
                % Dimensões consistentes com o gráfico (a)
                width=\linewidth,
                height=7cm,
                xlabel={$y$ (Real)},
                ylabel={$\hat{y}$ (Previsto)},
                zlabel={Perda},
                grid=major,
                view={150}{45}, % Mesmo ângulo de visão do seu template
                zmin=0, zmax=10.5, % Limite Z ajustado para Huber com domínio -5:5
                title style={font=\bfseries\small},
                label style={font=\small},
                tick label style={font=\scriptsize}
            ]
                % Gráfico da superfície da Huber Loss
                \addplot3[
                    mesh,           
                    color=orange,   % Cor consistente com o gráfico 2D
                    shader=interp,  
                    domain=-5:5,    % Mesmo domínio do seu template
                    domain y=-5:5,  % Mesmo domínio do seu template
                    samples=15      % Mesma resolução da malha
                ] { abs(x-y) <= \delta ? 0.5*(x-y)^2 : \delta*(abs(x-y) - 0.5*\delta) }; % A função Huber 3D
            \end{axis}
        \end{tikzpicture}
        \caption{Representação gráfica em três dimensões.} % Legenda da subfigura
        \label{fig:huber-3d}
    \end{subfigure}

    % --- Legenda e Fonte da Figura Principal ---
    \caption{Visualizações da função de perda de Huber (\textit{Huber Loss}, $\delta=1$) em duas e em três dimensões.}
    \label{fig:huber-loss} % Rótulo principal do seu gráfico
    \fonte{O autor (2025).}
\end{figure}

Um ponto a ser destacado ao utilizar a perda de Huber é com relação a escolha de valores para o parâmetro $\delta$. Um valor muito pequeno para $\delta$ faz com que a função se comporte mais como o erro absoluto médio, é possível ver essa situação na Figura \ref{fig:huber-comparacoes-mae}, já ao escolher um valor muito grande para $\delta$ faz com que a perda de Huber se assemelhe mais a função erro quadrático médio, essa sitação está na Figura \ref{fig:huber-comparacoes-mse}. \textcite{LossesArticle} explicam que a escolha de valores para $\delta$ pode ser feita de forma empírica, através de validação cruzada (\textit{cross-validation}). Assim, testes são recomendados a fim de escolher o melhor valor para $\delta$ no cenário em que está sendo trabalhado.

\begin{figure}[h!]
    \centering
    % Figura da Esquerda (Parecida com MAE)
    \begin{subfigure}[b]{0.48\textwidth}
        \centering
        \begin{tikzpicture}
            \def\delta{0.5} % Delta pequeno
            \begin{axis}[
                xlabel={Erro ($\hat{y} - y$)},
                ylabel={Perda Calculada},
                axis lines=middle,
                grid=major,
                grid style={dashed, gray!40},
                xmin=-4.5, xmax=4.5,
                ymin=-0.5, ymax=4.5,
                legend pos=north west,
                width=\textwidth,
                label style={font=\small},
                tick label style={font=\scriptsize},
                title style={font=\bfseries, yshift=-5pt},
            ]
                % Função Huber Loss
                \addplot[
                    domain=-4:4, 
                    samples=201, 
                    color=blue, 
                    very thick,
                ] {(abs(x) <= \delta) ? (0.5*x^2) : (\delta*(abs(x) - 0.5*\delta))};
            \end{axis}
        \end{tikzpicture}
        \caption{Perda Huber com $\delta = 0.5$.}
        \label{fig:huber-comparacoes-mae}
    \end{subfigure}
    \hfill % Espaço entre as figuras
    % Figura da Direita (Parecida com MSE)
    \begin{subfigure}[b]{0.48\textwidth}
        \centering
        \begin{tikzpicture}
            \def\delta{4.0} % Delta grande
            \begin{axis}[
                xlabel={Erro ($\hat{y} - y$)},
                ylabel={Perda Calculada},
                axis lines=middle,
                grid=major,
                grid style={dashed, gray!40},
                xmin=-4.5, xmax=4.5,
                ymin=-0.5, ymax=8.5,
                legend pos=north west,
                width=\textwidth,
                label style={font=\small},
                tick label style={font=\scriptsize},
                title style={font=\bfseries, yshift=-5pt},
            ]
                % Função Huber Loss
                \addplot[
                    domain=-4:4, 
                    samples=201, 
                    color=red, 
                    very thick
                ] {(abs(x) <= \delta) ? (0.5*x^2) : (\delta*(abs(x) - 0.5*\delta))};

            \end{axis}
        \end{tikzpicture}
        \caption{Perda Huber com $\delta = 4.0$.}
        \label{fig:huber-comparacoes-mse}
    \end{subfigure}
    
    \caption{Comparação da Perda de Huber com diferentes valores de $\delta$.}
    \label{fig:huber-delta-comparacoes}
    \fonte{O autor (2025).}
\end{figure}

Assim, nota-se que ao utilizar a perda de Huber em um problema de regressão, é nítido que o grau de complexidade do problema pode aumentar, pois haverá mais um hiperparâmetro para ser otimizado de forma manual. Isso pode não ser ideal para cenários em que já existem muitos hiperparâmetros. Para isso, \textcite{LossesArticle} explicam que essa função é comumumente utilizada em problemas de gressão robusta, como em regressões lineares e em previsão de séries temporais (\textit{time series forecasting}), em que \textit{outliers} e ruído podem estar presentes.

\medskip
\begin{center}
 * * *
\end{center}
\medskip

\textbf{Características da Perda de Huber}
\vspace{1em}

Conhecido o gráfico e sua equação, é possível agora discutir algumas das propriedades da perda de Huber, as quais são apresentadas a seguir:

\begin{itemize}
    \item \textbf{Robustez para \textit{outliers}:} Assim como o \textit{MAE}, a perda de Huber não penaliza de forma quadrática os erros como comparado com o erro quadrático médio, dessa forma, os \textit{outliers} não conseguem afetar drasticamente o cálculo da perda dependendo do valor de $\delta$ escolhido \parencite{LossesArticle};
    \item \textbf{Diferenciabilidade em $\delta$:} Um ponto a ser destacado ao se utilizar a perda Huber é que ela apresenta pontos de descontinuidade para o cenário em que $y - \hat{y} = \delta$, contudo a função é contínua em todo o resto, dessa forma, isso não a impede de ser utilizada em conjunto com otimizadores baseados em gradiente \parencite{LossesArticle};
    \item \textbf{Convexa:} Bem como as funções de perda vistas até agora, a perda de Huber também é uma função convexa. Isso pode ser visto nos gráficos da Figura \ref{fig:huber-loss}, note que ela segue a forma clássica de uma funil, comum em funções convexas. Essa característica é uma vantagem para a perda de Huber, pois facilita com que os otimizadores encontrem os pontos de mínimo. Contudo, em redes neurais isso é mais complicado, devido as transformações não-lineares que ocorrem no modelo, fazendo com que a função de perda deixe de ser convexa.
\end{itemize}

\medskip
\begin{center}
 * * *
\end{center}
\medskip

Cabe também discutir a diferenciabilidade da perda de Huber, e como é dado o cálculo do seu gradiente. Dito isso, como explicam \textcite{LossesArticle}, o gradiente da perda de Huber deve ser calculado por partes, sendo que é possível utilizar a Equação \ref{eq:huber-loss-derivada} como guia.

\begin{equacaodestaque}{Derivada Parcial da Perda de Huber (Huber \textit{loss}) em Relação a $\hat{y}$}
    \frac{\partial \Loss_{\delta}}{\partial \hat{y}} = 
    \begin{cases} 
        \hat{y} - y & \text{se } | y - \hat{y} | \le \delta \\
        \delta \cdot \text{sgn}(\hat{y} - y) & \text{se } | y - \hat{y} | > \delta
    \end{cases}
    \label{eq:huber-loss-derivada}
\end{equacaodestaque}

Mesmo possuindo esse problema de descontinuidade da função em $y - \hat{y} = \delta$, como foi visto isso não atrapalha o cálculo do gradiente. De forma semelhante, também não atrapalha a construção do gráfico da sua derivada, o qual pode ser visto na Figura \ref{fig:huber-derivada}, na Figura \ref{fig:huber-derivada-2d} é possível ver a representação em duas dimensões, já na Figura \ref{fig:huber-derivada-3d} está a representação da superfície no espaço.

\begin{figure}[h!]
    \centering % Centraliza a figura na página

    % --- SUBFIGURA (a): Gráfico 2D da Derivada da Huber Loss (delta=1.0) ---
    \begin{subfigure}[b]{0.48\textwidth}
        \centering
        \begin{tikzpicture}
            \begin{axis}[
                % Dimensões ajustadas para caber lado a lado
                width=\linewidth, 
                height=7cm,
                xlabel={Erro ($e = \hat{y} - y$)},
                ylabel={Gradiente ($\frac{\partial L}{\partial \hat{y}}$)},
                axis lines=middle,
                grid=major,
                grid style={dashed, gray!40},
                xmin=-3.5, xmax=3.5,      % Limites do gráfico (além de delta)
                ymin=-1.5, ymax=1.5,      % Limites (um pouco além de -delta e +delta)
                ytick={-1, 0, 1},         % Ticks em -delta, 0, +delta
                xtick={-3, -2, -1, 0, 1, 2, 3}, % Ticks incluindo -delta e +delta
                legend pos=north west,
                title style={font=\bfseries\small},
                label style={font=\small},
                tick label style={font=\scriptsize}
            ]
                % Parte constante negativa (e < -delta)
                \addplot[
                    domain=-3:-1, 
                    samples=10, 
                    color=red, 
                    very thick
                ] {-1}; % Valor -delta

                % Parte linear (e entre -delta e +delta)
                \addplot[
                    domain=-1:1, 
                    samples=10, 
                    color=red, 
                    very thick
                ] {x}; % Valor e

                % Parte constante positiva (e > delta)
                \addplot[
                    domain=1:3, 
                    samples=10, 
                    color=red, 
                    very thick
                ] {1}; % Valor +delta
                
                % Pontos sólidos para os "kinks" (onde a derivada é contínua, mas não suave)
                \addplot[only marks, mark=*, color=red, mark size=2pt] coordinates {(-1,-1) (1,1)};
            \end{axis}
        \end{tikzpicture}
        \caption{Visão 2D (Gradiente vs. Erro).} % Legenda da subfigura
        \label{fig:huber-derivada-2d}
    \end{subfigure}
    \hfill % Adiciona espaço horizontal flexível entre as subfiguras
    % --- SUBFIGURA (b): Gráfico 3D da Derivada da Huber Loss (delta=1.0) ---
    \begin{subfigure}[b]{0.48\textwidth}
        \centering
        \begin{tikzpicture}
            \begin{axis}[
                % Dimensões consistentes com o gráfico (a)
                width=\linewidth,
                height=7cm,
                xlabel={$y$ (Real)},
                ylabel={$\hat{y}$ (Previsto)},
                zlabel={Gradiente ($\frac{\partial L}{\partial \hat{y}}$)},
                grid=major,
                view={150}{45}, % Mesmo ângulo de visão do seu template
                zmin=-1.5, zmax=1.5, % Espelha o eixo Y do gráfico 2D
                ztick={-1, 0, 1}, % Consistente com o 2D
                title style={font=\bfseries\small},
                label style={font=\small},
                tick label style={font=\scriptsize}
            ]
                % Gráfico da superfície do gradiente: (y_hat - y) se |erro| <= delta, senão delta*sign(erro)
                \addplot3[
                    mesh, 
                    color=red,    % Cor consistente com o gráfico 2D
                    shader=interp, 
                    domain=-5:5,    % Mesmo domínio do seu template
                    domain y=-5:5,  % Mesmo domínio do seu template
                    samples=20,     % Aumentei um pouco os samples para definir melhor a parte linear
                    % PGFPlots usa 'x' para o primeiro domain e 'y' para o segundo
                    % e = y - x  (ou seja, y_previsto - y_real)
                    % delta = 1.0
                ] { ( abs(y-x) <= 1 ? (y-x) : (1 * sign(y-x)) ) }; 
            \end{axis}
        \end{tikzpicture}
        \caption{Superfície 3D completa.} % Legenda da subfigura
        \label{fig:huber-derivada-3d}
    \end{subfigure}

    % --- Legenda e Fonte da Figura Principal ---
    \caption{Visualizações da derivada (gradiente) da função de perda Huber (com $\delta = 1.0$).}
    \label{fig:huber-derivada} % Rótulo principal do seu gráfico
    \fonte{O autor (2025).}
\end{figure}

Perceba que dá para ver como os pontos de descontinuidade atrapalham o formato do gráfico. Note que existem dois pontos que formam um "bico", eles são justamente os pontos em que $y - \hat{y} = \delta$. A descontinuidade pode não atrapalhar a plotagem do gráfico de forma geral, contudo, é sempre importante destacar os detalhes em que esses pontos refletem no gráfico da derivada.

\medskip
\begin{center}
 * * *
\end{center}
\medskip

Criada por Huber como intuito de ser utilizada para fazer uma estimação robusta de parâmetros, a perda de Huber provou ser muito mais que uma solução isolada para resolver um problema específico. E com o passar do tempo, foi tornando-se uma excelente alternativa para ser utilizada ao construir modelos de aprendizado de máquina, servindo como uma função de perda para poder medir o erro do modelo que estaca sendo treinado.

Essa seção busca listar alguns trabalhos que fazem uso dessa função de perda para resolver problemas variados no contexto de inteligência artificial.

Dito isso, vale citar os trabalhos:

\textbf{Algumas Aplicações da Perda de Huber em Problemas de Regressão} \index{Aplicações práticas! Perda de Huber}
\vspace{1em}

\begin{itemize}
    \item \textbf{Previsão de custos (Saúde):} Em \textit{A Huber loss-based super learner with applications to healthcare expenditures}, \textcite{HuberLossSuperLearner} desenvolvem um algoritmo de \textit{ensemble} chamado de \textit{Super Leaner} que tem como um dos objetivos fazer a previsão de custos para a área da saúde. Para isso, os autores usam uma série de outros métodos para compor o \textit{Super Leaner}, como máquinas de vetores de suporte e florestas aleatórias, além disso, para calcular a perda é feito uso da perda de Huber \parencite{HuberLossSuperLearner};
    \item \textbf{Visão computacional de veículos autônomos (Automotiva):} Além disso, no trabalho \textit{Robust Aleatoric Modeling for Future Vehicle Localization},\textcite{RobustAleatoricModelingVehicleLocalization} apresentam uma rede \textit{feedfoward} para previsão robusta para fazer a localização de objetos com intuito de ser utilizada em veículos autônomos. Para fazer isso, os autores adotam a perda de Huber como função de perda para o modelo, como justificativa, eles explicam que ela apresenta a capacidade de treinar modelos de forma robusta contra caixas delimitadoras de referência (\textit{ground-truth}) anormais ou discrepantes;
    \item \textbf{Filtragem de tendências (Área):} Já no texto \textit{RobustTrend: A Huber Loss with a Combined First and Second Order Difference Regularization for Time Series Trend Filtering}, \textcite{RobustTrendHuberLoss} discutem um novo algoritmo de filtragem de tendências em séries temporais, de forma que o objetivo é extrair o sinal de tendência de uma série temporal mesmo quando houver \textit{outliers} ou variações abruptas na tendência. Com esse objetivo em mente, os autores utilizam a perda de Huber como a função de perda escolhida para ser otimizada \parencite{RobustTrendHuberLoss}. A escolha da perda de Huber é ideal para esse tipo de problema, dado que como os autores explicam, existe a presença de \textit{outliers} nos dados que estão sendo analisados;
    \item \textbf{Aplicação 4 (Área):} Por vim, vale citar o artigo \textit{Adaptive Huber Regression} dos pesquisadores \textcite{AdaptiveHuberRegression}, nele, é proposta uma mudança no parâmetro de robustificação da perda de Huber, o qual antes era fixo e agora passa a ser variável, considerando o tamanho da amostra, dimensão dos dados e outros parâmetros. Para testar essa nova forma de lidar com a perda de Huber, os autores utilizam o \textit{dataset} NCI-60, o qual possui 60 linhas de células de cancêr humano \parencite{AdaptiveHuberRegression}.
\end{itemize}

\medskip
\begin{center}
 * * *
\end{center}
\medskip

Com isso, foi possível ver que a perda de Huber é uma excelente alternativa para os casos em que deseja-se controlar os pontos em que o cálculo da perda deve atuar de forma severa (usando termos quadráticos, semelhante a perda L2) e a partir de quis casos esse cálculo pode ser menos punitivo (usando termos lineares, semelhante a perda L1). Contudo, foi visto que ela tem um problema, os pontos em que essas funções se juntam gera uma descontinuidade, atrabalhando a derivação dessas funções. Para resolver esse problema da descontinuidade, pode ser utilizada como alternativa a perda Log-Cosh, a qual será vista em seguida.

\subsection{Perda Log-Cosh (Log-Cosh Loss)} \index{Funções de Perda!Perda Log-Cosh (Log-Cosh Loss)}
\label{sec:log-cosh-loss}

A perda Log-Cosh é uma função de perda que vem ganhando popularidade entre os desenvolvedores, em \textit{Statistical Properties of the log-cosh Loss Function Used in Machine Learning}, \textcite{StatisticalPropetiesLogCosh} explicam que ela aparece em cenários de autoencoders variacionais, detecção de câncer, algortimos de aprendizado baseados em árvores (como o XGBoost) e também em regressão quantílica (\textit{quantile regression}).

Com relação a sua fórmula, é possível vê-la na Equação \ref{eq:log-cosh-loss}. Note que ela não adiciona nenhuma função nova, ela apenas faz a aplicação da função logaritmo que recebe como parâmetro de entrada a função cosseno hiperbólica. Essa combinação geram uma séries de propriedades interessantes, as quais serão discutidas depois ne analisar o seu gráfico.

\begin{equacaodestaque}{Perda Log-Cosh (\textit{Log-Cosh Loss})}
    \Loss_{\text{Log-Cosh}} = \sum_{j=1}^{N} \log(\cosh(y_j - \hat{y}_j))
    \label{eq:log-cosh-loss}
\end{equacaodestaque}

Já com relação ao gráfico dessa função, ele está presente na Figura \ref{fig:log-cosh-loss}. Note que a perda log-cosh atua de forma parecida com a perda de huber. Para valores em que a diferença da saída do modelo e o rôtulo real ($y_j - \hat{y}_j$) é pequena, ela tem um comportamento que lembra ao de uma função quadrática (como o \textit{MSE}). Além disso, conforme o resultado dessa diferença de valores aumenta, a perda log-cosh passa a assumir um comportamento parecido com o de uma função linear (como o \textit{MAE}). Em testes realizados por \textcite{StatisticalPropetiesLogCosh}, a perda log-cosh foi comparada com a perda de Huber, e foi verificado que as estimativas dessas funcoes bem como os erros padroes apresentam resultados similares. Com isso, ela pode ser uma alternativa a ser considerada caso sejam encontrados problemas ao utilizar a \textit{Huber Loss}, mas ainda é desejável manter a variação no cálculo do erro do modelo.

\begin{figure}[h!]
    \centering % Centraliza a figura na página

    % --- SUBFIGURA (a): Gráfico 2D da Log-Cosh Loss ---
    \begin{subfigure}[b]{0.48\textwidth}
        \centering
        \begin{tikzpicture}
            \begin{axis}[
                % Dimensões ajustadas para caber lado a lado
                width=\linewidth,  
                height=7cm,
                xlabel={Erro ($e = y - \hat{y}$)},
                ylabel={Perda Calculada},
                axis lines=middle,
                grid=major,
                grid style={dashed, gray!40},
                xmin=-4.5, xmax=4.5,        % Limites do seu gráfico
                ymin=-0.5, ymax=4.5,         % Limites do seu gráfico
                legend pos=north west,
                title style={font=\bfseries\small},
                label style={font=\small},
                tick label style={font=\scriptsize}
            ]
                % Gráfico da função ln(cosh(x))
                \addplot[
                    domain=-4:4, 
                    samples=101,
                    color=teal, 
                    very thick
                ] {ln(cosh(x))};
                
                \addlegendentry{$L = \log(\cosh(e))$}
            \end{axis}
        \end{tikzpicture}
        \caption{Representação gráfica em duas dimensões.} % Legenda da subfigura
        \label{fig:log-cosh-2d}
    \end{subfigure}
    \hfill % Adiciona espaço horizontal flexível entre as subfiguras
    % --- SUBFIGURA (b): Gráfico 3D da Log-Cosh Loss ---
    \begin{subfigure}[b]{0.48\textwidth}
        \centering
        \begin{tikzpicture}
            \begin{axis}[
                % Dimensões consistentes com o gráfico (a)
                width=\linewidth,
                height=7cm,
                xlabel={$y$ (Real)},
                ylabel={$\hat{y}$ (Previsto)},
                zlabel={Perda},
                grid=major,
                view={150}{45}, % Mesmo ângulo de visão do seu template
                zmin=0, zmax=10.5, % Limite Z ajustado para Log-Cosh com domínio -5:5
                title style={font=\bfseries\small},
                label style={font=\small},
                tick label style={font=\scriptsize}
            ]
                % Gráfico da superfície da Log-Cosh Loss
                \addplot3[
                    mesh,           
                    color=teal,     % Cor consistente com o gráfico 2D
                    shader=interp,  
                    domain=-5:5,    % Mesmo domínio do seu template
                    domain y=-5:5,  % Mesmo domínio do seu template
                    samples=15      % Mesma resolução da malha
                ] { ln(cosh(x - y)) }; % A função Log-Cosh 3D
            \end{axis}
        \end{tikzpicture}
        \caption{Representação gráfica em três dimensões.} % Legenda da subfigura
        \label{fig:log-cosh-3d}
    \end{subfigure}

    % --- Legenda e Fonte da Figura Principal ---
    \caption{Visualizações da função de perda Log-Cosh (\textit{Log-Cosh Loss}) em duas e em três dimensões.}
    \label{fig:log-cosh-loss} % Rótulo principal do seu gráfico
    \fonte{O autor (2025).}
\end{figure}

\medskip
\begin{center}
 * * *
\end{center}
\medskip

\textbf{Características da Perda Log-Cosh}
\vspace{1em}

Com relação as suas propriedades é possível discutir algumas a seguir:

\begin{itemize}
    \item \textbf{Convexa:} É possível ver pelo gráfico da Figura \ref{fig:log-cosh-loss} que ela é uma função convexa, apresentando um formato característico de funil, além de possuír um único ponto de mínimo global. Mas note que não é possível garantir essa propriedade em cenários em que ela está sendo aplicada em modelos de redes neurais densas, devido as transformações não-lineares que ocorrem.
    \item \textbf{Robustez para \textit{outliers}:} Assim como a perda de Huber e o erro absoluto médio, a perda log-cosh não pune de forma agressiva os erros cometidos pelo modelo ao ser treinado. Isso garante que essa função possa ser aplicada em cenários em que os dados possuem muitos \textit{outliers} sem afetar drasticamente o treinamento do modelo.
    \item \textbf{Continuidade e diferenciabilidade:} Diferente de a perda de Huber, que possui pontos de descontuidade na ligação da função linear com a quadrática, a perda log-cosh consegue ser contínua em todos os seus pontos. Além disso, isso também é uma vantagem sobre o erro absoluto médio, pois este também apresenta um ponto de descontinuidade em 0, o qual precisa do cálculo do subgradiente para garantir o aprendizado dos modelos que fazem uso de otimizadores baseados em gradiente. Dessa forma, além de ser contínua, a perda log-cosh pode ser derivada em todos os seus pontos, algo útil caso esteja sendo usados otimizadores que atuam como o método do gradiente.
\end{itemize}

\medskip
\begin{center}
 * * *
\end{center}
\medskip

Visto essas diferentes propriedades dessa função, cabe agora analisar a sua derivada, a qual será útil para a retropropagação e consequentemente o aprendizado do modelo. Para isso, ela pode ser vista na Equação \ref{eq:log-cosh-derivada}. Perceba que a derivada da perda log-cosh envolve o cálculo da tangente hiperbólica, que coincidementemente, também é utilizada em aprendizado de máquina como uma função de ativação, tendo como objetivo introduzir a não-liearidade para as saídas de uma camada densa.

\begin{equacaodestaque}{Derivada da Perda Log-Cosh}
    \frac{\partial \Loss_{\text{Log-Cosh}}}{\partial \hat{y}_j} = \tanh(\hat{y}_j - y_j)
    \label{eq:log-cosh-derivada}
\end{equacaodestaque}

Tendo a fórmula da sua derivada, o próximo passo é analisar o seu gráfico, o qual está representado na Figura \ref{eq:log-cosh-derivada}. Caso você leitor tenha lido o Capítulo \ref{cap:ativacao-sigmoidais}, você não verá nada novo aqui, é apenas o gráfico característico em formato de "S" que as funções sigmoidais possuem. Um ponto interessante a ser destacado ao analisar o gráfico é que as saídas dessa função serão em um intervalo $[-1, 1]$, o que podem fazer com que o sinal do gradiente fique alternando, garantindo uma convergência mais rápida em alguns casos.

\begin{figure}[h!]
    \centering % Centraliza a figura na página

    % --- SUBFIGURA (a): Gráfico 2D da Derivada da Log-Cosh ---
    \begin{subfigure}[b]{0.48\textwidth}
        \centering
        \begin{tikzpicture}
            \begin{axis}[
                % Dimensões ajustadas para caber lado a lado
                width=\linewidth,  
                height=7cm,
                xlabel={Erro ($e = \hat{y} - y$)},
                ylabel={Gradiente ($\frac{\partial L}{\partial \hat{y}}$)},
                axis lines=middle,
                grid=major,
                grid style={dashed, gray!40},
                xmin=-4.5, xmax=4.5,        % Limites do seu gráfico
                ymin=-1.5, ymax=1.5,         % Limites do seu gráfico
                ytick={-1, -0.5, 0, 0.5, 1}, % Marcas no eixo y
                legend pos=south east,
                title style={font=\bfseries\small},
                label style={font=\small},
                tick label style={font=\scriptsize}
            ]
                % Gráfico da função tanh(x)
                \addplot[
                    domain=-4:4, 
                    samples=101,
                    color=teal, 
                    very thick
                ] {tanh(x)};
                
                \addlegendentry{$L' = \tanh(e)$}
            \end{axis}
        \end{tikzpicture}
        \caption{Visão 2D (Gradiente vs. Erro).} % Legenda da subfigura
        \label{fig:log-cosh-derivada-2d}
    \end{subfigure}
    \hfill % Adiciona espaço horizontal flexível entre as subfiguras
    % --- SUBFIGURA (b): Gráfico 3D da Derivada da Log-Cosh ---
    \begin{subfigure}[b]{0.48\textwidth}
        \centering
        \begin{tikzpicture}
            \begin{axis}[
                % Dimensões consistentes com o gráfico (a)
                width=\linewidth,
                height=7cm,
                xlabel={$y$ (Real)},
                ylabel={$\hat{y}$ (Previsto)},
                zlabel={Gradiente ($\frac{\partial L}{\partial \hat{y}}$)},
                grid=major,
                view={150}{45}, % Mesmo ângulo de visão do seu template
                zmin=-1.5, zmax=1.5, % Espelha o eixo Y do gráfico 2D
                ztick={-1, 0, 1}, % Consistente com o 2D
                title style={font=\bfseries\small},
                label style={font=\small},
                tick label style={font=\scriptsize}
            ]
                % Gráfico da superfície do gradiente: tanh(y_hat - y)
                \addplot3[
                    mesh,           
                    color=teal,     % Cor consistente com o gráfico 2D
                    shader=interp,  
                    domain=-5:5,    % Mesmo domínio do seu template
                    domain y=-5:5,  % Mesmo domínio do seu template
                    samples=15      % Mesma resolução da malha
                ] { tanh(y - x) }; % A função da derivada: tanh(y_previsto - y_real)
            \end{axis}
        \end{tikzpicture}
        \caption{Superfície 3D completa.} % Legenda da subfigura
        \label{fig:log-cosh-derivada-3d}
    \end{subfigure}

    % --- Legenda e Fonte da Figura Principal ---
    \caption{Visualizações da derivada (gradiente) da função de perda Log-Cosh.}
    \label{fig:log-cosh-derivada} % Rótulo principal do seu gráfico
    \fonte{O autor (2025).}
\end{figure}

\medskip
\begin{center}
 * * *
\end{center}
\medskip

\textbf{Algumas Aplicações da Perda Log-Cosh em Problemas de Regressão} \index{Aplicações práticas! Perda Log-Cosh}
\vspace{1em}

\begin{itemize}
    \item \textbf{Aplicação 1 (Área):}
    \item \textbf{Aplicação 2 (Área):}
    \item \textbf{Aplicação 3 (Área):}
    \item \textbf{Aplicação 4 (Área):}
\end{itemize}

\medskip
\begin{center}
 * * *
\end{center}
\medskip

Vistas essas quatro funções: O erro quadrático médio (\textit{MSE}), o erro absoluto médio (\textit{MAE}), a perda de Huber e a perda log-cosh. Já é possível resolver a grande maioria dos problemas de regressão em aprendizado de máquina. Contudo, existem problemas que vão além dessas perdas, precisando de funções mais específicas para garantir uma melhor avaliação do erro e a partir dele, saber atualizar o gradiente e com isso o modelo aprender de fato.

Para isso, as próximas seções buscam explorar diferentes cenários em que as funções de perda para regressão vistas até agora não são a melhor escolha. Assim, serão exploradas mais três secões, a primeira focada no erro relativo, a segunda focada em funções de erro que não calculam a média dos diversos erros, e a última para casos em são utilizas distribuições específicas para os valores.

\section{Lidando com a Escala: Foco no Erro Relativo}

As funções de perda vistas até agora possuem um detalhe em comum: todas lidam com o erro de fazendo um cálculo absoluto. Para entender melhor essa frase é possível ilustrar isso com um exemplo. Considere que existem duas situações que está sendo previsto os valores de imóveis:

\begin{itemize}
    \item Cenário A: O modelo previu que uma casa vale 50.000 R\$, equanto no rótulo está que ela vale 10.000 R\$;
    \item Cenário B: O modelo previu que uma casa vale 950.000 R\$, equanto no rótulo está que ela vale 1.000.000 R\$.
\end{itemize}

Para fazer o trabalho dessa regressão foi utilizada a função de perda erro quadrático médio. Note que essa função calcula primeiro a diferença entre o valor previsto pelo modelo e o valor apresentado no rótulo. Com isso, essa diferença será a mesma para esses dois cenários, 50.000 R\$.

Essa é uma forma de analisar o problema. Mas também pode ser visto de forma relativa, veja que o modelo do cenário A previu que a casa vale apenas a metade do seu valor real, ele fez uma previsão subestimada. Por outro lado, o modelo do cenário B foi mais realista, ele sabe que a casa possui um valor alto, contudo, ainda sim ficou uma distância do valor real. 

O erro quadrático médio, e as outras três funções de perda vistas até agora tratam os erros dos cenários A e B como iguais. Mas e se você estivesse em uma situação em que o modelo subestimar os valores dos rôtulos seja considerada muito negativa, e por isso deve ser evitada?

Aí você precisaria de funções de perda que lidassem com o erro relativo das previsões do modelo. Para isso, esse capítulo busca explicar três funções de perda para serem utilizadas nesse cenário. O erro quadrático médio logarítico, que possui uma tendência de penalizar mais as subestimação. O erro percentual absoluto médio, que funciona de forma aposta o \textit{MSLE}, penalizando mais a superestimação, porém com um problema de descontinuidade que pode fazer com que o cálculo da métrica "exploda". E por fim, será visto o erro percentual absoluto médio simétrico, o qual busca corrigir o problema da descontinuidade da sua variante original.

\subsection{Erro Quadrático Médio Logarítmico (MSLE)} \index{Funções de Perda!Erro Quadrático Médio Logarítimico (MSLE)}

A primeira função a ser vista nessa nova categoria é o erro quadrático médio logarítimico, também chamado de \textit{Mean Squared Logarithmic Error} (\textit{MSLE}). A sua fórmula é dada pela Equação \ref{eq:msle-loss}. Perceba que diferente das funções vistas até agora, ela faz o cálculo do logaritmo natural dos valores reais ($y_j$) e dos valores previstos pelo modelo ($\hat{y}_j$). Além disso, um ponto que vale a pena ser destacado é com relação ao valor de 1 que é somado aos valores antes do cálculo do logaritmo. Isso ocorre para evitar com que esse resultado possa ser negativo ou zero, gerando uma indeterminação ao calcular o logaritimo.

\begin{equacaodestaque}{Erro Quadrático Médio Logarítmico (\textit{MSLE})}
    \Loss_{\text{MSLE}} = \frac{1}{N} \sum_{j=1}^{N} (\log(y_j + 1) - \log(\hat{y}_j + 1))^2
    \label{eq:msle-loss}
\end{equacaodestaque}

Com relação ao seu gráfico, ele pode visto na Figura \ref{fig:msle-loss}. Na Figura \ref{fig:msle-2d}, a esquerda, está a visão em duas dimensões dessa função, enquanto na Figura \ref{fig:msle-3d}, a direita, está a visão em três dimensões. Perceba que a adição do logaritmo para essa função traz alguns benefícios como a continuidade em seus pontos além de garantir uma superfície suave e diferenciável. Contudo, a pricipal característica dessa função é que ela é uma função assimétrica com relação ao eixo $y$ e por consequência, ela apresenta uma tendência de penalizar mais as subestimações feitas pelo modelo.

\begin{figure}[h!]
    \centering % Centraliza a figura na página

    % --- SUBFIGURA (a): Gráfico 2D da MSLE (Corte em y=10) ---
    \begin{subfigure}[b]{0.48\textwidth}
        \centering
        \begin{tikzpicture}
            \begin{axis}[
                % Dimensões ajustadas para caber lado a lado
                width=\linewidth,  
                height=7cm,
                xlabel={Valor Previsto ($\hat{y}$)},
                ylabel={Perda Calculada},
                axis lines=middle,
                grid=major,
                grid style={dashed, gray!40},
                xmin=-1, xmax=25,        % Limites do seu gráfico
                ymin=-0.5, ymax=6,         % Limites do seu gráfico
                legend pos=north west,
                title style={font=\bfseries\small},
                label style={font=\small},
                tick label style={font=\scriptsize}
            ]
                % Gráfico da função (ln(11) - ln(x+1))^2
                \addplot[
                    domain=0:25, 
                    samples=101,
                    color=green, 
                    very thick
                ] {(ln(10+1) - ln(x+1))^2};
                
                \addlegendentry{$L(\hat{y} | y=10)$}

                % Linha vertical para marcar o valor real
                \draw[dashed, gray] (axis cs:10, 0) -- (axis cs:10, 6);
                \node[above, gray!80, font=\tiny] at (axis cs:10, 6) {Valor Real ($y=10$)};
            \end{axis}
        \end{tikzpicture}
        \caption{Visão 2D (corte em $y=10$).} % Legenda da subfigura
        \label{fig:msle-2d}
    \end{subfigure}
    \hfill % Adiciona espaço horizontal flexível entre as subfiguras
    % --- SUBFIGURA (b): Gráfico 3D da MSLE ---
    \begin{subfigure}[b]{0.48\textwidth}
        \centering
        \begin{tikzpicture}
            \begin{axis}[
                % Dimensões consistentes com o gráfico (a)
                width=\linewidth,
                height=7cm,
                xlabel={$y$ (Real)},
                ylabel={$\hat{y}$ (Previsto)},
                zlabel={Perda},
                grid=major,
                view={150}{45}, % Mesmo ângulo de visão do seu template
                zmin=0, zmax=12, % Ajustado para (ln(26)-ln(1))^2
                title style={font=\bfseries\small},
                label style={font=\small},
                tick label style={font=\scriptsize}
            ]
                % Gráfico da superfície da MSLE
                \addplot3[
                    mesh,           
                    color=green,    % Cor consistente com o gráfico 2D
                    shader=interp,  
                    domain=0:25,    % Domínio > -1 (usando 0:25)
                    domain y=0:25,  % Domínio > -1 (usando 0:25)
                    samples=15      % Mesma resolução da malha
                ] { (ln(x+1) - ln(y+1))^2 }; % A função MSLE 3D
            \end{axis}
        \end{tikzpicture}
        \caption{Superfície 3D completa.} % Legenda da subfigura
        \label{fig:msle-3d}
    \end{subfigure}

    % --- Legenda e Fonte da Figura Principal ---
    \caption{Visualizações da função de perda MSLE (\textit{Mean Squared Logarithmic Error}).}
    \label{fig:msle-loss} % Rótulo principal do seu gráfico
    \fonte{O autor (2025).}
\end{figure}

\medskip
\begin{center}
 * * *
\end{center}
\medskip

\textbf{Características do Erro Quadrático Médio Logarítmico}
\vspace{1em}

Conhecendo a fórmula dessa função e suas representações gráficas, cabe agora discutir algumas das características dessa função:

\begin{itemize}
    \item \textbf{Tendência de penalizar subestimações:} Como pode ser visto nos gráficos da Figura \ref{fig:msle-loss}, a função \textit{MSLE} não forma uma curva convexa simétrica verticalmente, perceba que conforme os valores vão diminuindo, a perda aumenta consideravelmente. Enquanto isso, conforme os valores aumentam, a perda também aumenta, mas não de forma tão agressiva quanto no sentido inverso. Isso significa que quanto menor for a diferença entre o valor previsto $\hat{y}_j$ e o valor real $y_j$, existe uma tendência de que a perda será maior.
    \item \textbf{Continuidade e Diferencibilidade:} Ainda com relação aos seus gráficos é possível analisar a continuidade dessa função. Note que ela não apresenta nenhum ponto "problema" que poderia afetar o cálculo das derivadas e com isso atrabalhar o fluxo do gradiente. Além disso, o uso dos logaritmos para essa função permite que suas curvas sejam suaves, o que também ajuda na sua continuidade.
\end{itemize}

\medskip
\begin{center}
 * * *
\end{center}
\medskip

\begin{equacaodestaque}{Derivada do Erro Quadrático Médio Logarítmico (\textit{MSLE})}
    \frac{\partial \Loss_{\text{MSLE}}}{\partial \hat{y}_j} = - \frac{2}{N} \cdot \frac{\log(y_j + 1) - \log(\hat{y}_j + 1)}{\hat{y}_j + 1}
    \label{eq:msle-derivada}
\end{equacaodestaque}

\begin{figure}[h!]
    \centering % Centraliza a figura na página

    % --- SUBFIGURA (a): Gráfico 2D da Derivada da MSLE (Corte em y=10) ---
    \begin{subfigure}[b]{0.48\textwidth}
        \centering
        \begin{tikzpicture}
            \begin{axis}[
                % Dimensões ajustadas para caber lado a lado
                width=\linewidth,  
                height=7cm,
                xlabel={Valor Previsto ($\hat{y}$)},
                ylabel={Gradiente ($\frac{\partial L}{\partial \hat{y}}$)},
                axis lines=middle,
                grid=major,
                grid style={dashed, gray!40},
                xmin=-1, xmax=25,        % Limites do seu gráfico
                ymin=-1, ymax=5,         % Limites do seu gráfico
                legend pos=north east,
                title style={font=\bfseries\small},
                label style={font=\small},
                tick label style={font=\scriptsize}
            ]
                % Gráfico da derivada (fórmula corrigida para corresponder aos eixos)
                \addplot[
                    domain=0:25, 
                    samples=101,
                    color=orange, 
                    very thick
                ] {2 * (ln(10+1) - ln(x+1)) / (x+1)}; % Corrigido (removido o -)
                
                \addlegendentry{$L' \text{ para } y=10$}

                % Linha vertical para marcar o valor real
                \draw[dashed, gray] (axis cs:10, -1) -- (axis cs:10, 5);
                \node[above, gray!80, font=\tiny] at (axis cs:10, 5) {Valor Real};
            \end{axis}
        \end{tikzpicture}
        \caption{Visão 2D (corte em $y=10$).} % Legenda da subfigura
        \label{fig:msle-derivada-2d}
    \end{subfigure}
    \hfill % Adiciona espaço horizontal flexível entre as subfiguras
    % --- SUBFIGURA (b): Gráfico 3D da Derivada da MSLE ---
    \begin{subfigure}[b]{0.48\textwidth}
        \centering
        \begin{tikzpicture}
            \begin{axis}[
                % Dimensões consistentes com o gráfico (a)
                width=\linewidth,
                height=7cm,
                xlabel={$y$ (Real)},
                ylabel={$\hat{y}$ (Previsto)},
                zlabel={Gradiente ($\frac{\partial L}{\partial \hat{y}}$)},
                grid=major,
                view={150}{45}, % Mesmo ângulo de visão do seu template
                zmin=-1.5, zmax=6.5, % Ajustado para o pico do gradiente
                title style={font=\bfseries\small},
                label style={font=\small},
                tick label style={font=\scriptsize}
            ]
                % Gráfico da superfície da derivada da MSLE (fórmula corrigida)
                \addplot3[
                    mesh,           
                    color=orange,   % Cor consistente com o gráfico 2D
                    shader=interp,  
                    domain=0:25,    % Domínio de x (y_real)
                    domain y=0:25,  % Domínio de y (y_previsto)
                    samples=15      % Mesma resolução da malha
                ] { 2 * (ln(x+1) - ln(y+1)) / (y+1) }; % L' = 2(log(y+1)-log(y_hat+1))/(y_hat+1)
            \end{axis}
        \end{tikzpicture}
        \caption{Superfície 3D completa.} % Legenda da subfigura
        \label{fig:msle-derivada-3d}
    \end{subfigure}

    % --- Legenda e Fonte da Figura Principal ---
    \caption{Visualizações da derivada (gradiente) da função de perda MSLE.}
    \label{fig:msle-derivada} % Rótulo principal do seu gráfico
    \fonte{O autor (2025).}
\end{figure}

\medskip
\begin{center}
 * * *
\end{center}
\medskip

\textbf{Algumas Aplicações do Erro Quadrático Logarítimico Médio em Problemas de Regressão} \index{Aplicações práticas! Erro quadrático logarítimico médio}
\vspace{1em}

\begin{itemize}
    \item \textbf{Aplicação 1 (Área):}
    \item \textbf{Aplicação 2 (Área):}
    \item \textbf{Aplicação 3 (Área):}
    \item \textbf{Aplicação 4 (Área):}
\end{itemize}

\subsection{Erro Percentual Absoluto Médio (MAPE)} \index{Funções de Perda!Erro Percentual Absoluto Médio (MAPE)}

\begin{equacaodestaque}{Erro Percentual Absoluto Médio (\textit{MAPE})}
    \Loss_{\text{MAPE}} = \frac{1}{N} \sum_{j=1}^{N} \left| \frac{y_j - \hat{y}_j}{y_j} \right|
    \label{eq:mape-loss}
\end{equacaodestaque}

\begin{figure}[h!]
    \centering % Centraliza a figura na página

    % --- SUBFIGURA (a): Gráfico 2D da MAPE (Corte em y=10) ---
    \begin{subfigure}[b]{0.48\textwidth}
        \centering
        \begin{tikzpicture}
            \begin{axis}[
                % Dimensões ajustadas para caber lado a lado
                width=\linewidth,  
                height=7cm,
                title={(a) Perda vs. Valor Previsto (Visão 2D)},
                xlabel={Valor Previsto ($\hat{y}$)},
                ylabel={Perda Calculada},
                axis lines=middle,
                grid=major,
                grid style={dashed, gray!40},
                xmin=-1, xmax=25,        % Limites do seu gráfico
                ymin=-0.2, ymax=2,         % Limites do seu gráfico
                legend pos=north west,
                title style={font=\bfseries\small},
                label style={font=\small},
                tick label style={font=\scriptsize}
            ]
                % Gráfico da função abs((10 - x) / 10)
                \addplot[
                    domain=0:25, 
                    samples=101,
                    color=blue, 
                    very thick
                ] {abs((10 - x) / 10)};
                
                \addlegendentry{$L(\hat{y} | y=10)$}

                % Linha vertical para marcar o valor real
                \draw[dashed, gray] (axis cs:10, 0) -- (axis cs:10, 2);
                \node[above, gray!80, font=\tiny] at (axis cs:10, 2) {Valor Real ($y=10$)};
            \end{axis}
        \end{tikzpicture}
        \caption{Visão 2D (corte em $y=10$).} % Legenda da subfigura
        \label{fig:mape-2d}
    \end{subfigure}
    \hfill % Adiciona espaço horizontal flexível entre as subfiguras
    % --- SUBFIGURA (b): Gráfico 3D da MAPE ---
    \begin{subfigure}[b]{0.48\textwidth}
        \centering
        \begin{tikzpicture}
            \begin{axis}[
                % Dimensões consistentes com o gráfico (a)
                width=\linewidth,
                height=7cm,
                xlabel={$y$ (Real)},
                ylabel={$\hat{y}$ (Previsto)},
                zlabel={Perda},
                grid=major,
                view={150}{45}, % Mesmo ângulo de visão do seu template
                zmin=0, zmax=25, % Ajustado para o domínio
                title style={font=\bfseries\small},
                label style={font=\small},
                tick label style={font=\scriptsize}
            ]
                % Gráfico da superfície da MAPE
                \addplot3[
                    mesh,           
                    color=blue,     % Cor consistente com o gráfico 2D
                    shader=interp,  
                    domain=1:25,    % Domínio de x (y_real) > 0 para evitar divisão por zero
                    domain y=0:25,  % Domínio de y (y_previsto)
                    samples=15      % Mesma resolução da malha
                ] { abs((x - y) / x) }; % A função MAPE 3D: L = |(y - y_hat) / y|
            \end{axis}
        \end{tikzpicture}
        \caption{Superfície 3D completa ($y>0$).} % Legenda da subfigura
        \label{fig:mape-3d}
    \end{subfigure}

    % --- Legenda e Fonte da Figura Principal ---
    \caption{Visualizações da função de perda MAPE (\textit{Mean Absolute Percentage Error}).}
    \label{fig:mape-loss} % Rótulo principal do seu gráfico
    \fonte{O autor (2025).}
\end{figure}

\medskip
\begin{center}
 * * *
\end{center}
\medskip

\textbf{Características do Erro Percentual Absoluto Médio}
\vspace{1em}

\begin{itemize}
    \item \textbf{Característica 1:}
    \item \textbf{Característica 2:}
    \item \textbf{Característica 3:}
\end{itemize}

\medskip
\begin{center}
 * * *
\end{center}
\medskip

\begin{equacaodestaque}{Derivada do Erro Percentual Absoluto Médio (\textit{MAPE})}
    \frac{\partial \Loss_{\text{MAPE}}}{\partial \hat{y}_j} = - \frac{1}{N} \cdot \frac{\text{sgn}(y_j - \hat{y}_j)}{y_j}
    \label{eq:mape-derivada}
\end{equacaodestaque}

\begin{figure}[h!]
    \centering % Centraliza a figura na página

    % --- SUBFIGURA (a): Gráfico 2D da Derivada da MAPE (Corte em y=10) ---
    \begin{subfigure}[b]{0.48\textwidth}
        \centering
        \begin{tikzpicture}
            \begin{axis}[
                % Dimensões ajustadas para caber lado a lado
                width=\linewidth,  
                height=7cm,
                xlabel={Valor Previsto ($\hat{y}$)},
                ylabel={Gradiente ($\frac{\partial L}{\partial \hat{y}}$)},
                axis lines=middle,
                grid=major,
                grid style={dashed, gray!40},
                xmin=-1, xmax=25,        % Limites do seu gráfico
                ymin=-0.5, ymax=0.5,         % Limites do seu gráfico
                legend pos=north east,
                title style={font=\bfseries\small},
                label style={font=\small},
                tick label style={font=\scriptsize},
                declare function={sgn(\x) = (\x > 0) - (\x < 0);} % Função Sinal
            ]
                % Gráfico da derivada
                \addplot[
                    domain=0:25, 
                    samples=201,
                    color=red, 
                    very thick
                ] {-sgn(10-x)/10};
                
                \addlegendentry{$L'(\hat{y} | y=10)$}

                % Linha vertical
                \draw[dashed, gray] (axis cs:10, -0.5) -- (axis cs:10, 0.5);
                \node[above, gray!80, font=\tiny] at (axis cs:10, 0.5) {Valor Real};
            \end{axis}
        \end{tikzpicture}
        \caption{Visão 2D (corte em $y=10$).} % Legenda da subfigura
        \label{fig:mape-derivada-2d}
    \end{subfigure}
    \hfill % Adiciona espaço horizontal flexível entre as subfiguras
    % --- SUBFIGURA (b): Gráfico 3D da Derivada da MAPE ---
    \begin{subfigure}[b]{0.48\textwidth}
        \centering
        \begin{tikzpicture}
            \begin{axis}[
                % Dimensões consistentes com o gráfico (a)
                width=\linewidth,
                height=7cm,
                xlabel={$y$ (Real)},
                ylabel={$\hat{y}$ (Previsto)},
                zlabel={Gradiente ($\frac{\partial L}{\partial \hat{y}}$)},
                grid=major,
                view={150}{45}, % Mesmo ângulo de visão do seu template
                zmin=-1.5, zmax=1.5, % Gradiente máx/mín é +/- 1 (em y=1)
                ztick={-1, 0, 1},
                title style={font=\bfseries\small},
                label style={font=\small},
                tick label style={font=\scriptsize},
                declare function={sgn(\x) = (\x > 0) - (\x < 0);} % Função Sinal
            ]
                % Gráfico da superfície da derivada da MAPE
                \addplot3[
                    mesh,           
                    color=red,      % Cor consistente com o gráfico 2D
                    shader=interp,  
                    domain=1:25,    % Domínio de x (y_real) > 0
                    domain y=0:25,  % Domínio de y (y_previsto)
                    samples=15      % Mesma resolução da malha
                ] { -sgn(x - y) / x }; % L' = -sgn(y - y_hat) / y
            \end{axis}
        \end{tikzpicture}
        \caption{Superfície 3D completa ($y>0$).} % Legenda da subfigura
        \label{fig:mape-derivada-3d}
    \end{subfigure}

    % --- Legenda e Fonte da Figura Principal ---
    \caption{Visualizações da derivada (gradiente) da função de perda MAPE.}
    \label{fig:mape-derivada} % Rótulo principal do seu gráfico
    \fonte{O autor (2025).}
\end{figure}

\medskip
\begin{center}
 * * *
\end{center}
\medskip

\textbf{Algumas Aplicações do Erro Percentual Absoluto Médio em Problemas de Regressão} \index{Aplicações práticas! Erro percentual absoluto médio}
\vspace{1em}

\begin{itemize}
    \item \textbf{Aplicação 1 (Área):}
    \item \textbf{Aplicação 2 (Área):}
    \item \textbf{Aplicação 3 (Área):}
    \item \textbf{Aplicação 4 (Área):}
\end{itemize}

\subsection{Erro Percentual Absoluto Médio Simétrico (sMAPE)} \index{Funções de Perda!Erro Percentual Absoluto Médio Simétrico (sMAPE)}

\begin{equacaodestaque}{Erro Percentual Absoluto Médio Simétrico (\textit{sMAPE})}
    \Loss_{\text{sMAPE}} = \frac{1}{N} \sum_{j=1}^{N} \frac{|\hat{y}_j - y_j|}{(|\hat{y}_j| + |y_j|)/2}
    \label{eq:smape-loss}
\end{equacaodestaque}

\begin{figure}[h!]
    \centering % Centraliza a figura na página

    % --- SUBFIGURA (a): Gráfico 2D da sMAPE (Corte em y=10) ---
    \begin{subfigure}[b]{0.48\textwidth}
        \centering
        \begin{tikzpicture}
            \begin{axis}[
                % Dimensões ajustadas para caber lado a lado
                width=\linewidth,  
                height=7cm,
                xlabel={Valor Previsto ($\hat{y}$)},
                ylabel={Perda Calculada},
                axis lines=middle,
                grid=major,
                grid style={dashed, gray!40},
                xmin=-1, xmax=25,        % Limites do seu gráfico
                ymin=-0.2, ymax=2.2,         % Limites do seu gráfico
                legend pos=north west,
                title style={font=\bfseries\small},
                label style={font=\small},
                tick label style={font=\scriptsize}
            ]
                % Gráfico da função 2*abs(x - 10) / (abs(x) + 10)
                \addplot[
                    domain=0:25, 
                    samples=101,
                    color=purple, 
                    very thick
                ] {2*abs(x - 10) / (abs(x) + 10)};
                
                \addlegendentry{$L(\hat{y} | y=10)$}

                % Linha vertical para marcar o valor real
                \draw[dashed, gray] (axis cs:10, 0) -- (axis cs:10, 2.2);
                \node[above, gray!80, font=\tiny] at (axis cs:10, 2.2) {Valor Real ($y=10$)};
            \end{axis}
        \end{tikzpicture}
        \caption{Visão 2D (corte em $y=10$).} % Legenda da subfigura
        \label{fig:smape-2d}
    \end{subfigure}
    \hfill % Adiciona espaço horizontal flexível entre as subfiguras
    % --- SUBFIGURA (b): Gráfico 3D da sMAPE ---
    \begin{subfigure}[b]{0.48\textwidth}
        \centering
        \begin{tikzpicture}
            \begin{axis}[
                % Dimensões consistentes com o gráfico (a)
                width=\linewidth,
                height=7cm,
                xlabel={$y$ (Real)},
                ylabel={$\hat{y}$ (Previsto)},
                zlabel={Perda},
                grid=major,
                view={150}{45}, % Mesmo ângulo de visão do seu template
                zmin=0, zmax=2.2, % sMAPE é limitada entre 0 e 2
                title style={font=\bfseries\small},
                label style={font=\small},
                tick label style={font=\scriptsize}
            ]
                % Gráfico da superfície da sMAPE
                \addplot3[
                    mesh,           
                    color=purple,   % Cor consistente com o gráfico 2D
                    shader=interp,  
                    domain=0:25,    % Domínio de x (y_real)
                    domain y=0:25,  % Domínio de y (y_previsto)
                    samples=15      % Mesma resolução da malha
                ] { 2*abs(x - y) / (abs(x) + abs(y) + 1e-6) }; % L = 2|y - y_hat| / (|y| + |y_hat|)
            \end{axis}
        \end{tikzpicture}
        \caption{Superfície 3D completa.} % Legenda da subfigura
        \label{fig:smape-3d}
    \end{subfigure}

    % --- Legenda e Fonte da Figura Principal ---
    \caption{Visualizações da função de perda sMAPE (\textit{Symmetric Mean Absolute Percentage Error}).}
    \label{fig:smape-loss} % Rótulo principal do seu gráfico
    \fonte{O autor (2025).}
\end{figure}

\medskip
\begin{center}
 * * *
\end{center}
\medskip

\textbf{Características do Erro Percentual Absoluto Médio Simétrico}
\vspace{1em}

\begin{itemize}
    \item \textbf{Característica 1:}
    \item \textbf{Característica 2:}
    \item \textbf{Característica 3:}
\end{itemize}

\medskip
\begin{center}
 * * *
\end{center}
\medskip

\begin{equacaodestaque}{Derivada do Erro Percentual Absoluto Médio Simétrico (\textit{sMAPE})}
    \frac{\partial \Loss_{\text{sMAPE}}}{\partial \hat{y}_j} = \frac{2}{N} \cdot \frac{\text{sgn}(\hat{y}_j - y_j)(\hat{y}_j + y_j) - |\hat{y}_j - y_j|\text{sgn}(\hat{y}_j)}{(\hat{y}_j + y_j)^2}
    \label{eq:smape-derivada}
\end{equacaodestaque}

\begin{figure}[h!]
    \centering % Centraliza a figura na página

    % --- SUBFIGURA (a): Gráfico 2D da Derivada da sMAPE (Corte em y=10) ---
    \begin{subfigure}[b]{0.48\textwidth}
        \centering
        \begin{tikzpicture}
            \begin{axis}[
                % Dimensões ajustadas para caber lado a lado
                width=\linewidth,  
                height=7cm,
                xlabel={Valor Previsto ($\hat{y}$)},
                ylabel={Gradiente ($\frac{\partial L}{\partial \hat{y}}$)},
                axis lines=middle,
                grid=major,
                grid style={dashed, gray!40},
                xmin=-1, xmax=25,        % Limites do seu gráfico
                ymin=-0.5, ymax=0.5,         % Limites do seu gráfico
                legend pos=north east,
                title style={font=\bfseries\small},
                label style={font=\small},
                tick label style={font=\scriptsize},
                declare function={sgn(\x) = (\x > 0) - (\x < 0);} % Função Sinal
            ]
                % Gráfico da derivada
                \addplot[
                    domain=0:25, 
                    samples=201,
                    color=brown, 
                    very thick
                ] {2 * (sgn(x-10)*(x+10) - abs(x-10)) / ((x+10)^2)};
                
                \addlegendentry{$L' \text{ para } y=10$}

                % Linha vertical
                \draw[dashed, gray] (axis cs:10, -0.5) -- (axis cs:10, 0.5);
                \node[above, gray!80, font=\tiny] at (axis cs:10, 0.5) {Valor Real};
            \end{axis}
        \end{tikzpicture}
        \caption{Visão 2D (corte em $y=10$).} % Legenda da subfigura
        \label{fig:smape-derivada-2d}
    \end{subfigure}
    \hfill % Adiciona espaço horizontal flexível entre as subfiguras
    % --- SUBFIGURA (b): Gráfico 3D da Derivada da sMAPE ---
    \begin{subfigure}[b]{0.48\textwidth}
        \centering
        \begin{tikzpicture}
            \begin{axis}[
                % Dimensões consistentes com o gráfico (a)
                width=\linewidth,
                height=7cm,
                xlabel={$y$ (Real)},
                ylabel={$\hat{y}$ (Previsto)},
                zlabel={Gradiente ($\frac{\partial L}{\partial \hat{y}}$)},
                grid=major,
                view={150}{45}, % Mesmo ângulo de visão do seu template
                zmin=-4.2, zmax=1.2, % Gradiente é grande para y pequeno
                title style={font=\bfseries\small},
                label style={font=\small},
                tick label style={font=\scriptsize},
                declare function={sgn(\x) = (\x > 0) - (\x < 0);} % Função Sinal
            ]
                % Gráfico da superfície da derivada da sMAPE
                \addplot3[
                    mesh,           
                    color=brown,    % Cor consistente com o gráfico 2D
                    shader=interp,  
                    domain=1:25,    % Domínio de x (y_real) > 0
                    domain y=0:25,  % Domínio de y (y_previsto)
                    samples=15      % Mesma resolução da malha
                ] { 2 * (sgn(y-x)*(abs(y)+abs(x)) - abs(y-x)*sgn(y)) / ((abs(y)+abs(x))^2) }; % Derivada da sMAPE 3D
            \end{axis}
        \end{tikzpicture}
        \caption{Superfície 3D completa ($y>0$).} % Legenda da subfigura
        \label{fig:smape-derivada-3d}
    \end{subfigure}

    % --- Legenda e Fonte da Figura Principal ---
    \caption{Visualizações da derivada (gradiente) da função de perda sMAPE.}
    \label{fig:smape-derivada} % Rótulo principal do seu gráfico
    \fonte{O autor (2025).}
\end{figure}

\medskip
\begin{center}
 * * *
\end{center}
\medskip

\textbf{Algumas Aplicações do Erro Percentual Absoluto Médio em Problemas de Regressão} \index{Aplicações práticas! Erro percentual médio simétrico}
\vspace{1em}

\begin{itemize}
    \item \textbf{Aplicação 1 (Área):}
    \item \textbf{Aplicação 2 (Área):}
    \item \textbf{Aplicação 3 (Área):}
    \item \textbf{Aplicação 4 (Área):}
\end{itemize}

\section{Mudando o Objetivo da Previsão: Além da Média}

\subsection{Perda Quantílica} \index{Funções de Perda!Perda Quantílica}

\begin{equacaodestaque}{Perda Quantílica (\textit{Quantile Loss})}
    \Loss_{\tau}(y_j, \hat{y}_j) = 
    \begin{cases} 
        \tau (y_j - \hat{y}_j) & \text{se } y_j \ge \hat{y}_j \\
        (1 - \tau)(\hat{y}_j - y_j) & \text{se } y_j < \hat{y}_j
    \end{cases}
    \label{eq:quantile-loss}
\end{equacaodestaque}

\begin{figure}[h!]
    \centering % Centraliza a figura na página

    % --- SUBFIGURA (a): Gráfico 2D da Perda Quantílica ---
    \begin{subfigure}[b]{0.48\textwidth}
        \centering
        \begin{tikzpicture}
            \begin{axis}[
                % Dimensões ajustadas para caber lado a lado
                width=\linewidth,  
                height=7cm,
                xlabel={Erro ($e = y - \hat{y}$)},
                ylabel={Perda Calculada},
                axis lines=middle,
                grid=major,
                grid style={dashed, gray!40},
                xmin=-4.5, xmax=4.5,        % Limites do seu gráfico
                ymin=-0.5, ymax=3.5,         % Limites do seu gráfico
                legend pos=north west,
                title style={font=\bfseries\small},
                label style={font=\small},
                tick label style={font=\scriptsize}
            ]
                % Tau = 0.5 (MAE)
                \addplot[domain=-4:4, samples=5, color=gray, thick] {0.5*abs(x)};
                \addlegendentry{$\tau=0.5$ (Mediana)}
                
                % Tau = 0.9
                \addplot[domain=-4:4, samples=5, color=blue, very thick] {(x >= 0) ? (0.9*x) : ((1-0.9)*(-x))};
                \addlegendentry{$\tau=0.9$ (Percentil 90)}
                
                % Tau = 0.1
                \addplot[domain=-4:4, samples=5, color=red, very thick] {(x >= 0) ? (0.1*x) : ((1-0.1)*(-x))};
                \addlegendentry{$\tau=0.1$ (Percentil 10)}
            \end{axis}
        \end{tikzpicture}
        \caption{Visão 2D (Perda vs. Erro).} % Legenda da subfigura
        \label{fig:quantile-2d}
    \end{subfigure}
    \hfill % Adiciona espaço horizontal flexível entre as subfiguras
    % --- SUBFIGURA (b): Gráfico 3D da Perda Quantílica ---
    \begin{subfigure}[b]{0.48\textwidth}
        \centering
        \begin{tikzpicture}
            \begin{axis}[
                % Dimensões consistentes com o gráfico (a)
                width=\linewidth,
                height=7cm,
                xlabel={$y$ (Real)},
                ylabel={$\hat{y}$ (Previsto)},
                zlabel={Perda},
                grid=major,
                view={150}{45}, % Mesmo ângulo de visão do seu template
                zmin=0, zmax=10.5, % Ajustado para o domínio
                legend pos=north east,
                title style={font=\bfseries\small},
                label style={font=\small},
                tick label style={font=\scriptsize}
            ]
                % Tau = 0.5 (Plano de fundo como superfície)
                \addplot3[
                    surf,           % Superfície sólida
                    color=gray,
                    opacity=0.5,    % Semi-transparente
                    shader=interp,
                    domain=-5:5,
                    domain y=-5:5,
                    samples=30      % Mais samples para 'surf' suave
                ] { 0.5*abs(x - y) };
                \addlegendentry{$\tau=0.5$}

                % Tau = 0.9 (Malha)
                \addplot3[
                    mesh,           
                    color=blue,
                    domain=-5:5,
                    domain y=-5:5,
                    samples=15      % Menos samples para 'mesh'
                ] { (x - y >= 0) ? (0.9*(x - y)) : ((1-0.9)*(-(x - y))) };
                \addlegendentry{$\tau=0.9$}
                
                % Tau = 0.1 (Malha)
                \addplot3[
                    mesh,           
                    color=red,
                    domain=-5:5,
                    domain y=-5:5,
                    samples=15
                ] { (x - y >= 0) ? (0.1*(x - y)) : ((1-0.1)*(-(x - y))) };
                \addlegendentry{$\tau=0.1$}
            \end{axis}
        \end{tikzpicture}
        \caption{Superfície 3D completa.} % Legenda da subfigura
        \label{fig:quantile-3d}
    \end{subfigure}

    % --- Legenda e Fonte da Figura Principal ---
    \caption{Visualizações da Perda Quantílica (\textit{Pinball Loss}) em duas e em três dimensões.}
    \label{fig:quantile-loss} % Rótulo principal do seu gráfico
    \fonte{O autor (2025).}
\end{figure}

\medskip
\begin{center}
 * * *
\end{center}
\medskip

\textbf{Características da Perda Quantílica} 
\vspace{1em}

\begin{itemize}
    \item \textbf{Característica 1:}
    \item \textbf{Característica 2:}
    \item \textbf{Característica 3:}
\end{itemize}

\medskip
\begin{center}
 * * *
\end{center}
\medskip

\begin{equacaodestaque}{Derivada da Perda Quantílica}
    \frac{\partial \Loss_{\tau}}{\partial \hat{y}_j} = 
    \begin{cases} 
        -(1 - \tau) & \text{se } y_j < \hat{y}_j \text{ (superestimação)}\\
        -\tau & \text{se } y_j > \hat{y}_j \text{ (subestimação)}
    \end{cases}
    \label{eq:quantile-loss-derivada}
\end{equacaodestaque}

\begin{figure}[h!]
    \centering
    \begin{tikzpicture}
        \begin{axis}[
            xlabel={Erro ($y - \hat{y}$)},
            ylabel={Gradiente da Perda ($\frac{\partial L}{\partial (\text{erro})}$)},
            axis lines=middle,
            grid=major,
            grid style={dashed, gray!40},
            xmin=-4.5, xmax=4.5,
            ymin=-1.1, ymax=1.1,
            ytick={-0.9, -0.5, -0.1, 0, 0.1, 0.5, 0.9},
            legend pos=south east,
            width=12cm,
            height=9cm,
            title style={font=\bfseries},
            label style={font=\small},
            tick label style={font=\scriptsize}
        ]
            % Linhas para Tau = 0.9
            \addplot[const plot, color=blue, very thick] coordinates {(-4, 0.9-1) (0, 0.9-1)};
            \addplot[const plot, color=blue, very thick] coordinates {(0, 0.9) (4, 0.9)};
            \addlegendentry{$\tau=0.9$}
            
            % Linhas para Tau = 0.5
            \addplot[const plot, color=gray, thick] coordinates {(-4, 0.5-1) (0, 0.5-1)};
            \addplot[const plot, color=gray, thick] coordinates {(0, 0.5) (4, 0.5)};
            \addlegendentry{$\tau=0.5$}

            % Linhas para Tau = 0.1
            \addplot[const plot, color=red, very thick] coordinates {(-4, 0.1-1) (0, 0.1-1)};
            \addplot[const plot, color=red, very thick] coordinates {(0, 0.1) (4, 0.1)};
            \addlegendentry{$\tau=0.1$}

            % Círculos abertos para a descontinuidade
            \addplot[only marks, mark=o, color=blue, mark size=2pt] coordinates {(0, -0.1) (0, 0.9)};
            \addplot[only marks, mark=o, color=gray, mark size=2pt] coordinates {(0, -0.5) (0, 0.5)};
            \addplot[only marks, mark=o, color=red, mark size=2pt] coordinates {(0, -0.9) (0, 0.1)};
        \end{axis}
    \end{tikzpicture}
    \caption{Gráfico da derivada da Perda Quantílica (em relação ao erro). O gradiente é uma função de degrau assimétrica.}
    \label{fig:quantile-loss-derivada}
    \fonte{O autor (2025).}
\end{figure}

\medskip
\begin{center}
 * * *
\end{center}
\medskip

\textbf{Algumas Aplicações da Perda Quantília em Problemas de Regressão} \index{Aplicações práticas! Perda quantílica}
\vspace{1em}

\begin{itemize}
    \item \textbf{Aplicação 1 (Área):}
    \item \textbf{Aplicação 2 (Área):}
    \item \textbf{Aplicação 3 (Área):}
    \item \textbf{Aplicação 4 (Área):}
\end{itemize}

\subsection{Perda Epsilon-Insensível} \index{Funções de Perda!Perda Epsilon-Insensível}

\begin{equacaodestaque}{Perda Epsilon-Insensível (\textit{$\epsilon$-Insensitive Loss})}
    \Loss_{\epsilon}(y, \hat{y}) = 
    \begin{cases} 
        0 & \text{se } |y - \hat{y}| \le \epsilon \\
        |y - \hat{y}| - \epsilon & \text{se } |y - \hat{y}| > \epsilon
    \end{cases}
    \label{eq:epsilon-insensitive-loss}
\end{equacaodestaque}

\begin{figure}[h!]
    \centering % Centraliza a figura na página

    % --- SUBFIGURA (a): Gráfico 2D da Epsilon-Insensitive Loss ---
    \begin{subfigure}[b]{0.48\textwidth}
        \centering
        \begin{tikzpicture}
            % Define o valor de epsilon
            \def\epsilon{1.0}
            
            \begin{axis}[
                % Dimensões ajustadas para caber lado a lado
                width=\linewidth,  
                height=7cm,
                xlabel={Erro ($e = y - \hat{y}$)},
                ylabel={Perda Calculada},
                axis lines=middle,
                grid=major,
                grid style={dashed, gray!40},
                xmin=-4.5, xmax=4.5,        % Limites do seu gráfico
                ymin=-0.5, ymax=3.5,         % Limites do seu gráfico
                legend pos=north west,
                title style={font=\bfseries\small},
                label style={font=\small},
                tick label style={font=\scriptsize}
            ]
                % Gráfico da função
                \addplot[
                    domain=-4:4, 
                    samples=101,
                    color=purple, 
                    very thick
                ] {max(0, abs(x) - \epsilon)};
                
                \addlegendentry{$L_{\epsilon=1}(e)$}

                % Linhas tracejadas para marcar a margem epsilon
                \draw[dashed, gray] (axis cs:-\epsilon, -0.5) -- (axis cs:-\epsilon, 3.5);
                \draw[dashed, gray] (axis cs:\epsilon, -0.5) -- (axis cs:\epsilon, 3.5);
                \node[above, gray!80, font=\tiny] at (axis cs:0, 2) {Zona de Perda Zero};
            \end{axis}
        \end{tikzpicture}
        \caption{Visão 2D (Perda vs. Erro).} % Legenda da subfigura
        \label{fig:epsilon-2d}
    \end{subfigure}
    \hfill % Adiciona espaço horizontal flexível entre as subfiguras
    % --- SUBFIGURA (b): Gráfico 3D da Epsilon-Insensitive Loss ---
    \begin{subfigure}[b]{0.48\textwidth}
        \centering
        \begin{tikzpicture}
            % Define o valor de epsilon
            \def\epsilon{1.0}
            
            \begin{axis}[
                % Dimensões consistentes com o gráfico (a)
                width=\linewidth,
                height=7cm,
                xlabel={$y$ (Real)},
                ylabel={$\hat{y}$ (Previsto)},
                zlabel={Perda},
                grid=major,
                view={150}{45}, % Mesmo ângulo de visão do seu template
                zmin=0, zmax=10.5, % Ajustado para max(0, abs(10) - 1) = 9
                title style={font=\bfseries\small},
                label style={font=\small},
                tick label style={font=\scriptsize}
            ]
                % Gráfico da superfície da Epsilon-Insensitive Loss
                \addplot3[
                    mesh,           
                    color=purple,   % Cor consistente com o gráfico 2D
                    shader=interp,  
                    domain=-5:5,    % Mesmo domínio do seu template
                    domain y=-5:5,  % Mesmo domínio do seu template
                    samples=15      % Mesma resolução da malha
                ] { max(0, abs(x - y) - \epsilon) }; % A função Epsilon-Insensitive 3D
            \end{axis}
        \end{tikzpicture}
        \caption{Superfície 3D completa.} % Legenda da subfigura
        \label{fig:epsilon-3d}
    \end{subfigure}

    % --- Legenda e Fonte da Figura Principal ---
    \caption{Visualizações da função de perda Epsilon-Insensível (\textit{Epsilon-Insensitive Loss}, $\epsilon=1$) em duas e em três dimensões.}
    \label{fig:epsilon-insensitive-loss} % Rótulo principal do seu gráfico
    \fonte{O autor (2025).}
\end{figure}

\medskip
\begin{center}
 * * *
\end{center}
\medskip

\textbf{Características da Perda Epsilon-Insensível}
\vspace{1em}

\begin{itemize}
    \item \textbf{Característica 1:}
    \item \textbf{Característica 2:}
    \item \textbf{Característica 3:}
\end{itemize}

\medskip
\begin{center}
 * * *
\end{center}
\medskip

\begin{equacaodestaque}{Derivada da Perda Epsilon-Insensível}
    \frac{\partial \Loss_{\epsilon}}{\partial \hat{y}} = 
    \begin{cases} 
        -1 & \text{se } \hat{y} - y > \epsilon \\
        0 & \text{se } |\hat{y} - y| \le \epsilon \\
        1 & \text{se } \hat{y} - y < -\epsilon
    \end{cases}
    \label{eq:epsilon-insensitive-derivada}
\end{equacaodestaque}

\begin{figure}[h!]
    \centering
    \begin{tikzpicture}
        \begin{axis}[
            title={Derivada da Perda Epsilon-Insensível ($\epsilon=1$)},
            xlabel={Erro ($y - \hat{y}$)},
            ylabel={Gradiente da Perda ($\frac{\partial L}{\partial (\text{erro})}$)},
            axis lines=middle,
            grid=major,
            grid style={dashed, gray!40},
            xmin=-4.5, xmax=4.5,
            ymin=-1.5, ymax=1.5,
            ytick={-1, 0, 1},
            legend pos=north west,
            width=12cm,
            height=9cm,
            title style={font=\bfseries},
            label style={font=\small},
            tick label style={font=\scriptsize}
        ]
            % Define o valor de epsilon
            \def\epsilon{1.0}

            % Parte negativa da derivada (-1)
            \addplot[const plot, color=red, very thick] coordinates {(-4, -1) (-\epsilon, -1)};
            
            % Parte central da derivada (0)
            \addplot[const plot, color=gray, very thick] coordinates {(-\epsilon, 0) (\epsilon, 0)};
            
            % Parte positiva da derivada (+1)
            \addplot[const plot, color=blue, very thick] coordinates {(\epsilon, 1) (4, 1)};
            
            % Círculos abertos/fechados para as descontinuidades
            \addplot[only marks, mark=*, color=gray, mark size=1.5pt] coordinates {(-\epsilon, 0) (\epsilon, 0)};
            \addplot[only marks, mark=o, color=red, mark size=1.5pt] coordinates {(-\epsilon, -1)};
            \addplot[only marks, mark=o, color=blue, mark size=1.5pt] coordinates {(\epsilon, 1)};
            
        \end{axis}
    \end{tikzpicture}
    \caption{Gráfico da derivada da Perda Epsilon-Insensível. O gradiente é zero dentro da margem $\epsilon$.}
    \label{fig:epsilon-insensitive-derivada}
    \fonte{O autor (2025).}
\end{figure}

\medskip
\begin{center}
 * * *
\end{center}
\medskip

\textbf{Algumas Aplicações da Perda Epsilon-Insensível em Problemas de Regressão} \index{Aplicações práticas! Perda epsilon-insensível}
\vspace{1em}

\begin{itemize}
    \item \textbf{Aplicação 1 (Área):}
    \item \textbf{Aplicação 2 (Área):}
    \item \textbf{Aplicação 3 (Área):}
    \item \textbf{Aplicação 4 (Área):}
\end{itemize}

\section{Perdas Baseadas em Distribuições de Dados}

\subsection{Perda de Poisson} \index{Funções de Perda!Perda de Poisson}

\begin{equacaodestaque}{Perda de Poisson (\textit{Poisson Loss})}
    \Loss_{\text{Poisson}}(y_j, \hat{y}_j) = \hat{y}_j - y_j \log(\hat{y}_j)
    \label{eq:poisson-loss}
\end{equacaodestaque}

\begin{figure}[h!]
    \centering
    \begin{tikzpicture}
        \begin{axis}[
            xlabel={Valor Previsto ($\hat{y}$)},
            ylabel={Perda Calculada},
            axis lines=middle,
            grid=major,
            grid style={dashed, gray!40},
            xmin=-1, xmax=20,
            ymin=-2, ymax=12,
            legend pos=north west,
            width=12cm,
            height=9cm,
            title style={font=\bfseries},
            label style={font=\small},
            tick label style={font=\scriptsize}
        ]
            % Curva para y=2
            \addplot[domain=0.1:20, samples=101, color=red, thick] {x - 2*ln(x)};
            \addlegendentry{$y=2$}
            \draw[dashed, red!50] (axis cs:2, -2) -- (axis cs:2, {2-2*ln(2)});

            % Curva para y=5
            \addplot[domain=0.1:20, samples=101, color=blue, thick] {x - 5*ln(x)};
            \addlegendentry{$y=5$}
            \draw[dashed, blue!50] (axis cs:5, -2) -- (axis cs:5, {5-5*ln(5)});

            % Curva para y=10
            \addplot[domain=0.1:20, samples=101, color=green, thick] {x - 10*ln(x)};
            \addlegendentry{$y=10$}
            \draw[dashed, green!50] (axis cs:10, -2) -- (axis cs:10, {10-10*ln(10)});
            
        \end{axis}
    \end{tikzpicture}
    \caption{Gráfico da Perda de Poisson para diferentes valores reais de $y$. O mínimo de cada curva ocorre em $\hat{y}=y$.}
    \label{fig:poisson-loss}
    \fonte{O autor (2025).}
\end{figure}

\medskip
\begin{center}
 * * *
\end{center}
\medskip

\textbf{Características da Perda de Poisson}
\vspace{1em}

\begin{itemize}
    \item \textbf{Característica 1:}
    \item \textbf{Característica 2:}
    \item \textbf{Característica 3:}
\end{itemize}

\medskip
\begin{center}
 * * *
\end{center}
\medskip

\begin{equacaodestaque}{Derivada da Perda de Poisson}
    \frac{\partial \Loss_{\text{Poisson}}}{\partial \hat{y}_j} = 1 - \frac{y_j}{\hat{y}_j} = \frac{\hat{y}_j - y_j}{\hat{y}_j}
    \label{eq:poisson-loss-derivada}
\end{equacaodestaque}

\begin{figure}[h!]
    \centering
    \begin{tikzpicture}
        \begin{axis}[
            xlabel={Valor Previsto ($\hat{y}$)},
            ylabel={Gradiente da Perda ($\frac{\partial L}{\partial \hat{y}}$)},
            axis lines=middle,
            grid=major,
            grid style={dashed, gray!40},
            xmin=-1, xmax=20,
            ymin=-4, ymax=1.5,
            legend pos=south east,
            width=12cm,
            height=9cm,
            title style={font=\bfseries},
            label style={font=\small},
            tick label style={font=\scriptsize}
        ]
            % Curva para y=2
            \addplot[domain=0.5:20, samples=101, color=red, thick] {1 - 2/x};
            \addlegendentry{$y=2$}
            
            % Curva para y=5
            \addplot[domain=0.5:20, samples=101, color=blue, thick] {1 - 5/x};
            \addlegendentry{$y=5$}

            % Curva para y=10
            \addplot[domain=0.5:20, samples=101, color=green, thick] {1 - 10/x};
            \addlegendentry{$y=10$}
            
        \end{axis}
    \end{tikzpicture}
    \caption{Gráfico da derivada da Perda de Poisson. O gradiente é zero quando $\hat{y}=y$ e assintótico a 1 para $\hat{y} \to \infty$.}
    \label{fig:poisson-loss-derivada}
    \fonte{O autor (2025).}
\end{figure}

\medskip
\begin{center}
 * * *
\end{center}
\medskip

\textbf{Algumas Aplicações da Perda de Poisson em Problemas de Regressão} \index{Aplicações práticas! Perda de Poisson}
\vspace{1em}

\begin{itemize}
    \item \textbf{Aplicação 1 (Área):}
    \item \textbf{Aplicação 2 (Área):}
    \item \textbf{Aplicação 3 (Área):}
    \item \textbf{Aplicação 4 (Área):}
\end{itemize}

\subsection{Perda de Tweedie} \index{Funções de Perda!Perda de Tweedie}

\begin{equacaodestaque}{Perda de Tweedie (\textit{Tweedie Loss})}
    \Loss_{\text{Tweedie}}(y_j, \hat{y}_j; p) = -\frac{y_j \cdot \hat{y}_j^{1-p}}{1-p} + \frac{\hat{y}_j^{2-p}}{2-p}
    \label{eq:tweedie-loss}
\end{equacaodestaque}

\begin{figure}[h!]
    \centering
    \begin{tikzpicture}
        \begin{axis}[
            xlabel={Valor Previsto ($\hat{y}$)},
            ylabel={Perda Calculada},
            axis lines=middle,
            grid=major,
            grid style={dashed, gray!40},
            xmin=-1, xmax=15,
            ymin=0, ymax=15,
            legend pos=north west,
            width=12cm,
            height=9cm,
            title style={font=\bfseries},
            label style={font=\small},
            tick label style={font=\scriptsize}
        ]
            % Definição do parâmetro p
            \def\p{1.5}

            % Curva para y=1
            \addplot[domain=0.1:15, samples=101, color=red, thick] 
                { -1*x^(1-\p)/(1-\p) + x^(2-\p)/(2-\p) };
            \addlegendentry{$y=1$}
            \draw[dashed, red!50] (axis cs:1, 0) -- (axis cs:1, {-1*1^(1-\p)/(1-\p) + 1^(2-\p)/(2-\p)});

            % Curva para y=4
            \addplot[domain=0.1:15, samples=101, color=blue, thick] 
                { -4*x^(1-\p)/(1-\p) + x^(2-\p)/(2-\p) };
            \addlegendentry{$y=4$}
            \draw[dashed, blue!50] (axis cs:4, 0) -- (axis cs:4, {-4*4^(1-\p)/(1-\p) + 4^(2-\p)/(2-\p)});
            
        \end{axis}
    \end{tikzpicture}
    \caption{Gráfico da Perda de Tweedie para $p=1.5$. O mínimo de cada curva ocorre em $\hat{y}=y$.}
    \label{fig:tweedie-loss}
    \fonte{O autor (2025).}
\end{figure}

\medskip
\begin{center}
 * * *
\end{center}
\medskip

\textbf{Características da Perda de Tweedie}
\vspace{1em}

\begin{itemize}
    \item \textbf{Característica 1:}
    \item \textbf{Característica 2:}
    \item \textbf{Característica 3:}
\end{itemize}

\medskip
\begin{center}
 * * *
\end{center}
\medskip

\begin{equacaodestaque}{Derivada da Perda de Tweedie}
    \frac{\partial \Loss_{\text{Tweedie}}}{\partial \hat{y}_j} = \hat{y}_j^{-p}(\hat{y}_j - y_j)
    \label{eq:tweedie-loss-derivada}
\end{equacaodestaque}

\begin{figure}[h!]
    \centering
    \begin{tikzpicture}
        \begin{axis}[
            xlabel={Valor Previsto ($\hat{y}$)},
            ylabel={Gradiente da Perda ($\frac{\partial L}{\partial \hat{y}}$)},
            axis lines=middle,
            grid=major,
            grid style={dashed, gray!40},
            xmin=-1, xmax=15,
            ymin=-2, ymax=1.5,
            legend pos=south east,
            width=12cm,
            height=9cm,
            title style={font=\bfseries},
            label style={font=\small},
            tick label style={font=\scriptsize}
        ]
            % Definição do parâmetro p
            \def\p{1.5}
            
            % Curva para y=1
            \addplot[domain=0.2:15, samples=101, color=red, thick] 
                {x^(-\p)*(x-1)};
            \addlegendentry{$y=1$}
            
            % Curva para y=4
            \addplot[domain=0.2:15, samples=101, color=blue, thick] 
                {x^(-\p)*(x-4)};
            \addlegendentry{$y=4$}

            % Linhas verticais onde o gradiente é zero
            \draw[dashed, red!50] (axis cs:1, -2) -- (axis cs:1, 1.5);
            \draw[dashed, blue!50] (axis cs:4, -2) -- (axis cs:4, 1.5);
            
        \end{axis}
    \end{tikzpicture}
    \caption{Gráfico da derivada da Perda de Tweedie para $p=1.5$. O gradiente é zero quando a previsão é igual ao valor real.}
    \label{fig:tweedie-loss-derivada}
    \fonte{O autor (2025).}
\end{figure}

\medskip
\begin{center}
 * * *
\end{center}
\medskip

\textbf{Algumas Aplicações da Perda de Tweedie} \index{Aplicações práticas! Perda de Tweedie}
\vspace{1em}

\begin{itemize}
    \item \textbf{Aplicação 1 (Área):}
    \item \textbf{Aplicação 2 (Área):}
    \item \textbf{Aplicação 3 (Área):}
    \item \textbf{Aplicação 4 (Área):}
\end{itemize}

\subsection{Divergência Kullback-Leibler} \index{Funções de Perda!Divergência de Kullback-Leibler}

\begin{equacaodestaque}{Divergência KL entre duas Gaussianas}
    D_{KL}(P || Q) = \log\frac{\sigma_2}{\sigma_1} + \frac{\sigma_1^2 + (\mu_1 - \mu_2)^2}{2\sigma_2^2} - \frac{1}{2}
    \label{eq:kl-divergence-gaussiana}
\end{equacaodestaque}

\begin{figure}[h!]
    \centering
    \begin{tikzpicture}
        \begin{axis}[
            xlabel={Valor da Variável Contínua (y)},
            ylabel={Densidade de Probabilidade},
            axis lines=left,
            grid=major,
            grid style={dashed, gray!40},
            xmin=-4, xmax=6,
            ymin=0, ymax=0.5,
            legend pos=north east,
            width=12cm,
            height=9cm,
            title style={font=\bfseries},
            label style={font=\small},
            tick label style={font=\scriptsize}
        ]
            % Distribuição Real P
            \addplot[
                domain=-4:6, samples=101, color=blue, very thick,
                ] {exp(-(x-0)^2 / (2*1^2)) / (1 * sqrt(2*pi))};
            \addlegendentry{Distribuição Real $P \sim \mathcal{N}(\mu_1, \sigma_1^2)$}

            % Distribuição Prevista Q
            \addplot[
                domain=-4:6, samples=101, color=red, thick,
            ] {exp(-(x-1.5)^2 / (2*1.5^2)) / (1.5 * sqrt(2*pi))};
            \addlegendentry{Distribuição Prevista $Q \sim \mathcal{N}(\mu_2, \sigma_2^2)$}
            
        \end{axis}
    \end{tikzpicture}
    \caption{A Divergência KL mede a diferença entre a distribuição prevista pelo modelo ($Q$) e a distribuição real dos dados ($P$).}
    \label{fig:kl-divergence-concept-regressao}
    \fonte{O autor (2025).}
\end{figure}

\medskip
\begin{center}
 * * *
\end{center}
\medskip

\textbf{Características da Divergência de Kullback-Leibler} 
\vspace{1em}

\begin{itemize}
    \item \textbf{Característica 1:}
    \item \textbf{Característica 2:}
    \item \textbf{Característica 3:}
\end{itemize}

\medskip
\begin{center}
 * * *
\end{center}
\medskip

\begin{equacaodestaque}{Derivadas da Divergência KL (Gaussiana)}
    \frac{\partial D_{KL}}{\partial \mu_2} = \frac{\mu_2 - \mu_1}{\sigma_2^2}
    \\[10pt] % Espaçamento vertical
    \frac{\partial D_{KL}}{\partial \sigma_2} = \frac{1}{\sigma_2} - \frac{\sigma_1^2 + (\mu_1 - \mu_2)^2}{\sigma_2^3}
    \label{eq:kl-divergence-derivada-gaussiana}
\end{equacaodestaque}

\medskip
\begin{center}
 * * *
\end{center}
\medskip

\textbf{Algumas Aplicações da Divergência de Kullback-Leibler em Problemas de Regressão} \index{Aplicações práticas! Divergência de Kullback-Leibler}
\vspace{1em}

\begin{itemize}
    \item \textbf{Aplicação 1 (Área):}
    \item \textbf{Aplicação 2 (Área):}
    \item \textbf{Aplicação 3 (Área):}
    \item \textbf{Aplicação 4 (Área):}
\end{itemize}

\section{Comparativo: Funções de Perda para Regressão}

\section{Fluxograma: Escolhendo a Função de Perda Ideal}

% ===================================================================
% Arquivo: capitulos/parte-III-pilares/cap-10-perda-binaria.tex
% ===================================================================

\chapter{Funções de Perda para Classificação}
\label{cap:perda-classificacao}

\section{Exemplo Ilustrativo:}

\section{Funções de Perda para Classificação Binária}

\subsection{Entropia Cruzada Binária (Binary Cross-Entropy): A função de perda padrão}

\begin{equacaodestaque}{Entropia Cruzada Binária}
    L(y, \hat{y}) = -[y \log(\hat{y}) + (1 - y) \log(1 - \hat{y})]
    \label{eq:binary-cross-entropy}
\end{equacaodestaque}

\begin{tikzpicture}
    \begin{axis}[
        title={Função de Perda: Entropia Cruzada Binária},
        xlabel={Probabilidade Prevista ($\hat{y}$)},
        ylabel={Perda Calculada},
        axis lines=left,              % Eixos no canto inferior esquerdo
        grid=major,                   % Adiciona uma grade principal
        grid style={dashed, gray!40},   % Estilo da grade
        xmin=0, xmax=1,               % Limites do eixo x
        ymin=0, ymax=5,               % Limites do eixo y
        legend pos=north west,      % Posição da legenda
        width=12cm,                   % Largura do gráfico
        height=9cm,                   % Altura do gráfico
        title style={font=\bfseries},
        label style={font=\small},
        tick label style={font=\scriptsize}
    ]
        % Curva para a classe real y=1
        \addplot[
            domain=0.01:0.999, % Domínio para evitar log(0)
            samples=100,
            color=blue,
            very thick
        ] {-ln(x)};
        \addlegendentry{Classe Real = 1 ($L = -\log(\hat{y})$)}

        % Curva para a classe real y=0
        \addplot[
            domain=0.001:0.99, % Domínio para evitar log(0)
            samples=100,
            color=red,
            very thick
        ] {-ln(1-x)};
        \addlegendentry{Classe Real = 0 ($L = -\log(1-\hat{y})$)}
        
    \end{axis}
\end{tikzpicture}

\begin{equacaodestaque}{Derivada da Entropia Cruzada Binária}
    \frac{\partial L}{\partial \hat{y}} = \frac{\hat{y} - y}{\hat{y}(1 - \hat{y})}
    \label{eq:binary-cross-entropy-derivada}
\end{equacaodestaque}

\subsection{Perda Hinge (Hinge Loss)}

\begin{equacaodestaque}{Hinge Loss}
    L(y, f(x)) = \max(0, 1 - y \cdot f(x))
    \label{eq:hinge-loss}
\end{equacaodestaque}

\begin{tikzpicture}
    \begin{axis}[
        title={Função de Perda: Hinge Loss},
        xlabel={Saída Bruta do Modelo ($f(x)$)},
        ylabel={Perda Calculada},
        axis lines=middle,          % Eixos centrados em (0,0)
        grid=major,                 % Adiciona uma grade principal
        grid style={dashed, gray!40}, % Estilo da grade
        xmin=-3.5, xmax=3.5,        % Limites do eixo x
        ymin=-0.5, ymax=4.5,         % Limites do eixo y
        legend pos=north west,      % Posição da legenda
        width=12cm,                 % Largura do gráfico
        height=9cm,                 % Altura do gráfico
        title style={font=\bfseries},
        label style={font=\small},
        tick label style={font=\scriptsize}
    ]
        % Curva para a classe real y=+1
        \addplot[
            domain=-3:3, 
            samples=100, 
            color=blue, 
            very thick
        ] {max(0, 1-x)};
        \addlegendentry{Classe Real = 1 ($L=\max(0, 1-f(x))$)}

        % Curva para a classe real y=-1
        \addplot[
            domain=-3:3, 
            samples=100, 
            color=red, 
            very thick
        ] {max(0, 1+x)};
        \addlegendentry{Classe Real = -1 ($L=\max(0, 1+f(x))$)}
        
        % Opcional: Linhas tracejadas para marcar as margens
        \draw[dashed, gray!70] (axis cs:1, 0) -- (axis cs:1, 4.5);
        \draw[dashed, gray!70] (axis cs:-1, 0) -- (axis cs:-1, 4.5);
        \node[above, gray!80, font=\tiny, rotate=90] at (axis cs:1.1, 2) {Margem};
        \node[above, gray!80, font=\tiny, rotate=90] at (axis cs:-0.9, 2) {Margem};
        
    \end{axis}
\end{tikzpicture}

\begin{equacaodestaque}{Derivada da Hinge Loss}
    \frac{\partial L}{\partial f(x)} = 
    \begin{cases} 
      -y & \text{se } y \cdot f(x) < 1 \\
      0 & \text{se } y \cdot f(x) \ge 1
    \end{cases}
    \label{eq:hinge-loss-derivada}
\end{equacaodestaque}

\section{Funções de Perda para Classificação Multilabel}

\subsection{Entropia Cruzada Categórica (Categorical Cross-Entropy)} 

\begin{equacaodestaque}{Entropia Cruzada Categórica}
    L(y, \hat{y}) = - \sum_{c=1}^{C} y_c \log(\hat{y}_c)
    \label{eq:categorical-cross-entropy}
\end{equacaodestaque}

\begin{tikzpicture}
    \begin{axis}[
        title={Função de Perda: Entropia Cruzada Categórica},
        xlabel={Probabilidade Prevista para a Classe Correta ($\hat{y}_k$)},
        ylabel={Perda Calculada},
        axis lines=left,              % Eixos no canto inferior esquerdo
        grid=major,                   % Adiciona uma grade principal
        grid style={dashed, gray!40},   % Estilo da grade
        xmin=0, xmax=1.05,            % Limites do eixo x
        ymin=0, ymax=5,               % Limites do eixo y
        legend pos=north east,        % Posição da legenda
        width=12cm,                   % Largura do gráfico
        height=9cm,                   % Altura do gráfico
        title style={font=\bfseries},
        label style={font=\small},
        tick label style={font=\scriptsize}
    ]
        % Plota a função -log(y_k_hat)
        \addplot[
            domain=0.01:1, % Domínio para evitar log(0)
            samples=100,
            color=purple,
            very thick
        ] {-ln(x)};
        
        \addlegendentry{$L = -\log(\hat{y}_k)$}
        
    \end{axis}
\end{tikzpicture}

\begin{equacaodestaque}{Derivada da Entropia Cruzada Categórica}
    \frac{\partial L}{\partial z_i} = \hat{y}_i - y_i
    \label{eq:category-cross-entropy-derivada}
\end{equacaodestaque}

\subsection{Entropia Cruzada Categórica Esparsa (Sparse Categorical Cross-Entropy)}

\begin{equacaodestaque}{Entropia Cruzada Categórica Esparsa}
    L_i = - \log(\hat{y}_{i, y_i})
    \label{eq:sparse-categorical-cross-entropy}
\end{equacaodestaque}

\begin{equacaodestaque}{Derivada da Entropia Cruzada Categórica Esparsa}
    \frac{\partial L_i}{\partial z_{i,k}} = \hat{y}_{i,k} - y_{i,k}
    \label{eq:sparse-categorical-cross-entropy-derivada}
\end{equacaodestaque}

\section{Comparativo: Funções de Perda para Classificação}

\section{Fluxograma: Escolhendo a Função de Perda Ideal}
\chapter{Funções de Perda para Usos Específicos}
\label{cap:perdas-especificas}

\section{Focal Loss} \index{Funções de Perda!Focal Loss}

\begin{equacaodestaque}{Focal Loss}
    \Loss_{\text{FL}}(p_t) = -(1 - p_t)^\gamma \log(p_t)
    \label{eq:focal-loss}
\end{equacaodestaque}

\begin{figure}[h!]
    \centering
    \begin{tikzpicture}
        \begin{axis}[
            title={Comparação: Cross-Entropy vs. Focal Loss},
            xlabel={Probabilidade Prevista para a Classe Correta ($p_t$)},
            ylabel={Perda Calculada},
            axis lines=left,
            grid=major,
            grid style={dashed, gray!40},
            xmin=0, xmax=1.05,
            ymin=0, ymax=5,
            legend pos=north east,
            width=12cm,
            height=9cm,
            title style={font=\bfseries},
            label style={font=\small},
            tick label style={font=\scriptsize}
        ]
            % Cross-Entropy Padrão
            \addplot[
                domain=0.01:1, samples=100, color=gray, dashed, thick
            ] {-ln(x)};
            \addlegendentry{Cross-Entropy ($L = -\log(p_t)$)}

            % Focal Loss com gamma=2
            \addplot[
                domain=0.01:1, samples=100, color=purple, very thick
            ] {-(1-x)^2 * ln(x)};
            \addlegendentry{Focal Loss ($\gamma=2$)}
            
        \end{axis}
    \end{tikzpicture}
    \caption{Gráfico da Focal Loss em comparação com a Entropia Cruzada padrão. A perda para exemplos fáceis ($p_t \to 1$) é drasticamente reduzida.}
    \label{fig:focal-loss}
    \fonte{O autor (2025).}
\end{figure}

\begin{equacaodestaque}{Derivada da Focal Loss}
    \frac{\partial \Loss_{\text{FL}}}{\partial z} = 
    \begin{cases} 
        \hat{y}(\gamma(1-\hat{y})\log(\hat{y}) + \hat{y} - 1) & \text{se } y=1 \\
        (1-\hat{y})(\gamma\hat{y}\log(1-\hat{y}) + \hat{y}) & \text{se } y=0
    \end{cases}
    \label{eq:focal-loss-derivada}
\end{equacaodestaque}

\begin{figure}[h!]
    \centering
    \begin{tikzpicture}
        \begin{axis}[
            title={Derivada da Focal Loss ($\gamma=2$)},
            xlabel={Probabilidade Prevista ($\hat{y}$)},
            ylabel={Gradiente da Perda ($\frac{\partial L}{\partial z}$)},
            axis lines=middle,
            grid=major,
            grid style={dashed, gray!40},
            xmin=-0.1, xmax=1.1,
            ymin=-1.1, ymax=1.1,
            legend pos=south east,
            width=12cm,
            height=9cm,
            title style={font=\bfseries},
            label style={font=\small},
            tick label style={font=\scriptsize}
        ]
            % Derivada da Cross-Entropy padrão (y_hat - y)
            \addplot[domain=0:1, samples=10, color=gray, dashed, thick] {x-1};
            \addplot[domain=0:1, samples=10, color=gray, dashed, thick] {x};
            \addlegendentry{Derivada CE}
            
            % Derivada da Focal Loss para y=1
            \addplot[
                domain=0:1, samples=101, color=purple, very thick
            ] {x*(2*(1-x)*ln(x) + x - 1)};
            \addlegendentry{Derivada FL ($y=1$)}

            % Derivada da Focal Loss para y=0
            \addplot[
                domain=0:1, samples=101, color=orange, very thick
            ] {(1-x)*(2*x*ln(1-x) + x)};
            \addlegendentry{Derivada FL ($y=0$)}
            
        \end{axis}
    \end{tikzpicture}
    \caption{Gráfico da derivada da Focal Loss. O gradiente para exemplos fáceis (próximo das bordas 0 e 1) é suprimido em comparação com a derivada da Entropia Cruzada padrão.}
    \label{fig:focal-loss-derivada}
    \fonte{O autor (2025).}
\end{figure}

\medskip
\begin{center}
 * * *
\end{center}
\medskip

\textbf{Algumas Aplicações da \textit{Focal Loss}}
\vspace{1em}

\begin{itemize}
    \item \textbf{Aplicação 1 (Área):}
    \item \textbf{Aplicação 2 (Área):}
    \item \textbf{Aplicação 3 (Área):}
    \item \textbf{Aplicação 4 (Área):}
\end{itemize}

\medskip
\begin{center}
 * * *
\end{center}
\medskip

\section{Fluxograma:}

% ===================================================================
% Arquivo: capitulos/parte-III-pilares/cap-10-perda-binaria.tex
% ===================================================================

\chapter{Métricas de Avaliação}
\label{cap:metricas-de-avaliacao}

\section{Métricas de Avaliação}

\subsection{Acurácia}

\begin{equacaodestaque}{Acurácia}
    \text{Acurácia} = \frac{VP + VN}{VP + VN + FP + FN}
    \label{eq:acuracia}
\end{equacaodestaque}

\subsection{Precisão}

\begin{equacaodestaque}{Precisão (Precision)}
    \text{Precisão} = \frac{VP}{VP + FP}
    \label{eq:precisao}
\end{equacaodestaque}

\subsection{Revocação ou Sensibilidade}

\begin{equacaodestaque}{Revocação (Recall) ou Sensibilidade}
    \text{Revocação} = \frac{VP}{VP + FN}
    \label{eq:revocacao}
\end{equacaodestaque}

\subsection{F1-Score}

\begin{equacaodestaque}{F1-Score}
    \text{F1-Score} = 2 \times \frac{\text{Precisão} \times \text{Revocação}}{\text{Precisão} + \text{Revocação}}
    \label{eq:f1_score}
\end{equacaodestaque}

\subsection{Curva ROC e AUC}

\begin{equacaodestaque}{Taxa de Falsos Positivos (FPR)}
    \text{FPR} = \frac{FP}{FP + VN}
    \label{eq:fpr}
\end{equacaodestaque}

\subsection{Métricas Para Regressão ($R^2$)}

\begin{equacaodestaque}{Raiz do Erro Quadrático Médio (RMSE)}
    L_{\text{RMSE}} = \sqrt{\frac{1}{N} \sum_{i=1}^{N} (y_i - \hat{y}_i)^2}
    \label{eq:rmse}
\end{equacaodestaque}

\begin{equacaodestaque}{Coeficiente de Determinação (R²)}
    R^2 = 1 - \frac{\sum_{i=1}^{N}(y_i - \hat{y}_i)^2}{\sum_{i=1}^{N}(y_i - \bar{y})^2}
    \label{eq:r_quadrado}
\end{equacaodestaque}



\chapter{Autodiff}
\label{cap:autodiff}

\section{Técnicas de Diferenciação}

\section{Autodiff Direta}

\subsection{Autodiff Direta}

\subsection{Autodiff Direta Com Números Duais}

\section{Autodiff Reversa}
% ===================================================================
% Arquivo: capitulos/parte-III-pilares/cap-12-metaheuristicas.tex
% ===================================================================

\chapter{Metaheurísticas: Otimizando Redes Neurais Sem o Gradiente}
\label{cap:otimizacao-metaheuristicas}

O texto do seu capítulo começa aqui...

\section{Algoritmos Evolutivos}

\section{Inteligência de Enxame}

% =======================================================
% PARTE IV: APRENDIZADO DE MÁQUINA CLÁSSICO
% =======================================================
\part{Aprendizado de Máquina Clássico}

% O comando '\include' inicia uma nova página para cada capítulo e
% carrega o conteúdo do arquivo .tex especificado.
% ===================================================================
% Arquivo: capitulos/parte_IV_ml_classico/cap_13_regressao.tex
% ===================================================================

\chapter{Técnicas de Regressão}
\label{cap:regressao}

\section{Exemplo Ilustrativo}

\section{Regressão Linear}

\subsection{Função de Custo MSE}

\subsection{Equação Normal}

\subsection{Implementação em Python}

\section{Regressão Polininomial}

\subsection{Impletanção em Python}

\section{Regressão de Ridge}

\subsection{Implementação em Python}

\section{Regressão de Lasso}

\subsection{Implementação em Python}

\section{Elastic Net}

\subsection{Implementação em Python}

\section{Regressão Logística}

\subsection{Implementação em Python}

\section{Regressão Softmax}

\subsection{Implementação em Python}

\section{Outras Técnicas de Regressão}


% ===================================================================
% Arquivo: capitulos/parte_IV_ml_classico/cap_14_arvores.tex
% ===================================================================

\chapter{Árvores de Decisão e Florestas Aleatórias}
\label{cap:arvores}

\section{Exemplo Ilustrativo}

\section{Entendendo o Conceito de Árvores}

\subsection{Árvores Binárias}

\section{Árvores de Decisão}

\subsection{Implementação em Python}

\section{Florestas Aleatórias}

\subsection{Implementação em Python}
% ===================================================================
% Arquivo: capitulos/parte_IV_ml_classico/cap_15_svm.tex
% ===================================================================

\chapter{Máquinas de Vetores de Suporte}
\label{cap:svm}

O texto do seu capítulo começa aqui...
% ===================================================================
% Arquivo: capitulos/parte_IV_ml_classico/cap_16_emsamble.tex
% ===================================================================

\chapter{Ensamble}
\label{cap:ensamble}

\section{Exemplo Ilustrativo}
% ===================================================================
% Arquivo: capitulos/parte_IV_ml_classico/cap_17_dimensionalidade.tex
% ===================================================================

\chapter{Dimensionalidade}
\label{cap:dimensionalidade}

O texto do seu capítulo começa aqui...
% ===================================================================
% Arquivo: capitulos/parte_IV_ml_classico/cap_18_clusterizacao.tex
% ===================================================================

\chapter{Clusterização}
\label{cap:clusterizacao}

\section{Exemplo Ilustrativo}

\section{Aprendizado Não Supervisionado: Encontrando Grupos nos Dados}

\section{Clusterização Particional: K-Means}

\section{Clusterização Hierárquica}

\section{Clusterização Baseada em Densidade: DBSCAN}

% =======================================================
% PARTE V: REDES NEURAIS PROFUNDAS (DNNs)
% =======================================================
\part{Redes Neurais Profundas (DNNs)}

\include{capitulos/parte-V-deep-learning/cap-19-mlp}
% ===================================================================
% Arquivo: capitulos/parte_V_deep_learning/cap_20_ffn.tex
% ===================================================================

\chapter{Redes FeedForward (FFNs)}
\label{cap:ffn}

O texto do seu capítulo começa aqui...
% ===================================================================
% Arquivo: capitulos/parte_V_deep_learning/cap_21_dbn.tex
% ===================================================================

\chapter{Redes de Crença Profunda (DBNs) e Máquinas de Boltzmann Restritas}
\label{cap:dbn}

O texto do seu capítulo começa aqui...
% ===================================================================
% Arquivo: capitulos/parte_V_deep_learning/cap_22_cnn.tex
% ===================================================================

\chapter{Redes Neurais Convolucionais (CNN)}
\label{cap:cnn}

% ===================================================================
% Resumo do capítulo
% ===================================================================

\section{Exemplo Ilustrativo}

\section{Camadas Convolucionais: O Bloco Fundamental para as CNNs}

\subsection{Implementação em Python}

\section{Camadas de Poooling: Reduzindo a Dimensionalidade}

\subsection{Max Pooling}

\subsection{Avg Pooling}

\subsection{Global Abg Pooling}

\subsection{Implementação em Python}

\section{Camada Flatten: Achatando os Dados}

\subsection{Implementação em Python}

\section{Criando uma CNN}

\section{Detecção de Objetos}

\section{Redes Totalmente Convolucionais (FCNs)}

\section{You Only Look Once (YOLO)}

\section{Algumas Arquiteturas de CNNs}

\subsection{LeNet-5}

\subsection{AlexNet}

\subsection{GoogLeNet}

\subsection{VGGNet}

\subsection{ResNet}

\subsection{Xception}

\subsection{SENet}
% ===================================================================
% Arquivo: capitulos/parte_V_deep_learning/cap_23_resnet.tex
% ===================================================================

\chapter{Redes Residuais (ResNets)}
\label{cap:resnet}

O texto do seu capítulo começa aqui...
% ===================================================================
% Arquivo: capitulos/parte_V_deep_learning/cap_24_rnn.tex
% ===================================================================

\chapter{Redes Neurais Recorrentes (RNN)}
\label{cap:rnn}

O texto do seu capítulo começa aqui...
% ===================================================================
% Arquivo: capitulos/parte_V_deep_learning/cap_25_regularizacao.tex
% ===================================================================

\chapter{Técnicas para Melhorar o Desempenho de Redes Neurais}
\label{cap:regularizacao}

O texto do seu capítulo começa aqui...
% ===================================================================
% Arquivo: capitulos/parte_V_deep_learning/cap_26_transformers.tex
% ===================================================================

\chapter{Transformers}
\label{cap:transformers}

O texto do seu capítulo começa aqui...
\include{capitulos/parte-V-deep-learning/cap-27-gans}
% ===================================================================
% Arquivo: capitulos/parte_V_deep_learning/cap_28_moe.tex
% ===================================================================

\chapter{Mixture of Experts (MoE)}
\label{cap:moe}

O texto do seu capítulo começa aqui...
% ===================================================================
% Arquivo: capitulos/parte_V_deep_learning/cap_29_diffusion.tex
% ===================================================================

\chapter{Modelos de Difusão}
\label{cap:diffusion}

O texto do seu capítulo começa aqui...
\include{capitulos/parte-V-deep-learning/cap-30-gnns}
% Adicione o capítulo de otimizadores se ele estiver aqui


% --- APÊNDICES ---
% O comando \appendix muda a formatação dos capítulos para "Apêndice A", "Apêndice B", etc.
\appendix
\part{Apêndices}

\chapter{Comparativo dos Otimizadores}
\label{cap:comparativo-otimizadores}

A escolha do otimizador é um passo crucial no treinamento de redes neurais. Este capítulo apresenta os principais algoritmos de otimização baseados em gradiente, desde os métodos clássicos até as variantes adaptativas modernas, detalhando suas equações, ideias centrais e características.

% ===================================================================
% Otimizadores Clássicos
% ===================================================================
\section{Otimizadores Clássicos}

\subsection{Gradiente Descendente (GD)}

\textit{Atualiza os parâmetros na direção oposta ao gradiente, calculado sobre \textbf{todo} o conjunto de dados.}

\begin{equacaodestaque}{Atualização do Gradiente Descendente}
    \theta_{t+1} = \theta_t - \eta \nabla f(\theta_t)
\end{equacaodestaque}

\subsubsection*{Vantagens}
\begin{itemize}
    \item Garante uma convergência estável para um mínimo local (ou global, em funções convexas).
\end{itemize}

\subsubsection*{Desvantagens}
\begin{itemize}
    \item É computacionalmente caro e lento para datasets grandes, pois exige que todos os dados estejam na memória para cada atualização.
\end{itemize}

\subsection{Gradiente Descendente Estocástico (SGD)}

\textit{Atualiza os parâmetros usando o gradiente de \textbf{uma única amostra} aleatória por vez.}

\begin{equacaodestaque}{Atualização do Gradiente Descendente Estocástico}
    \theta_{t+1} = \theta_t - \eta \nabla f(\theta_t; x^{(i)})
\end{equacaodestaque}

\subsubsection*{Vantagens}
\begin{itemize}
    \item Muito mais rápido por iteração em comparação com o GD em lote.
    \item A natureza ruidosa das atualizações pode ajudar a escapar de mínimos locais rasos.
\end{itemize}

\subsubsection*{Desvantagens}
\begin{itemize}
    \item Apresenta uma trajetória de convergência ruidosa e com alta variância, podendo nunca se estabilizar no mínimo exato.
\end{itemize}

\subsection{Gradiente Descendente com Momento}

\textit{Adiciona "inércia" à atualização, acumulando uma média móvel dos gradientes passados para acelerar a descida.}

\begin{equacaodestaque}{Atualização com Momento}
    v_t = \beta v_{t-1} + \eta \nabla f(\theta_t) \\
    \theta_{t+1} = \theta_t - v_t
\end{equacaodestaque}

\subsubsection*{Vantagens}
\begin{itemize}
    \item Acelera a convergência, especialmente em direções onde o gradiente é consistente.
    \item Ajuda a amortecer oscilações em direções de alta curvatura.
\end{itemize}

\subsubsection*{Desvantagens}
\begin{itemize}
    \item Adiciona o hiperparâmetro de momento $\beta$, que precisa ser ajustado.
\end{itemize}

\subsection{Gradiente Acelerado de Nesterov (NAG)}

\textit{Um momento "mais inteligente" que calcula o gradiente em um ponto futuro estimado ("lookahead") para corrigir a direção da atualização.}

\begin{equacaodestaque}{Atualização com Momento de Nesterov}
    g_t = \nabla f(\theta_t - \beta v_{t-1}) \\
    v_t = \beta v_{t-1} + \eta g_t \\
    \theta_{t+1} = \theta_t - v_t
\end{equacaodestaque}

\subsubsection*{Vantagens}
\begin{itemize}
    \item Frequentemente converge mais rápido que o momento padrão.
    \item É mais eficaz em "antecipar" a curvatura, evitando ultrapassar o ponto de mínimo.
\end{itemize}

% ===================================================================
% Otimizadores Adaptativos Modernos
% ===================================================================
\section{Otimizadores Adaptativos Modernos}

\subsection{AdaGrad (Adaptive Gradient Algorithm)}

\textit{Adapta a taxa de aprendizado para cada parâmetro individualmente, diminuindo-a para parâmetros com gradientes grandes e frequentes.}

\begin{equacaodestaque}{Atualização do AdaGrad}
    \theta_{t+1} = \theta_t - \frac{\eta}{\sqrt{N_t + \epsilon}} g_t \\
    \text{(onde } N_t \text{ acumula os quadrados dos gradientes } g_t^2\text{)}
\end{equacaodestaque}

\subsubsection*{Vantagens}
\begin{itemize}
    \item É muito eficaz para lidar com dados esparsos, como em processamento de linguagem natural.
\end{itemize}

\subsubsection*{Desvantagens}
\begin{itemize}
    \item A taxa de aprendizado pode decair de forma muito agressiva e parar o treinamento prematuramente, pois o acumulador de gradientes no denominador só cresce.
\end{itemize}

\subsection{RMSProp (Root Mean Square Propagation)}

\textit{Resolve o problema do AdaGrad usando uma média móvel exponencial dos quadrados dos gradientes, o que evita que a taxa de aprendizado decaia para zero.}

\begin{equacaodestaque}{Atualização do RMSProp}
    \theta_{t+1} = \theta_t - \frac{\eta}{\sqrt{E[g^2]_t + \epsilon}} g_t \\
    \text{(onde } E[g^2]_t \text{ é uma média móvel de } g_t^2\text{)}
\end{equacaodestaque}

\subsubsection*{Vantagens}
\begin{itemize}
    \item Apresenta bom desempenho em problemas não-estacionários (onde a distribuição dos dados muda).
    \item É uma melhoria direta sobre o AdaGrad.
\end{itemize}

\subsection{Adam (Adaptive Moment Estimation)}

\textit{Calcula taxas de aprendizado adaptativas para cada parâmetro usando estimativas de primeiro (momento) e segundo (RMSProp) momentos dos gradientes.}

\begin{equacaodestaque}{Conceito do Adam}
    \text{Combina a inércia do \textbf{Momento} (1º momento, $m_t$)} \\
    \text{com a escala adaptativa do \textbf{RMSProp} (2º momento, $v_t$)} \\
    \text{e adiciona uma etapa de correção de viés.}
\end{equacaodestaque}

\subsubsection*{Vantagens}
\begin{itemize}
    \item Geralmente considerado o otimizador padrão para a maioria dos problemas.
    \item Combina os benefícios dos métodos de momento e de taxa de aprendizado adaptativa, sendo robusto e eficiente.
\end{itemize}

\subsection{AdaMax}

\textit{Generaliza o segundo momento do Adam, substituindo a média dos quadrados ($L_2$) pelo máximo ($L_\infty$) dos gradientes recentes, tornando a atualização mais estável.}

\begin{equacaodestaque}{Conceito do AdaMax}
    \text{Variante do Adam que usa a norma infinita ($L_\infty$) para o} \\
    \text{segundo momento, em vez da norma $L_2$.}
\end{equacaodestaque}

\subsubsection*{Vantagens}
\begin{itemize}
    \item Pode ser mais estável que o Adam, especialmente em cenários com gradientes ruidosos ou esparsos.
\end{itemize}

\subsection{Nadam (Nesterov-accelerated Adam)}

\textit{Incorpora o conceito de "lookahead" do Gradiente Acelerado de Nesterov (NAG) na estimativa do primeiro momento do Adam para uma atualização mais precisa.}

\begin{equacaodestaque}{Conceito do Nadam}
    \text{Combina o otimizador \textbf{Adam} com o momento de \textbf{Nesterov (NAG)}.}
\end{equacaodestaque}

\subsubsection*{Vantagens}
\begin{itemize}
    \item Frequentemente converge mais rápido que o Adam, especialmente em problemas com gradientes complexos e ruidosos.
\end{itemize}

\subsection{AdamW (Adam with Decoupled Weight Decay)}

\textit{Corrige a implementação da regularização L2 (decaimento de peso) no Adam, desacoplando-a da atualização do gradiente e aplicando-a diretamente aos pesos.}

\begin{equacaodestaque}{Atualização do AdamW}
    \theta_{t+1} = \theta_t - \eta \cdot (\text{update}_{Adam} + \lambda\theta_t)
\end{equacaodestaque}

\subsubsection*{Vantagens}
\begin{itemize}
    \item Melhora a generalização do modelo em comparação com o Adam padrão com regularização L2.
    \item Torna o ajuste da taxa de aprendizado e do decaimento de peso mais independente um do outro.
\end{itemize}
\chapter{Tabela das Funções de Ativação}

\begin{longtable}{@{} l p{0.25\linewidth} p{0.3\linewidth} p{0.3\linewidth} @{}}
    
    % --- TÍTULO (CAPTION) ---
    \caption{Comparativo das famílias de funções de ativação, suas propriedades, vantagens e desvantagens.}
    \label{tab:funcoes_comparativo_completo} \\

    % --- CABEÇALHO DA PRIMEIRA PÁGINA ---
    \toprule
    \textbf{Função} & \textbf{Equação e Derivada} & \textbf{Vantagens} & \textbf{Desvantagens} \\
    \midrule
    \endfirsthead

    % --- CABEÇALHO DAS PÁGINAS SEGUINTES ---
    \multicolumn{4}{l}{\small\textbf{Tabela \thetable{} – Continuação}} \\
    \toprule
    \textbf{Função} & \textbf{Equação e Derivada} & \textbf{Vantagens} & \textbf{Desvantagens} \\
    \midrule
    \endhead

    % --- RODAPÉ DE CONTINUAÇÃO ---
    \multicolumn{4}{r}{\small\textit{(Continua na próxima página)}} \\
    \endfoot

    % --- RODAPÉ FINAL (NA ÚLTIMA PÁGINA) ---
    \bottomrule
    \multicolumn{4}{l}{\parbox{\linewidth}{\small\textit{Fonte: O autor (2025).}}} \\
    \endlastfoot

    % --- CONTEÚDO DA TABELA (UNIFICADO) ---

    % --- Família Sigmoidal ---
    \textbf{Sigmoide} & 
    $\displaystyle \frac{1}{1 + e^{-z_i}}$ \newline\vspace{0.2cm}
    $\displaystyle \sigma(z_i)(1 - \sigma(z_i))$ 
    & 
    \begin{itemize}[noitemsep, topsep=0pt, partopsep=0pt, leftmargin=*]
        \item Saída no intervalo (0, 1), interpretável como probabilidade.
        \item Função suave e diferenciável.
    \end{itemize}
    &
    \begin{itemize}[noitemsep, topsep=0pt, partopsep=0pt, leftmargin=*]
        \item Não é centrada em zero.
        \item Sofre com o desvanecimento do gradiente.
    \end{itemize}
    \\ \addlinespace
    
    \textbf{Tangente hip.} & 
    $\displaystyle \frac{e^{z_i} - e^{-z_i}}{e^{z_i} + e^{-z_i}}$ \newline\vspace{0.2cm}
    $\displaystyle 1 - \tanh^2(z_i)$
    &
    \begin{itemize}[noitemsep, topsep=0pt, partopsep=0pt, leftmargin=*]
        \item Centrada em zero, acelera a convergência.
        \item Gradiente mais forte que a sigmoide.
    \end{itemize}
    &
    \begin{itemize}[noitemsep, topsep=0pt, partopsep=0pt, leftmargin=*]
        \item Ainda sofre com o desvanecimento do gradiente.
    \end{itemize}
    \\ \addlinespace
    
    \textbf{Softsign} &
    $\displaystyle \frac{z_i}{1 + |z_i|}$ \newline\vspace{0.2cm}
    $\displaystyle \frac{1}{(1 + |z_i|)^2}$
    &
    \begin{itemize}[noitemsep, topsep=0pt, partopsep=0pt, leftmargin=*]
        \item Computacionalmente eficiente.
        \item Satura mais lentamente que a Tanh.
    \end{itemize}
    &
    \begin{itemize}[noitemsep, topsep=0pt, partopsep=0pt, leftmargin=*]
        \item Derivada não pode ser expressa em termos da própria função.
    \end{itemize}
    \\ \addlinespace

    \textbf{Hard Sigmoid} &
    $\displaystyle \begin{cases} 0 & \text{se } z_i < -3 \\ z_i/6 + 0.5 & \text{se } -3 \le z_i \le 3 \\ 1 & \text{se } z_i > 3 \end{cases}$ \newline\vspace{0.2cm}
    $\displaystyle \begin{cases} 0 & \text{se } z_i < -3 \\ 1/6 & \text{se } -3 < z_i < 3 \\ 0 & \text{se } z_i > 3 \end{cases}$
    &
    \begin{itemize}[noitemsep, topsep=0pt, partopsep=0pt, leftmargin=*]
        \item Extremamente rápida e eficiente.
        \item Ideal para hardware com poucos recursos.
    \end{itemize}
    &
    \begin{itemize}[noitemsep, topsep=0pt, partopsep=0pt, leftmargin=*]
        \item Não é suave; pode "matar" gradientes.
        \item É uma aproximação.
    \end{itemize}
    \\ \addlinespace
    
    \textbf{Hard Tanh} &
    $\displaystyle \begin{cases} -1 & \text{se } z_i < -1 \\ z_i & \text{se } -1 \le z_i \le 1 \\ 1 & \text{se } z_i > 1 \end{cases}$ \newline\vspace{0.2cm}
    $\displaystyle \begin{cases} 0 & \text{se } z_i < -1 \\ 1 & \text{se } -1 < z_i < 1 \\ 0 & \text{se } z_i > 1 \end{cases}$
    &
    \begin{itemize}[noitemsep, topsep=0pt, partopsep=0pt, leftmargin=*]
        \item Extremamente rápida e centrada em zero.
        \item Ótima para hardware de baixo consumo.
    \end{itemize}
    &
    \begin{itemize}[noitemsep, topsep=0pt, partopsep=0pt, leftmargin=*]
        \item Não é suave; derivada nula em grande parte do domínio.
    \end{itemize}
    \\ \addlinespace

    % --- Família Retificadora ---
    \textbf{ReLU} & 
    $ \begin{cases}z_i, & \text{se } z_i > 0 \\0, & \text{se } z_i \leq 0\end{cases} $ \newline\vspace{0.2cm}
    $ \begin{cases}1, & \text{se } z_i > 0 \\0, & \text{se } z_i < 0 \\ \nexists & \text{se } z_i = 0\end{cases}$ 
    & 
    \begin{itemize}[noitemsep, topsep=0pt, partopsep=0pt, leftmargin=*]
        \item Computacionalmente eficiente.
        \item Evita o desvanecimento do gradiente.
        \item Promove esparsidade na rede.
    \end{itemize}
    &
    \begin{itemize}[noitemsep, topsep=0pt, partopsep=0pt, leftmargin=*]
        \item Não é centrada em zero.
        \item Pode "morrer" (Dying ReLU).
        \item Pode sofrer com a explosão de gradientes.
    \end{itemize}
    \\ \addlinespace

    \textbf{LReLU} & 
    $ \begin{cases}z_i, & \text{se } z_i \ge 0 \\ \alpha \cdot z_i, & \text{se } z_i < 0\end{cases} $ \newline\vspace{0.2cm}
    $\begin{cases}1, & \text{se } z_i > 0 \\ \alpha, & \text{se } z_i < 0 \\ \nexists, & \text{se } z_i = 0\end{cases}$
    &
    \begin{itemize}[noitemsep, topsep=0pt, partopsep=0pt, leftmargin=*]
        \item Resolve o problema da "Dying ReLU".
    \end{itemize}
    &
    \begin{itemize}[noitemsep, topsep=0pt, partopsep=0pt, leftmargin=*]
        \item O valor de $\alpha$ não é aprendido.
        \item Resultados podem ser inconsistentes.
    \end{itemize}
    \\ \addlinespace

    \textbf{PReLU} &
    $ \begin{cases}z_i, & \text{se } z_i \ge 0 \\ \alpha_i \cdot z_i, & \text{se } z_i < 0\end{cases} $ \newline\vspace{0.2cm}
    $\begin{cases}1, & \text{se } z_i > 0 \\ \alpha_i, & \text{se } z_i < 0 \\ \nexists, & \text{se } z_i = 0\end{cases}$
    &
    \begin{itemize}[noitemsep, topsep=0pt, partopsep=0pt, leftmargin=*]
        \item Variação da LReLU onde $\alpha$ é um parâmetro aprendido.
        \item Pode melhorar a performance.
    \end{itemize}
    &
    \begin{itemize}[noitemsep, topsep=0pt, partopsep=0pt, leftmargin=*]
        \item Risco de sobreajuste (overfitting) se os dados forem poucos.
    \end{itemize}
    \\ \addlinespace

    \textbf{ELU} &
    $\begin{cases}z_i, & \text{se } z_i \ge 0 \\ \alpha (e^{z_i} - 1) , & \text{se } z_i < 0\end{cases}$ \newline\vspace{0.2cm}
    $ \begin{cases}1, & \text{se } z_i > 0 \\ \alpha e^{z_i}, & \text{se } z_i < 0 \end{cases}$
    &
    \begin{itemize}[noitemsep, topsep=0pt, partopsep=0pt, leftmargin=*]
        \item Saídas com média próxima de zero.
        \item Mais robusta a ruído que LReLU/PReLU.
    \end{itemize}
    &
    \begin{itemize}[noitemsep, topsep=0pt, partopsep=0pt, leftmargin=*]
        \item Computacionalmente mais custosa (exponencial).
    \end{itemize}
    \\ \addlinespace

    \textbf{SELU} &
    $\lambda \begin{cases}z_i, & \text{se } z_i > 0 \\ \alpha (e^{z_i} - 1) , & \text{se } z_i \le 0\end{cases}$ \newline\vspace{0.2cm}
    $ \lambda \begin{cases}1, & \text{se } z_i > 0 \\ \alpha e^{z_i}, & \text{se } z_i \le 0\end{cases}$
    &
    \begin{itemize}[noitemsep, topsep=0pt, partopsep=0pt, leftmargin=*]
        \item Propriedades de autonormalização.
        \item Evita gradientes explosivos/desvanecentes em redes muito profundas.
    \end{itemize}
    &
    \begin{itemize}[noitemsep, topsep=0pt, partopsep=0pt, leftmargin=*]
        \item Requer inicialização de pesos específica (LeCun normal).
        \item Computacionalmente mais custosa.
    \end{itemize}
    \\ \addlinespace

    \textbf{GELU} &
    $z_i \cdot \Phi(z_i)$ \newline\vspace{0.2cm}
    $ \Phi(z_i) + z_i\phi(z_i)$
    &
    \begin{itemize}[noitemsep, topsep=0pt, partopsep=0pt, leftmargin=*]
        \item Suave e diferenciável em todos os pontos.
        \item Performance estado da arte em Transformers (BERT, GPT).
    \end{itemize}
    &
    \begin{itemize}[noitemsep, topsep=0pt, partopsep=0pt, leftmargin=*]
        \item Computacionalmente mais custosa que ReLU.
    \end{itemize}
    \\
    
\end{longtable}
\chapter{Comparativo das Funções de Perda}
\label{cap:comparativo-perda}

Este capítulo detalha as principais funções de perda utilizadas em tarefas de regressão e classificação, apresentando suas formulações matemáticas, principais vantagens e considerações de uso.

% ===================================================================
% Funções de Perda para Regressão
% ===================================================================
\section{Funções de Perda para Regressão}

\subsection{Erro Quadrático Médio (MSE)}

\textit{Uma das perdas mais comuns para regressão, que mede a média dos erros quadrados, penalizando fortemente previsões distantes do valor real.}

\begin{equacaodestaque}{Erro Quadrático Médio (MSE) e sua Derivada}
    \Loss = \frac{1}{N} \sum_{j=1}^{N} (y_j - \hat{y}_j)^2 \\
    \frac{\partial \Loss}{\partial \hat{y}_j} = \frac{2}{N}(\hat{y}_j - y_j)
\end{equacaodestaque}

\subsubsection*{Vantagens / Quando Usar}
\begin{itemize}
    \item Penaliza erros grandes de forma quadrática, sendo ideal para cenários onde grandes desvios são indesejáveis.
    \item É uma função convexa, o que garante um único mínimo global e facilita a otimização.
\end{itemize}

\subsubsection*{Desvantagens / Considerações}
\begin{itemize}
    \item É muito sensível a \textit{outliers}, que podem dominar o gradiente e prejudicar o treinamento.
    \item A unidade da perda (ex: metros quadrados) é diferente da unidade original dos dados (ex: metros), o que dificulta a interpretação direta do erro.
\end{itemize}

\subsection{Erro Absoluto Médio (MAE)}

\textit{Mede a média dos erros absolutos, sendo menos sensível a outliers e mais intuitiva que o MSE, pois mantém a unidade original dos dados.}

\begin{equacaodestaque}{Erro Absoluto Médio (MAE) e sua Derivada}
    \Loss = \frac{1}{N} \sum_{j=1}^{N} |y_j - \hat{y}_j| \\
    \frac{\partial \Loss}{\partial \hat{y}_j} = \text{sgn}(\hat{y}_j - y_j)
\end{equacaodestaque}

\subsubsection*{Vantagens / Quando Usar}
\begin{itemize}
    \item É robusta a \textit{outliers} devido à sua penalidade linear para os erros.
    \item A perda é intuitiva, pois está na mesma escala da variável alvo.
    \item Recomendada quando \textit{outliers} são esperados e não devem influenciar excessivamente o modelo.
\end{itemize}

\subsubsection*{Desvantagens / Considerações}
\begin{itemize}
    \item Não é diferenciável no ponto zero, embora isso seja contornável na prática com subgradientes.
    \item O gradiente é constante, o que pode dificultar a convergência para o mínimo exato, exigindo taxas de aprendizado menores no final do treino.
\end{itemize}

\subsection{Huber Loss}

\textit{Uma perda híbrida que combina o melhor do MSE para erros pequenos e do MAE para erros grandes, oferecendo robustez a outliers sem sacrificar a estabilidade perto do mínimo.}

\begin{equacaodestaque}{Huber Loss e sua Derivada}
    \Loss_{\delta}(y, \hat{y}) = \begin{cases} \frac{1}{2}(y - \hat{y})^2 & \text{se } |y - \hat{y}| \le \delta \\ \delta |y - \hat{y}| - \frac{1}{2}\delta^2 & \text{caso contrário} \end{cases} \\
    \frac{\partial \Loss_{\delta}}{\partial \hat{y}} = \begin{cases} \hat{y} - y & \text{se } |y - \hat{y}| \le \delta \\ \delta \cdot \text{sgn}(\hat{y} - y) & \text{caso contrário} \end{cases}
\end{equacaodestaque}

\subsubsection*{Vantagens / Quando Usar}
\begin{itemize}
    \item Combina a boa convergência do MSE perto do mínimo com a robustez do MAE para erros grandes.
    \item É diferenciável em todos os pontos, exceto em $\pm\delta$.
\end{itemize}

\subsubsection*{Desvantagens / Considerações}
\begin{itemize}
    \item Requer o ajuste do hiperparâmetro $\delta$, que define o limiar entre o comportamento quadrático e linear.
\end{itemize}

\subsection{Log-Cosh Loss}

\textit{Uma função de perda suave que se comporta como o MSE para erros pequenos e como o MAE para erros grandes, sendo uma alternativa à Huber Loss que não requer hiperparâmetros.}

\begin{equacaodestaque}{Log-Cosh Loss e sua Derivada}
    \Loss = \sum_{j=1}^{N} \log(\cosh(y_j - \hat{y}_j)) \\
    \frac{\partial \Loss}{\partial \hat{y}_j} = \tanh(\hat{y}_j - y_j)
\end{equacaodestaque}

\subsubsection*{Vantagens / Quando Usar}
\begin{itemize}
    \item É duplamente diferenciável em todos os pontos, o que a torna suave e bem-comportada para otimizadores baseados em gradiente.
    \item Não requer o ajuste de hiperparâmetros como a Huber Loss.
\end{itemize}

\subsubsection*{Desvantagens / Considerações}
\begin{itemize}
    \item É computacionalmente mais custosa que o MSE e o MAE devido às funções $\log$ e $\cosh$.
\end{itemize}

\subsection{Quantile Loss (Pinball Loss)}

\textit{Utilizada para prever um quantil específico (como a mediana ou o 90º percentil) em vez da média, sendo útil para estimar intervalos de incerteza.}

\begin{equacaodestaque}{Quantile Loss e sua Derivada}
    \Loss_{\tau}(y, \hat{y}) = \begin{cases} \tau (y - \hat{y}) & \text{se } y \ge \hat{y} \\ (1-\tau)(\hat{y}-y) & \text{caso contrário} \end{cases} \\
    \frac{\partial \Loss_{\tau}}{\partial \hat{y}} = \begin{cases} -(1-\tau) & \text{se } \hat{y}>y \\ -\tau & \text{se } \hat{y}<y \end{cases}
\end{equacaodestaque}

\subsubsection*{Vantagens / Quando Usar}
\begin{itemize}
    \item Permite a criação de modelos que preveem diferentes quantis, fornecendo uma visão mais completa da distribuição da variável alvo.
    \item Muito útil em finanças, meteorologia e análise de risco para estimar intervalos de confiança.
\end{itemize}

\subsubsection*{Desvantagens / Considerações}
\begin{itemize}
    \item Requer a definição do quantil $\tau$ como um hiperparâmetro.
    \item Não é diferenciável em zero, assim como o MAE.
\end{itemize}

% ===================================================================
% Funções de Perda para Classificação
% ===================================================================
\section{Funções de Perda para Classificação}

\subsection{Entropia Cruzada Binária (BCE)}

\textit{A função de perda padrão para problemas de classificação binária, que mede a "distância" entre a distribuição de probabilidade prevista e a distribuição real (0 ou 1).}

\begin{equacaodestaque}{Entropia Cruzada Binária (BCE) e sua Derivada}
    \Loss = -[y \log(\hat{y}) + (1-y)\log(1-\hat{y})] \\
    \frac{\partial \Loss}{\partial z} = \hat{y} - y \quad \text{(com Sigmoide na saída)}
\end{equacaodestaque}

\subsubsection*{Vantagens / Quando Usar}
\begin{itemize}
    \item É a escolha padrão e mais eficaz para tarefas de classificação binária.
    \item Otimiza diretamente a log-verossimilhança do modelo, resultando em previsões probabilísticas.
    \item Penaliza fortemente previsões que estão confiantes e erradas.
\end{itemize}

\subsubsection*{Desvantagens / Considerações}
\begin{itemize}
    \item Pode levar a modelos enviesados em cenários com classes muito desbalanceadas.
    \item Exige que a saída do modelo seja uma probabilidade no intervalo (0, 1), geralmente obtida com uma função Sigmoide.
\end{itemize}

\subsection{Hinge Loss}

\textit{Projetada para treinamento de classificadores de máxima margem, como as Support Vector Machines (SVMs), penalizando apenas previsões incorretas ou corretas mas com pouca confiança.}

\begin{equacaodestaque}{Hinge Loss e sua Derivada}
    \Loss = \max(0, 1 - y \cdot \hat{y}) \quad (y \in \{-1, 1\}) \\
    \frac{\partial \Loss}{\partial \hat{y}} = \begin{cases} -y & \text{se } y \cdot \hat{y} < 1 \\ 0 & \text{caso contrário} \end{cases}
\end{equacaodestaque}

\subsubsection*{Vantagens / Quando Usar}
\begin{itemize}
    \item Otimiza explicitamente para maximizar a margem de separação entre as classes.
    \item Não penaliza exemplos que já estão classificados corretamente e fora da margem, focando o aprendizado nos pontos difíceis.
\end{itemize}

\subsubsection*{Desvantagens / Considerações}
\begin{itemize}
    \item A saída do modelo não é uma probabilidade, mas sim uma pontuação de decisão.
    \item Pode ser mais sensível a \textit{outliers} do que perdas baseadas em entropia.
\end{itemize}

\subsection{Entropia Cruzada Categórica (CCE)}

\textit{A extensão da BCE para problemas de classificação multi-classe, comparando a distribuição de probabilidade prevista com o rótulo real em formato one-hot.}

\begin{equacaodestaque}{Entropia Cruzada Categórica (CCE) e sua Derivada}
    \Loss = - \sum_{c=1}^{C} y_{c} \log(\hat{y}_{c}) \\
    \frac{\partial \Loss}{\partial z_i} = \hat{y}_i - y_i \quad \text{(com Softmax na saída)}
\end{equacaodestaque}

\subsubsection*{Vantagens / Quando Usar}
\begin{itemize}
    \item É a função de perda padrão e mais eficaz para classificação multi-classe.
    \item Requer que os rótulos de destino estejam em formato \textit{one-hot}.
\end{itemize}

\subsubsection*{Desvantagens / Considerações}
\begin{itemize}
    \item Assume que as classes são mutuamente exclusivas (cada amostra pertence a apenas uma classe).
    \item É sensível ao desbalanceamento de classes, podendo favorecer a classe majoritária.
\end{itemize}

\subsection{Focal Loss}

\textit{Uma modificação da Entropia Cruzada que reduz a perda atribuída a exemplos fáceis e bem classificados, forçando o modelo a focar em exemplos difíceis e mal classificados.}

\begin{equacaodestaque}{Focal Loss}
    \Loss = -(1 - p_t)^\gamma \log(p_t) \\
    \text{onde } p_t \text{ é a probabilidade da classe correta}
\end{equacaodestaque}

\subsubsection*{Vantagens / Quando Usar}
\begin{itemize}
    \item Reduz o peso de exemplos fáceis, permitindo que o modelo se concentre nos erros mais significativos.
    \item É extremamente eficaz para treinar modelos em datasets com grande desbalanceamento de classes, como na detecção de objetos.
\end{itemize}

\subsubsection*{Desvantagens / Considerações}
\begin{itemize}
    \item Adiciona o hiperparâmetro de foco $\gamma$, que precisa ser ajustado para balancear o peso entre exemplos fáceis e difíceis.
\end{itemize}
\chapter{Comparativo das Métricas de Avaliação}
\label{cap:comparativo-metricas}

Após o treinamento de um modelo, é essencial avaliar seu desempenho de forma quantitativa. Este capítulo apresenta as métricas de avaliação mais comuns para tarefas de classificação e regressão, detalhando suas fórmulas, interpretações e contextos de aplicação.

% ===================================================================
% Métricas para Classificação
% ===================================================================
\section{Métricas para Classificação}

\subsection{Acurácia (Accuracy)}

\textit{Uma métrica direta que mede a proporção de previsões corretas sobre o total de previsões, oferecendo uma visão geral do desempenho do modelo.}

\begin{equacaodestaque}{Acurácia}
    \text{Acurácia} = \frac{VP + VN}{VP + VN + FP + FN}
\end{equacaodestaque}

\subsubsection*{Interpretação}
\begin{itemize}
    \item Representa o percentual de predições corretas (Verdadeiros Positivos + Verdadeiros Negativos) em relação ao total de amostras.
    \item Fornece uma medida geral e intuitiva da performance do classificador.
\end{itemize}

\subsubsection*{Quando Usar / Considerações}
\begin{itemize}
    \item É mais útil em cenários com classes bem balanceadas.
    \item Pode ser uma métrica enganosa em datasets com classes desbalanceadas. Por exemplo, se 95\% das amostras são da classe A, um modelo que sempre prevê A terá 95\% de acurácia, mas será inútil.
\end{itemize}

\subsection{Precisão (Precision)}

\textit{Avalia a exatidão das previsões positivas, respondendo à pergunta: "De todas as vezes que o modelo previu a classe positiva, quantas estavam corretas?".}

\begin{equacaodestaque}{Precisão}
    \text{Precisão} = \frac{VP}{VP + FP}
\end{equacaodestaque}

\subsubsection*{Interpretação}
\begin{itemize}
    \item Mede a proporção de Verdadeiros Positivos entre todas as predições que o modelo classificou como positivas.
    \item Indica a "qualidade" ou "confiabilidade" das predições positivas.
\end{itemize}

\subsubsection*{Quando Usar / Considerações}
\begin{itemize}
    \item É crucial quando o custo de um Falso Positivo (FP) é alto. Por exemplo, em um filtro de spam (marcar um e-mail importante como spam) ou em um diagnóstico médico (diagnosticar uma pessoa saudável com uma doença).
\end{itemize}

\subsection{Revocação (Recall / Sensibilidade)}

\textit{Mede a capacidade do modelo de encontrar todas as amostras positivas relevantes, respondendo à pergunta: "De todos os exemplos realmente positivos, quantos o modelo conseguiu identificar?".}

\begin{equacaodestaque}{Revocação}
    \text{Revocação} = \frac{VP}{VP + FN}
\end{equacaodestaque}

\subsubsection*{Interpretação}
\begin{itemize}
    \item Mede a proporção de Verdadeiros Positivos que foram corretamente identificados pelo modelo.
    \item Indica a "completude" ou a "abrangência" do classificador em relação à classe positiva.
\end{itemize}

\subsubsection*{Quando Usar / Considerações}
\begin{itemize}
    \item É crucial quando o custo de um Falso Negativo (FN) é alto. Por exemplo, na detecção de fraudes (não identificar uma transação fraudulenta) ou no diagnóstico de uma doença grave (não diagnosticar um paciente doente).
\end{itemize}

\subsection{F1-Score}

\textit{A média harmônica entre Precisão e Revocação, fornecendo uma única pontuação que equilibra o trade-off entre as duas métricas.}

\begin{equacaodestaque}{F1-Score}
    \text{F1-Score} = 2 \times \frac{\text{Precisão} \times \text{Revocação}}{\text{Precisão} + \text{Revocação}}
\end{equacaodestaque}

\subsubsection*{Interpretação}
\begin{itemize}
    \item É a média harmônica de Precisão e Revocação, o que significa que penaliza valores extremos de uma das métricas.
    \item Fornece uma única métrica que busca um balanço entre a qualidade (Precisão) e a completude (Revocação) das predições positivas.
\end{itemize}

\subsubsection*{Quando Usar / Considerações}
\begin{itemize}
    \item É especialmente útil em cenários com classes desbalanceadas, onde a Acurácia pode ser enganosa e é necessário um bom equilíbrio entre Precisão e Revocação.
\end{itemize}

\subsection{AUC-ROC}

\textit{Mede a capacidade geral de um modelo de distinguir entre as classes positiva e negativa, independentemente do limiar de classificação escolhido.}

\begin{equacaodestaque}{Curva ROC}
    \text{A curva ROC é plotada com a Taxa de Verdadeiros Positivos} \\
    \text{(Revocação) no eixo Y e a Taxa de Falsos Positivos no eixo X.} \\
    \text{FPR} = \frac{FP}{FP+VN}
\end{equacaodestaque}

\subsubsection*{Interpretação}
\begin{itemize}
    \item A AUC (Área Sob a Curva) representa a probabilidade de que o modelo classifique uma amostra positiva aleatória com uma pontuação maior do que uma amostra negativa aleatória.
    \item AUC = 1.0 indica um classificador perfeito.
    \item AUC = 0.5 indica um desempenho equivalente a um classificador aleatório.
\end{itemize}

\subsubsection*{Quando Usar / Considerações}
\begin{itemize}
    \item Para avaliar e comparar o desempenho geral de modelos de forma agnóstica ao limiar de decisão.
    \item É uma boa métrica agregada para problemas de classificação binária.
\end{itemize}

% ===================================================================
% Métricas para Regressão
% ===================================================================
\section{Métricas para Regressão}

\subsection{RMSE (Root Mean Square Error)}

\textit{Representa o desvio padrão dos erros de predição (resíduos), medindo a magnitude média dos erros na mesma unidade da variável alvo.}

\begin{equacaodestaque}{Raiz do Erro Quadrático Médio (RMSE)}
    \text{RMSE} = \sqrt{\frac{1}{N} \sum_{i=1}^{N} (y_i - \hat{y}_i)^2}
\end{equacaodestaque}

\subsubsection*{Interpretação}
\begin{itemize}
    \item É a raiz quadrada do MSE. Um RMSE de 10, por exemplo, significa que, em média, as previsões do modelo estão a 10 unidades de distância dos valores reais.
\end{itemize}

\subsubsection*{Quando Usar / Considerações}
\begin{itemize}
    \item Quando se deseja que o erro seja expresso na mesma unidade da variável alvo para facilitar a interpretação.
    \item Assim como o MSE, penaliza erros maiores mais fortemente que o MAE devido ao termo quadrático.
\end{itemize}

\subsection{R² (Coeficiente de Determinação)}

\textit{Indica a proporção da variância na variável alvo que é explicada pelo modelo, fornecendo uma medida da qualidade do ajuste.}

\begin{equacaodestaque}{Coeficiente de Determinação (R²)}
    R^2 = 1 - \frac{\sum_{i=1}^{N}(y_i - \hat{y}_i)^2}{\sum_{i=1}^{N}(y_i - \bar{y})^2}
\end{equacaodestaque}

\subsubsection*{Interpretação}
\begin{itemize}
    \item Um R² de 0.85 significa que 85\% da variabilidade da variável alvo é explicada pelas variáveis preditoras do modelo.
    \item Seus valores variam de $-\infty$ a 1. Quanto mais próximo de 1, melhor o modelo se ajusta aos dados. Um valor negativo indica que o modelo é pior que um modelo ingênuo que sempre prevê a média.
\end{itemize}

\subsubsection*{Quando Usar / Considerações}
\begin{itemize}
    \item Para entender o quão bem as variáveis de entrada explicam a variação da variável de saída.
    \item Cuidado: o valor do R² tende a aumentar à medida que mais variáveis são adicionadas ao modelo, mesmo que elas não sejam úteis. Nesses casos, o R² ajustado é uma métrica mais apropriada.
\end{itemize}


% \include{apendices/ap_B_guia_setup}


% --- ELEMENTOS PÓS-TEXTUAIS ---
% O comando \backmatter é usado para as seções finais do livro.
\backmatter

% Gera a lista de Referências a partir do arquivo 'bibliografia.bib',
% formatada no estilo ABNT pelo biblatex.
\printbibliography[title={Referências}]

% --- IMPRIMIR O GLOSSÁRIO ---
\cleardoublepage % Começar em página nova
\phantomsection % Ajuda links
\printglossaries % Este comando imprime TODAS as listas de glossário definidas
% Não precisa de \addcontentsline se usou a opção [toc] no \usepackage

\printindex


\end{document}
% ===================================================================
% FIM DO DOCUMENTO
% ===================================================================