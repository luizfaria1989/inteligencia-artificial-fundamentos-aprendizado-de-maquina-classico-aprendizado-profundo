% ===================================================================
% ARQUIVO MESTRE DO LIVRO
% Autor: Seu Nome
% Projeto: Um Mergulho Profundo no Aprendizado de Máquina
% ===================================================================

% --- CLASSE DO DOCUMENTO ---
% Usamos a classe 'abntex2' para seguir as normas da ABNT.
% As opções configuram o formato para um livro acadêmico padrão.
\documentclass[
    12pt,            % Tamanho da fonte do corpo do texto
    a4paper,         % Tamanho do papel
    book,            % Formato de livro
    openright,       % Força capítulos a começarem em páginas ímpares (da direita)
    twoside,         % Layout para impressão frente e verso (margens diferentes)
    brazil,          % Configurações para o idioma português do Brasil
    citacao=authoryear
]{abntex2}


% --- PREÂMBULO ---
% Importa todas as configurações, pacotes e comandos customizados
% do arquivo 'preambulo.tex'. Isso mantém este arquivo principal limpo
% e focado apenas na estrutura do conteúdo do livro.
% ===================================================================
% ARQUIVO DE PREÂMBULO
% Contém todas as configurações de pacotes e layout para o livro.
% ===================================================================


% --- CONFIGURAÇÕES ESSENCIAIS DO DOCUMENTO ---

% Codificacao da fonte. Essencial para a correta exibição de caracteres acentuados no PDF.
\usepackage[T1]{fontenc}

% Codificacao de entrada do arquivo .tex. Permite que você escreva com acentos diretamente.
\usepackage[utf8]{inputenc}

% Suporte ao idioma Português do Brasil.
% A classe abntex2 já cuida disso, mas é uma boa prática garantir.
% A opção 'brazil' ajusta hifenizacao, nomes de seções ("Sumário"), etc.
\usepackage[brazil]{babel}

% Para usar a fonte Palatino
\usepackage{mathpazo}

% --- PACOTES PARA CONTEÚDO TÉCNICO E MATEMÁTICO ---

% Pacotes padrao e essenciais para matemática avançada.
\usepackage{amsmath, amsfonts, amssymb}

% Pacote para inclusão de imagens.
\usepackage{graphicx}
% Define um ou mais diretórios padrão para suas imagens.
% Isso evita que você precise digitar "imagens/" toda vez.
\graphicspath{ {./imagens/} }

% Pacote para formatar e incluir snippets de código-fonte.
\usepackage{listings}
\usepackage{xcolor} % Necessário para definir cores no listings

% Define um estilo customizado para código Python, para ficar mais bonito.
\lstdefinestyle{PythonStyle}{
    language=Python,
    backgroundcolor=\color{gray!10},   % Cor de fundo leve
    basicstyle=\ttfamily\small,       % Fonte monoespaçada e pequena
    keywordstyle=\color{blue},        % Palavras-chave em azul
    commentstyle=\color{green!50!black},% Comentários em verde escuro
    stringstyle=\color{purple},       % Strings em roxo
    breaklines=true,                  % Quebra de linhas longas
    showstringspaces=false,           % Não mostra espaços em strings com um símbolo
    frame=single,                     % Adiciona uma moldura
    rulecolor=\color{black!30},       % Cor da moldura
    numbers=left,                     % Numeração de linhas à esquerda
    numberstyle=\tiny\color{gray},    % Estilo dos números de linha
}


% --- REFERÊNCIAS, CITAÇÕES E LINKS (PADRÃO ABNT) ---

% O sistema moderno e mais poderoso para bibliografia em LaTeX, já configurado para ABNT.
\usepackage[
    backend=biber,     % Ferramenta que processa a bibliografia (mais moderna que bibtex)
    style=abnt,        % Estilo de citação e referência da ABNT
    ittitles,          % Títulos de artigos, livros, etc., em itálico
    backref=true,      % Nas referências, mostra em quais páginas a citação aparece
]{biblatex}

\renewcommand*{\mkbibnamefamily}[1]{\MakeUppercase{#1}}
\renewcommand*{\mkbibnamelast}[1]{\MakeUppercase{#1}}

% Aponta para o arquivo que contém seu banco de dados de referências.
\addbibresource{bibliografia.bib}

% Melhora a formatação de URLs nas referências.
\usepackage{url}
\urlstyle{same} % Usa a mesma fonte do texto para as URLs.

% --- LAYOUT E ESTRUTURA ---

% Pacote para ajustar as margens e geometria da página.
% A classe abntex2 já define margens padrão ABNT, use para sobrescrever se necessário.
\usepackage{geometry}
\geometry{
    a4paper,
    left=3cm,
    right=2cm,
    top=3cm,
    bottom=2cm
}

% --- CONFIGURAÇÃO DO HYPERREF ---
% A classe abntex2 já carrega o pacote hyperref.
% Apenas usamos \hypersetup para definir as opções que queremos.

\hypersetup{
    pdftitle={O Título do Seu Livro},
    pdfauthor={Seu Nome},
    pdfsubject={Assunto do Livro},
    pdfkeywords={Palavra-chave1, Palavra-chave2},
    colorlinks=true,                % Habilita links coloridos
    linkcolor=blue,                 % Cor dos links internos (sumário, etc.)
    citecolor=red,                  % Cor das citações
    urlcolor=cyan,                  % Cor das URLs
    hidelinks=false,                % Se 'true', esconde as caixas e cores (bom para impressão)
    pdfstartview={FitH},            % Abre o PDF ajustado à largura da página
    bookmarksopen=true              % Abre o painel de bookmarks (índices)
}

% Pacote para customização avançada de cabeçalhos e rodapés.
% Descomente as linhas abaixo apenas se precisar de um layout diferente
% do padrão oferecido pelo abntex2.
%
% \usepackage{fancyhdr}
% \pagestyle{fancy}
% \fancyhf{} % Limpa todos os campos
% \fancyhead[LE,RO]{\nouppercase{\leftmark}} % Nome do capítulo no cabeçalho
% \fancyfoot[LE,RO]{\thepage} % Número da página no rodapé


% --- COMANDOS CUSTOMIZADOS (OPCIONAL) ---
% Aqui você pode definir seus próprios comandos para agilizar a escrita.
% Exemplo:
% \newcommand{\keras}{\texttt{Keras}}

\usepackage{tikz}
\usetikzlibrary{patterns, positioning, fit, shapes.geometric, arrows.meta, fpu, decorations.pathmorphing, shadows, shapes.symbols}
\usepackage{pgfplots}
\pgfplotsset{compat=1.18}

% --- Definindo as cores (hues) ---
\definecolor{darkblue}{rgb}{0.16, 0.32, 0.75}
\definecolor{darkorange}{rgb}{0.9, 0.35, 0.0}
\definecolor{pointred}{rgb}{0.85, 0.2, 0.2}

\usepackage[most]{tcolorbox}
\tcbuselibrary{skins} % Necessário para o estilo enhanced

\newtcolorbox{definicaomoderna}{
    enhanced,
    frame hidden, % Remove a moldura da caixa
    borderline west={2pt}{0pt}{gray!80}, % Cria uma linha de 2pt à esquerda (west)
    colback=gray!5, % Um fundo cinza muito sutil para diferenciar
    fonttitle=\bfseries,
    coltitle=black,
    sharp corners, % Cantos retos
    boxsep=5pt,
    left=10pt, % Espaço interno à esquerda
    attach boxed title to top left={yshift=-2mm, xshift=5mm}, % Posição do título
    boxed title style={
        frame hidden,
        colback=gray!5 % Fundo do título igual ao da caixa
    },
}

% --- Comando especial para equação destacada COM NOME ---
\newtcolorbox{equacaodestaque}[1]{ % O [1] indica que o comando aceita um argumento
    enhanced, ams equation,
    frame hidden,
    borderline west={2pt}{0pt}{gray!80},
    colback=gray!5,
    sharp corners,
    % --- Opções para o título ---
    fonttitle=\bfseries,
    coltitle=black,
    attach boxed title to top left={yshift=-2mm, xshift=5mm}, % Posiciona o título
    boxed title style={
        frame hidden,
        colback=gray!5 % Faz o fundo do título ser igual ao da caixa
    },
    title={#1} % Usa o argumento como título da caixa
}

\usepackage{xcolor} % Para definir cores customizadas
\usepackage{listings} % Para formatar o código
\usepackage[most]{tcolorbox} % Para criar a caixa estilizada
\usepackage{witharrows}

% --- 1. Definição das Cores para o Código ---
\definecolor{codegray}{rgb}{0.5,0.5,0.5}
\definecolor{codepurple}{rgb}{0.58,0,0.82}
\definecolor{codeblue}{rgb}{0,0,0.8}
\definecolor{codered}{rgb}{0.8,0,0}

% --- 2. Configuração Global do Pacote listings ---
\lstset{
    language=Python,
    backgroundcolor=\color{white},   
    commentstyle=\color{codegray}\itshape,
    keywordstyle=\color{codeblue}\bfseries,
    numberstyle=\tiny\color{codegray},
    stringstyle=\color{codered},
    basicstyle=\ttfamily\small,
    breakatwhitespace=false,         
    breaklines=true,                 
    captionpos=b,                    
    keepspaces=true,                 
    numbers=left,                    
    numbersep=5pt,                  
    showspaces=false,                
    showstringspaces=false,
    showtabs=false,                  
    tabsize=4
}

% --- Criação do Ambiente de Listagem de Código Estilizado (Fundo Branco) ---
\newtcblisting{codelisting}[2]{
  listing only,
  enhanced,
  colback=white,  % Fundo branco, igual ao da página
  colframe=gray!80,
  frame hidden,
  borderline west={2pt}{0pt}{gray!80},
  sharp corners,
  top=5pt,
  bottom=5pt,
  % --- Título e numeração (Listagem X: Título) ---
  fonttitle=\bfseries,
  coltitle=black,
  title={Bloco de Código~\thetcbcounter: #1},
  label={lst:#2},
  % --- Opções passadas para o pacote listings ---
  listing options={
    language=Python,
    numbers=left,
  }
}

\usepackage{algorithm} % Para o ambiente algorithm
\usepackage{algpseudocode} % Para os comandos do pseudocódigo

\algrenewcommand\algorithmicrequire{\textbf{Requer:}}
\algrenewcommand\algorithmicreturn{\textbf{Retorne:}}

\usepackage{threeparttable}

\usepackage{longtable}     % Para tabelas que ocupam várias páginas
\usepackage{booktabs}      % Para as linhas \toprule, \midrule, \bottomrule
\usepackage{amsmath}       % Para os ambientes matemáticos como 'cases'
\usepackage[labelfont=bf, singlelinecheck=false, labelsep=endash]{caption} % Para formatar o título

\usepackage{makecell}

\numberwithin{figure}{chapter}
\numberwithin{table}{chapter}
\numberwithin{equation}{chapter}

\usepackage{nomencl}
\usepackage{etoolbox} % Para customizações avançadas

% --- Comandos Customizados para Funções ---
\newcommand{\fativ}[1]{\ensuremath{\mathcal{A}_{#1}}}
\newcommand{\fperd}[1]{\ensuremath{\mathcal{L}_{#1}}}

\usepackage{ragged2e}   % Para um melhor alinhamento do texto justificado

% ===================================================================
% FIM DO PREÂMBULO
% ===================================================================


% --- INFORMAÇÕES DO DOCUMENTO (para a folha de rosto) ---
\title{INTELIGÊNCIA ARTIFICIAL \\ 
    \large Fundamentos Teóricos, Técnicas de Aprendizado de Máquina Clássico e Aprendizado Profundo}
\author{Luiz Guilherme Morais da Costa Faria}
\date{\today} % Usa a data atual no momento da compilação

% Informações adicionais que o abntex2 usa (opcional)
\instituicao{Universidade de Brasília}
\local{Brasília, DF}
\orientador{Nome do Orientador/Revisor (se aplicável)}


% ===================================================================
% INÍCIO DO DOCUMENTO
% ===================================================================
\begin{document}

\begin{acronym}
    % --- Siglas Gerais ---
    \acro{IA}{Inteligência Artificial}
    \acro{ML}{Aprendizado de Máquina (Machine Learning)} % Embora não explícito, é o tema central
    \acro{RNA}{Rede Neural Artificial} % Implícito no contexto de redes neurais
    \acro{DNN}{Rede Neural Profunda (Deep Neural Network)}
    \acro{GPU}{Unidade de Processamento Gráfico (Graphics Processing Unit)} % Mencionada no contexto de processamento

    % --- Otimizadores (Cap. 6 e Apêndice A) ---
    \acro{GD}{Gradiente Descendente (Gradient Descent)}
    \acro{SGD}{Gradiente Descendente Estocástico (Stochastic Gradient Descent)}
    \acro{NAG}{Gradiente Acelerado de Nesterov (Nesterov Accelerated Gradient)}
    \acro{AdaGrad}{Adaptive Gradient Algorithm}
    \acro{RMSProp}{Root Mean Square Propagation}
    \acro{Adam}{Adaptive Moment Estimation}
    \acro{AdaMax}{Adaptive Moment Estimation based on the infinity norm}
    \acro{Nadam}{Nesterov-accelerated Adaptive Moment Estimation}
    \acro{AdamW}{Adam with Decoupled Weight Decay}
    \acro{RAdam}{Rectified Adam}

    % --- Funções de Ativação (Cap. 7, 8, 9 e Apêndice B) ---
    \acro{ReLU}{Unidade Linear Retificada (Rectified Linear Unit)}
    \acro{LReLU}{Leaky Rectified Linear Unit}
    \acro{PReLU}{Parametric Rectified Linear Unit}
    \acro{RReLU}{Randomized Leaky Rectified Linear Unit}
    \acro{ELU}{Unidade Linear Exponencial (Exponential Linear Unit)}
    \acro{SELU}{Unidade Linear Exponencial Escalonada (Scaled Exponential Linear Unit)}
    \acro{GELU}{Unidade Linear de Erro Gaussiano (Gaussian Error Linear Unit)}
    \acro{SiLU}{Sigmoid Linear Unit}
    \acro{h-swish}{Hard-Swish}
    \acro{h-mish}{Hard-Mish}

    % --- Funções de Perda (Cap. 10, 11, 12 e Apêndice C) ---
    \acro{MSE}{Erro Quadrático Médio (Mean Squared Error)}
    \acro{MAE}{Erro Absoluto Médio (Mean Absolute Error)}
    \acro{MSLE}{Erro Quadrático Médio Logarítmico (Mean Squared Logarithmic Error)}
    \acro{MAPE}{Erro Percentual Absoluto Médio (Mean Absolute Percentage Error)}
    \acro{sMAPE}{Erro Percentual Absoluto Médio Simétrico (Symmetric Mean Absolute Percentage Error)}
    \acro{BCE}{Entropia Cruzada Binária (Binary Cross-Entropy)}
    \acro{WCE}{Entropia Cruzada Ponderada Binária (Binary Weighted Cross-Entropy)}
    \acro{CCE}{Entropia Cruzada Categórica (Categorical Cross-Entropy)}
    \acro{WCCE}{Weighted Categorical Cross-Entropy}
    \acro{KL}{Divergência Kullback-Leibler (Kullback-Leibler Divergence)}
    \acro{FL}{Focal Loss}

    % --- Métricas (Cap. 13 e Apêndice D) ---
    \acro{VP}{Verdadeiro Positivo (True Positive)}
    \acro{VN}{Verdadeiro Negativo (True Negative)}
    \acro{FP}{Falso Positivo (False Positive)}
    \acro{FN}{Falso Negativo (False Negative)}
    \acro{ROC}{Característica de Operação do Receptor (Receiver Operating Characteristic)}
    \acro{AUC}{Área Sob a Curva (Area Under the Curve)}
    \acro{RMSE}{Raiz do Erro Quadrático Médio (Root Mean Square Error)}
    \acro{R2}{Coeficiente de Determinação (Coefficient of Determination)}

    % --- Redes Neurais e Componentes (Parte V) ---
    \acro{MLP}{Perceptron Multicamadas (Multi-Layer Perceptron)}
    \acro{FFN}{Rede FeedForward (FeedForward Network)}
    \acro{DBN}{Rede de Crença Profunda (Deep Belief Network)}
    \acro{RBM}{Máquina de Boltzmann Restrita (Restricted Boltzmann Machine)}
    \acro{CNN}{Rede Neural Convolucional (Convolutional Neural Network)}
    \acro{FCN}{Rede Totalmente Convolucional (Fully Convolutional Network)}
    \acro{YOLO}{You Only Look Once}
    \acro{ResNet}{Rede Residual (Residual Network)}
    \acro{SENet}{Squeeze-and-Excitation Network}
    \acro{RNN}{Rede Neural Recorrente (Recurrent Neural Network)}
    \acro{LSTM}{Memória Longa de Curto Prazo (Long Short-Term Memory)}
    \acro{GRU}{Unidade Recorrente Gated (Gated Recurrent Unit)}
    \acro{ViT}{Vision Transformer}
    \acro{GAN}{Rede Adversária Generativa (Generative Adversarial Network)}
    \acro{MoE}{Mistura de Especialistas (Mixture of Experts)}
    \acro{GNN}{Rede Neural de Grafos (Graph Neural Network)}

    % --- Outras ---
    \acro{PCA}{Análise de Componentes Principais (Principal Component Analysis)}
    \acro{SVD}{Decomposição em Valores Singulares (Singular Value Decomposition)}
    \acro{t-SNE}{t-Distributed Stochastic Neighbor Embedding}
    \acro{UMAP}{Uniform Manifold Approximation and Projection}
    \acro{DBSCAN}{Density-Based Spatial Clustering of Applications with Noise}
    \acro{SVM}{Máquina de Vetores de Suporte (Support Vector Machine)}
    \acro{ILSVRC}{Desafio de Reconhecimento Visual em Larga Escala ImageNet (ImageNet Large Scale Visual Recognition Challenge)}
    \acro{SIFT}{Scale-Invariant Feature Transform}
    \acro{FVs}{Fisher Vectors}
    \acro{SNN}{Rede Neural Auto-Normalizadora (Self-Normalizing Neural Network)}
    \acro{WER}{Taxa de Erro de Palavra (Word Error Rate)}
    \acro{SWBD}{Switchboard (dataset)}
    \acro{CH}{CallHome (dataset)}
    \acro{EV}{Conjunto de Avaliação (Evaluation set)}
    \acro{LAD}{Mínimo Desvio Absoluto (Least Absolute Deviation)}
    \acro{EW-RSM}{Erro Quadrático Médio Exponencialmente Ponderado (Exponentially Weighted Mean Squared Error)}
    \acro{fMAE}{Erro Absoluto Médio Ponderado pela Frequência (Frequency-weighted Mean Absolute Error)}
    \acro{RDA}{Regularized Dual Averaging}
    \acro{FB}{Forward-Backward splitting} 
    \acro{PA}{Passive-Aggressive algorithms} 
    \acro{AROW}{Adaptive Regularization of Weight Vectors} 
\end{acronym}

\newglossaryentry{labelUnica}{
    name={Nome do Termo},
    description={A explicação clara e concisa do que diabos esse termo significa. Pode ser longa, pode ter \textit{itálico}, o que você precisar.},
    symbol={\ensuremath{\eta}} % Opcional: Se tiver um símbolo associado, use \ensuremath{} para modo matemático
}

% Exemplo Concreto do seu livro:
\newglossaryentry{retropapagacao}{
    name={Retropropagação},
    description={Um dos principais algoritmos para o treinamento de redes neurais. Permite o aprendizado atráves do ajuste sucessivo dos parâmetros da rede com auxílio do cálculo do gradiente da perda propagado das camadas finais para as camadas iniciais.} \index{Glossário!Retropropagação} % Indexar no índice remissivo também? Boa ideia!
}

\newglossaryentry{gradiente-descendente}{ 
    name={Gradiente descendente},
    description={Método de otimização iterativo que tem como objetivo encontrar pontos de mínimo de uma função (geralmente funções de perda) dando pequenos "passos" na direção contrária do vetor gradiente.} \index{Glossário!Gradiente Descendente}
}

\newglossaryentry{sigmoide}{
    name={Sigmoide logística},
    description={Função de ativação do tipo sigmoidal que foi comumente empregada em redes neurais antes da popularização das funções retificadoras. Pode ser utilizada na camada de saída de um modelo para que a sua saída fique limitada em um intervalo [0,1], sendo útil para problemas de classificação binárias.} \index{Glossário!Sigmoide Logística}
}

\newglossaryentry{tangente-hiperbolica}{
    name={Tangente hiperbólica},
    description={Função de ativação do tipo sigmoidal que, assim como a sigmoide, foi amplamente utilizada em redes neurais antes do surgimento das funções retificadoras. Ela é uma função limitada em um intervalo de [-1,1], que tem como característica empurrar os valores de sua entrada para esses extremos.} \index{Glossário!Tangente Hiperbólica}
}

% --- ELEMENTOS PRÉ-TEXTUAIS ---
% O comando \frontmatter inicia a contagem de páginas em algarismos romanos (i, ii, ...)
% e desativa a numeração de capítulos para os elementos iniciais.
\frontmatter

% Gera a capa e a folha de rosto com base nas informações acima,
% seguindo o padrão ABNT.
\imprimircapa
\imprimirfolhaderosto

% Páginas opcionais como Dedicatória, Agradecimentos, Epígrafe...
% \begin{dedicatoria}
%    \vspace*{\fill} % Centraliza verticalmente
%    \noindent
%    \textit{Para ...}
%    \vspace*{\fill}
% \end{dedicatoria}

% Gera o Sumário automaticamente com base nos comandos \part, \chapter, \section, etc.
\tableofcontents

% --- IMPRIMIR A LISTA DE SIGLAS ---
\cleardoublepage
\phantomsection
\chapter*{Lista de Siglas}
\addcontentsline{toc}{chapter}{Lista de Siglas}

% --- ELEMENTOS TEXTUAIS (O CONTEÚDO PRINCIPAL DO LIVRO) ---
% O comando \mainmatter reinicia a contagem de páginas em algarismos arábicos (1, 2, ...)
% e reativa a numeração de capítulos.
\mainmatter

% =======================================================
% PARTE I: HISTÓRIA DA IA E DO COMPUTADOR
% =======================================================
\part{História da IA e do Computador}

% ===================================================================
% Arquivo: capitulos/parte-01-historia/cap-01-historia-do-computador.tex
% ===================================================================

\chapter{Uma Breve História do Computador}
\label{cap:historia-computador}

\section{A Necessidade de Contar ao Longo das Eras}

\subsection{Ábaco}

\subsection{Régua de Cálculo}

\subsection{Bastões de Napier}

\subsection{Pascalina}
% ===================================================================
% Arquivo: capitulos/parte-01-historia/cap-01-historia-da-ia.tex
% ===================================================================

\chapter{Uma Breve História da Inteligência Artificial}
\label{cap:historia-ia}

O texto do seu capítulo começa aqui...

% =======================================================
% PARTE II: Conceitos Matemáticos
% =======================================================
\part{Conceitos Matemáticos}

% ===================================================================
% Arquivo: capitulos/parte-II-matematica/cap-03-calculo.tex
% ===================================================================

\chapter{Cálculo para Aprendizado de Máquina}
\label{cap:calculo-ia}

O texto do seu capítulo começa aqui...
% ===================================================================
% Arquivo: capitulos/parte-II-matematica/cap-04-algebra-linear.tex
% ===================================================================

\chapter{Álgebra Linear para Aprendizado de Máquina}
\label{cap:algebra-linear-ia}

O texto do seu capítulo começa aqui...
% ===================================================================
% Arquivo: capitulos/parte-II-matematica/cap-05-probabilidade-e-estatistica.tex
% ===================================================================

\chapter{Probabilidade e Estatística para Aprendizado de Máquina}
\label{cap:probabilidade-e-estatistica-ia}

O texto do seu capítulo começa aqui...

% =======================================================
% PARTE III: Pilares das Redes Neurais
% =======================================================
\part{Pilares das Redes Neurais}

% ===================================================================
% Arquivo: capitulos/parte-III-pilares/cap-06-retropropagacao-e-gradiente.tex
% ===================================================================

\chapter{O Algoritmo da Repropropagação e Os Otimizadores Baseados em Gradiente}
\label{cap:retropropagacao-gradiente}

% ===================================================================
% Resumo do capítulo
% ===================================================================

% ===================================================================
% Método do Gradiente Descendente
% ===================================================================

\section{O Método do Gradiente Descendente}

\subsection{Exemplo Ilustrativo}

\subsection{O Método em Si}

\subsection{Implentação em Python}

% ===================================================================
% A Retropropagação
% ===================================================================

\section{A Retropropagação: Aprendendo com os Erros}

% ===================================================================
% Otimizadores Baseados em Gradiente
% ===================================================================

\section{Otimizadores Baseados em Gradiente}

\subsection{Método do Gradiente Estocástico}

\subsection{Método do Gradiente com Momentum}

\subsection{Nesterov}

\subsection{AdaGrad}

\subsection{RMSProp}

\subsection{Adam}

\subsection{Nadam}

% ===================================================================
% Método de Newton
% ===================================================================

\section{O Método de Newton: Indo Além do Gradiente}


% ===================================================================
% Arquivo: capitulos/parte-III-pilares/cap-07-sigmoidais.tex
% ===================================================================

\chapter{Funções de Ativação Sigmoidais}
\label{cap:ativacao-sigmoidais}

% ===================================================================
% Resumo do capítulo
% ===================================================================

% ===================================================================
% Teoremas da Aproximação Universal
% ===================================================================

\begin{flushright}
\textit{"Ué, cadê o gradiente que estava aqui?"} \\
--- Estagiário descobrindo o problema do desaparecimento de gradientes
\end{flushright}

Conhecendo como uma rede neural aprende, é possível agora entender outros pontos que são essenciais para o seu funcionamento, sendo um deles as funções de ativação. O principal objetivo de se adicionar uma função de ativação como a sigmoide após a camada densa de modelo está voltado a introdução da não-lineridade. Dessa forma, um modelo de rede neural torná-se mais capaz de aproximar uma gama de funções que antes não eram possíveis com funções de ativação lineares.

Esse capítulo se inicia justamente discutindo a importância de introduzir a não-linearidade para uma rede que está sendo criada, como justificativa são citados alguns dos diferentes teoremas da aproximação universal. O restante do captítulo é dedicado as funções sigmoidais, começando pela sigmoide, e como elas surgiram no cenário de aprendizado de máquina como uma forma de replicar o comportamento dos neurônios humanos em redes artificiais. 

Além disso, são discutidas outras funções, como a tangente hiperbólica (tanh) e a \textit{softsign}. Seguindo adiante, é explicada uma das principais desvantagens de se utilizar funções sigmoidais em uma rede: o problema do desaparecimento do gradiente. E como ele se tornou um empecilho para a criação de redes neurais mais profundas. No fim do capítulo, está uma tabela resumindo as principais características das funções de ativação que foram apresentadas.

\section{Teoremas da Aproximação Universal: Introduzindo a Não-Linearidade}

Pense que você tem uma função matemática, como $f(t) = 40t + 12$, a qual representa o deslocamento em quilômetros de um carro em uma cidade, onde $t$ é o tempo em horas. Caso você queira encontrar o valor de deslocamento quando o carro tiver andando por 3 horas, basta substituir a variável $t$ por 3 e resolver a expressão. Assim temos:

\[\begin{WithArrows}
    f(t) & = 40t + 12 \Arrow{Quando t = 3} \\
    f(3) & = 40\cdot 3 + 12 = 132 km
\end{WithArrows}\]

Esse cenário é o mais comum quando estamos estudando, contudo existe um segundo cenário que também é possível de acontecer. Isso ocorre quando temos um conjunto de pontos e, com base neles, queremos encontrar uma função que descreva o comportamento desses pontos.

Pense que temos os pontos dispostos no gráfico da Figura \ref{fig: pontos_addplot} e queremos encontrar uma reta que tente passar o mais próximo de cada um deles.

\begin{figure}[h!]
    \centering
    \begin{tikzpicture}
        \begin{axis}[
            title={Pontos},
            xlabel={$x$},
            ylabel={$\sigma(x)$},
            xmin=-1, xmax=5,
            ymin=-1, ymax=5,
            axis lines=middle,
            grid=major,
            enlarge x limits=0.1, 
            enlarge y limits=0.1,
        ]
        \addplot[
            only marks,                     
            mark=*,                       
            mark size=2pt,              
            nodes near coords,             
            point meta=explicit symbolic,   
            nodes near coords align={above right}, 
        ] table [meta=label] { 
            x y label
            1 1 $P_1$
            1.7 2 $P_2$
            3.2 3 $P_3$
            4.3 4 $P_4$ 
        };
        \end{axis}
    \end{tikzpicture}
    \caption{Conjunto de pontos dispostos no plano cartesiano.}
    \label{fig: pontos_addplot}
    \fonte{O autor (2025).}
\end{figure}

Existe uma técnica que permite que nos façamos isso, ela se chama regressão linear (a qual é um dos tópicos discutidos no Capítulo \ref{cap:regressao}), e com base nela, é possível dado um conjunto de pontos em um plano, traçar uma reta que se aproxime igualmente de cada um desses pontos. Existem diferentes técnicas de regressão linear, neste caso aplicaremos a dos mínimos quadrados e encontramos a expressão:

\[ y=0.8623x+0.3011 \]

Com base nessa função que encontramos, podemos desenhá-la junto ao gráfico dos pontos e vermos se ela é realmente uma boa aproximação, assim temos a Figura \ref{fig: regressao-linear}

\begin{figure}[h!]
    \centering
    \begin{tikzpicture}
        \begin{axis}[
            title={Pontos},
            xlabel={$x$},
            ylabel={$\sigma(x)$},
            xmin=-1, xmax=5,
            ymin=-1, ymax=5,
            axis lines=middle,
            grid=major,
            enlarge x limits=0.1, 
            enlarge y limits=0.1,
        ]
        \addplot[
            only marks,                     
            mark=*,                       
            mark size=2pt,              
            nodes near coords,             
            point meta=explicit symbolic,   
            nodes near coords align={above right}, 
        ] table [meta=label] { 
            x y label
            1 1 $P_1$
            1.7 2 $P_2$
            3.2 3 $P_3$
            4.3 4 $P_4$ 
        };
        \addplot[blue, thick, domain=-8:8, samples=100] {0.8623(x) + 0.3011};
        \end{axis}
    \end{tikzpicture}
    \caption{Conjunto de pontos dispostos no plano cartesiano.}
    \label{fig: regressao-linear}
    \fonte{O autor (2025).}
\end{figure}

Existem diversas aproximações além da regressão linear, se quisermos, podemos tentar aproximar esses pontos utilizando uma função quadrática, cúbica ou até mesmo exponencial.

Esse tema parece não ter uma conexão com esse capítulo de funções de ativação, mas na realidade, o que muitas das vezes é feito por uma rede neural é justamente esse trabalho de encontrar uma função que aproxima o comportamento de um conjunto de pontos. Só que neste caso, não teremos um conjunto de pontos, vamos ter várias informações em uma base de dados, como imagens de exames médicos ou informações sobre o valor de imóveis e queremos encontrar de alguma forma uma conexão entre esses dados.

Para isso, existe um conjunto de teoremas que servem justamente para provar que uma determinada rede neural criada será capaz de encontrar uma função que descreva o comportamento que você esteja estudando. Eles são os teoremas da aproximação universal.

No livro \textit{Deep Learning}, \textcite{DeepLearningBook} dedicam uma seção explicando esses teoremas. Segundo os autores, o teorema da aproximação universal, introduzido \textcite{Cybenko1989} para a comunidade científica no texto \textit{Approximation by Superpositions of a Sigmoidal Function}, afirma que uma rede \textit{feedforward} com uma camada de saída linear e no mínimo uma camada oculta com qualquer função que possui a propriedade de "esmagamento", como a sigmoide logística, é capaz de aproximar qualquer função mensurável de Borel de um espaço de dimensão finita para outro com qualquer quantidade de erro diferente de zero desejada desde que essa rede neural possua unidades ocultas suficientes.

Para entendermos esse teorema, devemos primeiro entender o conceito de mensurabilidade de Borel, segundo \textcite{DeepLearningBook} uma função contínua em um subconjunto fechado e limitado de $\mathbb{R}^N$ é mensurável por Borel. Assim esse tipo de função pode ser aproximada por uma rede neural. Além disso, os autores ressaltam que por mais que o teorema original tenha conseguido provar apenas para as funções que saturam tanto para termos muito negativos ou termos muito positivos, diversos outros autores como \textcite{Leshno1993}, no texto \textit{Multilayer feedforward networks with a nonpolynomial activation function can approximate any function}, foram capazes de provar que o teorema da aproximação universal pode funcionar para outras funções, no caso de Leshno, eles provaram para funções não polinomiais, como a \textit{Rectified Linear Unit} (\textit{ReLU}), a qual é o tema central do Capítulo \ref{cap:ativacao-retificadoras}.

Basicamente os teoremas da aproximação universal reforçam o uso de funções de ativação para permitir que as redes neurais resolvam os problemas propostos por meio da aproximação de uma função. Isso acontece porque essas funções introduzem a não-linearidade para a rede, como nos vimos na equação do neurônio de uma rede neural, uma rede neural é composta por diferentes camadas de neurônios que são capazes de pegar valores de entrada, multiplicar por um peso dado e somar com um viés, esse resultado passa então por uma função de ativação.

\[ x_j = (\sum_i y_i \cdot w_{ji}) + b_j \]

Se nos tivéssemos uma rede neural em que os neurônios não possuíssem uma função de ativação, ou, fosse uma função linear, mesmo juntando todas essas camadas de neurônios que trazem expressões lineares, a junção disso, ainda seria uma expressão linear. Mas quando introduzimos uma função não linear, como a sigmoide, o teorema da aproximação universal, nos garante que somos capazes de encontrar qualquer função que estivermos procurando, desde que ela seja mensurável de Borel.

\section{Propriedades das Funções de Ativação: Escolhendo Uma Boa Função de Ativação}

Visto um dos motivos de se utilizar funções de ativação em uma rede neural, cabe agora entender um pouco dessa classe de funções. Em \textit{A Survey on Activation Functions and their relation with Xavier and He Normal Initialization}, \textcite{PropriedadesFuncoesDeAtivacao} discute algumas das propriedades das funções de ativação, dizendo que é esperado que essas funções sejam: não-lineares, diferenciáveis, contínuas, limitadas, centradas em zero. Além disso, o autor discute também que é interessante considerar também o custo computacional dessas funções que estão sendo aplicadas em uma rede neural \parencite{PropriedadesFuncoesDeAtivacao}.

Dessa forma, é possível discutir elaborar pouco de cada uma dessas propriedades em seguida.

\textbf{Não-linearidade}

Como foi visto na seção anterior, o principal objetivo de uma função de ativação ser não linear é justamente relacionado ao fato delas permitirem que um modelo seja capaz de aproximar diversos tipos de funções desde que sejam mensuradas por Borel. Além disso, \textcite{PropriedadesFuncoesDeAtivacao} justifica que caso tenha uma rede como um \textit{Perceptron} com múltiplas camadas ocultadas pode ser facilmente comprimido em um \textit{percetron} de apenas uma camada. Isso acontece porque a várias transformações lineares (uma para cada camada) em cima de outras também lineares continuam formando uma transformação linear quando serem consideradas por inteiro. Com isso, o principal benefício de uma rede neural, que é a profundidade, acaba por não ter tanto efeito em casos em que são aplicadas funções de ativação lineares.

\textbf{Diferenciabilidade}

O principal motivo de considerar a diferenciabilidade ao se escolher uma função de ativação está intrisicamente ligado em como a retropropagação do erro. Como visto no Capítulo \ref{cap:retropropagacao-gradiente}, a retropropagação utiliza das derivadas parciais, do vetor gradiente e também da derivada da função de ativação para calcular o erro e com base neste, as atualizações dos pesos e vieses da rede. Perceba que escolher uma função que tenha uma derivada complexa ou não seja derivada em grande parte do seu domínio, acaba por atrasar o algoritmo da retropropagação, que levará mais tempo tendo que calcular a derivada da função de ativação ou terão que ser feitas adaptações na derivada da função original para adequá-la ao novo algoritmo.

\textbf{Continuidade}

A continuidade está ligada com a diferencibilidade. Pela definição de derivada, não é possível derivar uma função em pontos de descotinuidade. Dessa forma, é importante considerar como o a função é representada graficamente, garantindo que não haja "quinas" no desenho da função e que o gráfico pode ser desenhado "sem tirar o lápis do papel". Técnicas como essas ajudam a descobrir ao olhar o gráfico, se existem pontos de descontinuidade os quais podem afetar a derivada da função de ativação.

\textbf{Limitada}

Considerar que uma função de ativação seja limitada ao escolher uma função de ativação para construir uma rede neural está ligado com os problemas do gradiente. Neste caso, uma função que não é limitada pode sofrer do problema do gradiente explosivo, o qual é explicado com maiores detalhes no Capítulo \ref{cap:ativacao-retificadoras}. Nesse sentido, o problema do gradiente explosivo o gradiente de uma rede neural começa a crescer de forma acelerada afetando o aprendizado da rede e as atualizações de pesos e vieses.

\textbf{Centrada em zero}

Uma função que não é centrada em zero, que é sempre positiva ou sempre negativa, afeta como a saída ocorre \parencite{PropriedadesFuncoesDeAtivacao}. Como explica \textcite{PropriedadesFuncoesDeAtivacao}, um cenário em que isso acontece, significa que a saída está sendo sempre movida para os valores positivos ou negativos, e como resultado, o vetor de pesos precisa de uma quantidade maior de iterações para ser melhor ajustado. Dessa forma, ao escolher uma função que é centrada em zero, é possível garantir um treinamento com menos iterações mas ainda assim garantir uma boa convergência do modelo que está sendo treinado.

\textbf{Custo computacional}

Cabe considerar também o custo computacional de uma função de ativação. Funções mais simples, como a degrau unitário e a \textit{ReLU}, as quais fazem apenas comparações com o valor da entrada, apresentam um custo computacional bem menor quando comparadas com funções mais complexas, como aquelas que fazem uso de muitos exponenciais em sua fórmula. Com isso, ao utilizar uma função de ativação muito "cara" em termos computacionais, isso acaba por gerar um gargalo no tempo de treino do modelo em desenvolvimento.

Vale destacar também que muitas dessas propriedades não estão presentes em todas as funções de ativação. A sigmoide por exemplo é uma função que não está centrada em zero, possuindo apenas saídas positivas. Já a \textit{ReLU} apresenta um ponto de descontinuidade na origem, mas isso não atrapalha essas funções de ativação de serem utilizadas ao construir uma rede neural. As propriedades servem mais como um guia, destacando vantagens que se pode ter ao utilizar determinada função de ativação.

Considerando isso, é possível finalmente entrar no tópico principal desse capítulo: as funções sigmoidais. Para isso, será explicado antes um exemplo ilustrativo destacando como essas funções atuam ao receberem suas entradas.

\section{Exemplo Ilustrativo: Empurrando para Extremos}

Imagine que você está trabalhando para uma empresa na área de marketing e precisa analisar como foi a recepção do público para um novo produto anunciado. Para isso, você ficou responsável por classificar os comentários do público sobre esse produto. Você precisa colocar eles em duas categorias: avaliação positiva ou negativa. Então você precisa ler cada comentário e colocar ele em uma dessas categorias. 

No começo foi fácil, mas depois de um tempo foi ficando repetitivo, então você teve a ideia de automatizar esse processo. Assim, você decide criar um diagrama de uma “caixa-preta” responsável por classificar automaticamente esses comentários da mesma forma que estava fazendo. Essa “caixa” irá receber uma entrada, neste caso, o texto do comentário sobre o produto, e irá retornar uma saída, uma classificação positiva ou negativa sobre o comentário. 

Na matematica, as funções do tipo sigmoide são excelentes para esse tipo de problema, pois possuem uma propriedade muito interessante: dado um valor de entrada, elas são capazes de "empurrar" esse valor para dois diferentes extremos. No caso da sigmoide logísitca, a função que dá nome a essa família, ela é capaz de empurrar essa entrada para valores próximos de zero ou um. Se nós consideramos que zero é uma avaliação negativa e um é uma avaliação positiva, essa função se torna perfeita para resolver o seu problema de classificar comentários.

\section{A Sigmoide Logística: Ótima para Classificações Binárias}

Por mais que a sigmoide hoje em dia seja bem comum em redes neurais, seu uso não começou nesse cenário. A sigmoide tem suas origens a analise de crescimento populacional e demografia, ela nao surgiu em um artigo em especifico, sendo mais uma evolução presente em vários artigos do matemático belga Pierre François Verhulst dentre os anos de 1838 e 1847. Contudo, existe um artigo desse matemático, intitulado \textit{Recherches mathématiques sur la loi d'accroissement de la population} (Pesquisas matemáticas sobre a lei de crescimento da população), em que \textcite{SigmoidVerhulst1845} propõe a função logística como um modelo para descrever o crescimento populacional, levando em consideração a capacidade de suporte de um ambiente, isso gerou a curva em "S" característica da sigmoide. Contudo, foi somente no próximo século, que sigmoide passou a ser utilizada na area da ciência da computação.

Nos anos 1980, estavam ocorrendo mudanças com as funções que eram utilizadas para construir uma rede neural, um desses motivos foi a introdução da retropropagação pelos pesquisadores Daviel Rumelhart, Geoffrey Hinton e Ronald Williams. A retropropagação era uma técnica que permitia que um modelo aprendesse com base nos seus erros, ajustando automaticamente os seus parâmetros em busca de conseguir uma melhor acurácia \parencite{BackpropagationArticle}. 

Além disso, na pesquisa que introduz a retropropagação, \textit{Learning representations by back-propagating errors} de 1986, os cientistas propõem o uso da função sigmoide logística como uma das candidatas para ser utilizada junto com a retropropagação como uma função de ativação, como justificativa, \textcite{BackpropagationArticle} citam o fato dela ser uma função que é capaz de introduzir a não-linearidade para o modelo, permitindo que ele aprenda padrões mais complexos, e também por possui uma derivada limitada.

Mas esse não foi o único fator que fez com que a sigmoide e sua familia se tornassem funções populares para a época. Pouco antes da criação da retropropagação, na década passada, haviam cientistas estudando o comportamento dos neurônios humanos como inspiração para a criação de redes neurais artificiais. Um exemplo desse caso foi o dos cientistas \textcite{SigmoidWilsonCowan}, no início dos anos 70 eles publicaram um artigo intitulado \textit{excitatory and Inhibitory interactions in localized populations of model neurons}, em que buscam estudar como os neurônios respondiam a determinados estímulos.

No artigo, Hugh e Cowan buscam analisar o comportamento de populações localizadas de neurônios excitatórios (denotados pela função $E(t)$) e inibitórios (representados por $I(t)$) e como as duas interagem entre si, para isso, eles utilizam como variável a proporção de células em uma subpopulação que dispara/reage por unidade de tempo \parencite{SigmoidWilsonCowan}. Para modelar essa atividade, \textcite{SigmoidWilsonCowan} fizeram o uso uma variação da função sigmoide, representada para Equação \ref{eq:neuronio-de-wilson-cowan}, que era capaz de descrever o comportamento dos neurônios a certos estímulos.

\begin{equacaodestaque}{Neurônio de Wilson e Cowan}
    \mathcal{S}(x) = \frac{1}{1 + e^{-a(x - \theta)}} - \frac{1}{1 + e^{a\theta}}
    \label{eq:neuronio-de-wilson-cowan}
\end{equacaodestaque}


Nessa equação, o parâmetro $a$ representa a inclinação, a qual foi ajustada para passar pela origem ($\mathcal{S}(0) = 0$) e $\theta$ 
é o limiar. Para o plotar o gráfico da Figura \ref{fig:sigmoide-wilson-e-cowan}, foi utilizado os mesmos valores escolhidos pelos cientistas na pesquisa, assim $a = 1.2$ e $\theta = 2.8$.

\begin{figure}[h!]

    \pgfmathdeclarefunction{wilson_sigmoid}{3}{% 
        \pgfmathparse{1/(1+exp(-#1*(#3-#2))) - 1/(1+exp(#1*#2))}%
    }

    \centering
    \begin{tikzpicture}
        \begin{axis}[
            title={Função Sigmoide de Wilson-Cowan},
            xlabel={$x$ (Estímulo)},
            ylabel={$s(x)$ (Proporção de Ativação)},
            xmin=-2, xmax=10, % Ajusta o eixo x para centralizar a curva
            ymin=-0.1, ymax=1.1, % Ajusta o eixo y
            axis lines=middle,
            grid=major,
            legend pos=north west,
            % Define os parâmetros para a função
            /pgf/declare function={a=1.2; theta=2.8;}
            ]
            \addplot[blue, thick, domain=-2:10, samples=150] {wilson_sigmoid(a, theta, x)};
        \end{axis}
    \end{tikzpicture}
    \caption{Gráfico da função sigmoide conforme proposta por Wilson e Cowan (1972), com parâmetros de exemplo $a=1.2$ e $\theta=2.8$.}
    \label{fig:sigmoide-wilson-e-cowan}
\end{figure}

\textcite{SigmoidWilsonCowan} demonstram que a população de neurônios reage de formas distintas quando sofrem determinados estímulos, os níveis baixos de excitação não conseguem ativar a população, porém, existe uma região de alta sensibilidade, na qual pequenos aumentos no estímulo geram um grande aumento na atividade. Além disso, existe um terceiro nível, o de saturação, em que níveis muito altos de estímulos são capazes de ativar todas as células e a partir disso, a resposta da população atinge o comportamento de uma função constante, indicando que ela saturou \parencite{SigmoidWilsonCowan}. Ao juntar todos esses três níveis, tem-se a famosa curva em "S", caraterística da função sigmoide.

Com isso, ao consideramos esses dois cenários: a criação da retropropagação e busca na natureza para inspiração na criação de redes neurais. A sigmoide, junto com a sua família, se tornaram funções muito populares para a época, estando presentes em varias redes neurais criadas. Um desses exemplos é a rede de Elman, um tipo de rede neural recorrente criada para aprender e representar estruturas em dados sequenciais, \textcite{ElmanNetwork} cita em seu estudo \textit{Finding structure in time} que era ideal o uso de uma função de ativação com valores limitados entre zero e um. Um cenário perfeito para o uso da sigmoide logística.

Cabe destacar também que, as sigmoidais nao foram as primeiras funções de ativação a serem utilizadas na criação de uma rede neural. Nesse cenário de redes neurais, existem sempre funções que são mais populares, e que com o tempo e o surgimento de novas pesquisas, são deixadas de lado para novas funções mais interessantes. Uma função que era muito utilizada era a \textit{heavside}, ou degrau unitário em português, ela está representada na Figura \ref{fig:degrau-unitario}. Essa função inclusive esteve presente na produção da rede neural \textit{Perceptron} criada por \textcite{PerceptronRosenblatt} e introduzida para a comunidade científica no artigo \textit{The Perceptron: A probabilistic model for informations storage and organization in the brain} no final dos anos 50.

\begin{figure}[h!]
    \centering
    \begin{tikzpicture}
        \begin{axis}[
            xlabel={$x$},
            ylabel={$H(x)$},
            xmin=-8.5, xmax=8.5,
            ymin=-0.3, ymax=1.1,
            axis lines=middle,
            grid=major,
            ytick={0,1}, % Define os ticks no eixo y para 0 e 1
        ]
        % Desenha a parte da função para x < 0
        \addplot[blue, thick, domain=-8:0] {0};
        % Desenha a parte da função para x >= 0
        \addplot[blue, thick, domain=0:8] {1};

        % Adiciona os marcadores para a descontinuidade em x=0
        % Círculo aberto em (0,0) para indicar que o ponto não pertence a essa parte
        \addplot[only marks, mark=o, mark size=1.5pt, blue, fill=white] coordinates {(0,0)};
        % Círculo fechado em (0,1) para indicar que o ponto pertence a essa parte
        \addplot[only marks, mark=*, mark size=1.5pt, blue] coordinates {(0,1)};
        \end{axis}
    \end{tikzpicture}
    \caption{Gráfico da função degrau unitário (\textit{heaviside}).}
    \label{fig:degrau-unitario}
    \fonte{O autor (2025).}
\end{figure}

Ao comparar a degrau unitário com a sigmoide, é possível notar uma diferença crucial, a sigmoide é uma função contínua em todos os pontos, podemos dizer que para desenhar seu gráfico não precisamos tirar o lápis do papel nenhuma vez, algo que não ocorre com a \textit{heavside}. Além disso, a derivada da \textit{heavside} é zero em quase todos os seus pontos, por esse motivo, ela impossibilitava a retropropagação do erro, uma vez que quando fossemos calcular o gradiente para uma parte de um modelo que usasse essa função, ele provavelmente seria zero.

A função sigmoide é escrita com uma exponencial, como na Equação \ref{eq:sigmoide}. Como explica \textcite{ActivationFunctionsLederer}, a sigmoide ela é uma função limitada, diferenciável e monotônica, o que significa que conforme os valores de $x$ aumentam os valores de $f(x)$ também aumentam.

\begin{equacaodestaque}{Sigmoide Logística}
    \sigma(y_j) = \frac{1}{1 + e^{-y_j}}
    \label{eq:sigmoide}
\end{equacaodestaque}

O gráfico da sigmoide está presente na Figura \ref{fig:sigmoide}, note que ele possui o formato de um "S" deitado. Assim, também é possível dizer que a função sigmoide é uma função suave, contínua (o que a possibilita de ser derivável em todos os pontos) e também é não-linear. \textcite{ActivationFunctionsLederer} explica que uma das propriedades interessantes da sigmoide está no fato dela empurrar os valores de entrada para dois extremos, neste caso, 0 e 1. Essa propriedade de retornar valores em um intervalo de 0 a 1 é bem útil quando queremos fazer uma classificação binária, para isso, utilizamos a sigmoide aplicada na última camada densa de neurônios, limitando os intervalos, de forma que podemos interpretá-los como probabilidades, em que quanto mais próximo de 1, mais próximo aquele resultado está de uma classe A, por exemplo, já valores mais distantes pertenceriam a uma classe B.

\begin{figure}[h!]
    \centering
    \begin{tikzpicture}
        \begin{axis}[
            xlabel={$y_j$},
            ylabel={$\sigma(y_j)$},
            xmin=-8.5, xmax=8.5,
            ymin=-0.3, ymax=1.1,
            axis lines=middle,
            grid=major,
        ]
        \addplot[blue, thick, domain=-8:8, samples=100] {1/(1+exp(-x))};
    \end{axis}
    \end{tikzpicture}
    \caption{Gráfico da função de ativação sigmoide logística.}
    \label{fig:sigmoide}
    \fonte{O autor (2025).}
\end{figure}

Um ponto a ser destacado nas funções de ativação, e que começou a ser visto no capítulo anterior (Capítulo \ref{cap:retropropagacao-gradiente}), é que se estamos construindo um modelo que aprende por meio da retropropagação, nós estamos também interessados em entender como essas funções de ativação se comportam em suas derivadas, pois ela é um dos componentes básicos para se calcular o gradiente retropropagado.

Assim, ao derivar a sigmoide logística é possível encontrar a Equação \ref{eq:sigmoide-derivada}.

\begin{equacaodestaque}{Sigmoide Logística Derivada}
    \frac{d}{dy_j}[\sigma](y_j) = \frac{e^{-y_j}}{(1 + e^{-y_j})^2}
    \label{eq:sigmoide-derivada}
\end{equacaodestaque}

Um ponto a ser destacado é que a sigmoide também pode ser expressa de uma forma recurssiva, algo que é muito útil pois ajuda a poupar cálculos a serem feitos. Dessa forma, também tem-se a Equação \ref{eq:sigmoide-derivada-recursiva}.

\begin{equacaodestaque}{Sigmoide Logística Derivada Recursivamente}
    \frac{d}{dy_j}[\sigma](y_j) = \sigma(y_j)(1 - \sigma(y_j))
    \label{eq:sigmoide-derivada-recursiva}
\end{equacaodestaque}

Tendo a sua derivada, cabe também plotar o seu gráfico para ver o seu comportamento. Com isso, a derivada da sigmoide logística pode ser vista na Figura \ref{fig:sigmoide-derivada}. Perceba que o valor máximo que a derivada da sigmoide retorna é pouco maior que 0.2, esse é um dos pontos que mais atrapalha as redes que fazem uso de muitas funções sigmoidais, porque acaba por gerar um problema conhecido por desaparacimento do gradiente. Mas é importante destacar o quão importante são as derivadas das funções de ativação, pois muitas vezes elas acabam por gerar diversos problemas.

\begin{figure}[h!]
    \centering
    \begin{tikzpicture}
        \begin{axis}[
            xlabel={$y_j$},
            ylabel={$\sigma'(y_j)$},
            xmin=-8.3, xmax=8.3,
            ymin=-0.05, ymax=0.35,
            axis lines=middle,
            grid=major,
        ]
        \addplot[red, thick, domain=-8:8, samples=100] {exp(-x)/((1+exp(-x))^2)};
        \end{axis}
    \end{tikzpicture}
    \caption{Gráfico da derivada da função de ativação sigmoide logística.}
    \label{fig:sigmoide-derivada}
    \fonte{O autor (2025).}
\end{figure}

Conhecendo a sigmoide logística, a função que dá nome a essa família de funções de ativação, é possível conhecer agora outras funções dessa mesma família. Essas funções apresentam propriedades semelhantes com a sigmoide, como a caracterítisca curva em "S" mas com algumas variações em sua estrutura ou fórmula, as quais podem ser úteis em alguns cenários.

Para isso, será vista primeira a tangente hiperbólica, e como ela melhora uma das desvantagens da sigmoide, com o fato dessa nova função ser centrada em zero.

\section{Tangente Hiperbólica: A Pioneira nas Redes Convolucionais}

Assim como a função sigmoide, a tangente hiperbólica não possui suas origens voltadas para o uso em redes neurais. Neste caso, um dos matemáticos que ficou reconhecido por criar a notação das funções hiperbólicas, seno, cosseno e tangente, foi o suíço Johann Heinrich Lambert no trabalho de 1769 \textit{Mémoire sur quelques propriétés remarquables des quantités transcendantes circulaires et logarithmiques} (Memória sobre algumas propriedades notáveis de grandezas transcendentais circulares e logarítmicas), em que provou que muitas das identidades trigonométricas possuíam suas equivalentes hiperbólicas \parencite{TanhLambert}.

Passado mais de dois séculos, a tangente hiperbólica já estava sendo utilizada em diversas redes neurais, ela inclusive fez parte da primeira rede neural convolucional criada, estando presente na \textit{Le-Net-5}, uma rede neural criada para identificar e classificar imagens de cheques em caixas eletrônicos \parencite{LecunLeNet1998}. No artigo acadêmico \textit{Gradient-based learning applied to document recognition} de 1998, os cientistas \textcite{LecunLeNet1998} explicam a criação dessa rede além de destacar suas métricas alcançadas.

É possível escrever a tangente hiperbólica utilizando a definição de tangente, que é o quociente a função seno com a função cosseno, só que neste caso usando as funções hiperbólicas. Assim ela é representada pela Equação \ref{eq:tanh}.

\begin{equacaodestaque}{Tangente Hiperbólica (tanh)}
    \mathcal{A}_{\tanh(y_j)} = \frac{\sinh(y_j)}{\cosh(y_j)} = \frac{e^y_j - e^{-y_j}}{e^y_j + e^{-y_j}}
    \label{eq:tanh}
\end{equacaodestaque}

Semelhante a sigmoide, a tangente hiperbólica possui várias propriedades parecidas. Como afirma \textcite{ActivationFunctionsLederer}, a tangente hiperbólica é infinitamente diferenciável, sendo uma versão escalada e rotacionada da sigmoide logística. Assim, veja no gráfico da Figura \ref{fig:tanh} que ela é uma função que está centrada em zero, e seus valores variam agora em um intervalo de -1 a 1, diferente da sigmoide, que varia somente de 0 a 1.

\begin{figure}[h!]
    \centering
    \begin{tikzpicture}
        \begin{axis}[
            xlabel={$y_j$},
            ylabel={$\tanh(y_j)$},
            xmin=-5, xmax=5,
            ymin=-1.2, ymax=1.2,
            axis lines=middle,
            grid=major,
        ]
        \addplot[blue, thick, domain=-5:5, samples=100] {tanh(x)};
        \end{axis}
    \end{tikzpicture}
    \caption{Gráfico da função de ativação tangente hiperbólica (tanh).}
    \label{fig:tanh}
    \fonte{O autor (2025).}
\end{figure}

Como foi visto em seções anteriores, é interessante ao escolher uma função de ativação verificar se ela é centrada em zero, como no caso da tangente hiperbólica. Isso acontece pois funções desse tipo permitem uma convergência mais rápida do modelo para os pontos de mínimo, garantindo um melhor desempenho com menos iterações, quando comparadas com funções que não são centradas em zero, como a sigmoide.

De forma análoga a feita na sigmoide logística, vale a pena calcular a derivada da tangente hiperbólica para entender como o gradiente é propagado para as camadas anteriores ao utilizar essa função de ativação. Dessa forma, a sua derivada está representada na Equação \ref{eq:tanh-derivada}.

\begin{equacaodestaque}{Tangente Hiperbólica (tanh) Derivada}
    \frac{d}{dy_j}[\mathcal{A}_{\tanh}](y_j) = \text{sech}^2(y_j)
    \label{eq:tanh-derivada}
\end{equacaodestaque}

Assim como a sigmoide logística, a tangente hiperbólica possui uma vantagem, ela pode ser escrita de forma recursiva, o que ajuda a poupar cálculos para a o \textit{backward pass}, pois é possível simplemente armazenar o resultado dessa função que estava calculado no \textit{forward pass} e reutilizá-lo ao calcular a sua derivada. Isso é muito útil pois quanto menos calculos forem feitos, maior é a tendência que o modelo será mais rápido e com isso irá convergir em menos tempo \footnote{Note que tanto a sigmoide quanto a tangente hiperbólica fazem uso de exponenciais em suas fórmulas, essas operações acabam por ser caras (em termos de poder de processamento) quando comparadas com as simples operações de comparação da hard tanh, a qual será vista em seções futuras.}. Dessa forma, tem-se que a derivada da tangente hiperbólica escrita de forma recursiva pode ser expressa pela Equação \ref{eq:tanh-derivada-recursiva}

\begin{equacaodestaque}{Tangente Hiperbólica (tanh) Derivada Recursivamente}
    \frac{d}{dy_j}[\mathcal{A}_{\tanh}](y_j) = 1 - \tanh^2(y_j)
    \label{eq:tanh-derivada-recursiva}
\end{equacaodestaque}

Tendo as equações da derivada da tangente hiperbólica, o próximo passo é plotar também o seu gráfico, o qual está presenta na Figura \ref{fig:tanh-derivada}. Note mais uma vez que a tangente hiperbólica possui o mesmo problema que a sigmoide, os valores máximos que ela retorna para a derivada são muito pequenos, esse é um fator recorrente nas funções sigmoidais.

\begin{figure}[h!]
    \centering
    \begin{tikzpicture}
        \begin{axis}[
            xlabel={$y_j$},
            ylabel={$\tanh'(y_j)$},
            xmin=-5, xmax=5,
            ymin=-0.2, ymax=1.2,
            axis lines=middle,
            grid=major,
        ]
        \addplot[red, thick, domain=-5:5, samples=100] {1-(tanh(x))^2};
        \end{axis}
    \end{tikzpicture}
    \caption{Gráfico da derivada da função de ativação tangente hiperbólica (tanh).}
    \label{fig:tanh-derivada}
    \fonte{O autor (2025).}
\end{figure}

Semelhante a sigmoide, os valores da tangente hiperbólica também se aproximam de zero conforme aumentam ou diminuem na sua derivada. Eles atingem um pico de 1, e conforme as entradas se aproximam de $\pm 4$ a saída da derivada da tangente hiperbólica também fica próxima de zero. Mais uma vez, nota-se que a tangente hiperbólica não atinge valores muito altos em sua derivada, isso acaba por gerar um problema conhecido como desaparacimento do gradiente, o qual será explicado em seções futuras.

Por fim, perceba que a tanto a tangente hiperbólica quando a sigmoide logística são funções de ativação "caras", por utilizarem exponenciais em sua composição. Assim, a próxima função a ser vista busca justamente corrigir esse problema, possuindo a mesma proposta das sigmoidais, porém, sendo mais "barata" em termos de custo computacional.

\section{Softsign: Uma Sigmoidal Mais Barata}

A próxima função sigmoidal a ser analisada é a \textit{softsign}, diferente da tangente hiperbólica e da sigmoide que tiveram suas origens em outros campos diferentes da ciência da computação, a \textit{softsign} foi criada com o intuito de ser trabalhada em uma rede neural. Ela foi introduzida no artigo \textit{A Better Activation Function for Artificial Neural Networks} de 1993, do cientista \textcite{Softsign1998}, no texto o autor propõe a \textit{softsign} como uma alternativa para as funções sigmoidais tradicionais.

A principal diferença da \textit{softsign} com as outras sigmoidais está na sua fórmula, como pode ser visto na Equação \ref{eq:softsign} é que ela não utiliza nenhum exponencial para compor sua função. Isso faz com que ela seja uma função mais "barata" em termos de custo computacional para ser implementada em redes neurais. Assim, é possível obter resultados parecidos porem utilizando cálculos menos complexos e com isso encontrar redes mais rápidas de serem treinadas.

\begin{equacaodestaque}{\textit{Softsign}}
    \mathcal{A}_{\text{Softsign}}(y_j) = \frac{y_j}{1 + |y_j|}
    \label{eq:softsign}
\end{equacaodestaque}

A \textit{softsign} também possui uma representação gráfica, como vista na Figura \ref{fig:softsign}, ela possui o formato em "S" característico das sigmoidais além de ser centrada em zero como a tangente hiperbólica. Além disso, como \textcite{Softsign1998} destaca em seu texto, ela é uma função que é diferenciável em toda a reta possuindo também a mesma propriedade das outras sigmoidais de empurrar os valores de entrada para os seus extremos. Nota-se também pelo seu gráfico que ela também uma função contínua, suave e não-linear.

\begin{figure}[h!]
    \centering
    % Gráfico da função Softsign usando PGFPlots
    \begin{tikzpicture}
        \begin{axis}[
            xlabel={$y_j$},
            ylabel={$\text{softsign}(y_j)$},
            xmin=-15, xmax=15,
            ymin=-1.2, ymax=1.2,
            axis lines=middle,
            grid=major,
        ]
        % A função softsign(x) = x / (1 + abs(x))
        \addplot[blue, thick, domain=-10:10, samples=101] {x / (1 + abs(x))};
        \end{axis}
    \end{tikzpicture}
    \caption{Gráfico da função de ativação \textit{softsign}.}
    \label{fig:softsign}
    \fonte{O autor (2025).}
\end{figure}

Com relação a sua diferenciabilidade, a \textit{softsign} pode ser derivada em todos os seus pontos, e sua derivada pode ser vista na Equação \ref{eq:softsign-derivada}. Com base nela, é possível notar uma outra diferença da \textit{softsign} com a tangente hiperbólica e a sigmoide. Diferente das outras duas, a derivada da \textit{softsign} não pode ser expressa em termos da sua própria função. Assim, enquanto nas outras funções é feito um cálculo complexo na função e um simples na derivada, pois é aproveitado o resultado, na \textit{softsign} isso não ocorre.

\begin{equacaodestaque}{\textit{Softsign} Derivada}
    \frac{d}{dy_j}[\mathcal{A}_{\text{softsign}}](y_j) = \frac{1}{(1 + |y_j|)^2}
    \label{eq:softsign-derivada}
\end{equacaodestaque}

Tendo a sua derivada, cabe também plotar o gráfico da derivada da \textit{softsign}, apresentado na Figura \ref{fig:softsign-derivada}. Perceba que o valor de sua derivada começa a ficar proximo de zero quando $x$ se aproxima de $\pm 10$, indicando que ela começa a saturar nesse ponto. Com isso, nota-se que ela demora mais para saturar quando comparada com as outras sigmoidais vistas até o momento.

\begin{figure}[h!]
    \centering
    % Gráfico da derivada da função Softsign
    \begin{tikzpicture}
        \begin{axis}[
            xlabel={$y_j$},
            ylabel={$\text{softsign}'(y_j)$},
            xmin=-10, xmax=10,
            ymin=-0.2, ymax=1.2,
            axis lines=middle,
            grid=major,
        ]
        % A derivada da softsign é 1 / (1 + |x|)^2
        \addplot[red, thick, domain=-10:10, samples=101] {1/((1+abs(x))^2)};
        \end{axis}
    \end{tikzpicture}
    \caption{Gráfico da derivada da função de ativação \textit{softsign}.}
    \label{fig:softsign-derivada}
    \fonte{O autor (2025).}
\end{figure}

Além disso, essa função também sofre do mesmo problema das outras duas funções de ativação vistas até agora, o problema do desaparecimento do gradiente. Esse é um ponto que é corrigido com uma outra família de funções de ativação, as retificadoras, as quais são tema principal do Capítulo \ref{cap:ativacao-retificadoras}.

Seguindo na proposta de ver funções sigmoidais que são mais "baratas", a próxima seção busca explorar isso com outras duas funções: a \textit{hard sigmoid} e a \textit{hard tanh}. Essas funções tem como intuito imitar o comportamento das sigmoidais porém, escrito em forma de retas, dessa forma, elas sacrificam a suavidade das sigmoidais para garantir um menor custo computacional.

\section{Hard Sigmoid e Hard Tanh: O Sacrifício da Suavidade em Prol do Desempenho}

Agora será visto duas funções sigmoidais, criadas no contexto de redes neurais, cujo o seu intuito é ser trazer velocidade para o modelo que está sendo criado, elas são a \textit{hard sigmoid} e \textit{hard tanh}. Elas são inspiradas nas suas versões originais ou \textit{soft}, com o mesmo intuito de variar até um certo ponto e depois saturar. Contudo, não garantem a mesma suavidade que as outras funções.

A primeira é a \textit{hard sigmoid}. Como pode ser visto na Equação \ref{eq:hard-sigmoid}, a qual está presente na documentação do \textcite{PyTorchHardSigmoid}, ela pode ser escrita juntando 3 diferentes funções, duas delas sendo funções constantes e uma terceira sendo a função identidade. Note que ela perde a suavidade da função seno, possuindo bicos que a impedem de ser derivada em todos os seus pontos, contudo, seus cálculos são bem mais simples quando comparados com a sigmoide tradicional, não tem nenhuma exponencial para atrasar as respostas da função. Mas note que também existe um crescimento linear quando os valores estão entre -3 e 3.

\begin{equacaodestaque}{\textit{Hard Sigmoid}}
        \mathcal{A}_{\text{Hard sigmoid}}(y_j) = \begin{cases} 0 & \text{se } y_j < -3 \\ y_j/6 + 0.5 & \text{se } -3 \le y_j \le 3 \\ 1 & \text{se } y_j > 3 \end{cases}
    \label{eq:hard-sigmoid}
\end{equacaodestaque}

Além disso, sabendo a sua fórmula, é possível também plotar o seu gráfico, o qual está representado na Figura \ref{fig:hard-sigmoid}. Ele lembra uma função sigmoide logística, porém, sem todas as curvas suaves daquela função, apresentando no lugar um conjunto de três diferentes retas, a primeira constante em zero, a segunda a função identidade e a terceira também constante em um. 

\begin{figure}[h!]
    \centering
    \begin{tikzpicture}
        \begin{axis}[
            xlabel={$y_j$},
            ylabel={$\text{hard sigmoid}(y_j)$},
            xmin=-5, xmax=5,
            ymin=-0.2, ymax=1.2,
            axis lines=middle,
            grid=major,
            domain=-5:5,
            samples=200, % More samples for a smoother piecewise look
            restrict y to domain*=-0.2:1.2 % Keep plot within y limits
        ]
        % Define the piecewise function using if-else logic for plotting
        \addplot[blue, thick] {
            (x < -3) * 0 +
            (x >= -3 && x <= 3) * (x/6 + 0.5) +
            (x > 3) * 1
        };
        \end{axis}
    \end{tikzpicture}
    \caption{Gráfico da função de ativação \textit{hard sigmoid}.}
    \label{fig:hard-sigmoid}
    \fonte{O autor (2025).}
\end{figure}

Com relação a derivada da \textit{hard sigmoid}, para obtê-lá, deve-se derivar todas as três expressões construindo a sua derivada. Note que os pontos que as retas se tocam no gráfico da Figura \ref{fig:hard-sigmoid} são pontos de descontinuidade, pois quando calculados os limites laterais nesses pontos eles serão diferentes. Isso faz com que essa função não possa ser derivada desses locais. Considerando isso, é possível expressar a derivada da \textit{hard sigmoid} com a Equação \ref{eq:hard-sigmoid-derivada}.

\begin{equacaodestaque}{\textit{Hard Sigmoid} Derivada}
        \frac{d}{dy_j}[\mathcal{A}_{\text{Hard sigmoid}}](y_j) = \begin{cases} 0 & \text{se } y_j < -3 \\ 1/6 & \text{se } -3 < y_j < 3 \\ 0 & \text{se } y_j > 3 \end{cases}
    \label{eq:hard-sigmoid-derivada}
\end{equacaodestaque}

Tendo a sua derivada, cabe também plotar o seu gráfico, o qual está presente na figura \ref{fig:hard-sigmoid-derivada}. Perceba mais uma vez que o gráfico não é suave que nem na sigmoide logística, perceba que ele também tenta imitar o comportamento do gráfico da derivada da sigmoide logística, neste caso, com retas ao invés de curvas. Mas ainda sim, essa derivada retorna valores muito pequenos, ficando susceptível ao problema do desaparecimento do gradiente assim como a sua variante \textit{soft}.

\begin{figure}[h!]
    \centering
    \begin{tikzpicture}
        \begin{axis}[
            xlabel={$y_j$},
            ylabel={$\text{hard sigmoid}'(y_j)$},
            xmin=-5, xmax=5,
            ymin=-0.2, ymax=0.8,
            axis lines=middle,
            grid=major,
            domain=-5:5,
            samples=200, % More samples to show step clearly
            restrict y to domain*=-0.2:0.8
        ]
        % Plot the derivative: 1/6 between -3 and 3, 0 otherwise
        \addplot[red, thick] {
            (x > -3 && x < 3) * (1/6)
        };
        % Add lines for the jumps (optional, but makes it clearer)
        \draw[dashed, blue] (axis cs:-3, 0) -- (axis cs:-3, 1/6);
        \draw[dashed, blue] (axis cs:3, 0) -- (axis cs:3, 1/6);
        \end{axis}
    \end{tikzpicture}
    \caption{Gráfico da derivada da função de ativação \textit{hard sigmoid}.}
    \label{fig:hard-sigmoid-derivada}
    \fonte{O autor (2025).}
\end{figure}

Além disso, também existe a função \textit{hard tanh}, que apresenta a mesma proposta da \textit{hard sigmoid}, mas, nesse sentido, busca imitar o comportamento da função de ativação tangente hiperbólica. Para isso, ela segue uma ideia parecida com a da equação da \textit{hard sigmoid}, escrevendo a sua equação em forma de condicionais, mas ajustando os seus valores para corresponder ao funcionamento da tanh. Essa função está expressa na Equação \ref{eq:hard-tanh}.

\begin{equacaodestaque}{\textit{Hard Tanh}}
        \mathcal{A}_{\text{Hard tanh}}(y_j) = \begin{cases} -1 & \text{se } y_j < -1 \\ y_j & \text{se } -1 \le y_j \le 1 \\ 1 & \text{se } y_j > 1 \end{cases}
    \label{eq:hard-tanh}
\end{equacaodestaque}

O seu gráfico também é parecido com o da \textit{hard sigmoid}. Como é possível vez na Figura \ref{fig:hard-tanh}, ele é composto por três retas, sendo duas delas constantes e uma terceira sendo a função identidade. Note que ele também possui "bicos", o que o impede essa função de ser derivada nesses pontos, uma vez que ela não é contínua nessas regiões.

\begin{figure}[h!]
    \centering
    \begin{tikzpicture}
        \begin{axis}[
            xlabel={$y_j$},
            ylabel={$\text{hard tanh}(y_j)$},
            xmin=-5, xmax=5,
            ymin=-1.2, ymax=1.2,
            axis lines=middle,
            grid=major,
            domain=-5:5,
            samples=200, % More samples for clarity
            restrict y to domain*=-1.2:1.2 % Keep plot within y limits
        ]
        % Define the piecewise function for plotting
        \addplot[blue, thick] {
            (x < -1) * -1 +
            (x >= -1 && x <= 1) * x +
            (x > 1) * 1
        };
        \end{axis}
    \end{tikzpicture}
    \caption{Gráfico da função de ativação \textit{hard tanh}.}
    \label{fig:hard-tanh}
    \fonte{O autor (2025).}
\end{figure}

Perceba também que o gráfico, bem como sua equação, é bem mais simples que a tangente hiperbólica tradicional, isso faz com que a \textit{hard tanh} seja uma função mais barata para ser empregada ao desenvolver um modelo de rede neural. Isso signica que ela pode ser uma função ideal para ser utilizada naqueles sistemas que possuem recursos limitados, como uma quantidade pequena de memória ram, ou seu a possibilidade de utilizar uma placa gráfica dedicada para o processamento dos dados. O mesmo vale para a \textit{hard sigmoid}.

Agora é possível também considerar também a derivada da \textit{hard tanh}, a qual é calculada de forma semelhante a da \textit{hard sigmoid}, derivando as três expressões de sua equação separadamente. Feito isso, a derivada da \textit{hard tanh} está expressa na Equação \ref{eq:hard-tanh-derivada}.

\begin{equacaodestaque}{\textit{Hard Tanh} Derivada}
        \frac{d}{dy_j}[\mathcal{A}_{\text{Hard tanh}}](y_j) = \begin{cases} 0 & \text{se } y_j < -1 \\ 1 & \text{se } -1 < y_j < 1 \\ 0 & \text{se } y_j > 1 \end{cases}
    \label{eq:hard-tanh-derivada}
\end{equacaodestaque}

Tendo a derivada, o próximo passo é plotar o seu gráfico, que pode ser visto na Figura \ref{fig:hard-tanh-derivada}. Note mais uma vez que ele também tenta imitar o comportamento da derivada da tangente hiperbólica, porém, sendo composto por retas ao invés de uma curva suave. Novamente, é possível perceber que a \textit{hard tanh} também irá sofrer do problema do desaparecimento do gradiente, uma vez que retorna valores muito pequenos para a sua derivada.

\begin{figure}[h!]
    \centering
    \begin{tikzpicture}
        \begin{axis}[
            xlabel={$y_j$},
            ylabel={$\text{hard tanh}'(y_j)$},
            xmin=-5, xmax=5,
            ymin=-0.2, ymax=1.2,
            axis lines=middle,
            grid=major,
            domain=-5:5,
            samples=200, % More samples to show step clearly
            restrict y to domain*=-0.2:1.2
        ]
        % Plot the derivative: 1 between -1 and 1, 0 otherwise
        \addplot[red, thick] {
            (x > -1 && x < 1) * 1
        };
        % Add lines for the jumps (optional, but makes it clearer)
        \draw[dashed, blue] (axis cs:-1, 0) -- (axis cs:-1, 1);
        \draw[dashed, blue] (axis cs:1, 0) -- (axis cs:1, 1);
        \end{axis}
    \end{tikzpicture}
    \caption{Gráfico da derivada da função de ativação \textit{hard-tanh}.}
    \label{fig:hard-tanh-derivada}
    \fonte{O autor (2025).}
\end{figure}

Conhecendo todas essas diferentes funções de ativação, é possível finalmente entender o que é o problema do desaparecimento de gradientes e como ele ocorre em uma rede neural, além de relacionar a atuação das funções sigmoidais com o aparecimento desse fenônemo. Para isso, é possível ver essas explicações na seção a seguir.

\section{O Desaparecimento de Gradientes}

Mesmo possuindo muitas propriedades atrativas para a utilização da familia sigmoidal em redes neurais, como a continuidade em todos os pontos e suavidade, além de que suas derivadas podem ser feitas com as próprias funções (no caso da sigmoide e da tangente hiperbólica), essa familia de funções trouxe um problema para os cientistas da época.

Como foi destacado ao discutir o gráfico das derivadas dessas funções, é possível notar que para valores extremos, seja eles positivos ou negativos, a derivada dessas funções fica bem próxima de zero. Isso significa que quando essas funções recebem como entrada um valor alto no \textit{forward pass}, na retropropagação, por pegarmos esse valor e calcularmos a derivada da função de ativação naquele ponto, irá retornar um valor baixo.

Para explicar melhor essa condição será utilizado um problema como base.

Como foi visto no capítulo anterior, o gradiente retropropagado para camadas anteriores de uma rede neural é proporcional a multiplicação da perda, com a derivada da função de ativação no ponto e o valor do resultado da camada anterior de neurônios. 

\[
    \delta^{(L)} = \left( \left( \textbf{W}^{(L+1)} \right)^T \delta^{(L+1)} \right)  \odot \mathcal{A}'(x^{(L)})
\]

Em que: 

\begin{itemize}
    \item $L$: Representa o índice de uma camada, podendo ser um valor entre $1$ (indicando que é uma camada de entrada) ou $n$ (indicando que é uma camada de saída);
    \item $\textbf{W}^{(L)}$: Representa a matriz dde pesos que conecta a camada $L - 1$ à camada $L$;
    \item $b^{(L)}$: Representa o vetor de viés da camada $L$;
    \item $x^{(L)}$: Representa o vetor de entradas totais para os neurônios da camada $L$ antes da ativação;
    \item $y^{(L)}$: Representa o vetor de saídas da camada $L$
    \item $\delta^{(L)}$: Representa o vetor do gradienye na camada $L$;
    \item $\mathcal{A}'(x^{(L)})$: Representa o vetor contendo a derivada da função de ativação para cada neurônio da camada $L$;
    \item $\odot$: O produto de Hadamard, que significa multiplicação elemento a elemento.
\end{itemize}

Considerando isso, imagine que temos uma rede composta por quatro camadas densas e cada camada tem apenas um neurônio com pesos iguais a 1. Dessa forma, é possível simplificar a fórmula vista para a Equação \ref{eq:gradiente-retropropagado-simplificado}.

\begin{equation}
        \delta^{(L)} =  \delta^{(L+1)} \times \mathcal{A}'(x^{(L)})
        \label{eq:gradiente-retropropagado-simplificado}
\end{equation}

Considere também que as camadas da rede possuem a seguinte configuração:

\begin{itemize}
    \item Sigmoide da primeira camada: tem como resultado da derivada $\sigma'(y_j) = 0.2$
    \item Sigmoide da camada densa 2: tem como resultado da derivada $\sigma'(y_j) = 0.05$
    \item Sigmoide da camada densa 3: tem como resultado da derivada $\sigma'(y_j) = 0.1$
    \item Sigmoide da camada de saída: tem como resultado da derivada $\sigma'(y_j) = 0.08$
\end{itemize}

Além disso, você sabe também que o gradiente inicial na camada de saída está sendo de 1. Com isso é possível calcular o gradiente retropropagado para a primeira camada, comecando pela terceira, já que já temos o valor do gradiente para a camada de saída, dessa forma temos que:

\[\begin{WithArrows}
    \delta^{(3)} & = \delta^{(4)} \times \sigma'(x^{(3)}) \Arrow{Subtituindo os valores} \\
    \delta^{(3)} & = 1 \times 0.1 = 0.1
\end{WithArrows}\]

Seguindo adiante, é possível fazer o mesmo procedimento para encontrar $\delta^{(2)}$ agora já tendo $\delta^{(3)}$, dessa forma:

\[\begin{WithArrows}
    \delta^{(2)} & = \delta^{(3)} \times \sigma'(x^{(2)}) \Arrow{Subtituindo os valores} \\
    \delta^{(2)} & = 0.1 \times 0.05 = 0.005
\end{WithArrows}\]

De forma semelhante, é finalmente possível encontrar o gradiente retropropagado para a primeira camada:

\[\begin{WithArrows}
    \delta^{(1)} & = \delta^{(2)} \times \sigma'(x^{(1)}) \Arrow{Subtituindo os valores} \\
    \delta^{(1)} & = 0.005 \times 0.2 = 0.001
\end{WithArrows}\]

Note que o gradiente que entrou ao ser calculado pela perda era de 1, no final da camada chegou apenas 0.001, ou seja, ele diminui mil vezes. Se considerarmos uma situação como esta, em que a derivada função de ativação irá retornar uma valor baixo, esse valor será multiplicado com os outros termos da expressão fazendo com que o valor total do gradiente retropropagado naquela camada seja baixo. Nós também vimos que redes em que é utilizada a retropropagação, cálculo do gradiente é utilizado como forma de fazer com que os pesos e vieses dos neurônios se atualizem e com isso a rede aprenda. E se os pesos são atualizados com uma variação muito pequena quando comparamos com seus valores anteriores, isso significa que essa rede estaria dando pequenos passos para encontrar a sua função. 

Ao passar valores muito extremos para uma função de ativação sigmoide, estamos prejudicando o aprendizado de uma rede neural, pois isso implica em derivadas com valores pequenos e consequentemente gradientes retropropagados pequenos. Agora imagine que em uma rede neural pode existir diversas camadas densas que usem a função sigmoide, se cada vez que o gradiente passar para a camada anterior ele diminuir, isso signica que na primeira camada, a última a ter seus pesos atualizados, o gradiente será tão pequeno que pode ser que não contribua para que a rede aprenda corretamente. É como se toda vez que passasse um valor muito extremo para a rede, o tamanho do passo que ela dá em direção a função procurada diminuísse. Assim, tem-se o problema do desaparecimento do gradiente.

\begin{definicaomoderna}{\textbf{Definição:}}
Quando o gradiente é muito pequeno em valor absoluto, ou até mesmo igual a zero, a atualização do gradiente apresenta quase nenhum impacto nos parâmetros de uma rede neural, fazendo com que não haja progresso nos parâmetros de aprendizado, dessa forma o \textbf{desaparecimento do gradiente} é quando esse fenômeno acontece repetidamente por várias pares de entrada e saída. \parencite{ActivationFunctionsLederer}.
\end{definicaomoderna}

Isso se torna um problema, pois, quando criamos uma rede neural, utilizamos as primeiras camadas para que elas sejam responsáveis por aprender características básicas/simples de uma determinada amostra de dados. Se o gradiente é próximo de zero, o calculo da atualização dos pesos e dos vises irá gerar valores muito próximos dos originais. Como esses valores não irão atualizar corretamente, a rede neural não irá aprender características básicas de um cenário.

Nesse sentido, o Capítulo \ref{cap:ativacao-retificadoras} busca explicar as funções retificadoras, começando pela \textit{ReLU}, e como elas vieram como uma alternativa para contornar o problema do desaparecimento do gradiente. Contudo, elas também não foram perfeitas, e acabaram por introduzir os seus próprios problemas ao serem aplicadas em uma rede neural. Neste caso, um dos problemas que pode ocorrer ao se utilizar uma função retificadora é o problema da explosão do gradiente, e considerando a \textit{ReLU}, ela também pode causar um problema conhecido como \textit{ReLUs} agonizantes.

Assim, o último passo ao conhecer todas essas funções, é resumir o seu conteúdo para melhor entendimento. Este resumo pode ser visto na próxima seção.

\section{Comparativo: Funções Sigmoidais}

Tendo visto diferentes funções de ativação sigmoidais, é possível compilá-las em na Tabela \ref{tab:comparativo-funcoes-sigmoidais}, destacando as suas fórmulas, vantagens e desvantagens ao serem utilizadas em uma rede neural.

\begin{table}[htbp]
    \centering
    \begin{threeparttable}
        \caption{Comparativo das funções de ativação sigmoidais}
        \label{tab:comparativo-funcoes-sigmoidais}
        % p{3.2cm} define uma largura fixa para a primeira coluna.
        % As 3 colunas 'X' restantes dividem o espaço que sobra de forma flexível.
        % >{\raggedright\arraybackslash} alinha o texto à esquerda para melhor leitura.
        \begin{tabularx}{\textwidth}{p{3.2cm} *{2}{>{\raggedright\arraybackslash}X}}
            \toprule
            \textbf{Função} & \textbf{Vantagem} & \textbf{Desvantagem} \\
            \midrule
            Sigmoide logística & Pode ser interpretada como uma probabilidade, podendo ser aplicada nas camadas de saída para classificação binária. & Não é centrada em zero, fazendo com que a convergência de modelos que usem essa função seja um pouco mais lenta. \\
            \addlinespace
            Tangente hiperbólica & Centrada em zero, garantindo uma convergência mais rápida de modelos que usem essa função. & Possui muitos exponenciais, sendo uma função "cara" em termos de custo computacional. \\
            \addlinespace
            \textit{Softsign} & É uma sigmoidal mais "barata" em termos de custo computacional, permitindo a criação de redes mais rápidas. & Sua derivada não pode ser escrita de forma recursiva. \\
            \addlinespace
            \textit{Hard Sigmoid} & É uma versão mais "barata" da sigmoide logística. & Não é uma função suave, possuindo "quinas", as quais impedem essa função de ser derivada nesses pontos. \\
            \addlinespace
            \textit{Hard Tanh} & É uma versão mais "barata" da tangente hiperbólica. & Assim como a \textit{hard sigmoid}, a \textit{hard tanh} apresenta "quinas", as quais impedem essa função de ser derivada nesses pontos. \\
        \end{tabularx}
        
        \begin{tablenotes}[para]
            \small
            \item[] Fonte: O autor (2025).
        \end{tablenotes}

    \end{threeparttable}
\end{table}
% ===================================================================
% Arquivo: capitulos/parte-III-pilares/cap-08-retificadoras.tex
% ===================================================================

\chapter{Funções de Ativação Retificadoras}
\label{cap:ativacao-retificadoras}

\begin{flushright}
\textit{"caramba! A perda do meu modelo está em 31.415"} \\
--- Estagiário conhecendo o problema dos gradientes explosivos
\end{flushright}

No Capítulo \ref{cap:ativacao-sigmoidais} foi apresentado as funções de ativação sigmoidais. Essas funções estiveram presentes em um grande quantidade de redes neurais criadas até os anos 2010, sendo consideradas padrões para se utilizarem ao construir uma RNA. Contudo, essas funções são susceptíveis ao problema do desaparecimento do gradiente, um problema que afetou consideravelmente como as redes neurais eram construídas ao utilizar esse tipo de função de ativação, pois não era possível construir redes muito profundas.

Nesse cenário surgem as funções retificadoras, como uma solução para contornar esse problema. Essas funções são o tópico central desse capítulo. Para isso, primeiro será vista a \textit{rectfied linear unit}, também conhecida como \textit{ReLU}, conhecendo as suas origens, propriedades, fórmulas e como ela permitiu a criação de redes neurais mais profundas. Em seguida, será visto um problema crônico dessa função de ativação: os \textit{ReLUs} agonizantes.

Para contornar esse problema, surgem então variantes da \textit{ReLU} com vazamento, permitindo que um pouco de gradiente flua pela rede para os casos em que a entrada dessa função seja negativa. Serão vistas nessa seção três funções: a \textit{Leaky ReLU}, a \textit{Parametric ReLU}, e a \textit{Randomized Leaky ReLU}. Seguindo adiante serão vistas também variantes que apresentam curvas suaves em seus gráficos, como a \textit{ELU} e a \textit{SELU}. Para todas essas funções serão apresentados diversos comparativos para entender melhor o seu desempenho e em quais cenários elas são ideais.

Essas funções também não são perfeitas, assim como a sigmoidais, elas também acabaram por introduzir uma nova classe de problemas. Neste caso, um problema que pode ocorrer ao se utilizar uma função retificadora é o do chamado gradiente explosivo, que, assim como o desaparecimento do gradiente, impede o aprendizado dos neurônios da rede. O capítulo termina com uma tabela, compilando as principais características dessa família de funções de ativação.

% ===================================================================
% Resumo do capítulo
% ===================================================================

\section{Exemplo Ilustrativo: Vendendo Pipoca}

Imagine que você está querendo ganhar dinheiro e decidiu vender pipoca em uma praça da sua cidade. Você comprou milho, óleo, sal e manteiga, um carrinho para poder levar e fazer as pipocas, além disso, você também comprou vários pacotes para poder colocar as pipocas para vender.

Nisso, você teve que estipular um valor para vender essas pipocas, após pensar um pouco e analisar todos os seus gastos, você estimou que um valor de R\$ 5,00 seria ideal, pois conseguiria pagar os seus gastos mas você ainda ia obter lucro dos seus clientes.

Agora você está pronto para vender, começou a fritar o milho e colocou uma plaquinha com o preço ao lado do seu carrinho. Então chega uma pessoa com R\$ 6,00 e decide comprar um pacote, você vende e entrega um real de troco. Logo em seguida aparece uma segunda pessoa com R\$ 4,99 e decide negociar com você, ela afirma que é quase R\$ 5,00, e por isso, você deveria vender a pipoca para ela, mas você explica que só vende pelo valor de R\$ 5,00.

Com base nisso, nós podemos chegar em uma situação em que um pacote de pipoca será vendido somente se uma pessoa possuir R\$ 5,00 no bolso, ou mais. Podemos então escrever algo como o da equação \ref{eq: VendaPipoca}. Em casos em que uma venda ocorre, você poderá vender mais um pacote, para isso, o seu comprador deverá possuir pelo menos R\$ 10,00, assim, $x$ que indica a quantidade de pacotes vendido seguirá a lei de formação $x = 5 \mod d$, em que $d$ é o dinheiro que a pessoa possui.

\begin{equation}
    \text{Número de Pacotes} = \begin{cases} 0 & \text{quando } R\$ \leq  4,99 \\ x & \text{quando } R\$ > 4,99 \end{cases}
    \label{eq: VendaPipoca}
\end{equation}

Saindo do assunto da pipoca e voltando para o tema deste texto, existe uma família de funções de ativação que funciona de forma semelhante a lógica de venda dos pacotes de pipoca, elas são as unidades lineares retificadoras. A ReLU, que dá nome a essa família, funciona de forma semelhante a essa venda, ela tem um comportamento de "tudo ou nada", em que irá comandar quando um neurônio de uma rede neural irá disparar seu resultado.

% ===================================================================
% ReLU
% ===================================================================

\section{Rectified Linear Unit (ReLU): A Revolução Retificadora} \index{Funções de Ativação!Rectified Linear Unit (ReLU)}

Como foi visto anteriormente no Capítulo \ref{cap:ativacao-sigmoidais}, as funções sigmoidais surgiram com inspiração nos neurônios humanos e como eles se comportam com determinados estímulos. Mas essas não foram as únicas funções que tiveram essa origem. Na década de 40, o pesquisador Alton Householder estava estudando um cenário parecido em seu trabalho \textit{A theory of steady-state activity in nerve fiber network: I. Definition of mathematical biofysics}, nele o autor analisou o comportamento de fibras nervosas e quando elas irão assumir caráter excitatório ou inibitório, para isso ele apresentou a Equação \ref{eq:fibra-nervosa-householder} \parencite{Householder1941}.

\begin{equation}
    a_{ij} = \begin{cases} 0 & \text{quando } \eta_i \le h_{ij} \\ a_{ij}, & \text{quando } \eta_i > h_{ij} \end{cases}
    \label{eq:fibra-nervosa-householder}
\end{equation}

Essa equação mostra quando uma fibra nervosa irá disparar, para isso, deve-se olhar o limiar da fibra $h_{ij}$ e o estímulo total $\eta_i$, com base nesses valores e no que a fórmula apresenta, uma fibra irá disparar quando o estímulo total for maior que o seu limiar, quando isso não ocorrer, ela não irá disparar \parencite{Householder1941}. Além disso, \textcite{Householder1941} explica também sobre o termo $a_{ij}$, a saída dessa função, segundo o autor ele é utilizado para representar o parâmetro de atividade, sendo um valor diferente de zero, podendo ser positivo (quando a fibra possui ação excitatória), ou negativo (apresentando caráter inibitório).

Essa equação criada por Householder, lembra bastante a expressão da função \textit{ReLU}, a qual é denotada pelas Equações \ref{eq:relu}.

\begin{equacaodestaque}{\textit{Rectified Linear Unit} (\textit{ReLU})}
    \mathcal{A}_{\text{ReLU}}(y_j) = \begin{cases}y_j, & \text{se } y_j > 0 \\0, & \text{se } y_j \leq 0\end{cases} \quad \text{ou} \quad \mathcal{A}_{\text{ReLU}}(y_j) = \max(0, y_j)
    \label{eq:relu}
\end{equacaodestaque}

Dito isso, mesmo com ela existindo a mais de 80 anos, ela só passou a ser amplamente utilizada nos anos 2010, antes disso, as sigmoides eram a grande maioria quando o assunto era função de ativação. Contudo, as sigmoides eram funções saturantes, e isso fazia com que sua derivada retornasse muitos valores pequenos ao longo da função. Ao multiplicar vários valores pequenos na retropropagação do gradiente, o vetor gradiente ia diminuindo até chegar um ponto em que ele não conseguia atualizar os pesos e vieses das redes neurais de forma eficiente, assim, tínhamos o problema do desaparecimento do gradiente. As funções retificadoras, sendo a principal delas a \textit{ReLU}, surgem para corrigir esse problema crônico. 

Dessa forma, antes de conhecer de fato a \textit{ReLU} e suas propriedades, é preciso entender o cenário que ela se popularizou, com os cientistas buscando novos tipos de funções de ativação que substituísse as sigmoides, funções saturantes, por outro tipo de função que resolvesse o problema do desaparecimento do gradiente.

Nesse cenário, artigos como \textit{Rectified Linear Units Improve Restricted Boltzmann Machines} foram essenciais para popularizar a \textit{ReLU} como uma função de ativação interessante para se utilizar em redes neurais. No trabalho, \textcite{Nair2010} foram responsáveis por demonstrar propriedades úteis das funções retificadoras, como a capacidade da \textit{NReLU} de auxiliar em reconhecimentos de objetos por possuir equivariância de intensidade(\textit{intensity equivarience}), o que significa que se a intensidade da entrada de uma função for alterada por um determinado fator a intensidade de sua saída será alterada pelo mesmo fator. Essa propriedade se torna bastante útil em casos que queremos preservar informações, como ao comparar imagens, garantindo melhor precisão por exemplo em situações de baixa luz quando comparados com cenários em que possuem muita luz nas imagens.

Além disso, no trabalho \textit{Deep Sparse Rectifier Neural Networks} dos autores \textcite{Glorot}, o uso de unidades retificadoras não lineares são propostos como alternativas para a tangente hiperbólica e sigmoide em redes neurais profundas, mas também os pesquisadores são capazes de demonstrar que as unidades retificadoras se aproximam melhor do comportamento de neurônios biológicos. Um ponto chave desse texto é que os autores destacam características importantes que a esparsidade traz para uma rede neural possibilitada pelo uso de funções retificadoras \parencite{Glorot}. Entre elas estão:

\begin{itemize}
    \item \textbf{Desembaraçamento de Informações:} Um dos principais objetivos dos algoritmos de aprendizado profundo é desembaraçar os fatores que explicam as variações nos dados, assim, existem diferentes tipos de representações, uma representação densa é altamente emaranhada porque quase qualquer mudança na entrada modifica a maior parte as entradas no vetor de representação, contudo, se tivermos uma representação esparsa e robusta a pequenas mudanças na entrada, o conjunto de características diferentes de zero é quase sempre aproximadamente conservado por pequenas mudanças na entrada \parencite{Glorot};
    \item \textbf{Representação eficiente de tamanho variável}. Diferentes entradas podem conter diferentes quantidades de informação e seriam mais convenientemente representadas usando uma estrutura de dados de tamanho variável, o que é comum em representações computacionais de informação, assim é interessante poder variar o número de neurônios ativos permitindo que um modelo controle a dimensionalidade efetiva da representação para uma determinada entrada e a precisão necessária \parencite{Glorot};
    \item \textbf{Separabilidade linear}. Representações esparsas também são mais propensas a serem linearmente separáveis, ou mais facilmente separáveis com menos maquinário não linear, simplesmente porque a informação é representada em um espaço de alta dimensão, além disso, isso pode refletir o formato original dos dados \parencite{Glorot};
    \item \textbf{Distribuídas, mas esparsas}. Representações densamente distribuídas são as representações mais ricas, sendo potencialmente exponencialmente mais eficientes do que as puramente locais, além disso a eficiência das representações esparsas ainda é exponencialmente maior, com a potência do expoente sendo o número de características diferentes de zero, elas podem representar uma boa compensação em relação aos critérios acima \parencite{Glorot}.
\end{itemize}

\index{AlexNet}

Por fim, um último trabalho que colaborou para a popularização da \textit{ReLU} foi a \textit{AlexNet}, de \textcite{AlexNet}, essa rede neural convolucional (\textit{CNN}) foi capaz ganhar o Desafio de Reconhecimento Visual em Larga Escala \textit{ImageNet} (\textit{ILSVRC}) sendo treinada para classificar 1.2 milhões de imagens de alta resolução e classificá-las em 1000 diferentes classes. Para isso, essa \textit{CNN} foi construída utilizando 8 camadas com pesos, sendo as primeiras 5 camadas convolucionais, enquanto as três últimas são camadas totalmente conectadas, a última camada de neurônios faz uso da função de ativação \textit{softmax} para fazer a distribuição em 1000 diferentes classes, por último mas não menos importante, a \textit{AlexNet} fez uso da ReLU em sua arquitetura \parencite{AlexNet}.

Assim, como pode ser visto na tabela \ref{tab:desempenho-alexnet}, a \textit{AlexNet} foi capaz de alcançar uma taxa de erro de 15.3\% na fase de testes, note com base na variação de camadas convolucionais, que essa é uma rede que se beneficia da sua profundidade, algo que provavelmente só foi capaz de ocorrer devido ao uso da \textit{ReLU} como função de ativação, por não gerar o problema do desaparecimento de gradientes como nas sigmoidais. Além disso, a rede \textit{SIFT + FVs} (\textit{Scale-Invariant Feature Transform + Fisher Vectors}) é mostrada na tabela como base de comparativo, perceba que a \textit{AlexNet} foi capaz de diminuir com mais de 10\% dos erros que essa rede gerava.

Além disso, o modelo \textit{SIFT + FVs} (\textit{Scale-Invariant Feature Transform + Fisher Vectors},) o qual é apresentado como base de comparação, apresenta uma taxa de erro de 26.2\%, um aumento de 10 pontos percentuais quando comparado com o melhor modelo da \textit{AlexNet} de 15.3\%.

\begin{table}[ht]
    \centering
    \begin{threeparttable}
        \caption{Comparação das Taxas de Erro no \textit{AlexNet}}
        \label{tab:desempenho-alexnet}
        \begin{tabular}{lccc}
            \toprule
            \textbf{Modelo} & \textbf{Top-1 (validação)} & \textbf{Top-5 (validação)} & \textbf{Top-5 (teste)} \\
            \midrule
            SIFT + FVs & -    & -      & 26.2\% \\
            1 CNN      & 40.7\% & 18.2\% & -      \\
            5 CNNs     & 38.1\% & 16.4\% & 16.4\% \\
            1 CNN\textsuperscript{a}      & 39.0\% & 16.6\% & -      \\
            7 CNNs\textsuperscript{a}     & 36.7\% & 15.4\% & \textbf{15.3\%} \\
            \bottomrule
        \end{tabular}
        
        \begin{tablenotes}[para] % Ambiente para as notas e fonte
            \small % Define o tamanho da fonte
            \item[] Nota: A tabela compara as taxas de erro de diferentes modelos nos conjuntos de validação e teste do \textit{ILSVRC-2012}. Os valores em negrito indicam o melhor resultado. \textsuperscript{a}Modelos que foram pré-treinados para classificar todo o conjunto de dados ImageNet 2011 Fall.
            \item[] Fonte: Adaptado de "ImageNet Classification with Deep Convolutional Neural Networks", por A. Krizhevsky, I. Sutskever, \& G. E. Hinton, 2012, \textit{Advances in Neural Information Processing Systems, 25}.
        \end{tablenotes}

    \end{threeparttable}
\end{table}

A definição da \textit{ReLU} pode ser interpretada como uma pergunta, ao receber um número como entrada a \textit{ReLU} questiona: "esse número é menor que zero?", se a resposta for sim, ela retorna como resultado o número zero, se a resposta for não, ela irá retornar o próprio número como sua saída. Neste caso estão sendo considerados números, mas a analogia utilizada no início do texto em que o pacote de pipoca só é vendido caso a pessoa tenha mais de R\$ 5,00 também pode ser utilizada, em que o resultado seria um valor booleano, indicando se a pessoa vende ou não a pipoca.

Além de sua fórmula, é possível plotar o seu gráfico, que está presente na Figura \ref{fig:relu}, ele é bem mais simples quando comparado com a sigmoide, por exemplo, sendo apenas a junção de duas retas, uma delas uma função constante que irá retornar sempre zero e a outra a função identidade. Essa simplicidade da \textit{ReLU} é algo muito atrativo para os desenvolvedores, pois, ao utilizá-la ao invés de uma função mais complexa como a sigmoide ou a tangente hiperbólica, estamos diminuindo a complexidade da rede neural, se essa rede se torna mais simples, a tendência é de que ela possua um custo de poder de processamento menor permitindo que um volume maior de dados seja processado em menos tempo e com isso seu tempo de treinamento seja menor. Note que, antes da \textit{ReLU} surgir, muitos das funções de ativação faziam uso de exponenciais, a \textit{ReLU} não só resolvia o problema do desaparecimento do gradiente mas também era muito mais "barata".

\begin{figure}{h!}
    \centering
    \begin{tikzpicture}
        \begin{axis}[
            xlabel={$y_j$},
            ylabel={$\text{ReLU}(y_j)$},
            xmin=-2.3, xmax=2.3,
            ymin=-0.8, ymax=2.3,
            axis lines=middle,
            grid=major,
        ]
        \addplot[blue, thick, domain=-2:2] {max(0, x)};
        \end{axis}
    \end{tikzpicture}
    \caption{Gráfico da função de ativação \textit{Rectified Linear Unit} (\textit{ReLU}).}
    \label{fig:relu}
    \fonte{O autor (2025).}
\end{figure}

\medskip
\begin{center}
 * * *
\end{center}
\medskip

\textbf{Características da Rectified Linear Unit}
\vspace{1em}

\begin{itemize}
    \item \textbf{Característica 1:}
    \item \textbf{Característica 2:}
    \item \textbf{Característica 3:}
\end{itemize}

\medskip
\begin{center}
 * * *
\end{center}
\medskip

ao trabalhar com redes neurais, uma das maiores vantagens destas é o fato de “aprenderem” com base na retropropagação do gradiente nas camadas da rede. Assim, ao calcular o gradiente para fazer a retropropagação do erro e ajustar os pesos e vieses das camadas, é necessário ter em mente também a derivada daquela função de ativação que que será aplicada em uma camada de neurônios da rede, dado que ela entrará no \textit{backward pass} do modelo.

Para achar a derivada da \textit{ReLU}, deve-se derivar as duas condicionais dela, assim, quando $x$ for maior que zero, a saída será 1, já quando $x$ for menor que zero, a saída será zero. Mas você vai encontrar um problema nisso, a derivada dessa função não existe quando $x$ é 0, pois o limite lateral à esquerda dessa função é zero, enquanto o limite lateral a direita dela é um. Isso passa a ser um problema quando queremos calcular o valor de saída justamente quando aquele valor de entrada é zero. Na prática, esse problema é fácil de resolver, ao desenvolver o código dessa função, escolher qual será o resultado da \textit{ReLU} quando esse valor de entrada for zero. Podemos dizer que ele será um ou zero, isso irá depender somente da nossa implementação da derivada da \textit{ReLU}.

Assim, a derivada da \textit{ReLU} é dada pela Equação \ref{eq:relu-derivada}.

\begin{equacaodestaque}{\textit{Rectified Linear Unit} (\textit{ReLU} Derivada)}
    \frac{d}{dy_j} [\mathcal{A}_{ReLU}](y_j) = \begin{cases}1, & \text{se } y_j > 0 \\0, & \text{se } y_j \leqslant 0 \end{cases}
    \label{eq:relu-derivada}
\end{equacaodestaque}

Esse detalhe da descontinuidade da \textit{ReLU} no ponto zero foi algo que acabou mudando em funções futuras, que buscam corrigir erros da \textit{ReLU} e melhorá-la, assim, com o passar do tempo foram surgindo outras alternativas que também trabalhassem com os atributos da ReLU, mas que fossem contínuas em toda a reta, permitindo a sua derivação também em todos os pontos. Uma dessas funções a \textit{ELU}, ela será explicada mais em frente.

Além da sua representação em forma de equação, é possível fazer também o seu gráfico na Figura \ref{eq:relu-derivada}, note que ele é ainda mais simples que a própria função de ativação, são só duas retas constantes que irão retornar zero quando o número for menor que zero, ou irão retornar 1 quando a entrada for um número maior que zero.

\begin{figure}[h!] % Use [htbp] para dar flexibilidade ao LaTeX
    \centering % Centraliza o gráfico na página
    \begin{tikzpicture}
        \begin{axis}[
            xlabel={$y_j$},
            ylabel={$\text{ReLU}'(y_j)$},
            xmin=-2.3, xmax=2.3,
            ymin=-0.8, ymax=1.8,
            axis lines=middle,
            grid=major
        ]
        \addplot[red, thick, domain=-2:0] {0};
        \addplot[red, thick, domain=0:2] {1};
        \addplot[red, only marks, mark=o, mark size=1.5pt] coordinates {(0,0)};
        \addplot[red, only marks, mark=*, mark size=1.5pt] coordinates {(0,1)};
        \end{axis}
    \end{tikzpicture}
        \caption{Gráfico da derivada da função de ativação Rectified Linear Unit (ReLU).}
    \label{fig:relu-derivada}
    \fonte{O autor (2025).}
\end{figure}

\medskip
\begin{center}
 * * *
\end{center}
\medskip

\textbf{Algumas Aplicações da Rectified Linear Unit em Redes Neurais} \index{Aplicações práticas! ReLU}
\vspace{1em}

\begin{itemize}
    \item \textbf{Aplicação 1 (Área):}
    \item \textbf{Aplicação 2 (Área):}
    \item \textbf{Aplicação 3 (Área):}
    \item \textbf{Aplicação 4 (Área):}
\end{itemize}

\medskip
\begin{center}
 * * *
\end{center}
\medskip

Assim, com toda essa simplicidade e versatilidade, a \textit{ReLU} se tornou uma função que é considerada padrão para a maioria das redes neurais \textit{feedforward} \parencite{DeepLearningBook}. Contudo, ela também apresenta problemas assim como as sigmoidais, sendo um desses problemas o dos \textit{ReLUs} agonizantes, o qual será explicado em sequência.

\section{O Problema dos ReLUs Agonizantes} \index{ReLUs agonizantes}

Mesmo apresentando tantas propriedades úteis, como o fato de impedir o problema do desaparecimento de gradientes além de ser uma função computacionalmente barata, a \textit{ReLU} não é perfeita. Ela é responsável por causar um problema conhecido como \textit{ReLUs} agonizantes (\textit{Dying ReLUs problem}), que será explicado nessa seção. Para isso, é preciso lembrar primeiro da equação da camada densa:

\[
    y = W^T  X + b
\]

Imagine que a \textit{ReLU} é a função de ativação que está sendo utilizada para introduzir a não-linearidade após essa camada, quando uma variável $y$ passar para uma função $\max(0, y)$, e condição mais comum dessa comparação for os casos em que $y < 0$ isso pode afetar negativamente o aprendizado do modelo na etapa do \textit{backward pass}, uma vez que a derivada será utilizada nessa parte, ela também será zero para os casos em que $y < 0$, e como foi visto no capítulo \ref{cap:retropropagacao-gradiente}, a derivada será utilizada para a multiplicação do gradiente, se a derivada é zero, o gradiente também é zero, e se o gradiente é zero, isso resulta em cenários nos quais os neurônios não vão ter seus pesos atualizados e portanto não irão aprender \parencite{DyingReluDouglas}. Como explica \textcite{DyingReluDouglas}, os neurônios irão morrer, passando apenas a retornar zeros independente de sua entrada.

Essa condição de vários neurônios morrendo causada pela \textit{ReLU} acabou por gerar um novo conjunto de funções, as quais possuem propriedades comuns da \textit{ReLU}, como a não linearidade e a simplicidade nos cálculos mas que buscam resolver ou amenizar esse problema em uma rede neural. Uma das funções que busca resolver esse problema é a \textit{Leaky ReLU} \parencite{DyingReluDouglas}.

\section{As Variantes com Vazamento: Corrigindo o Problema do ReLUs agonizantes}

Diferente da \textit{ReLU} tradicional que retorna zero para os casos em que sua entrada é negativa e por isso na sua derivada irá também retornar zero nestes casos, as variantes com vazamento atuam de outra forma, elas retornam um valor muito pequeno como 0.1, multiplicado pela entrada da função quando ela é negativa. Por isso, a sua derivada será algo também 0.1 (ou valores muito pequenos), isso permite um "vazamento" do gradiente em cenários nos quais a entrada do neurônio será negativa.

Como foi visto, que a causa do \textit{ReLUs} agonizantes era justamente isso: muitas situações em que a entrada era negativa, que gerava um gradiente nulo e consequentemente impedia os neurônios de terem seus pesos e vieses ajustados, e futuramente morrendo, retornando zero independente de qual fosse a sua entrada.

Assim, essas variantes, como a \textit{Leaky ReLU} e a \textit{PReLU} buscam tentar corrigir um amenizar esse problema da \textit{ReLU} mas mantendo algumas de suas principais propriedades, como a não linearidade, a capacidade de ser escrita compondo duas retas permitindo a criação de uma função simples e rápida de ser computada em uma rede neural.

\subsection{Leaky ReLU (LReLU)} \index{Funções de Ativação!Leaky ReLU (LReLU)}

Seguindo adiante, é possível analisar agora a \textit{Leaky ReLU}, ela é uma variante da \textit{ReLU} que foi criada com intuito de corrigir o problema do \textit{ReLUs} agonizantes. Assim como a \textit{ReLU}, que foi explicada com a analogia do vendedor de pipoca, é possível extender essa explicação para essa nova função, antes o limiar para comprar um pacote de pipoca era de R\$ 5,00, quem tivesse menos que isso não comprava nada. Mas agora, para garantir que todos possam comprar pipoca, você como vendedor definiu que quando uma pessoa tiver menos que R\$ 5,00 ela também será capaz de comprar pipoca, só que neste caso ela comprará um punhado de pipoca que será proporcional ao dinheiro que ela tem multiplicado por uma constante $\alpha$. Assim, uma pessoa com um valor próximo de R\$ 5,00 pode sair com um punhado de pipoca quase igual ao do pacote original se essa constante $\alpha$ for um valor muito proximo de um. Com isso, você como vendedor consegue obter lucro com uma nova clientela além de não perder clientes por não possuírem o valor total do pacote de pipoca. A \textit{Leaky ReLU} traz uma proposta parecida para resolver com o problema do \textit{ReLUs} agonizantes.

Essa função de ativação foi apresentada no artigo \textit{Recfier Nonlinearites Improve Neural Networks Acustic Models}, em que os autores exploram o uso de redes retificadoras profundas como modelos acústicos para a tarefa de reconhecimento de fala conversacional \textit{switchboard} \parencite{LeakyReLUArticle}. Além disso, a sua principal diferença, como explicam \textcite{LeakyReLUArticle}, está no fato dela permitir que um pequeno gradiente diferente de zero flua quando a unidade está saturada e não ativa. Esse gradiente diferente de zero que flui quando a unidade está saturada e não ativa são os seus compradores de pipoca que não possuem o valor total mas são capazes de comprar um punhado dela, neste caso a unidade estará não ativa pois o valor de entrada é negativo mas irá retornar um valor diferente de zero, algo que não acontecia na \textit{ReLU}.

Também é possível discutir a expressão matemática da \textit{Leaky ReLU}, a qual é dada pela Equação \ref{eq:leaky-relu}, que é bem parecida com a \textit{ReLU}, porém, ela também irá retornar valores negativos quando a sua entrada for um valor negativo, diferente da \textit{ReLU}, que iria retornar como saída zero. A constante $\alpha$, no texto original é dada por 0.1 fazendo com que os valores negativos sejam pequenos mas ainda sim, diferentes de zero quando passam pela entrada \parencite{LeakyReLUArticle} \footnote{Cabe destacar que, essa constante $\alpha$ pode ser ajustada para diferentes cenários, podendo ser valores diferentes de 0.1 como foram propostos no texto original, é possível ver isso acontecendo em comparativos ao longo desse capítulo, em que diferentes autores optam por valores diferentes de $\alpha$ para melhor ajustar ao problema que está sendo analisado.}. 

\begin{equacaodestaque}{\textit{Leaky ReLU} (\textit{LReLU})}
    \mathcal{A}_{\text{LReLU}}(y_j) = \begin{cases}y_j, & \text{se } y_j \ge 0 \\ \alpha \cdot y_j, & \text{se } y_j < 0\end{cases} \quad \text{ou} \quad \mathcal{A}_{\text{LReLU}}(y_j) = \max(0, \alpha y_j)
    \label{eq:leaky-relu}
\end{equacaodestaque}

Em que a constante $\alpha$ representa uma constante pré-definida pelo programador ao desenvolver a rede neural. Por padrão, em bibliotecas como o \textcite{PyTorchLeakyReLU} essa constante tem valor de 0.1.

Já para a sua representação gráfica, ela está presente na Figura \ref{fig:leaky-relu}. Perceba que a \textit{Leaky ReLU} possui características muito semelhantes com a \textit{ReLU}, como o fato dela assumir o comportamento de uma função identidade para para valores positivos em sua entrada, mas, quando é analisado os seus valores negativos é possível ver uma diferença, agora eles são dados por um gráfico de uma função do primeiro grau, diferente da \textit{ReLU} que era uma função constante em zero. Além disso, a \textit{LReLU}, também é uma função assimétrica e não linear, bem como apresenta um ponto de descontinuidade em zero, pois ao traçar os seus limites laterais, eles apresentam valores diferentes, por isso ela não pode ser derivada nesse ponto, assim como a \textit{ReLU} vista anteriormente.

\begin{figure}[h!]
    \centering
    \begin{tikzpicture}
        \begin{axis}[
            xlabel={$y_j$},
            ylabel={$\text{LReLU}(y_j)$},
            xmin=-2.3, xmax=2.3,
            ymin=-0.8, ymax=2.3,
            axis lines=middle,
            grid=major,
        ]
        % \leakyalpha é o comando que você definiu no preâmbulo (0.1)
        \addplot[blue, thick, domain=-2:2] {x > 0 ? x : 0.1*x};
        \end{axis}
    \end{tikzpicture}
    \caption{Gráfico da função de ativação \textit{Leaky ReLU} (\textit{LReLU}) com $\alpha = 0.1$.}
    \label{fig:leaky-relu}
    \fonte{O autor (2025).}
\end{figure}

\medskip
\begin{center}
 * * *
\end{center}
\medskip

\textbf{Características da Leaky ReLU}
\vspace{1em}

\begin{itemize}
    \item \textbf{Característica 1:}
    \item \textbf{Característica 2:}
    \item \textbf{Característica 3:}
\end{itemize}

\medskip
\begin{center}
 * * *
\end{center}
\medskip

Sabendo de sua expressão e seu gráfico, é possível agora calcular sua derivada, para isso, deve-se derivar as duas condicionais que estão na função da \textit{Leaky ReLU}. Assim, quando a entrada dessa função for maior que zero, essa função será $x$ que derivada é 1, já quando a entrada for menor que zero, a função será $\alpha x$, que quando derivada tem como resultado a própria constante $\alpha$. Contudo, como dito anteriormente, a derivada da \textit{Leaky ReLU} não existe quando a entrada é exatamente zero, mas na prática, ao trabalhar com a sua definição na retropropagação, é possível definir um valor para a derivada nesse ponto, assim como foi feito com a \textit{ReLU} tradicional. Com isso em mente, tem-se então a Equação \ref{eq:leaky-relu-derivada}, a qual representa a derivada da \textit{Leaky ReLU}.

\begin{equacaodestaque}{\textit{Leaky ReLU} (\textit{LReLU} Derivada)}
    \frac{d}{dy_j} [\mathcal{A}_{LReLU}](y_j) = \begin{cases}1, & \text{se } y_j > 0 \\ \alpha, & \text{se } y_j \leqslant  0 \end{cases}
    \label{eq:leaky-relu-derivada}
\end{equacaodestaque}

Já que a sua derivada é conhecida, pode-se também plotar o seu gráfico, o qual é dado pela figura \ref{fig:leaky-relu-derivada}. Ele também é parecido com o gráfico da \textit{ReLU} visto anteriormente, sendo composto por duas retas constantes, para valores positivos ele retorna 1 (assim como a \textit{ReLU}), e para valores negativos ou nulos ele irá sempre retornar a constante $\alpha$, diferente da \textit{ReLU}, que iria retornar zero, indicando que neste caso o neurônio não está passando nenhuma informação na na retropropagação do gradiente. Por esse motivo que e a \textit{Leaky ReLU} tem esse nome, pois \textit{leaky} em inglês significa "vazamento", e neste caso, como a derivada dela é diferente de zero, mesmo quando a entrada for negativa, ela irá passar informações durante a retropropagação, isso permite que o neurônio não morra, como acontecia em algumas redes que faziam uso da \textit{ReLU} tradicional, e por esse motivo continue aprendendo pelo fato do gradiente continuar fluindo pela rede e consequentemente atualizando os pesos e vieses dos neurônios.

\begin{figure}[h!]
    \centering
    \begin{tikzpicture}
        \begin{axis}[
            xlabel={$y_j$},
            ylabel={$\text{LReLU}'(y_j)$},
            xmin=-2.3, xmax=2.3,
            ymin=-0.2, ymax=1.2,
            axis lines=middle,
            grid=major
        ]
        \def\alphaVal{0.1} % Define alpha for the derivative graph

        \addplot[red, thick, domain=-2:0] {\alphaVal};
        \addplot[red, thick, domain=0:2] {1};
        \addplot[red, only marks, mark=o, mark size=1.5pt] coordinates {(0,\alphaVal)};
        \addplot[red, only marks, mark=*, mark size=1.5pt] coordinates {(0,1)};
        \end{axis}
    \end{tikzpicture}
    \caption{Gráfico da derivada da função de ativação \textit{Leaky ReLU} (\textit{LReLU}) com $\alpha = 0.1$.}
    \label{fig:leaky-relu-derivada}
    \fonte{O autor (2025).}
\end{figure}

Conhecendo a \textit{leaky ReLU} e suas propriedades, é possível agora entender como essa função se comporta quando comparada com outras em testes.

Em testes de desempenho realizados por \textcite{LeakyReLUArticle} em seu trabalho, eles foram capazes de analisar como uma rede neural que faz uso dessa função pode performar quando comparada com a \textit{ReLU} tradicional e também com redes que fazem uso da tangente hiperbólica, esse comparativo pode ser visto na Tabela \ref{tab:leaky-relu-desempenho}. Note que as redes neurais que fizeram uso da \textit{Leaky ReLU} como função de ativação obtiveram melhores resultandos quando comparadas com as redes que utilizaram a \textit{ReLU} tradicional ou mesmo a tangente hiperbólica, veja que a rede que foi construída com 3 camadas utilizando a \textit{LReLU} foi capaz de obter a menor taxa de erro de palavra (\textit{WER}) no conjunto \textit{SWBD}, com 17.8\%, esse resultado 0.3 pontos percentuais menor quando comparado com uma mesma rede de três camadas que utilizou a \textit{ReLU} tradicional.

Além disso, ainda na tabela \ref{tab:leaky-relu-desempenho}, nas redes compostas por três camadas, a rede que fez uso da \textit{LReLU} também foi melhor que suas outras concorrentes que fizeram uso da \textit{ReLU} e da tanh, sendo capaz de ter a menor taxa de erro de palavra no conjunto de avaliação (\textit{EV}), com 24.3\%, uma diferença de 0.1 pontos percentuais quando comparada com a \textit{ReLU} de 24.4\%, já quando essa rede é comparada com a tangente hiperbólica, a diferença é ainda maior, sendo de 2.1 pontos percentuais, indicando que a \textit{LReLU} traz resultados melhores quando comparada com essas duas funções de ativação.

\begin{table}[ht]
    \centering
    \begin{threeparttable}
        \caption{Comparativo de Desempenho de Redes Neurais para Reconhecimento de Fala}
        \label{tab:leaky-relu-desempenho}
        \begin{tabular}{lccccc}
            \toprule
            \textbf{Modelo} & \textbf{Dev CrossEnt} & \textbf{Dev Acc (\%)} & \textbf{SWBD WER} & \textbf{CH WER} & \textbf{EV WER} \\
            \midrule
            
            GMM Baseline & N/A & N/A & 25.1 & 40.6 & 32.6 \\ 
            \addlinespace % Adiciona um espaço para separar o baseline dos outros
            2 Camadas Tanh  & 2.09 & 48.0 & 21.0 & 34.3 & 27.7 \\
            2 Camadas ReLU  & 1.91 & 51.7 & 19.1 & 32.3 & 25.7 \\
            2 Camadas LReLU & 1.90 & 51.8 & 19.1 & 32.1 & 25.6 \\ 
            \addlinespace
            3 Camadas Tanh  & 2.02 & 49.8 & 20.0 & 32.7 & 26.4 \\
            3 Camadas ReLU  & 1.83 & 53.3 & 18.1 & 30.6 & 24.4 \\
            3 Camadas LReLU & 1.83 & 53.4 & \textbf{17.8} & 30.7 & \textbf{24.3} \\ 
            \addlinespace
            4 Camadas Tanh  & 1.98 & 49.8 & 19.5 & 32.3 & 25.9 \\
            4 Camadas ReLU  & 1.79 & 53.9 & 17.3 & 29.9 & 23.6 \\
            4 Camadas LReLU & \textbf{1.78} & 53.9 & 17.3 & 29.9 & 23.7 \\
            
            \bottomrule
        \end{tabular}
        
        \begin{tablenotes}[para]
            \small
            \item[] Nota: Comparação de métricas de erro para sistemas de redes neurais profundas (DNN) em reconhecimento de fala. As métricas de quadro a quadro (frame-wise) foram avaliadas em um conjunto de desenvolvimento, e as taxas de erro de palavra (WER) no conjunto de avaliação Hub5 2000 e seus subconjuntos. Abreviações: Dev CrossEnt = Entropia Cruzada no conjunto de desenvolvimento; Dev Acc = Acurácia no conjunto de desenvolvimento; WER = Taxa de Erro de Palavra (Word Error Rate); SWBD = Switchboard; CH = CallHome; EV = Evaluation set. Valores em negrito indicam os melhores resultados para modelos de 3 e 4 camadas.
            \item[] Fonte: Adaptado de "Rectifier Nonlinearities Improve Neural Network Acoustic Models", por A. L. Maas, A. Y. Hannun, \& A. Y. Ng, 2013, \textit{In Proceedings of the 30th International Conference on Machine Learning, Workshop on Deep Learning for Audio, Speech and Language Processing}.
        \end{tablenotes}
        
    \end{threeparttable}
\end{table}

Agora comparando as redes que fazem uso de quatro camadas, cabe destacar os resultados da entropia cruzada no conjunto de desenvolvimento (\textit{Dev CrossEnt}), que é uma métrica responsável por medir a diferença entre duas distribuições de probabilidade, neste cenário: a distrubição de probabilidade prevista pelo modelo e a distribuição de probabilidade real, com base nesses dois valores, a entropia cruzada consegue medir o quão bem o modelo de rede neural criado pelos pesquisadores está prevendo a transcrição correta da fala durante a fase de treinamento e ajuste, para isso, é utilizado o conjunto de dados de desenvolvimento (\textit{Dev Set}). Tendo isso em mente, o modelo de quatro camadas que fez uso da \textit{Leaky ReLU} em sua arquitetura obteve o melhor resultado dos seus outros dois concorrentes, sendo assim, ele teve como resultado uma entropia cruzada de 1.78, 0.01 menor que o modelo que fez uso da \textit{ReLU} tradicional (que obteve 1.79) e 0.2 menor que o modelo que fez uso da tangente hiperbólica (que obteve 1.98).

Ainda no grupo de redes que possuem quatro camadas, é possível ver um empate ao analisar a acurácia no conjunto de desenvolvimento (\textit{Dev Acc}), que é uma métrica responsável por medir a proporção das previsões corretas feitas pelo modelo quando comparadas com o total de previsões feitas. Assim, perceba que na tabela \ref{tab:leaky-relu-desempenho}, as redes que fizeram uso tanto da \textit{ReLU} quanto da \textit{Leaky ReLU} obtiveram a mesma acurácia de 53.9\%, já quando comparadas com a tangente hiperbólica, é possível ver uma diferença de 4.1 pontos percentuais, indicando as funções retificadoras acabam sendo mais precisas para essa análise. Não somente elas são mais precisas, mas quando comparadas com a \textit{Leaky ReLU}, nota-se outros ganhos também, como menores taxas de erro de palavra (\textit{WER}) tanto no conjunto \textit{Switchboard} (\textit{SWBD}) quanto no conjunto de avaliação (\textit{EV}).

\medskip
\begin{center}
 * * *
\end{center}
\medskip

\textbf{Algumas Aplicações da Leaky ReLU em Redes Neurais} \index{Aplicações práticas! Leaky ReLU (LReLU)}
\vspace{1em}

\begin{itemize}
    \item \textbf{Aplicação 1 (Área):}
    \item \textbf{Aplicação 2 (Área):}
    \item \textbf{Aplicação 3 (Área):}
    \item \textbf{Aplicação 4 (Área):}
\end{itemize}

\medskip
\begin{center}
 * * *
\end{center}
\medskip

Assim, a \textit{Leaky ReLU} já é uma evolução quando comparamos com a ReLU tradicional, mas, é possível ir além e encontrar funções ainda mais complexas que também buscam assim como a \textit{LReLU} resolver o problema dos \textit{ReLUs} agonizantes com um vazamento de gradiente nos casos negativos, uma dessas funções é a \textit{PReLU}, a qual será vista em seguida.

\subsection{Parametric ReLU (PReLU)} \index{Funções de Ativação!Parametric ReLU (PReLU)}

Continuando nas analogias do vendedor de pipoca para explicar as funções retificadoras, podemos também pensar uma para a \textit{Parametric ReLU}. Na \textit{LReLU} nós tínhamos uma constante fixa, que valia para todos os valores de entrada e não mudava, era como se o vendedor de pipoca definisse um valor para a proporção de pipoca que a pessoa irá receber quando tiver com uma quantia menor de dinheiro que o limiar da venda. Mas agora, este vendedor está mais experiente, e sabe que pode ajustar essa constante sempre que quiser, assim, quando estiverem muitas pessoas na praça em que está vendendo pipoca, ele poderá colocar uma constante que será capaz de dar uma quantidade ainda maior de pipoca para aqueles que não possuem o valor total de um pacote, o que incentivaria a venda para as pessoas. Já quando estivesse em um lugar mais vazio, colocaria uma constante que daria menos pipoca, para maximizar o seu lucro. A Diferença da \textit{PReLU} para a \textit{LReLU} está justamente nessa constante e como ela irá se comportar.

Proposta por \textcite{PReLUArticle} no artigo \textit{Delving Deep into Rectifiers: Surpassing Human Level Performance on Image Net Classification}, a \textit{PReLU} surgiu como uma variação não somente da \textit{ReLU}, mas também uma evolução da \textit{Leaky ReLU} que foi vista anteriormente, isso ocorre, pois diferente da \textit{LReLU} que possuía uma constante $\alpha$ fixa que multiplicava o valor da entrada nos casos negativos, a \textit{PReLU} trás essa mesma constante, mas neste caso ela é adaptável, se ajustando as particularidades de cada problema que uma rede neural está tentando resolver. Assim, a \textit{PReLU} é como o pipoqueiro mais experiente, que ajusta como vai vender o seu punhado de pipoca em cada uma das situações para poder maximizar os seus lucros mas ao mesmo tempo garantir mais clientes para si.

A fórmula matemática da \textit{PReLU} é dada pela Equação \ref{eq:prelu}. Como explicam \textcite{PReLUArticle}, a \textit{PReLU} generaliza a tradicional \textit{ReLU}, além de melhorar o \textit{model fitting} apresentando quase nenhum custo computacional extra e com um baixo risco de sobreajuste (\textit{overfitting}). Este coeficiente $\alpha$, que ela apresenta assim como a \textit{Leaky ReLU} que foi visto anteriormente, é aprendível, e não uma constante fixa, isso indica ser otimizado utilizando a retropropagação do gradiente de forma simultânea com as outras camadas da rede neural criada \parencite{PReLUArticle}. Assim, por esse fato tem-se uma melhor eficiência no aprendizado e no tempo da rede, dado que não precisamos criar uma nova etapa só para ajustar os valores de \textit{alpha} das camadas densas que fazem o uso da \textit{Parametric ReLU}.

\begin{equacaodestaque}{\textit{Parametric ReLU} (\textit{PReLU})}
    \mathcal{A}_{\text{PReLU}}(y_j) = \begin{cases}y_j, & \text{se } y_j \ge 0 \\ \alpha_i \cdot y_j, & \text{se } y_j < 0\end{cases} \quad \text{ou} \quad \mathcal{A}_{\text{PReLU}}(y_j) = \max(0, \alpha y_j)
    \label{eq:prelu}
\end{equacaodestaque}

Com base em sua equação, tem-se o gráfico da \textit{PReLU} na Figura \ref{fig:prelu}, note que caso este coeficiente for igual a zero, nos temos então a \textit{ReLU} tradicional, já quando ele for igual a 0.1, tem-se a \textit{LReLU}, no gráfico $\alpha$ está com valor de 0.2, mas ele irá variar conforme a rede aprende e para cada problema, podendo apresentar diferentes valores em diferentes situações. Com isso, é possível perceber que a \textit{PReLU} é composta por duas funções do primeiro grau, sendo assimétrica, e também apresentando um ponto de descontinuidade em zero, o que impede de ser derivada neste ponto.

\begin{figure}[h!]
    \centering
    \begin{tikzpicture}
        \begin{axis}[
            xlabel={$y_j$},
            ylabel={$\text{PReLU}(y_j)$},
            xmin=-2.3, xmax=2.3,
            ymin=-0.5, ymax=2.3,
            axis lines=middle,
            grid=major,
        ]
        % Define um valor exemplo de alpha para o gráfico
        \def\alphaVal{0.2} 
        \addplot[blue, thick, domain=-2:2] {x >= 0 ? x : \alphaVal*x};
        \end{axis}
    \end{tikzpicture}
    \caption{Gráfico da função de ativação \textit{Parametric ReLU} (\textit{PReLU}) com $\alpha=0.2$.}
    \label{fig:prelu}
    \fonte{O autor (2025).}
\end{figure}

\medskip
\begin{center}
 * * *
\end{center}
\medskip

\textbf{Características da Parametric ReLU}
\vspace{1em}

\begin{itemize}
    \item \textbf{Característica 1:}
    \item \textbf{Característica 2:}
    \item \textbf{Característica 3:}
\end{itemize}

\medskip
\begin{center}
 * * *
\end{center}
\medskip

Conhecendo a sua fórmula e como ela se comporta, cabe também derivar a \textit{PReLU}, para isso, deve-se derivar cada uma das expressões dos condicionais de forma separada, assim como foi feito com a \textit{Leaky ReLU} e a \textit{ReLU} anteriormente. Como a fórmula da \textit{PReLU} é igual a \textit{LReLU}, podemos apenas nos lembrar dela e citar a Equação \ref{eq:prelu-derivada} como sua derivada. Note que, essa derivada também não existe quando a entrada é exatamente zero, mas assim como na \textit{Leaky ReLU}. É possível "corrigir" isso dizendo que ela será igual a $\alpha$ nesse ponto, para que o gradiente continue fluindo durante o \textit{backward pass}. Vale lembrar, que essa reta em $\alpha$ irá variar conforme a rede neural aprende as características e se ajusta ao problema que está tentando resolver.

\begin{equacaodestaque}{\textit{Parametric ReLU} (\textit{PReLU} Derivada)}
    \frac{d}{dy_j} [\mathcal{A}_{PReLU}](y_j) = \begin{cases}1, & \text{se } y_j > 0 \\ \alpha_i, & \text{se } y_j \le 0 \end{cases}
    \label{eq:prelu-derivada}
\end{equacaodestaque}

Sabendo a fórmula da derivada da \textit{PReLU}, é possível plotar o seu gráfico, o qual é dado pela Figura \ref{fig:prelu-derivada}, ele é semelhante ao gráfico da \textit{Leaky ReLU} visto na seção anterior, mas agora, a constante $\alpha$ está com valor em 0.2. Note que, o gráfico da derivada é também muito simples, sendo apenas duas funções constantes, uma que vale 1 para os valores de entrada positivos e outra que irá valer 0,2 para os outros valores. Assim, a \textit{PReLU} consegue manter a simplicidade da \textit{ReLU}, mas ao mesmo tempo fazendo pequenos ajustes garantindo melhorias de desempenho em redes mais profundas, como as que foram apresentas por \textcite{PReLUArticle}.

\begin{figure}[h!]
    \centering
    \begin{tikzpicture}
        \begin{axis}[
            xlabel={$y_j$},
            ylabel={$\text{PReLU}'(y_j)$},
            xmin=-2.3, xmax=2.3,
            ymin=-0.2, ymax=1.2,
            axis lines=middle,
            grid=major
        ]
        % Define um valor exemplo de alpha para o gráfico
        \def\alphaVal{0.2}

        \addplot[red, thick, domain=-2:0] {\alphaVal};
        \addplot[red, thick, domain=0:2] {1};
        \addplot[red, only marks, mark=o, mark size=1.5pt] coordinates {(0,\alphaVal)};
        \addplot[red, only marks, mark=*, mark size=1.5pt] coordinates {(0,1)};
        \end{axis}
    \end{tikzpicture}
    \caption{Gráfico da derivada da função de ativação \textit{Parametric ReLU} (\textit{PReLU}) com $\alpha=0.2$.}
    \label{fig:prelu-derivada}
    \fonte{O autor (2025).}
\end{figure}

Ainda em \textit{Delving Deep into Rectifiers: Surpassing Human Level Performance on Image Net Classification}, os autores realizam testes comparando a \textit{ReLU} tradicional com a \textit{PReLU} utilizando como base o \textit{dataset} de 1000 classes do \textit{ImageNet} 2012, o qual contêm cerca de 1.2 milhões de imagens de treino, 50.000 imagens de validação e 100.000 imagens de teste sem rótulos publicados \parencite{PReLUArticle}. Para isso, \textcite{PReLUArticle} criaram três modelos diferentes (A, B e C), baseados na arquitetura \textit{VGG-16} mas com variações entre si como diferentes números de camadas convolucionais e consequentemente complexidades distintas para cada algoritmo. Esses modelos podem ser vistos na Tabela \ref{tab:arquitetura-prelu}. 

\begin{table}
    \centering
    \begin{threeparttable}
        \caption{Arquiteturas dos modelos grandes}
        \label{tab:arquitetura-prelu}
        \begin{tabular}{cllll}
            \toprule
            \textbf{Input Size} & \textbf{VGG-19 [25]} & \textbf{Model A} & \textbf{Model B} & \textbf{Model C} \\
            \midrule
            224
              & \makecell[l]{3$\times$3, 64 \\ 3$\times$3, 64 \\ \addlinespace 2$\times$2 maxpool, /2}
              & 7$\times$7, 96, /2
              & 7$\times$7, 96, /2
              & 7$\times$7, 96, /2 \\
            \midrule
            112
              & \makecell[l]{3$\times$3, 128 \\ 3$\times$3, 128 \\ \addlinespace 2$\times$2 maxpool, /2}
              & 2$\times$2 maxpool, /2
              & 2$\times$2 maxpool, /2
              & 2$\times$2 maxpool, /2 \\
            \midrule
            56
              & \makecell[l]{3$\times$3, 256 \\ 3$\times$3, 256 \\ 3$\times$3, 256 \\ 3$\times$3, 256 \\ \addlinespace 2$\times$2 maxpool, /2}
              & \makecell[l]{3$\times$3, 256 \\ 3$\times$3, 256 \\ 3$\times$3, 256 \\ 3$\times$3, 256 \\ \addlinespace 2$\times$2 maxpool, /2}
              & \makecell[l]{3$\times$3, 256 \\ 3$\times$3, 256 \\ 3$\times$3, 256 \\ 3$\times$3, 256 \\ \addlinespace 2$\times$2 maxpool, /2}
              & \makecell[l]{3$\times$3, 384 \\ 3$\times$3, 384 \\ 3$\times$3, 384 \\ 3$\times$3, 384 \\ \addlinespace 2$\times$2 maxpool, /2} \\
            \midrule
            28
              & \makecell[l]{3$\times$3, 512 \\ 3$\times$3, 512 \\ 3$\times$3, 512 \\ 3$\times$3, 512 \\ \addlinespace 2$\times$2 maxpool, /2}
              & \makecell[l]{3$\times$3, 512 \\ 3$\times$3, 512 \\ 3$\times$3, 512 \\ 3$\times$3, 512 \\ \addlinespace 2$\times$2 maxpool, /2}
              & \makecell[l]{3$\times$3, 512 \\ 3$\times$3, 512 \\ 3$\times$3, 512 \\ 3$\times$3, 512 \\ \addlinespace 2$\times$2 maxpool, /2}
              & \makecell[l]{3$\times$3, 768 \\ 3$\times$3, 768 \\ 3$\times$3, 768 \\ 3$\times$3, 768 \\ \addlinespace 2$\times$2 maxpool, /2} \\
            \midrule
            14
              & \makecell[l]{3$\times$3, 512 \\ 3$\times$3, 512 \\ 3$\times$3, 512 \\ 3$\times$3, 512 \\ \addlinespace 2$\times$2 maxpool, /2}
              & \makecell[l]{3$\times$3, 512 \\ 3$\times$3, 512 \\ 3$\times$3, 512 \\ 3$\times$3, 512 \\ \addlinespace spp, \{7, 3, 2, 1\}}
              & \makecell[l]{3$\times$3, 512 \\ 3$\times$3, 512 \\ 3$\times$3, 512 \\ 3$\times$3, 512 \\ \addlinespace spp, \{7, 3, 2, 1\}}
              & \makecell[l]{3$\times$3, 896 \\ 3$\times$3, 896 \\ 3$\times$3, 896 \\ 3$\times$3, 896 \\ \addlinespace spp, \{7, 3, 2, 1\}} \\
            \midrule
            fc\textsubscript{1} & \multicolumn{4}{l}{4096} \\
            fc\textsubscript{2} & \multicolumn{4}{l}{4096} \\
            fc\textsubscript{3} & \multicolumn{4}{l}{1000} \\
            \midrule
            depth (conv+fc) & 19 & 19 & 22 & 22 \\
            complexity (ops., $\times 10^{10}$) & 1.96 & 1.90 & 2.32 & 5.30 \\
            \bottomrule
        \end{tabular}
        
        \begin{tablenotes}[para]
            \small
            \item[] Nota: Comparação detalhada das arquiteturas de rede. A notação "/2" denota um stride de 2. Cada coluna representa um modelo diferente. A tabela descreve a sequência de camadas convolucionais (no formato \texttt{kernel $\times$ kernel, n° de filtros}), de pooling e totalmente conectadas (fc). Os modelos A, B e C são variações da estrutura VGG-19, propostas pelos autores para avaliar a função de ativação PReLU. As linhas finais comparam a profundidade total (conv+fc) e a complexidade computacional de cada modelo.
            \item[] Fonte: Adaptado de "Delving Deep into Rectifiers: Surpassing Human-Level Performance on ImageNet Classification", por K. He, X. Zhang, S. Ren, \& J. Sun, 2015, \textit{Proceedings of the IEEE International Conference on Computer Vision (ICCV)}, pp. 1026-1034.
        \end{tablenotes}
        
    \end{threeparttable}
\end{table}

 Com isso, é possível ver na Tabela \ref{tab:prelu-desempenho} as métricas utilizadas pelos autores. É medido o erro \textit{Top-1} e o erro \textit{Top-5}, o erro \textit{Top-1} mostrando quão preciso é o modelo em seu melhor chute, já o erro \textit{Top-5} mostra se a resposta correta estava entre os top 5 melhores chutes feito pelo modelo. Assim, conhecendo esses parâmetros de medida, é possível chegar na conclusão de que quanto menor esses valores, melhor o modelo está performando, seguindo essa lógica, nota-se que todos os modelos que fazem uso de funções retificadoras, seja a \textit{PReLU} ou mesmo a \textit{ReLU} tradicional, são capazes de performar melhor que os modelos \textit{VGG-16} e \textit{GoogleLet} que fazem uso de outras funções de ativação.

\begin{table}[ht]
    \centering
    \begin{threeparttable}
        \caption{Resultados de Erro Top-1 e Top-5 no Conjunto de Validação ImageNet 2012}
        \label{tab:prelu-desempenho}
        \begin{tabular}{lcc}
            \toprule
            \textbf{Modelo} & \textbf{Erro Top-1 (\%)} & \textbf{Erro Top-5 (\%)} \\
            \midrule
            
            MSRA       & 29.68 & 10.95 \\
            VGG-16     & 28.07\textsuperscript{a} & 9.33 \\
            GoogleLeNet&  -    & 9.15 \\
            \addlinespace
            A, ReLU    & 26.48 & 8.59 \\
            A, PReLU   & 25.59 & 8.23 \\
            B, PReLU   & 25.53 & 8.13 \\
            C, PReLU   & \textbf{24.27} & \textbf{7.38} \\ 
            
            \bottomrule
        \end{tabular}
        
        \begin{tablenotes}[para]
            \small
            \item[] Nota: Resultados de erro (\%) para um único modelo com a técnica de 10-view no conjunto de validação do ImageNet 2012. Os modelos A, B e C são variações da arquitetura VGG-16 modificada pelos autores. Os valores em negrito indicam os melhores resultados. \textsuperscript{a}Resultado baseado em testes realizados pelos autores do artigo original.
            \item[] Fonte: Adaptado de "Delving Deep into Rectifiers: Surpassing Human-Level Performance on ImageNet Classification", por K. He, X. Zhang, S. Ren, \& J. Sun, 2015, \textit{Proceedings of the IEEE International Conference on Computer Vision (ICCV)}, pp. 1026-1034.
        \end{tablenotes}

    \end{threeparttable}
\end{table}

Ao comparar o modelo C, que faz uso da \textit{PReLU} e possui mais camadas convolucionais, com o modelo A que faz uso da \textit{ReLU}, é possível notar uma diferença de 2,21 pontos percentuais no erro \textit{Top-1}, já ao analisar o erro \textit{Top-5}, essa diferença é de 1,21 pontos percentuais, o que indica que a \textit{PReLU} é capaz de trazer melhores resultados quando comparada com redes que fazem uso da \textit{ReLU} tradicional. Já ao comparar com o \textit{VGG-16}, essa diferença de desempenho é ainda maior, sendo de 3,8 pontos percentuais no erro \textit{Top-1} e 1,95 pontos percentuais no erro \textit{Top-5}, note que a \textit{VGG-16}, a qual é indicada na Tabela \ref{tab:arquitetura-prelu} possui bem menos camadas convolucionais que o modelo C, é possível notar notar também a sua complexidade computacional, que é menos da metade da do modelo C, isso indica que a \textit{PReLU}, por ser uma função não saturante e consequentemente corrigir o problema do desaparecimento do gradiente, é capaz de criar redes neurais que se beneficiam melhor com uma maior profundidade, sendo capazes de extrair mais informações e com isso performar melhor.

Um feito importante que deve ser destacado sobre a \textit{PReLU}, que inclusive é o nome do artigo que foi responsável por introduzir essa função para a comunidade científica, é de que durante a pesquisa do texto \textit{Delving Deep into Rectifiers: Surpassing Human Level Performance on Image Net Classification}, \textcite{PReLUArticle} foram capazes de criar uma rede neural capaz de superar a capacidade humana de reconhecer diferentes conjuntos de imagens no \textit{ImagneNet} Classification, isso ocorreu porque a um humano ao analisar o \textit{ImageNet} apresenta uma taxa de erro \textit{Top-5} de 5.1\% e em um dos testes realizados pelos autores, uma rede neural alcançou uma taxa de erro \textit{Top-5} de 4.94\%, superando assim a capacidade humana de reconhecimento de imagens. 

Assim, é nítido destacar que não somente a \textit{PReLU}, mas as funções retificadoras de forma geral, foram capazes de trazer melhorias significativas para os modelos de aprendizado profundo quando comparadas com as funções sigmoides, as quais foram o padrão da indústria por muitos anos. De fato, a resolução do problema do desaparecimento do gradiente, possibilitou a criação de redes neurais ainda mais profundas, e com isso, sendo capazes de extrair mais informações e consequente melhores métricas, sendo capazes até de superar a capacidade humanas em algumas tarefas como no artigo de introdução da \textit{PReLU}.

\medskip
\begin{center}
 * * *
\end{center}
\medskip

\textbf{Algumas Aplicações da Parametric ReLU em Redes Neurais} \index{Aplicações práticas! Parametric ReLU (PReLU)}
\vspace{1em}

\begin{itemize}
    \item \textbf{Aplicação 1 (Área):}
    \item \textbf{Aplicação 2 (Área):}
    \item \textbf{Aplicação 3 (Área):}
    \item \textbf{Aplicação 4 (Área):}
\end{itemize}

\subsection{Randomized Leaky ReLU (RReLU)} \index{Funções de Ativação!Randomized Leaky ReLU (RReLU)}

Anteriormente, na \textit{Parametric ReLU}, existia uma padrão aprendível que era atualizado ao longo da retropropagação da rede, nós o comparamos com o caso do vendedor de pipoca ficando mais experiente para as vendas. No caso da \textit{RReLU} temos um vendedor um tanto quanto instável, ele se baseia na sorte/aleatoriedade para definir qual será o punhado de pipoca que cada pessoa irá receber ao comprar com um valor abaixo do limiar de venda. Para isso, não existe mais um padrão, algo como o horário ou a quantidade de pessoas na praça para fazer aumentar o diminuir a quantidade de pipoca que tera em um punhado. Essa estratégia parece caótica, mas pode ser interessante caso voce queira instigar as vendas e deixa-las divertidas, você pode comprar um punhado de pipoca, mas nao ira saber quanto irá receber, é uma grande aposta.

Segundo \textcite{XuRReLU}, em \textit{Empirical Evaluation of Rectified Activations in Convolutional Network}, a \textit{Randomized Leaky ReLU} foi proposta pela primeira vez em uma competição do \textit{Kaggle NDSB}, ela era uma função semelhante a \textit{leaky ReLU} mas que o seu coeficiente $\alpha$ é um número aleatório dado pela distribuição normal da forma $U(l, u)$, nessa mesma competição os valores escolhidos para essa distribuição foram de $U(3, 8)$.

A fórmula da \textit{RReLU} é dada pela Equação \ref{eq:rrelu}, note que é a mesma expressão da \textit{Leaky ReLU} e da \textit{PReLU}, mas o que muda é o significado do termo $\alpha$ em cada uma delas. Neste caso: $\alpha \sim U (l, u)$ em que $l < u$  e $l, u \in [0, 1)$ 

\begin{equacaodestaque}{\textit{Randomized Leaky ReLU} (\textit{RReLU})}
    \mathcal{A}_{\text{RReLU}}(y_j) = \begin{cases} y_j, & \text{se } y_j > 0 \\ \alpha_i y_j, & \text{se } y_j \leq 0 \end{cases} \quad \text{ou} \quad \mathcal{A}_{\text{RReLU}}(y_j) = \max(0, \alpha y_j)
    \label{eq:rrelu}
\end{equacaodestaque}

Na fase de testes, deve-se calcular a média de todos os valores de $\alpha$ durante o treino, e com isso $\alpha$ se torna uma constante fixa do tipo $(l+u)/2$ de forma que com isso seja possível obter um resultado determinístico, no artigo, os autores utilizam a fórmula \ref{eq:equacao-rrelu-teste} para calcular a \textit{RReLU} durante o teste do modelo \parencite{XuRReLU}.

\begin{equation}
    \mathcal{A}_{\text{RReLU}}(y_j) = \frac{y_j}{\frac{l + u}{2}}
    \label{eq:equacao-rrelu-teste}
\end{equation}

O gráfico da \textit{RReLU} está presente na Figura \ref{fig:rrelu}. Na representação, é possível ver várias retas, isso ocorre pois elas irão variar de caso a caso, e como a \textit{RReLU} é uma função que utiliza de conceitos probabilísticos, não é possível garantir um gráfico exato de como ela seria pois não temos os valores de $\alpha$ até que a distribuição seja feita. Note que mesmo com essa particularidade, ela ainda continua sendo uma função bem simples, sendo a construção de duas retas originarias de equações do primeiro grau, a primeira delas sendo a própria função identidade, para os casos em que a entrada é positiva, e a outra é dada pela variável da entrada multiplicada pela constante $\alpha$, para os casos em que a saída é negativa. Além disso, deve-se atentar também para a sua descontinuidade no ponto zero, é o mesmo problema que acontece com outras variantes, como a \textit{ReLU} e a \textit{Leaky ReLU}.

\begin{figure}[h!]
    \centering
    \begin{tikzpicture}
        \begin{axis}[
            xlabel={$y_j$},
            ylabel={$\text{RReLU}(y_j)$},
            xmin=-2.3, xmax=2.3,
            ymin=-0.8, ymax=2.3,
            axis lines=middle,
            grid=major,
            legend pos=north west,
            legend style={font=\tiny}
        ]
        \addplot[blue, thick, domain=0:2.3] {x};
        \addplot[red, dashed, domain=-2.3:0, samples=2] {0.1*x};
        \addplot[red, dashed, domain=-2.3:0, samples=2] {0.25*x};
        \addplot[red, dashed, domain=-2.3:0, samples=2] {0.4*x};
        \end{axis}
    \end{tikzpicture}
    \caption{Gráfico da função de ativação \textit{Randomized Leaky ReLU} (\textit{RReLU}) com diferentes inclinações aleatórias para a parte negativa.}
    \label{fig:rrelu}
    \fonte{O autor (2025).}
\end{figure}

\medskip
\begin{center}
 * * *
\end{center}
\medskip

\textbf{Características da Randomized Leaky ReLU}
\vspace{1em}

\begin{itemize}
    \item \textbf{Característica 1:}
    \item \textbf{Característica 2:}
    \item \textbf{Característica 3:}
\end{itemize}

\medskip
\begin{center}
 * * *
\end{center}
\medskip

Conhecendo como a \textit{RReLU} se comporta, é possível também calcular a sua derivada, a qual será de extrema utilidade durante a retropropagação, fazendo que os pesos e vieses do modelo sejam ajustados e com base nisso ele consiga aprender melhor o problema que está sendo analisado. Como a \textit{RReLU} utiliza a mesma fórmula que funções como a \textit{Leaky ReLU} e a \textit{PReLU}, pode-se apenas repetir a expressão de sua derivada novamente, a qual será dada pela Equação \ref{eq:rrelu-derivada}. Note que mesmo compartilhando a mesma fórmula, o termo $\alpha$ possui significado distintos em cada uma dessas funções, neste caso, ele é um valor aleatório dado pela distribuição $U(l, u)$.

Com relação ao problema da descontinuidade no ponto zero, é possível apenas escolher para qual valor essa função irá retornar neste caso, assim, vamos considerar que quando a sua entrada for zero, ela irá retornar o segundo caso, em que é a própria constante $\alpha$

\begin{equacaodestaque}{\textit{Randomized Leaky ReLU} (\textit{RReLU}) Derivada}
    \frac{d}{dy_j} [\mathcal{A}_{\text{RReLU}}](y_j) = \begin{cases}1, & \text{se } y_j > 0 \\ \alpha_i, & \text{se } y_j \leqslant  0 \end{cases}
    \label{eq:rrelu-derivada}
\end{equacaodestaque}

Esse detalhe da aleatoriedade da constante $\alpha$ afetou o desenho do gráfico da \textit{RReLU}, e com isso, ele também afeta a plotagem de sua derivada. Como pode ser visto na Figura \ref{fig:rrelu-derivada}, existem um conjunto de retas em um intervalo, neste caso, está sendo considerado a distribuição como sendo de $U(0.1, 0.3)$, mas para cada uma dessas distribuições, terá um conjunto de retas diferentes e com isso gráficos distintos para cada um dos problemas.

\begin{figure}[h!]
    \centering
    \begin{tikzpicture}
        \begin{axis}[
            xlabel={$y_j$},
            ylabel={$\text{RReLU}'(y_j)$},
            xmin=-2.3, xmax=2.3,
            ymin=-0.2, ymax=1.2,
            axis lines=middle,
            grid=major,
            legend pos=north west,
            legend style={font=\scriptsize}
        ]
        % Define l e u para a distribuição uniforme U(l,u)
        \def\lVal{0.1}
        \def\uVal{0.3}

        % Plota a derivada para z > 0
        \addplot[red, thick, domain=0:2.1] {1};
        \addlegendentry{$f'(y_j) = 1$}

        % Plota a região hachurada para z < 0
        \addplot[
            pattern=north east lines, 
            pattern color=blue!50,
            draw=none
        ] coordinates {(-2.1, \lVal) (0, \lVal) (0, \uVal) (-2.1, \uVal)} -- cycle;
        \addlegendentry{$\alpha_i \sim U(l,u)$}

        % Marcadores na descontinuidade
        \addplot[blue, only marks, mark=o, mark size=1.5pt, forget plot] coordinates {(0,\lVal)};
        \addplot[blue, only marks, mark=o, mark size=1.5pt, forget plot] coordinates {(0,\uVal)};
        \addplot[red, only marks, mark=*, mark size=1.5pt, forget plot] coordinates {(0,1)};
        
        \end{axis}
    \end{tikzpicture}
    \caption{Gráfico da derivada da função de ativação \textit{Randomized Leaky ReLU} (\textit{RReLU}) com $l=0.1, u=0.3$.}
    \label{fig:rrelu-derivada}
    \fonte{O autor (2025).}
\end{figure}

Ainda no artigo, \textcite{XuRReLU} investigam a performance de diferentes funções retificadoras em uma rede neural convolucional, cuja arquitetura pode ser vista na Tabela \ref{tab:nin_arquitetura}, para a classificação de imagens, os autores analisaram a \textit{ReLU} tradicional, a \textit{Leaky ReLU}, a \textit{Parametric ReLU} e a \textit{Randomized Leaky ReLU}. Para fazer a análise dos modelos, foram escolhidos os datasets \textit{CIFAR-10} e \textit{CIFAR-100}, e o desempenho dessas funções nos respectivos datasets pode ser visto na Tabela \ref{tab:rrelu-cifar-10} e na Tabela \ref{tab:rrelu-cifar-100}.

\begin{table}[ht]
    \centering
    \begin{threeparttable}
        \caption{Estrutura da Rede "Network in Network" (NIN) para CIFAR-10/100}
        \label{tab:nin_arquitetura}
        \begin{tabular}{ll}
            \toprule
            \textbf{Tamanho da Entrada} & \textbf{Camada / Operação} \\
            \midrule
            
            $32 \times 32$ & Conv 5x5, 192 canais \\
            $32 \times 32$ & Conv 1x1, 160 canais \\
            $32 \times 32$ & Conv 1x1, 96 canais \\
            $32 \times 32$ & Max Pooling 3x3, stride /2 \\
            \addlinespace % Adiciona espaço para separar os blocos
            
            $16 \times 16$ & Dropout, taxa 0.5 \\
            $16 \times 16$ & Conv 5x5, 192 canais \\
            $16 \times 16$ & Conv 1x1, 192 canais \\
            $16 \times 16$ & Conv 1x1, 192 canais \\
            $16 \times 16$ & Avg Pooling 3x3, stride /2 \\
            \addlinespace
            
            $8 \times 8$ & Dropout, taxa 0.5 \\
            $8 \times 8$ & Conv 3x3, 192 canais \\
            $8 \times 8$ & Conv 1x1, 192 canais \\
            $8 \times 8$ & Conv 1x1, 10 ou 100 canais\textsuperscript{a} \\
            $8 \times 8$ & Global Avg Pooling 8x8 \\
            \addlinespace
            
            10 ou 100 & Softmax \\
            
            \bottomrule
        \end{tabular}
        
        \begin{tablenotes}[para]
            \small
            \item[] Nota: Descrição da arquitetura da rede convolucional "Network in Network" (NIN). A coluna "Tamanho da Entrada" indica a dimensão espacial dos mapas de características em cada estágio. A coluna "Camada / Operação" detalha a sequência de operações da rede. \textsuperscript{a}O número de canais na última camada convolucional corresponde ao número de classes do dataset (10 para CIFAR-10 ou 100 para CIFAR-100), funcionando como uma etapa de classificação antes do Global Average Pooling.
            \item[] Fonte: Adaptado de "Empirical Evaluation of Rectified Activations in Convolutional Network", por B. Xu, N. Wang, T. Chen, \& M. Li, 2015, \textit{arXiv preprint arXiv:1505.00853}.
        \end{tablenotes}
        
    \end{threeparttable}
\end{table}

Analisando a Tabela \ref{tab:rrelu-cifar-10}, que mostra a taxa de erro das funções retificadoras na rede \textit{NIN} para o \textit{dataset CIFAR-10}, é possível notar que a \textit{Randomized Leaky ReLU} foi a função que performou melhor, com um total de 11.19\% de erro nos casos de teste, enquanto a \textit{leaky ReLU} com $a = 100$ obteve o pior resultado. Contudo, essa diferença de resultado é pequena, indicando que caso essas funções sejam muito mais complexas quando comparadas com a \textit{ReLU} na hora de treinar o modelo, pode ser melhor optar por uma função mais "barata" mas com uma taxa de erro um pouco maior.

\begin{table}[ht]
    \centering
    \begin{threeparttable}
        \caption{Taxa de Erro de Diferentes Funções de Ativação na Rede NIN para o CIFAR-10}
        \label{tab:rrelu-cifar-10}
        \begin{tabular}{lcc}
            \toprule
            \textbf{Função de Ativação} & \textbf{Erro de Treino} & \textbf{Erro de Teste (\%)} \\
            \midrule
            
            ReLU                      & 0.00318 & 12.45 \\
            \addlinespace % Adiciona um espaço para agrupar as Leaky ReLUs
            Leaky ReLU ($a=100$)      & 0.00310 & 12.66 \\
            Leaky ReLU ($a=5.5$)      & 0.00362 & 11.20 \\
            \addlinespace
            PReLU                     & 0.00178 & 11.79 \\
            \addlinespace
            RReLU\textsuperscript{a}  & 0.00550 & \textbf{11.19} \\
            
            \bottomrule
        \end{tabular}
        
        \begin{tablenotes}[para]
            \small
            \item[] Nota: Comparação da taxa de erro da rede "Network in Network" (NIN) treinada no conjunto de dados CIFAR-10 com diferentes funções de ativação retificadoras. Os valores de erro de teste são apresentados em porcentagem (\%). O valor em negrito indica o melhor resultado (menor erro de teste). \textsuperscript{a}Randomized Leaky ReLU (RReLU), onde o coeficiente de vazamento é amostrado de uma distribuição uniforme durante o treino. A fórmula na tabela original representa uma parametrização específica testada.
            \item[] Fonte: Adaptado de "Empirical Evaluation of Rectified Activations in Convolutional Network", por B. Xu, N. Wang, T. Chen, \& M. Li, 2015, \textit{arXiv preprint arXiv:1505.00853}.
        \end{tablenotes}

    \end{threeparttable}
\end{table}

Já ao analisar a Tabela \ref{tab:rrelu-cifar-100}, que mostra a taxa de erro dessas funções no \textit{dataset CIFAR-10}, é possível ver que assim como no \textit{CIFAR-10}, a \textit{RReLU} foi a função que obteve melhor resultado, neste caso há uma diferença de 2.65 pontos percentuais, quando comparada com a \textit{ReLU} tradicional. Outro ponto interessante a ser destacado ao analisar essa tabela é de que provavelmente a rede que utilizou a \textit{Parametric ReLU} sofreu um sobreajuste (\textit{overfitting}) fazendo com que ela decorasse o dados de treino e com isso conseguisse uma taxa de erro consideravelmente menor, já quando ela foi apresentada para o conjunto de testes houve uma grande disparidade das taxas de erro.

\begin{table}[ht]
    \centering
    \begin{threeparttable}
        \caption{Taxa de Erro de Funções de Ativação na Rede NIN para o CIFAR-100}
        \label{tab:rrelu-cifar-100}
        \begin{tabular}{lcc}
            \toprule
            \textbf{Função de Ativação} & \textbf{Erro de Treino (\%)} & \textbf{Erro de Teste (\%)} \\
            \midrule
            
            ReLU                      & 13.56 & 42.90 \\
            \addlinespace
            Leaky ReLU ($a=100$)      & 11.55 & 42.05 \\
            Leaky ReLU ($a=5.5$)      & 8.54  & 40.42 \\
            \addlinespace
            PReLU                     & \textbf{6.33}  & 41.63 \\
            \addlinespace
            RReLU\textsuperscript{a}  & 11.41 & \textbf{40.25} \\
            
            \bottomrule
        \end{tabular}
        
        \begin{tablenotes}[para]
            \small
            \item[] Nota: Comparação da taxa de erro (\%) da rede "Network in Network" (NIN) treinada no conjunto de dados CIFAR-100. Os valores em negrito indicam o melhor resultado (menor erro) em cada coluna. \textsuperscript{a}Randomized Leaky ReLU (RReLU), onde o coeficiente de vazamento é amostrado de uma distribuição uniforme durante o treino.
            \item[] Fonte: Adaptado de "Empirical Evaluation of Rectified Activations in Convolutional Network", por B. Xu, N. Wang, T. Chen, \& M. Li, 2015, \textit{arXiv preprint arXiv:1505.00853}.
        \end{tablenotes}
        
    \end{threeparttable}
\end{table}

Ainda em \textit{Empirical Evaluation of Rectified Activations in Convolutional Network} \textcite{XuRReLU} explicam que a \textit{RReLU} é uma função que ajuda a combater o sobreajuste (overfitting) do modelo, mas ainda devem ser feitos mais testes para descobrir como a aleatoriedade afeta os processos de treino e teste \parencite{XuRReLU}. Essa característica de ajudar a combater o sobreajuste é uma vantagem que a \textit{RReLU} possui, permitindo com que modelos maiores e mais profundos possam ser criados e mesmo assim obtenham resultados significativos. Provavelmente, um dos motivos dela possuir essa função está no fato de que ela introduz uma maior aleatoriedade para o modelo, ajudando a impedir que ele decore os padrões, como em imagens dos conjuntos \textit{CIFAR-10} e \textit{CIFAR-100}.

\medskip
\begin{center}
 * * *
\end{center}
\medskip

\textbf{Algumas Aplicações da Randomized Leaky ReLU em Redes Neurais} \index{Aplicações práticas! Randomized Leaky ReLU (RReLU)}
\vspace{1em}

\begin{itemize}
    \item \textbf{Aplicação 1 (Área):}
    \item \textbf{Aplicação 2 (Área):}
    \item \textbf{Aplicação 3 (Área):}
    \item \textbf{Aplicação 4 (Área):}
\end{itemize}

\section{As Variantes Não Lineares: Em Busca da Suavidade}

Seguindo adiante, agora será visto um novo conjunto de variantes da \textit{ReLU} tradicional, elas incluem funções que apresentam gráficos com curvas mais suaves, como é o caso da \textit{ELU}, que faz uso de funções exponenciais para a sua composição e com isso consegue não só resolver o problema dos \textit{ReLUs} agonizantes, mas também sendo uma função contínua na origem e portanto derivável em todo o seu domínio.

Além disso, é possível conhecer também uma variante da \textit{ELU}, a \textit{Scaled Exponetial Linear Unit}, uma função que é utilizada para construir redes capazes de se autonormalizarem, ademais será visto a \textit{Noisy ReLU}, outra variante da \textit{ReLU}, mas que dessa vez adiciona ruído em sua saída a fim de garantir uma melhor performance quando comparada com a sua função original.

\subsection{Exponential Linear Unit (ELU)} \index{Funções de Ativação!Exponential Linear Unit (ELU)}

Continuando com as analogias do vendedor de pipoca, o vendedor de pipoca que faz uso da \textit{ELU} para as suas vendas trabalha de forma diferente. Ao invés de vender um punhado de pipoca de forma linear com base no dinheiro que o cliente tem quando ele não quer comprar o pacote inteiro por não possuir o valor total, ele adota uma curva exponencial como base, assim, clientes com valores muito próximos de R\$ 5,00 recebem uma quantia muito grande de pipoca, quase equivalente ao pacote total, enquanto aqueles que possuem valores pequenos irão receber um punhado pequeno de pipoca. Talvés seja uma forma de incentivar aqueles que quase tem o valor total para comprar uma pipoca, mas ainda sim garantir a clientela dos que possuem pouco dinheiro e aumentando o seu lucro como vendendor.

A \textit{ELU} ou \textit{Exponential Linear Unit} foi introduzida no artigo \textit{Fast and Accurate Deep Networks Leaning By Exponential Linear Units (ELUs)}, sendo uma variação que acelera o aprendizado de uma rede neural densa e apresentando uma maior acurácia em problemas de classificação \parencite{ELUArticle}.

A \textit{Exponential Linear Unit} pode é descrita utilizando a Equação \ref{eq:elu}. Note que há uma grande diferença dela quando comparamos com a \textit{ReLU}, a \textit{ELU} faz uso de funções exponenciais, algo que é computacionalmente mais "caro" para um computador quando comparado com apenas cálculos simples como uma função identidade.

\begin{equacaodestaque}{\textit{Exponential Linear Unit} (\textit{ELU})}
    \mathcal{A}_{\text{ELU}}(y_j) = \begin{cases}y_j, & \text{se } y_j \ge 0 \\ \alpha \cdot (e^{y_j} - 1), & \text{se } y_j < 0\end{cases}
    \label{eq:elu}
\end{equacaodestaque}

Conhecendo como é a fórmula da \textit{ELU}, é possível também plotar o seu gráfico, o qual está presente na Figura \ref{fig:elu}. Ao analisar, é possível ver uma diferença notável quando o comparada com as funções vistas anterioemente, a \textit{ELU} não apresenta um bico no ponto de origem, ela é uma função bem mais suave. Além disso, note que ela segue o mesmo padrão das outras funções: ela retorna a função identidade nos casos em que a entrada é maior que zero, assim como as outras variantes, mas quando é tratado dos casos em que a entrada é negativa, nota-se que ela utiliza uma funcão exponencial, o que garante a suavidade vista no gráfico. Perceba também que ela possui valores negativos, assim como as variantes com vazamento, indicando que ela também pode ser capaz de lidar com o problema do \textit{ReLUS} agonizantes causado pela \textit{ReLU} tradicional e que vem sendo mitigado com outras variantes como a \textit{Leaky ReLU}.

\begin{figure}[h!]
    \centering
    \begin{tikzpicture}
        \begin{axis}[
            xlabel={$y_j$},
            ylabel={$\text{ELU}(y_j)$},
            xmin=-2.3, xmax=2.3,
            ymin=-1.2, ymax=2.3,
            axis lines=middle,
            grid=major,
        ]
        % Define alpha para o gráfico. O valor comum para ELU é 1.
        \def\alphaVal{1} 
        \addplot[blue, thick, domain=-2:2] {x >= 0 ? x : \alphaVal*(exp(x) - 1)};
        \end{axis}
    \end{tikzpicture}
    \caption{Gráfico da função de ativação \textit{Exponential Linear Unit} (\textit{ELU}) com $\alpha=1$.}
    \label{fig:elu}
    \fonte{O autor (2025).}
\end{figure}

\medskip
\begin{center}
 * * *
\end{center}
\medskip

\textbf{Características da Exponential Linear Unit}
\vspace{1em}

\begin{itemize}
    \item \textbf{Característica 1:}
    \item \textbf{Característica 2:}
    \item \textbf{Característica 3:}
\end{itemize}

\medskip
\begin{center}
 * * *
\end{center}
\medskip

Agora que a sua fórmula e seu gráfico são conhecidos, cabe também calcular a sua derivada, a qual será útil na retropropagação do modelo. Para isso, é possível seguir a mesma estratégia vista até agora, derivando a função em cada um dos casos, gerando assim a sua derivada. Nos cenários em que a entrada é positiva, a derivada será sempre 1, pois quando derivamos a expressão $x$, ela nos irá retornar 1. Já quando temos o cenário negativo, teremos como resultado da derivação da expressão $alpha \cdot (e^{y_j} - 1)$ o termo $alpha \cdot e^{y_j}$. Um ponto a ser destacado é de que a \textit{ELU} é contínua na origem, assim, não tendo que preocupar em escolher um valor da derivada quando o seu valor de entrada for zero. Assim, tem-se como resultado final a Equação \ref{eq:elu-derivada}

\begin{equacaodestaque}{\textit{Exponential Linear Unit} (\textit{ELU}) Derivada}
    \frac{d}{dy_j} [\mathcal{A}_{ELU}](y_j) = \begin{cases}1, & \text{se } y_j > 0 \\ \alpha \cdot e^{y_j}, & \text{se } y_j \le 0 \end{cases}
    \label{eq:elu-derivada}
\end{equacaodestaque}

Sabendo a sua derivada, pode-se também plotar o seu gráfico, para isso, ele está representado na Figura \ref{fig:elu-derivada}. Note que ele é composto de duas partes diferentes, sendo a primeira delas, para os casos em que a entrada é negativa, a função constante em um, e para os casos em que a entrada é negativa, tem-se uma curva exponencial. Perceba também que a sua derivada irá sempre retornar valores positivos quando é calculada para qualquer ponto do seu domínio.

\begin{figure}[h!]
    \centering
    \begin{tikzpicture}
        \begin{axis}[
            xlabel={$y_j$},
            ylabel={$\text{ELU}'(y_j)$},
            xmin=-2.3, xmax=2.3,
            ymin=-0.2, ymax=1.2,
            axis lines=middle,
            grid=major
        ]
        % Define alpha para o gráfico
        \def\alphaVal{1} 

        % Plota a derivada usando uma única expressão condicional
        \addplot[red, thick, domain=-2.3:2.3, samples=100] {x > 0 ? 1 : \alphaVal*exp(x)};
        
        \end{axis}
    \end{tikzpicture}
    \caption{Gráfico da derivada da função de ativação \textit{Exponential Linear Unit} (\textit{ELU}) com $\alpha=1$.}
    \label{fig:elu-derivada}
    \fonte{O autor (2025).}
\end{figure}

Um dos testes realizados pelos autores para analisar o desempenho na \textit{ELU}, foi na criação de uma \textit{CNN} com 18 camadas convolucionais para fazer a classifição dos datasets \textit{CIFAR-10} e \textit{CIFAR-100}, para isso, outras técnicas foram utilizas em conjunto como o decaimento do peso L2 e reduções das taxas de aprendizado \parencite{ELUArticle}.

Com base nessa \textit{CNN} e seus experimentos, é possível ver os resultados na Tabela \ref{tab:elu-cifar-comparativo}, em que \textcite{ELUArticle} comparam a \textit{ELU} com outras redes no mesmo problema, como a \textit{AlexNet} que foi vista anterioremente na explicação do surgimento da \textit{ReLU}. Ao analisar esses resultados, nota-se que a \textit{ELU} obteve um desempenho excelente no dataset \textit{CIFAR-100}, com uma diferença de 21.52 pontos percentuais quando comparada com a \textit{AlexNet}, que ficou em último lugar. Já ao considerar o seu desempenho para um problema de classificação mais simples, como o \textit{CIFAR-10}, ela ficou em segundo lugar, estando atrás apenas da \textit{Fract. Max-Pooling}, mas ainda sim, apresentando uma diferença considerável de 2.05 pontos percentuais a mais de erro. 

\begin{table}[ht]
    \centering
    \begin{threeparttable}
        \caption{Comparativo da Taxa de Erro de Redes Neurais nos Datasets CIFAR}
        \label{tab:elu-cifar-comparativo}
        \begin{tabular}{lccc}
            \toprule
            \textbf{Arquitetura da Rede} & \textbf{CIFAR-10 Erro (\%)} & \textbf{CIFAR-100 Erro (\%)} & \textbf{Augmentation} \\
            \midrule
            
            AlexNet              & 18.04          & 45.80          &                \\ 
            \addlinespace
            DSN                  & 7.97           & 34.57          & v              \\
            NiN                  & 8.81           & 35.68          & v              \\
            Maxout               & 9.38           & 38.57          & v              \\
            All-CNN              & 7.25           & 33.71          & v              \\
            Highway Network      & 7.60           & 32.24          & v              \\
            Fract. Max-Pooling   & \textbf{4.50}  & 27.62          & v              \\
            ELU-Network          & 6.55           & \textbf{24.28} & v              \\
            
            \bottomrule
        \end{tabular}
        
        \begin{tablenotes}[para]
            \small
            \item[] Nota: Comparação da taxa de erro de classificação (\%) no conjunto de teste para diversas arquiteturas de redes neurais convolucionais (CNNs). Os valores em negrito indicam o melhor resultado em cada dataset. A coluna "Augmentation" indica se foram utilizadas técnicas de aumento de dados (data augmentation) durante o treinamento, marcado com "v".
            \item[] Fonte: Adaptado de "Fast and Accurate Deep Network Learning by Exponential Linear Units (ELUs)", por D. Clevert, T. Unterthiner, \& S. Hochreiter, 2015, \textit{arXiv preprint arXiv:1511.07289}.
        \end{tablenotes}
        
    \end{threeparttable}
\end{table}

Isso indica que a \textit{ELU} é uma excelente opção para problemas de classificação, especialmente se há um grande número de classes a ser analisadas. Contudo, uma rede neural convolucional que apresenta 18 camadas de convolução pode ser um tanto quanto custosa para ser processada por um computador, assim, faz se necessário o uso de unidades de \textit{GPUs} para o processamento de uma rede como essa, para que mesmo sendo pesada para ser processada, os resultados possam sair um pouco mais rápidos.

Por fim, cabe destacar algumas afirmações realizadas pelos autores ainda em \textit{Fast and Accurate Deep Networks Leaning By Exponential Linear Units (ELUs)}, segundo \textcite{ELUArticle}, ao comparar a \textit{ELU} com funções como a \textit{ReLU tradicional} e \textit{Leaky ReLU}, pode-se notar um melhor e mais rápido aprendizado, além de que a \textit{Exponetial Linear Unit} é capaz de garantir uma melhor generalização quando passa a ser utilizada em redes com mais de cinco camadas. Outro ponto destacado pelos autores, está no fato da \textit{ELU} garantir \textit{noise rebust deactivation states}, algo que mesmo com a \textit{Leaky ReLU} e \textit{Parametric ReLU} possuindo valores negativos, não são capazes de garantir ao serem utilizadas para construir uma rede neura \parencite{ELUArticle}.

Mesmo apresentando um grande salto, quando comparada com a \textit{ReLU} tradicional, a \textit{ELU} também pode ser modificada para atender outros casos. Para isso, ela também tem variações, sendo uma delas a \textit{SELU}, a qual adiciona a \textit{ELU} um termo $\lambda$ para garantir uma autonormalização da rede que está sendo criada. Essa função será vista em seguida.

\medskip
\begin{center}
 * * *
\end{center}
\medskip

\textbf{Algumas Aplicações da Exponential Linear Unit em Redes Neurais} \index{Aplicações práticas! Exponential Linear Unit (ELU)}
\vspace{1em}

\begin{itemize}
    \item \textbf{Aplicação 1 (Área):}
    \item \textbf{Aplicação 2 (Área):}
    \item \textbf{Aplicação 3 (Área):}
    \item \textbf{Aplicação 4 (Área):}
\end{itemize}

\subsection{Scaled Exponential Linear Unit (SELU)} \index{Funções de Ativação!Scaled Exponential Linear Unit (SELU)}

A próxima função é uma variante da \textit{ELU}, a \textit{Scaled Exponential Linear Unit}, ou \textit{SELU}. Ela se distingue da ELU tradicional pelo fato de que ela é capaz de implemtar propriedades auto-normalizadoras em uma rede que faz uso dessa função, como explicam os autores no artigo de sua introdução \textit{Self-Normalizing Neural Networks} \parencite{SELUArticle}.

Os autores apresentam a Equação \ref{eq:selu} para calcular essa função, em que o termo $\lambda$ é uma constante que será maior que 1 \parencite{SELUArticle}. Note que essa fórmula é bem parecida com a da \textit{Exponential Linear Unit}, a única diferença é que ela estará sendo multiplicada pela constante $\lambda$, assim pode-se escrever também que $\text{SELU}(y_j) = \lambda \text{ELU}(y_j)$.

\begin{equacaodestaque}{\textit{Scaled Exponential Linear Unit} (\textit{SELU})}
    \mathcal{A}_{\text{SELU}}(y_j) = \lambda \begin{cases}y_j, & \text{se } y_j > 0 \\ \alpha \cdot (e^{y_j} - 1), & \text{se } y_j \le 0\end{cases} \quad \text{ou} \quad \mathcal{A}_{\text{SELU}}(y_j) = \lambda \text{ELU}(y_j)
    \label{eq:selu}
\end{equacaodestaque}

Com relação ao seu gráfico, ele está presente na Figura \ref{fig:selu}, neste caso, está sendo considerado que as constantes $\alpha$ e $\lambda$ são dadas por 1.7 e 1.05 respectivamente. Note que é um gráfico que lembra bastante a \textit{Leaky ReLU}, mas que neste caso, quando a função recebe valores negativos, ela não estará mais assumindo o comportamento de uma reta, e sim o de uma curva exponencial, já para os cenários em que a entrada é postiva os resultados serão próximos os de uma função identidade, mas com uma reta um pouco mais inclinada. 

Pelo fato da \textit{SELU} ter um comportamento que também retorna valores para a saída quando a sua entrada é negativa, ela consegue combater o problema dos \textit{ReLUs} agonizantes, causado pela \textit{ReLU}, além de que também não é uma função saturante, como a sigmoide, o que também ajuda a resolver o problema no desaparecimento do gradiente. Mas, por não ser uma função saturante, e pelo fato de que sua saída vai para valores infinitos conforme os valores de sua entrada aumentam, ela está sujeita ao problema da explosão de gradientes.

\begin{figure}[h!]
    \centering
    \begin{tikzpicture}
        \begin{axis}[
            xlabel={$y_j$},
            ylabel={$\text{SELU}(y_j)$},
            xmin=-2.3, xmax=2.3,
            ymin=-2, ymax=2.5,
            axis lines=middle,
            grid=major,
        ]
        % Define as constantes da SELU para o gráfico
        \def\alphaVal{1.67326}
        \def\lambdaVal{1.0507}
        \addplot[blue, thick, domain=-2:2, samples=100] {x > 0 ? \lambdaVal*x : \lambdaVal*\alphaVal*(exp(x) - 1)};
        \end{axis}
    \end{tikzpicture}
    \caption{Gráfico da função de ativação \textit{Scaled Exponential Linear Unit} (\textit{SELU}) com $\alpha \approx 1.67 e \lambda \approx 1.05$.}
    \label{fig:selu}
    \fonte{O autor (2025).}
\end{figure}

\medskip
\begin{center}
 * * *
\end{center}
\medskip

\textbf{Características da Scaled Exponential Linear Unit}
\vspace{1em}

\begin{itemize}
    \item \textbf{Característica 1:}
    \item \textbf{Característica 2:}
    \item \textbf{Característica 3:}
\end{itemize}

\medskip
\begin{center}
 * * *
\end{center}
\medskip

Considerando agora como é a equação da \textit{SELU} e qual é o seu comportamento no gráfico, é possível também calcular a sua derivada para ser utilizada na retroproapagação do gradiente, para isso, deve-se derivar a Equação \ref{eq:selu}, considerando os dois cenários, em que a sua entrada será positiva e quando sua entrada for negativa ou zero. Um ponto que ajuda bastante ao calcular a derivada da \textit{SELU} está no fato dela ser composta pela função \textit{ELU} multiplicada por uma constante, se utilizarmos regras de derivação para esse cenário, precisaremos apenas derivar a \textit{ELU} e depois adicionar a constante $\lambda$ multiplicando-a. Como a derivada da \textit{ELU} já foi calculada, ela pode ser aproveitada agora. Você pode ver então a derivada da \textit{Scaled Exponential Linear Unit} na Equação \ref{eq:selu-derivada}.

\begin{equacaodestaque}{\textit{Scaled Exponential Linear Unit} (\textit{SELU}) Derivada}
    \frac{d}{dy_j} [\mathcal{A}_{SELU}](y_j) = \lambda \begin{cases}1, & \text{se } y_j > 0 \\ \alpha \cdot e^{y_j}, & \text{se } y_j \le 0\end{cases}
    \label{eq:selu-derivada}
\end{equacaodestaque}

Sabendo a sua derivada, é possível também plotar o seu gráfico, para isso, ele está na Figura \ref{fig:selu-derivada}. Note, que o gráfico da derivada da \textit{SELU} também possui grandes similaridades com a \textit{ELU} original mas também com as outras retificadoras, pois também pode ser dividido em duas partes principais. A primeira parte, para os casos em que a entrada é negativa segue o comportamento de uma curva exponencial, enquanto a segunda parte é semelhante a uma reta constante com inclinação zero. 

Um ponto interessante dessa reta da segunda parte é que ela retorna justamente um valor bem próximo de um, assim como nas retificadoras, isso trás como benefício uma menor chance para ocorrer casos de desaparecimento do gradiente, pois ele não estará sendo constantemente sendo multiplicado por valores pequenos e com isso reduzindo o seu valor. Mas, por outro lado, isso também acaba colaborando para que gradientes explosivos possam ocorrer.

\begin{figure}[h!]
    \centering
    \begin{tikzpicture}
        \begin{axis}[
            xlabel={$y_j$},
            ylabel={$\text{SELU}'(y_j)$},
            xmin=-2.3, xmax=2.3,
            ymin=-0.2, ymax=2.0, % Ajustado para lambda > 1
            axis lines=middle,
            grid=major
        ]
        % Define as constantes da SELU para o gráfico
        \def\alphaVal{1.67326}
        \def\lambdaVal{1.0507}

        \addplot[red, thick, domain=-2:2, samples=100] {x > 0 ? \lambdaVal*1 : \lambdaVal*\alphaVal*exp(x)};
        \end{axis}
    \end{tikzpicture}
    \caption{Gráfico da derivada da função de ativação Scaled Exponential Linear Unit (SELU) com $\alpha \approx 1.67, \lambda \approx 1.05$.}
    \label{fig:selu-derivada}
    \fonte{O autor (2025).}
\end{figure}

No texto da introdução da \textit{SELU}, \textcite{SELUArticle}, comparam as redes neurais criadas por eles, as quais são chamadas de \textit{Self-Normalizing Neural Networks} (\textit{SNN}), com outras redes \textit{feedforward}, como \textit{MSRAinit} (uma \textit{FNN} que não possui técnicas de normalização, com funções de ativação \textit{ReLU}, e que faz uso do \textit{Microsoft weight initialization}), a \textit{BatchNorm} (uma \textit{FNN} com normalização em lote), a \textit{LayerNorm} (uma \textit{FNN} com normalização nas camadas), a \textit{WightNorm} (uma \textit{FNN} com normalização nos pesos), a \textit{Highway} e também com redes residuais \textit{ResNet}. Para comparar essas redes, os autores escolhem 121 \textit{datasets} do \textit{UCI}, em que são apresentadas áreas de aplicação diversas como física e biologia, nesses \textit{datasets} o seus tamanhos podem variar de 10 até 130.000 pontos de dados com o número de features variando de 4 a 250. Na Tabela \ref{tab:comparativo-selu}, é possível ver o ranking médio entre as \textit{SNNs} e as outras diferentes arquiteturas em 75 tarefas de classificação.

\begin{table}[ht]
    \centering
    \begin{threeparttable}
        \caption{Comparativo do Rank Médio entre SNNs e Outras Arquiteturas de Redes Neurais}
        \label{tab:comparativo-selu}
        \begin{tabular}{llcc}
            \toprule
            \textbf{Grupo do Método} & \textbf{Método} & \textbf{Rank Médio} & \textbf{Valor-p} \\
            \midrule
            
            SNN & SNN & 9.6 & $3.8 \times 10^{-1}$ \\
            MSRAinit & MSRAinit & 11.0 & $4.0 \times 10^{-2}$ \\
            LayerNorm & LayerNorm & 11.3 & $7.2 \times 10^{-2}$ \\
            Highway & Highway & 11.5 & $8.9 \times 10^{-3}$ \\
            ResNet & ResNet & 12.3 & $3.5 \times 10^{-3}$ \\
            BatchNorm & BatchNorm & 12.6 & $4.9 \times 10^{-4}$ \\
            WeightNorm & WeightNorm & 13.0 & $8.3 \times 10^{-5}$ \\

            \bottomrule
        \end{tabular}
        
        \begin{tablenotes}[para]
            \small
            \item[] Nota: Comparação do rank médio de diferentes arquiteturas de redes neurais em 75 tarefas de classificação do repositório UCI. O "Rank Médio" é a média das classificações de acurácia entre as tarefas. O "Valor-p" corresponde ao teste de Wilcoxon pareado para avaliar se a diferença para o método de melhor desempenho é significativa.
            \item[] Fonte: Adaptado de "Self-Normalizing Neural Networks", por G. Klambauer, T. Unterthiner, A. Mayr, \& S. Hochreiter, 2017, \textit{arXiv preprint arXiv:1706.02515}.
        \end{tablenotes}
        
    \end{threeparttable}
\end{table}

Para analisar essa tabela, pode-se primeiro olhar o ranking médio de cada uma desses modelos, que é dado pela média de como esses modelos performaram nos diferentes datasets, considerando isso, note que as redes que fazem uso da \textit{SELU} em sua composição, as \textit{SNNs}, são as melhores, por uma diferença de 1.4 pontos quando comparadas com o segundo colocado, isso indica que a \textit{SELU} pode ser uma ótima alternativa quando ainda não se sabe exatamente qual será o conjunto de dados que será trabalhado, se ele será de conceitos como física ou geologia, assim, elas garantem uma maior versatilidade para encarar diversos problemas. Além disso, ao olhar também o seu \textit{p-value}, que vem de um teste de Wilcoxon pareado para verificar se a diferença em relação ao melhor colocado é signifitiva, percebe-se que as \textit{SNNs} continuam se destacando, com o \textit{p-value} mais alto, indicando que existe uma diferença que não é estatisticamente significativa quando ela é comparada com o modelo \textit{SVM}, mostrando que podemos considerar as \textit{SNNs} como se tivessem empatadas com o campeão.

O interessante dessa comparação é analisar ela considerando aquelas redes que fazem uso de técnicas de normalização para conseguir uma maior desempenho, como a \textit{BatchNorm} e a \textit{LayerNorm}, cada uma delas utiliza uma técnica de normalização diferente de forma a garantir que que problemas como os gradientes explosivos e desaparacimento do gradiente não ocorra com tanta frequência e com isso permitindo um melhor convergência do modelo que está sendo treinado. Ao olhar por essa ótica, pode-se chegar a conclusão que criar uma rede neural utilizando a \textit{SELU} não só irá garantir uma maior versatilidade para a resolução de problemas, como também você não terá que se preocupar em aplicar técnicas de normalização ao construir essa rede.

Voltando para o seu texto de introdução, os autores destacam propriedades importantes dessa nova função de atiavção criada, como o fato de que de que elas possbilitam a criação de redes neurais mais profundas além de serem capazes de aplicar fortes esquemas de regularização \parencite{SELUArticle}. Por favorecer a criação de RNAs mais profundas, como consequência, a \textit{SELU} se torna uma excelente alternativa para ser utilizada em problemas complexos, que possuem muitas características e por essa razação necessitam de que mais camadas sejam construídas a fim de garantir um melhor processamento e aprendizado dos dados e com base nisso, alcançar métricas maiores, como uma maior acurácia indicando uma gerenalização maior e também uma perda menor, indicando que o gradiente conseguiu uma convergência melhor.

\medskip
\begin{center}
 * * *
\end{center}
\medskip

\textbf{Algumas Aplicações da Scaled Exponential Linerar Unit em Redes Neurais} \index{Aplicações práticas! Scaled Exponential Linear Unit (SELU)}
\vspace{1em}

\begin{itemize}
    \item \textbf{Aplicação 1 (Área):}
    \item \textbf{Aplicação 2 (Área):}
    \item \textbf{Aplicação 3 (Área):}
    \item \textbf{Aplicação 4 (Área):}
\end{itemize}

\subsection{Noisy ReLU (NReLU)} \index{Funções de Ativação!Noisy ReLU (NReLU)}

Seguindo adiante, é possível conhecer a \textit{Noisy ReLU}, também conhecida como \textit{NReLU}. Um dos trabalhos que explora as característica dessa função e como ela pode ser aplicada em uma RNA é o \textit{Rectified Linear Units Improve Restricted Boltzmann Machines} dos autores \textcite{Nair2010}. Nesse texto, os autores comparam o desenpenho dessa função com a função binária, dada pela equação \ref{eq:equacao-binary}, que era a opção mais comum para ser utilizada na construção de máquinas restritas de Boltzmann.

\begin{equacaodestaque}{Função de Ativação Binária (\textit{Binary})}
    \mathcal{A}(y_j) = \begin{cases} 
    1 & \text{se } y_j \ge \theta \\ 
    0 & \text{se } y_j < \theta 
    \end{cases}
    \label{eq:equacao-binary}
\end{equacaodestaque}

No texto, \textcite{Nair2010}, apresentam a \textit{NReLU} como sendo dada pela equação $\max{0, y_j + \mathcal{N}(0, \sigma(y_j))}$, em que $\mathcal{N}(0, V)$ representa o ruído Gaussiano (\textit{Gaussian noise} em inglês), com méida zero e vairância dada por $V$. Também é possível expressar a \textit{Noisy ReLU} com a Equação \ref{eq:nrelu}.

\begin{equacaodestaque}{\textit{Noisy ReLU} (\textit{NReLU})}
    \mathcal{A}_{\text{NReLU}}(y_j) = \begin{cases} 
    y_j + \mathcal{N} (0, \sigma(y_j)) & \text{se } y_j > 0 \\
    0 & \text{se } y_j \le 0
    \end{cases}
    \label{eq:nrelu}
\end{equacaodestaque}

Antes de seguir em frente, é útil entende primeiro o que é ruído gaussiano, e para isso, é preciso entender antes a distribuição gaussiana. Segundo \textcite{DeepLearningBook}, a distribuição gaussiana é a distribuição mais utilizada para números reais, ela também é conhecida por ser chamada de distribuição normal, ela é dada pela Equação \ref{eq:distribuicao-gaussiana}.

\begin{equation}
    \mathcal{N}(x; \mu, \sigma^2) = \sqrt{\frac{1}{2\pi\sigma^2}} \exp\left( -\frac{1}{2\sigma^2}(x - \mu)^2 \right)
    \label{eq:distribuicao-gaussiana}
\end{equation}

Nessa equação, os dois parâmetros $\mu \in \mathbb{R}$ e $\sigma (0, \infty)$ controlam como a distribuição normal funciona; o termo $\mu$ é responsável por dar as coordenadas para o pico do centro, que é também a média da distribuição $\mathbb{E}[x] = \mu$, já o desvio padrão é dado por $\sigma$, equanto a variância é denotada por $\sigma^2$ \parencite{DeepLearningBook}. Para chegarmos no ruído gaussiano, é preciso então adicionar como parâmetros da equação da distribução gaussiana (Equação \ref{eq:distribuicao-gaussiana}) os termos que são dados pela Equação \ref{eq:nrelu}. 

A distribuição normal, com os parâmetros $\mu = 0$ e $\sigma=1$, é responsável por gerar um gráfico em formato de sino, como é mostrado na Figura \ref{fig:distribuicao-normal-padrao}. Esse gráfico indica quais são os casos que possuem uma maior probabilidade de acontecer, os casos que estão no centro, onde, $p(x)$ possuem uma maior probabilidade de acontecer, já conforme eles se distânciam desse centro essa propabilidade diminui.

\begin{figure}[htbp]
    \centering
    \begin{tikzpicture}
        \begin{axis}[
            xlabel={$x$},
            ylabel={$p(x)$}, % p(x) é a densidade de probabilidade
            xmin=-4, xmax=4,
            ymin=0, ymax=0.5,
            axis lines=middle,
            grid=major,
            samples=200, % Aumenta o número de pontos para uma curva mais suave
            domain=-4:4,
        ]
        
        % Declara a função da distribuição normal para facilitar o uso
        \def\normaldist#1#2{1/(#2*sqrt(2*pi))*exp(-((x-#1)^2)/(2*#2^2))}
        
        % Adiciona a área sombreada para +/- 1 desvio padrão
        \addplot[fill=blue!20, draw=none, domain=-1:1] {\normaldist{0}{1}} \closedcycle;

        % Plota a curva da distribuição normal padrão (mu=0, sigma=1)
        \addplot[blue, thick] {\normaldist{0}{1}};

        % Adiciona linhas verticais para marcar a média e os desvios padrão
        \draw[dashed, gray] (axis cs:0, 0) -- (axis cs:0, 0.45);
        \draw[dashed, gray] (axis cs:1, 0) -- (axis cs:1, 0.24);
        \draw[dashed, gray] (axis cs:-1, 0) -- (axis cs:-1, 0.24);

        % Adiciona os labels para a média e desvio padrão
        \node[below] at (axis cs:0, 0) {$\mu=0$};
        \node[below] at (axis cs:1, 0) {$\sigma$};
        \node[below] at (axis cs:-1, 0) {$-\sigma$};
        
        \end{axis}
    \end{tikzpicture}
    \caption{Gráfico da Distribuição Gaussiana (ou Normal) para o caso padrão, com média 0 ($\mu = 0$) e desvio padrão 1 ($\sigma = 1$).}
    \label{fig:distribuicao-normal-padrao}
    \fonte{O autor (2025).}
\end{figure}

Entendendo a estrutura e como funciona a \textit{Noisy ReLU}, é possível plotar o seu gráfico, o qual é dado pela Figura \ref{fig:nrelu}. Note que ela compartilha a suavidade das funções \textit{ELU} e \textit{SELU}, apresentando uma curva para os casos em que a entrada é negativa, o que é bom, pois ela permite um vazamento de gradiente, evitando assim o o problema do \textit{ReLUs} agonizantes. Já para os cenários que a entrada é positiva ela assume o comportamento de uma função identidade, lembrando bastante as outras variantes da \textit{ReLU}. Contudo, como os autores destacam no texto, ela não é capaz de resolver o problema da descontinuidade em zero, para isso, em sua derivada que é utilizada na retropropagação, os casos em que sua entrada é zero irão retornar zero como saída \parencite{Nair2010}.

\begin{figure}[h!]
    \centering
    \begin{tikzpicture}
        \begin{axis}[
            xlabel={$y_j$},
            ylabel={$\text{NReLU}(y_j)$},
            xmin=-1.3, xmax=2.3,
            ymin=-2.3, ymax=2.3,
            axis lines=middle,
            grid=major,
            domain=-0.999:2, % Domínio para evitar o log de zero ou negativo
        ]
        % A função ln(1+x) está definida para x > -1
        \addplot[blue, thick] {x >= 0 ? x : ln(1+x)};
        % Linha vertical para mostrar a assíntota em z = -1
        \draw[dashed, gray] (axis cs:-1, -2.3) -- (axis cs:-1, 2.3);
        \end{axis}
    \end{tikzpicture}
    \caption{Gráfico da função de ativação \textit{Noisy ReLU} (\textit{NReLU}).}
    \label{fig:nrelu}
    \fonte{O autor (2025).}
\end{figure}

\medskip
\begin{center}
 * * *
\end{center}
\medskip

\textbf{Características da Noisy ReLU}
\vspace{1em}

\begin{itemize}
    \item \textbf{Característica 1:}
    \item \textbf{Característica 2:}
    \item \textbf{Característica 3:}
\end{itemize}

\medskip
\begin{center}
 * * *
\end{center}
\medskip

Ao calcular a derivada da \textit{NReLU} você encontrará um problema que não havia aparecido nas outras funções. O termo de ruído $\mathcal{N}$ é um termo não determísitico, o que significa que mesmo que tivéssemos a mesma entrada para a função \textit{Noisy ReLU} duas ou mais vezes, não poderíamos afirmar com certeza de que essas saídas seriam iguais. Para resolver esses problemas, os autores consideram para a \textit{NReLU} que sua função para o \textit{backward pass} será irá retornar zero quando o valor de entrada for negativo ou nulo, e irá retornar um, quando o valor de entrada for positivo \parencite{Nair2010}. Então a expressão que representa a \textit{NReLU} para a retropropagação é dada pela Equação \ref{eq:nrelu-derivada}, note que ela é a mesma expressão da derivada da \textit{ReLU} tradicional.

\begin{equacaodestaque}{\textit{Noisy ReLU} (\textit{NReLU}) para o \textit{Backward Pass}}
    \frac{d}{dy_j}{\mathcal{A}_{\text{NReLU}}}(y_j) = \begin{cases} 
        1 & \text{se } y_j > 0 \\
        0 & \text{se } y_j \le 0
    \end{cases}
    \label{eq:nrelu-derivada}
\end{equacaodestaque}

Se a função da \textit{Noisy ReLU} no \textit{backward pass} será a mesma que a derivada da \textit{ReLU}, então é possível utilizar como base o gráfico da derivada da \textit{ReLU}. Para isso, tem-se então a Figura \ref{fig:nrelu-derivada}.

\begin{figure}[htbp] % Use [htbp] para dar flexibilidade ao LaTeX
    \centering % Centraliza o gráfico na página
    \begin{tikzpicture}
        \begin{axis}[
            xlabel={$y_j$},
            ylabel={$\text{NReLU}'(y_j)$},
            xmin=-2.3, xmax=2.3,
            ymin=-0.8, ymax=1.8,
            axis lines=middle,
            grid=major
        ]
        \addplot[red, thick, domain=-2:0] {0};
        \addplot[red, thick, domain=0:2] {1};
        \addplot[red, only marks, mark=o, mark size=1.5pt] coordinates {(0,0)};
        \addplot[red, only marks, mark=*, mark size=1.5pt] coordinates {(0,1)};
        \end{axis}
    \end{tikzpicture}
    \caption{Gráfico da função \textit{Backward Pass} da função de ativação \textit{Noisy ReLU}.}
    \label{fig:nrelu-derivada}
    \fonte{O autor (2025).}
\end{figure}

No trabalho \textit{Rectified Linear Units Improve Restricted Boltzmann Machines}, \textcite{Nair2010} exploram o desempenho da \textit{NReLU} e \textit{ReLU} utilizando o \textit{dataset NORB}, que é um \textit{dataset} para o reconhecimento de objetos 3D sintéticos que contém cinco classes: humanos, animais, carros, aviões e caminhões. No texto, os autores utilizam a versão \textit{Jittered-Cluttered NORB}, uma variante que tem imagens estereoscópicas em tons de cinza com fundo desorganizado e um objeto central que é aleatoriamente instável em posição, tamanho, intensidade de pixels \parencite{Nair2010}. O desempenho dessas funções nesse \textit{dataset} pode ser visto pela Tabela \ref{tab:norb-error-rate} e pela Tabela \ref{tab:nrelu-norb-comparativo}.

A Tabela \ref{tab:norb-error-rate}, mostra a taxa de erro dos diferentes modelos no \textit{dataset}, para isso, são construídos modelos com 4000 unidades ocultas treinados com imagens de dimensões 32x32x2. Note ao analisar a tabela que é nítido que os modelos que são pré-treinados utilizando máquinas de Boltzmann restritas apresentam um melhor desempenho de forma geral, da mesma forma que a \textit{NReLU} oferece uma perda menor quando comparada com a função binária em ambos os casos: pré-treinado ou não. 

Mas, vale a pena fazer uma comparação ainda mais interessante, o modelo que faz uso da \textit{Noisy ReLU} que não foi pré-treinado apresenta um resultado com uma diferença de 0.9 pontos quando comparado com o modelo treinado que faz uso da função binária. Isso é interessante porque indica que é possível conseguir um resultado muito melhor utilizando a \textit{Noisy ReLU} mediante a função binária mesmo quando não tiver condições de pré-treinar uma rede.

\begin{table}[ht]
    \centering
    \begin{threeparttable}
        \caption{Taxa de Erro de Classificadores no Dataset NORB Jittered-Cluttered}
        \label{tab:norb-error-rate}
        \begin{tabular}{lcc}
            \toprule
            \textbf{Pré-treinado?} & \textbf{NReLU (\%)} & \textbf{Binary (\%)} \\
            \midrule
            
            Não & 17.8 & 23.0 \\
            Sim & \textbf{16.5} & \textbf{18.7} \\
            
            \bottomrule
        \end{tabular}
        
        \begin{tablenotes}[para]
            \small
            \item[] Nota: Taxas de erro no conjunto de teste para classificadores com 4000 unidades ocultas. Os valores em negrito indicam a menor taxa de erro (melhor resultado) em cada coluna. Os modelos foram treinados com imagens de 32x32x2 do dataset NORB Jittered-Cluttered. A coluna "NReLU" refere-se a unidades de ativação ReLU com ruído (Noisy ReLU), enquanto "Binary" refere-se a unidades binárias tradicionais. A coluna "Pré-treinado?" indica se o modelo utilizou uma Máquina de Boltzmann Restrita para pré-treinamento.
            \item[] Fonte: Adaptado de "Rectified Linear Units Improve Restricted Boltzmann Machines", por V. Nair \& G. E. Hinton, 2010, \textit{Proceedings of the 27th International Conference on Machine Learning (ICML-10)}.
        \end{tablenotes}
        
    \end{threeparttable}
\end{table}

Seguindo adiante, na Tabela \ref{tab:nrelu-norb-comparativo} também é possível analisar as taxas de erro dos classicadores, mas, neste caso, eles possuem duas camadas ao invés de somente uma como mostrado da comparação da Tabela \ref{tab:norb-error-rate}, assim a primeira camada é composta de 4000 unidades ocultas (assim como no primeiro caso), enquanto a segunda camada possui 2000 unidades ocultas. 

Com base essa tabela, é possível notar que o desempenho dos modelos que fazem uso da \textit{Noisy ReLU} melhorou tanto nos casos em que as camadas não foram pré-treinadas, quanto nos casos em que uma ou ambas foram, quando se compara com os resultados da Tabela \ref{tab:norb-error-rate}. Um ponto interessante disso é que no caso em que somente uma das camadas foi pré-treinada no modelo que usa a \textit{NReLU} o seu resultado foi igual ao do primeiro caso onde existia somente uma camada, o que pode indicar que talvez não seja vantajoso adicionar mais camadas em uma rede caso não esteja disposto a pré-treiná-las. Note também que esse cenário com a unidade binária foi ainda pior, mostrando que não existe um grande ganho em adicionar mais camadas em uma rede que faz uso dessa função.

Esse resultado prová-se ainda mais nítido quando é analisado o caso em que ambas as camadas foram pré-treinadas, na unidade binária não houve nenhuma diminuição na sua perda, ela inclusive é pior que a do modelo que faz uso de apenas uma camada. Já quando é observada a \textit{Noisy ReLU} e seus resultados, nota-se um cenário bem diferente, os modelos que fazem uso dela apresentam um desempenho melhor quando são adicionadas mais camadas na rede, fazendo com que a perda da rede diminua, indicando que essa função permite a criação de redes mais profundas e com isso, desempenhos melhores possam ser alcançados.

\begin{table}[ht]
    \centering
    \begin{threeparttable}
        \caption{Taxas de Erro de Classificadores no Dataset Jittered-Cluttered NORB}
        \label{tab:nrelu-norb-comparativo}
        \begin{tabular}{llcc}
            \toprule
            \multicolumn{2}{c}{\textbf{Camadas Pré-treinadas}} & \multicolumn{2}{c}{\textbf{Taxa de Erro de Teste (\%)}} \\
            \cmidrule(r){1-2} \cmidrule(r){3-4} % Linhas parciais para agrupar colunas
            \textbf{Camada 1} & \textbf{Camada 2} & \textbf{NReLU} & \textbf{Binary} \\
            \midrule
            
            Não & Não & 17.6 & 23.6 \\
            Sim & Não & 16.5 & \textbf{18.8} \\
            Sim & Sim & \textbf{15.2} & \textbf{18.8} \\
            
            \bottomrule
        \end{tabular}
        
        \begin{tablenotes}[para]
            \small
            \item[] Nota: Taxas de erro (\%) de teste para classificadores com duas camadas ocultas (4000 unidades na primeira, 2000 na segunda), treinados em imagens do dataset Jittered-Cluttered NORB de 32x32x2. A tabela compara modelos com unidades retificadoras ruidosas (NReLU) e unidades binárias tradicionais (Binary), avaliando o impacto do pré-treinamento de cada camada. Os valores em negrito indicam os melhores resultados para cada tipo de unidade.
            \item[] Fonte: Adaptado de "Rectified Linear Units Improve Restricted Boltzmann Machines", por V. Nair \& G. E. Hinton, 2010, \textit{Proceedings of the 27th International Conference on Machine Learning (ICML-10)}, pp. 807-814.
        \end{tablenotes}
        
    \end{threeparttable}
\end{table}

Esse tópico de permitir a criação de redes mais profundas, que aconteceu justamente ao optar por funções retificadoras como a \textit{ReLU} e a \textit{Noisy ReLU} ao invés de funções sigmoidais, está intrinsicamente relacionado ao problema da desaparecimento do gradiente. Em redes que faziam uso de funções sigmoidais, os programadores e pesquisadores estavam constantemente correndo o risco de que ao adicionar mais camadas a fim de que essa rede pudesse alcançar melhores métricas, o problema do desaparecimento do gradiente viesse a tona. Isso acontece, porque ao adicionar mais camadas, existe uma maior chance de que esse vetor seja mais uma vez multiplicado por valores pequenos e com isso diminuísse o seu valor.

Ainda em \textit{Rectified Linear Units Improve Restricted Boltzmann Machines}, \textcite{Nair2010} citam que uma das propriedades interessantes da \textit{NReLU} é a \textit{intensity equivarience} (equivariância de intensidade), a qual é bem útil para o reconhecimento de objetos. No texto, os autores destacam que um dos principais objetivos ao contruir um sistema que faça o reconhecimento de objetos, é garantir que a saída seja invariante às propriedades da sua entrada, como localização, escala, iluminação e orientação, e a NReLU é uma das funções que quando adicionada em uma rede neural, garante que isso possa ser atingido \parencite{Nair2010}.

\medskip
\begin{center}
 * * *
\end{center}
\medskip

\textbf{Algumas Aplicações da Noisy ReLU em Redes Neurais} \index{Aplicações práticas! Noisy ReLU (NReLU)}
\vspace{1em}

\begin{itemize}
    \item \textbf{Aplicação 1 (Área):}
    \item \textbf{Aplicação 2 (Área):}
    \item \textbf{Aplicação 3 (Área):}
    \item \textbf{Aplicação 4 (Área):}
\end{itemize}

\section{O Problema dos Gradientes Explosivos} \index{Gradientes explosivos}

Anteriormente, ao conhecer as sigmoidais no Capítulo \ref{cap:ativacao-sigmoidais}, foi possível ver que elas eram comumente utilizadas como as funções de ativação padrão de uma rede neural antes das retificadoras. Mas elas possuíam um problema, o do desaparacimento do gradiente. Esse problema acontecia porque essas funções retornavam sempre números muito pequenos em suas derivadas, que consequentemente eram multiplicadas no \textit{backward pass} com o gradiente retropropagado gerando como produto um número pequeno, esse número era então novamente multiplicado por outra constante de baixo valor e por aí vai, como resultado, o gradiente retropropagado que chegava nas primeiras camadas para atualizar os pesos e vieses da rede possuía um valor tão pequeno que muitas vezes não resultava em uma atualização capaz de gerar impacto no aprendizado da rede. Assim, tínhamos o problema do desaparecimento do gradiente.

Já neste capítulo, foram conhecidas as funções retificadoras, e como elas surgiram como uma alternativa para contornar esse problema. Contudo, elas também apresentam problemas, sendo um delos o dos gradientes explosivos, o qual será explicado nessa seção.

Para explicar melhor essa condição será utilizado um exemplo como base.

Como foi visto no Capítulo \ref{cap:retropropagacao-gradiente}, o gradiente retropropagado para camadas anteriores de uma rede neural é proporcional a multiplicação da perda, com a derivada da função de ativação no ponto e o valor do resultado da camada anterior de neurônios. 

\[
    \delta^{(L)} = \left( \left( \textbf{W}^{(L+1)} \right)^T \delta^{(L+1)} \right)  \odot \mathcal{A}'(x^{(L)})
\]

Em que: 

\begin{itemize}
    \item $L$: Representa o índice de uma camada, podendo ser um valor entre $1$ (indicando que é uma camada de entrada) ou $n$ (indicando que é uma camada de saída);
    \item $\textbf{W}^{(L)}$: Representa a matriz dde pesos que conecta a camada $L - 1$ à camada $L$;
    \item $b^{(L)}$: Representa o vetor de viés da camada $L$;
    \item $x^{(L)}$: Representa o vetor de entradas totais para os neurônios da camada $L$ antes da ativação;
    \item $y^{(L)}$: Representa o vetor de saídas da camada $L$
    \item $\delta^{(L)}$: Representa o vetor do gradienye na camada $L$;
    \item $\mathcal{A}'(x^{(L)})$: Representa o vetor contendo a derivada da função de ativação para cada neurônio da camada $L$;
    \item $\odot$: O produto de Hadamard, que significa multiplicação elemento a elemento.
\end{itemize}

Considerando isso, imagine que temos uma rede composta por quatro camadas densas e cada camada tem apenas um neurônio com pesos iguais a 1. Dessa forma, é possível simplificar a fórmula vista para a Equação \ref{eq:gradiente-retropropagado-simplificado}.

\begin{equation}
        \delta^{(L)} =  \delta^{(L+1)} \times \sigma'(x^{(L)})
        \label{eq:gradiente-retropropagado-simplificado}
\end{equation}

Considere também que as camadas da rede possuem a seguinte configuração:

\begin{itemize}
    \item SELU da primeira camada: tem como resultado da derivada $\mathcal{A}_{\text{SELU}}(y_j) = 1.5$
    \item SELU da camada densa 2: tem como resultado da derivada $\mathcal{A}_{\text{SELU}}(y_j) = 1.4$
    \item SELU da camada densa 3: tem como resultado da derivada $\mathcal{A}_{\text{SELU}}(y_j) = 1.45$
    \item SELU da camada de saída: tem como resultado da derivada $\mathcal{A}_{\text{SELU}}(y_j)= 1.5$
\end{itemize}

Além disso, você sabe também que o gradiente inicial na camada de saída está sendo de 25. Com isso é possível calcular o gradiente retropropagado para a primeira camada, comecando pela terceira, já que já temos o valor do gradiente para a camada de saída, dessa forma temos que:

\[\begin{WithArrows}
    \delta^{(3)} & = \delta^{(4)} \times \mathcal{A}_{\text{SELU}}'(x^{(3)}) \Arrow{Subtituindo os valores} \\
    \delta^{(3)} & = 25 \times 1.45 = 36.25
\end{WithArrows}\]

Seguindo adiante, é possível fazer o mesmo procedimento para encontrar $\delta^{(2)}$ agora já tendo $\delta^{(3)}$, dessa forma:

\[\begin{WithArrows}
    \delta^{(2)} & = \delta^{(3)} \times \mathcal{A}_{\text{SELU}}'(x^{(2)}) \Arrow{Subtituindo os valores} \\
    \delta^{(2)} & = 36.25 \times 1.4 = 50.75
\end{WithArrows}\]

De forma semelhante, é finalmente possível encontrar o gradiente retropropagado para a primeira camada:

\[\begin{WithArrows}
    \delta^{(1)} & = \delta^{(2)} \times \mathcal{A}_{\text{SELU}}'(x^{(1)}) \Arrow{Subtituindo os valores} \\
    \delta^{(1)} & = 50.125 \times 1 = 76.125
\end{WithArrows}\]

Perceba que o gradiente que antes era de 25, mais que triplicou, passando para 76.125. Caso você tivesse uma rede neural mais profunda, com 10 ou mais camadas, por exemplo, e todas essas camadas fizessem uso da \textit{SELU}, existe uma chance de que o gradiente que chegasse para as primeiras camadas tivesse um valor muito alto. Esse valor muito elevado pode afetar diretamente como os pesos e vieses da rede são atualizados, impedindo que a rede aprenda nas primeiras camadas. E como as primeiras camadas geralmente são responsáveis por aprender características mais básicas do problema, todo o aprendizado da rede sofre com isso. 

Esse é o problema do gradiente explosivo, e pode ser definido como:

\begin{definicaomoderna}{\textbf{Definição:}}
    Quando o erro é retropropagado por uma rede neural, ele pode aumentar exponencialmente de camada para camada. Nesses casos, o gradiente em relação aos parâmetros em camadas inferiores pode ser exponencialmente maior do que o gradiente em relação aos parâmetros em camadas superiores. Isso torna a rede difícil de treinar se ela for suficientemente profunda. Tendo então o problema do \textbf{gradiente explosivo} \parencite{ExplodingGradient}.
\end{definicaomoderna}

Assim, mesmo as retificadoras corrigindo o problema do desaparecimento do gradiente, ela acabou por introduzir uma nova categoria de problemas para uma rede neural. Acontecimentos assim são comuns, muitas vezes queremos concertar algo mas acabamos por atrapalhar outra parte de um projeto de rede neural, por isso, devemos escolher com calma quais funções serão utilizadas além de realizar testes para garantir uma melhor performance do modelo que está sendo criado.

Perceba também que se você tiver uma função como a \textit{ReLU}, em que o maior valor retornado por sua derivada é 1, ainda sim isso pode ser um problema. Pois, caso o gradiente que é calculado para a última camada seja muito alto, ele vai voltar para as primeiras camadas também com o mesmo valor, considerando que todas as funções \textit{ReLU} retornem 1 em suas derivadas. Dessa forma, ainda sim, há um problema em como o gradiente é propagado pela rede.

Conhecidas todas essas diferentes funções de ativação retificadoras, começando pela \textit{ReLU}, com as suas propriedades e como ela foi importante para o desenvolvimento de redes neurais mais profundas. E, terminando explicando o problema do gradiente explosivo, cabe então sumarizar o conteúdo visto. Para isso, é possível ver esse resumo na próxima seção.

\section{Comparativo: Funções Retificadoras} \index{Comparativos!Funções retificadoras}

Por fim, visto todas essas funções, é possível compilá-las na Tabela \ref{tab:comparativo-funcoes-retificadoras}. A qual é responsável por destacar a principal característica de cada uma das funções retificadoras, bem como suas vantagens e desvantagens de forma resumida.

\begin{table}[htbp]
    \centering
    \begin{threeparttable}
        \caption{Comparativo das funções de ativação retificadoras}
        \label{tab:comparativo-funcoes-retificadoras}
        % p{3.2cm} define uma largura fixa para a primeira coluna.
        % As 3 colunas 'X' restantes dividem o espaço que sobra de forma flexível.
        % >{\raggedright\arraybackslash} alinha o texto à esquerda para melhor leitura.
        \begin{tabularx}{\textwidth}{p{3.2cm} *{3}{>{\raggedright\arraybackslash}X}}
            \toprule
            \textbf{Função} & \textbf{Principal característica} &\textbf{Vantagem} & \textbf{Desvantagem} \\
            \midrule
            \textit{Rectfied Linear Unit} (\textit{ReLU}) & Retorna $y_j$ quando $y_j > 0$, caso contrário, retorna zero. & É a função padrão para ser utilizada em uma rede \textit{feedforwad}. & Sobre do problema dos \textit{ReLUs} agonizantes. \\
            \addlinespace
            \textit{Leaky ReLU} (\textit{LReLU}) & Variante da \textit{ReLU} que apresenta um coeficiente $\alpha$, permitindo um pequeno "vazamento" de gradiente em situações que $y_j < 0$ & Consegue mitigar o problema dos \textit{ReLUs} agonizantes. & Adiciona mais um hiperparâmetro que precisa ser ajustado manualmente. \\
            \addlinespace
            \textit{Parametric ReLU} (\textit{PReLU}) & Variante da \textit{Leaky ReLU} em que $\alpha$ é um parâmetro aprendível & Por possuir um coeficiente aprendível, a \textit{PReLU} apresenta uma tendência maior de se adptar aos dados. & Apresenta mais parâmetros, aumentando o grau de complexidade da rede neural. \\
            \addlinespace
            \textit{Randomized Leaky ReLU} (\textit{RReLU}) & Versão da \textit{Leaky ReLU} em que $\alpha$ é dado por um número aleatório dado pela distribuição normal da forma $U(l, u)$ & Adiciona maior aleatoriedade para a rede, ajudando a combater o sobreajuste. & Por possuir uma natureza aleatória pode deixar o treinamento menos determinístico.  \\
            \addlinespace
            \textit{Exponential Linear Unit} (\textit{ELU}) & Versão exponencial que busca imitar o comportamento da \textit{ReLU} & É uma função suave e derivável em todos os seus pontos. & Apresenta exponenciais em sua fórmula, sendo mais "cara" em termos de custo computacional. \\
            \addlinespace
            \textit{Scale Exponential Linear Unit} (\textit{SELU}) & Versão escalada da \textit{ELU}. & Possui propriedades auto-normalizadoras, permitindo criar as \textit{SNNs} & Assim como a \textit{ELU}, a \textit{SELU} apresenta exponenciais em sua composição, sendo mais "cara" que outras funções dessa tabela.\\
            \addlinespace
            \textit{Noisy ReLU} (\textit{NReLU}) & Versão da \textit{ReLU} que adiciona ruído Gaussiano em sua fórmula. & Apresenta \textit{intensity equivariance}, ajudando a reconhecer padrões em diferentes situações. & Não pode ser derivada, sendo usada para o \textit{backward pass} a derivada da \textit{ReLU}. \\
            \addlinespace
        \end{tabularx}
        
        \begin{tablenotes}[para]
            \small
            \item[] Fonte: O autor (2025).
        \end{tablenotes}

    \end{threeparttable}
\end{table}
% ===================================================================
% Arquivo: capitulos/parte-III-pilares/cap-09-modernas.tex
% ===================================================================

\chapter{Funções de Ativação Modernas e Outras Funções de Ativação}
\label{cap:ativacao-modernas-outras}

\section{Funções Modernas: O Estado-da-Arte das Funções de Ativação}

\subsection{Gaussian Error Linear Unit (GELU)}

Agora chegamos na última função que iremos ver neste texto, ela é a Gaussian Error Linear Unit, ou GELU. Ela é uma das mais diferentes quando nós passamos a analisar como é sua fórmula, a qual veremos mais em frente. A GELU foi introduzida no artigo \textit{Gaussian Error Linear Units (GELUs)} dos autores \textcite{GELUArticle}, nele, os pesquisadores apresentam uma nova função de ativação de alta performance para ser utilizada na construção de redes neurais.

No texto, os autores, avaliam a GELU e outras variantes como a Exponential Linear Unit e a ReLU tradicional no dataset MNIST, o qual apresenta 10 classes de imagens em escala de cinza sendo 60.000 imagens de treino e 10.000 de teste, os resultados de como a perda e robustez a ruído dessas funções se comporta o longo do experimento podem ser vistos nas figuras \ref{fig:log-loss-gelu} e \ref{fig:accuracy-gelu} respectivamente \parencite{GELUArticle}. Além disso, \textcite{GELUArticle}, avaliaram essas funções em outros conjuntos como o MNIST autoencoding, Tweet part-of-speech tagging, TIMIT frame recognition além dos datasets de imagens CIFAR-10/100. 

Analisando a figura \ref{fig:log-loss-gelu}, podemos compreender como essas diferentes funções contribuem para a diminuição da perda no modelo criado para a classificação do dataset MNIST. Nos vemos que a GELU é a função que apresenta a menor perda, mas não somente isso, ela é a que contribui para que ela diminua mais rapidamente. Nos casos em que foi utilizado o dropout nas camadas os resultados são ainda mais mais significativos, com a ReLU apresentando a maior perda dentre as três funções, a ELU em segundo, e GELU com uma diferença considerável em relação as suas concorrentes. Isso pode nos indicar que em redes neurais cujo técnicas como o dropout de neurônios nas camadas não seja uma prática viável, pode ser interessante utilizar como estratégia a GELU como função de ativação, pois mesmo sem o dropout, ela consegue manter um bom valor para a perda do modelo que está sendo desenvolvido.

Já na figura \ref{fig:accuracy-gelu}, podemos ver como a acurácia dos modelos se comporta quando é adicionado é adicionado ruído as dados de teste. Para isso, nós notamos que os todos os modelos possuem a mesma tendência de decrescer a sua acurácia ao longo do aumento do ruído, vemos que o modelo que é mais afetado com essa transformação é o que faz uso da exponential linear unit, enquanto os que fazem uso da ReLU e da GELU, mesmo encontrando grandes dificuldades para identificar coretamente as imagens, conseguem manter uma acurácia de quase 0.1 a mais que a ELU. Ainda na figura \ref{fig:accuracy-gelu}, também vemos como a perda no conjunto de testes de comporta ao aumentar o ruído, aqui vemos uma situação diferente, vemos que o modelo que menos se adequou ao que estava analisando foi o que fazia uso da ReLU, pois, encontrou difuculdades em tentar minizar o cálculo da perda, por outro lado temos a GELU, que mesmo aumentando o ruído, conseguiu manter uma diferença considerável quando comparada a essas outras duas funções.

Além disso, como dito anteriormente, os autores também fazem testes comparando a GELU com outras funções de ativação utilizando também o dataset CIFAR-10, o qual vem sendo discutido em seções anteriores deste texto, assim, temos a figura \ref{fig:gelu-cifar-10}, que nos mostra esse comparativo com a taxa de erro \parencite{GELUArticle}. Com base nessa análise, nós podemos concluir que a GELU é a melhor alternativa dentre essas três funções para a rede que foi criada, apresentando a menor taxa de erro, tanto no conjunto de dados de treino quanto no conjunto de dados de teste. Uma observação interessante a ser feita com base neste gráfico é de que esses modelos foram treinados por 200 épocas no total, e como a GELU é uma função bem mais complexa que a ReLU, o tempo de treino do modelo que fez uso dessa função foi provavelmente bem maior, algo que pode ser levado em consideração caso seja necessário criar uma rede que seja treinada mais rapidamente mas que ainda sim tenha uma taxa de erro baixa.

Conhecendo um pouco como a GELU atua em uma rede neural, podemos agora conhecer ela por meio da sua fórmula, a qual é dada pela expressão \ref{eq:gelu}, a qual é um tanto diferente das outras expressões que vimos até agora. Note que nós não temos dessa vez uma expressão condicional como na ReLU e suas outras variantes, temos uma expressão única, que pode ser reescrita utilizando outras expressões diferentes. Mais a esquerda, temos o termo $\Phi(z_i)$ que representa o standard Gaussian cumalative distribution function, já na expressão mais a direita, temos o uso da função erro, uma função bem comum de ser utilizada quando estamos trabalhando com conceitos probabilísticos.

\begin{equacaodestaque}{Gaussian Error Linear Unit}
    \mathcal{A}_{\text{GELU}}(z_i) = z_i P(X \leqslant z_i) = z_i \Phi (z_i) = z_i \frac{1}{2} \left[ 1 + \text{erf} (z_i/ \sqrt{2}) \right]
    \label{eq:gelu}
\end{equacaodestaque}

Por apresentar cálculos mais complexos em sua composição, como o uso da função erro para encontrar o satrd gaussian cumative distribution function os autores também apresentam aproximações para a GELU, elas são dadas pelas equações \ref{eq:EquacaoGELUAprox} e \ref{eq:EquacaoGELUAprox2} \parencite{GELUArticle}. Essas aproximações facilitam não somente os cálculos mas também na hora de implementar essa função em Python, garantindo algoritmos mais curtos e fáceis de serem implementados.

Na expressão \ref{eq:gelu-aproximacoes} vemos que ela pode ser aproximada utilizando a função tangente como um dos componentes usados, já na expressão \ref{eq:gelu-aproximacoes-sigmoide}, os autores utilizam como base a Sigmoid Linear Unit (SiLU), a qual é dada pela fórmula $SiLU = x\sigma(x)$ para criar uma expressão que seja capaz de aproximar como a GELU se comporta mas trazer uma maior velocidade de processamento por apresentar cálculos mais simples em sua composição.
    
\begin{equacaodestaque}{Gaussian Error Linear Unit Aproximações}
    \mathcal{A}_{\text{GELU}(x)} \approx 0.5x \left(1 + \tanh\left[\sqrt{\frac{2}{\pi}} \left(x + 0.044715x^3\right)\right]\right) \\
    \label{eq:gelu-aproximacoes}
\end{equacaodestaque}

\begin{equacaodestaque}{Gaussian Error Linear Unit Aproximação com Sigmoide}
    \mathcal{A}_{\text{GELU}}(x) \approx x \sigma(1.702x)
    \label{eq:gelu-aproximacoes-sigmoide}
\end{equacaodestaque}

Se sabemos a sua fórmula, podemos também plotar o seu gráfico, para isso temos a figura \ref{fig:gelu}, esse gráfico é uma aproximação, tendo como base a expressão \ref{eq:gelu-aproximacoes}. Note que ela é uma função assimétrica, que possui o comportamento quase que de uma função identidade para os casos em que a sua entrada é maior que zero, além disso, podemos ver que ela retorna valores não nulos quando a entrada é negativa, mas apenas até um certo ponto, depois ela assume o comportamento de um função constante. O fato dela retornar valores quando alguns valores da entrada são negativos pode acabar contribuindo para que essa função diminua o problema do neurônios agonizantes, além disso, por ser uma variante da ReLU, ela também é capaz de resolver o problema do gradiente em fuga.

\begin{figure}[htbp]
    \centering
    \begin{tikzpicture}
        \begin{axis}[
            xlabel={$z_i$},
            ylabel={$\text{GELU}(z_i)$},
            xmin=-3.5, xmax=3.5,
            ymin=-0.5, ymax=3.5,
            axis lines=middle,
            grid=major,
            legend pos=outer north east,
            legend style={font=\scriptsize}
        ]
        % Plota a APROXIMAÇÃO da função GELU(x) usando tanh
        \addplot[red, thick, domain=-3.5:3.5, samples=100, smooth] 
            {0.5*x*(1 + tanh(sqrt(2/pi)*(x + 0.044715*x^3)))};
        \addlegendentry{GELU($x$) (Aproximação)}
        
        \end{axis}
    \end{tikzpicture}
    \caption{Gráfico da função de ativação Gaussian Error Linear Unit (GELU) usando aproximação.}
    \label{fig:gelu}
    \fonte{O autor (2025).}
\end{figure}

Considerando as suas expressões e como a GELU se comporta, podemos também calcular a sua derivada para ser utilizada na retropropagação do modelo. Para isso, temos a expressão \ref{eq:gelu-derivada}, a qual pode ser expandida em uma equação mais completa, resultando então na fórmula da expressão \ref{eq:gelu-derivada-completa}.

\begin{equacaodestaque}{Derivada Gaussian Error Linear Unit}
    \frac{d}{dz_i} [\mathcal{A}_{\text{GELU}}](z_i) = \Phi(z_i) + z_i\phi(z_i)
    \label{eq:gelu-derivada}
\end{equacaodestaque}

\begin{equacaodestaque}{Derivada Completa Gaussian Error Linear Unit}
    \frac{d}{dx} [\mathcal{A}_{\text{GELU}}](z_i) = \Phi(z_i) + \frac{z_i}{\sqrt{2\pi}} e^{-\frac{z_i^2}{2}}
    \label{eq:gelu-derivada-completa}
\end{equacaodestaque}

Para plotarmos o gráfico de sua derivada podemos fazer uma aproximação utilizando $\phi(x) ~= 1/(1+exp(-1.702*x))$, com base nela, encontramos como resultado a figura \ref{fig:GraficoGELUDerivada}. Note que ele também é diferente dos gráficos que estávamos vendo até agora, ele é contínuo em todo o seu domínio, diferente das funções mais simples, com a ReLU e a Leaky ReLU, além de possuir um carater saturante, assim, para valores acima de três o ele irá retornar valores próximos de um, indicando que o não irá colaborar para que o problema do gradiente em fuga ocorra como no caso das funções sigmodais. Além disso, para valores abaixo de -3 a derivada da GELU retorna valores próximos de zero, algo que tem em comum com a ReLU e algumas de suas variantes.

\begin{figure}[htbp]
    \centering
    \begin{tikzpicture}
        \begin{axis}[
            xlabel={$z_i$},
            ylabel={$\text{GELU}'(o_I)$},
            xmin=-3.5, xmax=3.5,
            ymin=-0.3, ymax=1.3,
            axis lines=middle,
            grid=major,
            legend pos=outer north east,
            legend style={font=\scriptsize}
        ]
        
        % Plota a derivada usando uma aproximação para Phi(x)
        % Phi(x) ~= 1/(1+exp(-1.702*x))
        \addplot[blue, thick, domain=-3.5:3.5, samples=100, smooth] 
            {(1/(1+exp(-1.702*x))) + x * (1/sqrt(2*pi)) * exp(-x^2/2)};
        \addlegendentry{Derivada: $\Phi(x) + x\phi(x)$}
        
        \end{axis}
    \end{tikzpicture}
    \caption{Gráfico da derivada da função de ativação Gaussian Error Linear Unit (GELU).}
    \label{fig:gelu-derivada}
    \fonte{O autor (2025).}
\end{figure}

Ainda no artigo \textit{Gaussian Error Linear Units (GELUs)}, os autores discutem outras informações úteis da Gaussian Error Linear Unit para serem considerados ao construir uma rede neural com essa função de ativação, o primeiro deles é de que é recomendado o uso de um otimizador com momentum quando estiver treinando uma rede com a GELU \parencite{GELUArticle}. Em segundo lugar, \textcite{GELUArticle}, destacam que é importante utilizar uma aproximação próxima da distribuição acumulativa da distribuição gaussiana, entretanto, funções como a sigmoide, são uma aproximação acumulativa da distribuição normal, contudo, a SiLU mesmo performando pior que a GELU, ainda sim, é capaz de performar melhor que outras retificadoras como a ELU e a ReLU nos testes realizados pelos autores, assim, faz necessário o uso de novas aproximações, como as vistas nas expressões \ref{eq:gelu-aproximacoes} e \ref{eq:gelu-aproximacoes-sigmoide}.

\medskip
\begin{center}
 * * *
\end{center}
\medskip

\textbf{Algumas Aplicações da Gaussian Error Linear Unit em Redes Neurais}

\begin{itemize}
    \item \textbf{Aplicação 1 (Área):}
    \item \textbf{Aplicação 2 (Área):}
    \item \textbf{Aplicação 3 (Área):}
    \item \textbf{Aplicação 4 (Área):}
\end{itemize}

\medskip
\begin{center}
 * * *
\end{center}
\medskip

\subsection{Swish}

\begin{equacaodestaque}{Swish}
    \mathcal{A}_{\text{Swish}}(z_i) = z_i \cdot \sigma(z_i) = z_i \frac{1}{1 + e^{-z_i}}
    \label{eq:swish}
\end{equacaodestaque}

\begin{figure}[htbp]
    \centering
    \begin{tikzpicture}
        \begin{axis}[
            xlabel={$z_i$},
            ylabel={$\text{Swish}(z_i)$},
            xmin=-5.5, xmax=5.5,
            ymin=-1.5, ymax=5.5,
            axis lines=middle,
            grid=major,
            legend pos=outer north east,
            legend style={font=\scriptsize}
        ]
        % Plota a função Swish: x * (1 / (1 + exp(-x)))
        \addplot[blue, thick, domain=-5.5:5.5, samples=100, smooth] 
            {x / (1 + exp(-x))};
        \addlegendentry{Swish($z_i$) = $z_i \cdot \sigma(z_i)$}
        
        \end{axis}
    \end{tikzpicture}
    \caption{Gráfico da função de ativação Swish.}
    \label{fig:swish}
    \fonte{O autor (2025).}
\end{figure}

\begin{equacaodestaque}{Derivada Swish}
    \frac{d}{dz_i} [\mathcal{A}_{\text{Swish}}](z_i) = \text{Swish}(z_i) + \sigma(z_i)(1 - \text{Swish}(z_i))
    \label{eq:swish-derivada}
\end{equacaodestaque}

\begin{figure}[htbp]
    \centering
    \begin{tikzpicture}
        \begin{axis}[
            xlabel={$z_i$},
            ylabel={$\text{Swish}'(z_i)$},
            xmin=-5.5, xmax=5.5,
            ymin=-0.3, ymax=1.3,
            axis lines=middle,
            grid=major,
            legend pos=outer north east,
            legend style={font=\scriptsize}
        ]
        
        % Plota a derivada da função Swish
        % Swish'(x) = Swish(x) + sigmoid(x)*(1 - Swish(x))
        \addplot[orange, thick, domain=-5.5:5.5, samples=101, smooth] 
            { (x / (1 + exp(-x))) + (1 / (1 + exp(-x))) * (1 - (x / (1 + exp(-x)))) };
        \addlegendentry{$\text{Swish}'(z_i) = \text{Swish}(z_i) + \sigma(z_i)(1-\text{Swish}(z_i))$}
        
        \end{axis}
    \end{tikzpicture}
    \caption{Gráfico da derivada da função de ativação Swish.}
    \label{fig:swish-derivada}
    \fonte{O autor (2025).}
\end{figure}

\medskip
\begin{center}
 * * *
\end{center}
\medskip

\textbf{Algumas Aplicações da Swish em Redes Neurais}

\begin{itemize}
    \item \textbf{Aplicação 1 (Área):}
    \item \textbf{Aplicação 2 (Área):}
    \item \textbf{Aplicação 3 (Área):}
    \item \textbf{Aplicação 4 (Área):}
\end{itemize}

\medskip
\begin{center}
 * * *
\end{center}
\medskip

\subsection{Hard-Swish (h-swish)}

\begin{equacaodestaque}{Hard-Swish}
    \mathcal{A}_{\text{h-Swish}}(z_i) = z_i \cdot \frac{\text{ReLU6}(z_i + 3)}{6}
    \label{eq:h-swish}
\end{equacaodestaque}

\begin{figure}[htbp]
    \centering
    \begin{tikzpicture}
        \begin{axis}[
            xlabel={$z_i$},
            ylabel={$\text{h-Swish}(z_i)$},
            xmin=-5.5, xmax=5.5,
            ymin=-1.5, ymax=5.5,
            axis lines=middle,
            grid=major,
            legend pos=outer north east,
            legend style={font=\scriptsize}
        ]
        % Plota a função Hard-Swish: x * ReLU6(x + 3) / 6
        % ReLU6(x) = min(max(0, x), 6)
        \addplot[purple, thick, domain=-5.5:5.5, samples=100, smooth] 
            {x * min(max(0, x + 3), 6) / 6};
        \addlegendentry{h-Swish($z_i$) = $z_i \cdot \frac{\text{ReLU6}(z_i + 3)}{6}$}
        
        \end{axis}
    \end{tikzpicture}
    \caption{Gráfico da função de ativação Hard-Swish.}
    \label{fig:h-swish}
    \fonte{O autor (2025).}
\end{figure}

\begin{equacaodestaque}{Derivada Hard-Swish}
    \frac{d}{dz_i} [\mathcal{A}_{\text{h-Swish}}](z_i) = \begin{cases} 0 & \text{se } z_i \le -3 \\ \frac{2z_i + 3}{6} & \text{se } -3 < z_i < 3 \\ 1 & \text{se } z_i \ge 3 \end{cases}
    \label{eq:h-swish-derivada}
\end{equacaodestaque}

\begin{figure}[htbp]
    \centering
    \begin{tikzpicture}
        \begin{axis}[
            xlabel={$z_i$},
            ylabel={$\text{h-Swish}'(z_i)$},
            xmin=-5.5, xmax=5.5,
            ymin=-0.3, ymax=1.3,
            axis lines=middle,
            grid=major,
            legend pos=outer north east,
            legend style={font=\scriptsize}
        ]
        
        % Plota a derivada da função Hard-Swish
        \addplot[orange, thick, domain=-5.5:5.5, samples=101, smooth] 
            { (x <= -3) ? 0 : ((x >= 3) ? 1 : ( (2*x + 3) / 6 )) };
        \addlegendentry{$\text{h-Swish}'(z_i)$}
        
        \end{axis}
    \end{tikzpicture}
    \caption{Gráfico da derivada da função de ativação Hard-Swish.}
    \label{fig:h-swish-derivada}
    \fonte{O autor (2025).}
\end{figure}

\medskip
\begin{center}
 * * *
\end{center}
\medskip

\textbf{Algumas Aplicações da Hard Swish em Redes Neurais}

\begin{itemize}
    \item \textbf{Aplicação 1 (Área):}
    \item \textbf{Aplicação 2 (Área):}
    \item \textbf{Aplicação 3 (Área):}
    \item \textbf{Aplicação 4 (Área):}
\end{itemize}

\medskip
\begin{center}
 * * *
\end{center}
\medskip

\subsection{Mish}

\begin{equacaodestaque}{Mish}
    \mathcal{A}_{\text{Mish}}(z_i) = z_i \cdot \tanh(\text{softplus}(z_i)) = z_i \cdot \tanh(\ln(1 + e^{z_i}))
    \label{eq:mish}
\end{equacaodestaque}

\begin{figure}[htbp]
    \centering
    \begin{tikzpicture}
        \begin{axis}[
            xlabel={$z_i$},
            ylabel={$\text{Mish}(z_i)$},
            xmin=-5.5, xmax=5.5,
            ymin=-1.5, ymax=5.5,
            axis lines=middle,
            grid=major,
            legend pos=outer north east,
            legend style={font=\scriptsize}
        ]
        % Plota a função Mish: x * tanh(ln(1 + exp(x)))
        \addplot[purple, thick, domain=-5.5:5.5, samples=100, smooth] 
            {x * tanh(ln(1 + exp(x)))};
        \addlegendentry{Mish($z_i$) = $z_i \cdot \tanh(\text{softplus}(z_i))$}
        
        \end{axis}
    \end{tikzpicture}
    \caption{Gráfico da função de ativação Mish.}
    \label{fig:mish}
    \fonte{O autor (2025).}
\end{figure}

\begin{equacaodestaque}{Derivada Mish}
    \frac{d}{dz_i} [\mathcal{A}_{\text{Mish}}](z_i) = \tanh(\omega) + z_i \sigma(z_i) \text{sech}^2(\omega) \\
    \text{onde } \omega = \text{softplus}(z_i) \text{ e } \sigma(z_i) \text{ é a função sigmoide.}
    \label{eq:mish-derivada}
\end{equacaodestaque}

\begin{figure}[htbp]
    \centering
    \begin{tikzpicture}
        \begin{axis}[
            xlabel={$z_i$},
            ylabel={$\text{Mish}'(z_i)$},
            xmin=-5.5, xmax=5.5,
            ymin=-0.3, ymax=1.3,
            axis lines=middle,
            grid=major,
            legend pos=outer north east,
            legend style={font=\scriptsize}
        ]
        
        % Plota a derivada da função Mish
        % Mish'(x) = tanh(sp(x)) + x * sigmoid(x) * sech^2(sp(x))
        % sp(x) = ln(1 + exp(x))
        % sigmoid(x) = 1 / (1 + exp(-x))
        % sech^2(x) = (1 / cosh(x))^2
        \addplot[red, thick, domain=-5.5:5.5, samples=101, smooth] 
            { tanh(ln(1 + exp(x))) + x * (1 / (1 + exp(-x))) * (1 / cosh(ln(1 + exp(x))))^2 };
        \addlegendentry{$\text{Mish}'(z_i)$}
        
        \end{axis}
    \end{tikzpicture}
    \caption{Gráfico da derivada da função de ativação Mish.}
    \label{fig:mish-derivada}
    \fonte{O autor (2025).}
\end{figure}

\medskip
\begin{center}
 * * *
\end{center}
\medskip

\textbf{Algumas Aplicações da Mish em Redes Neurais}

\begin{itemize}
    \item \textbf{Aplicação 1 (Área):}
    \item \textbf{Aplicação 2 (Área):}
    \item \textbf{Aplicação 3 (Área):}
    \item \textbf{Aplicação 4 (Área):}
\end{itemize}

\medskip
\begin{center}
 * * *
\end{center}
\medskip

\subsection{Hard-Mish (h-mish)}

\begin{equacaodestaque}{Hard-Mish}
    \mathcal{A}_{\text{h-Mish}}(z_i) = \frac{z_i}{2} \cdot \min(\max(z_i + 2, 0), 2)
    \label{eq:h-mish}
\end{equacaodestaque}

\begin{figure}[htbp]
    \centering
    \begin{tikzpicture}
        \begin{axis}[
            xlabel={$z_i$},
            ylabel={$\text{h-Mish}(z_i)$},
            xmin=-5.5, xmax=5.5,
            ymin=-1.5, ymax=5.5,
            axis lines=middle,
            grid=major,
            legend pos=outer north east,
            legend style={font=\scriptsize}
        ]
        % Plota a função Hard-Mish: x/2 * min(max(x + 2, 0), 2)
        \addplot[brown, thick, domain=-5.5:5.5, samples=100, smooth] 
            {x * min(max(0, x + 2), 2) / 2};
        \addlegendentry{h-Mish($z_i$)}
        
        \end{axis}
    \end{tikzpicture}
    \caption{Gráfico da função de ativação Hard-Mish.}
    \label{fig:h-mish}
    \fonte{O autor (2025).}
\end{figure}

\begin{equacaodestaque}{Derivada Hard-Mish}
    \frac{d}{dz_i} [\mathcal{A}_{\text{h-Mish}}](z_i) = \begin{cases} 0 & \text{se } z_i \le -2 \\ z_i + 1 & \text{se } -2 < z_i < 0 \\ 1 & \text{se } z_i \ge 0 \end{cases}
    \label{eq:h-mish-derivada}
\end{equacaodestaque}

\begin{figure}[htbp]
    \centering
    \begin{tikzpicture}
        \begin{axis}[
            xlabel={$z_i$},
            ylabel={$\text{h-Mish}'(z_i)$},
            xmin=-5.5, xmax=5.5,
            ymin=-0.3, ymax=1.3,
            axis lines=middle,
            grid=major,
            legend pos=outer north east,
            legend style={font=\scriptsize}
        ]
        
        % Plota a derivada da função Hard-Mish
        \addplot[red, thick, domain=-5.5:5.5, samples=101, smooth] 
            { (x <= -2) ? 0 : ((x >= 0) ? 1 : (x + 1)) };
        \addlegendentry{$\text{h-Mish}'(z_i)$}
        
        \end{axis}
    \end{tikzpicture}
    \caption{Gráfico da derivada da função de ativação Hard-Mish.}
    \label{fig:h-mish-derivada}
    \fonte{O autor (2025).}
\end{figure}

\medskip
\begin{center}
 * * *
\end{center}
\medskip

\textbf{Algumas Aplicações da Hard Mish em Redes Neurais}

\begin{itemize}
    \item \textbf{Aplicação 1 (Área):}
    \item \textbf{Aplicação 2 (Área):}
    \item \textbf{Aplicação 3 (Área):}
    \item \textbf{Aplicação 4 (Área):}
\end{itemize}

\medskip
\begin{center}
 * * *
\end{center}
\medskip

\section{Funções Para Camadas de Saída}

\subsection{Softmax}

\begin{equacaodestaque}{Softmax}
    \mathcal{A}_{\text{Softmax}}(z_i) = \frac{e^{z_i}}{\sum_{j=1}^{K} e^{z_j}}
    \label{eq:softmax}
\end{equacaodestaque}

\begin{equacaodestaque}{Derivada Softmax}
    \frac{\partial [\mathcal{A}_{\text{Softmax}}](z_i)}{\partial z_j} = 
    \begin{cases} 
      \text{Softmax}(z_i)(1 - \text{Softmax}(z_i)) & \text{se } i = j \\
      - \text{Softmax}(z_i)\text{Softmax}(z_j) & \text{se } i \neq j
    \end{cases}
    \label{eq:softmax-derivada}
\end{equacaodestaque}

\subsection{Identidade (Linear)}

\begin{equacaodestaque}{Identidade}
    \mathcal{A}_{\text{Linear}}(z_i) = z_i
    \label{eq:linear}
\end{equacaodestaque}

\begin{figure}[htbp]
    \centering
    \begin{tikzpicture}
        \begin{axis}[
            xlabel={$z_i$},
            ylabel={$\text{Linear}(z_i)$},
            xmin=-5.5, xmax=5.5,
            ymin=-5.5, ymax=5.5, % Ajustado para y=x
            axis lines=middle,
            grid=major,
            legend pos=outer north east,
            legend style={font=\scriptsize}
        ]
        % Plota a função Identidade: y = x
        \addplot[blue, thick, domain=-5.5:5.5, samples=10, smooth] 
            {x};
        \addlegendentry{Linear($z_i$) = $z_i$}
        
        \end{axis}
    \end{tikzpicture}
    \caption{Gráfico da função de ativação Linear (Identidade).}
    \label{fig:linear}
    \fonte{O autor (2025).}
\end{figure}

\begin{equacaodestaque}{Derivada Identidade}
    \frac{d}{dz_i} [\mathcal{A}_{\text{Linear}}](z_i) = 1
    \label{eq:linear-derivada}
\end{equacaodestaque}

\begin{figure}[htbp]
    \centering
    \begin{tikzpicture}
        \begin{axis}[
            xlabel={$z_i$},
            ylabel={$\text{Linear}'(z_i)$},
            xmin=-5.5, xmax=5.5,
            ymin=-0.3, ymax=1.3,
            ytick={1}, % Útil para mostrar que é constante
            axis lines=middle,
            grid=major,
            legend pos=outer north east,
            legend style={font=\scriptsize}
        ]
        
        % Plota a derivada da função Linear: y = 1
        \addplot[orange, thick, domain=-5.5:5.5, samples=10, smooth] 
            {1};
        \addlegendentry{$\text{Linear}'(z_i) = 1$}
        
        \end{axis}
    \end{tikzpicture}
    \caption{Gráfico da derivada da função de ativação Linear.}
    \label{fig:linear-derivada}
    \fonte{O autor (2025).}
\end{figure}

\subsection{Softplus}

\begin{equacaodestaque}{Softplus}
    \mathcal{A}_{\text{Softplus}}(z_i) = \ln(1 + e^{z_i})
    \label{eq:softplus}
\end{equacaodestaque}

\begin{figure}[htbp]
    \centering
    \begin{tikzpicture}
        \begin{axis}[
            xlabel={$z_i$},
            ylabel={$\text{Softplus}(z_i)$},
            xmin=-5.5, xmax=5.5,
            ymin=-0.5, ymax=5.5, % Eixo do template funciona bem
            axis lines=middle,
            grid=major,
            legend pos=outer north east,
            legend style={font=\scriptsize}
        ]
        % Plota a função Softplus: ln(1 + exp(x))
        \addplot[green, thick, domain=-5.5:5.5, samples=100, smooth] 
            {ln(1 + exp(x))};
        \addlegendentry{Softplus($z_i$) = $\ln(1 + e^{z_i})$}
        
        \end{axis}
    \end{tikzpicture}
    \caption{Gráfico da função de ativação Softplus.}
    \label{fig:softplus}
    \fonte{O autor (2025).}
\end{figure}

\begin{equacaodestaque}{Derivada Softplus}
    \frac{d}{dz_i} [\mathcal{A}_{\text{Softplus}}](z_i) = \frac{e^{z_i}}{1 + e^{z_i}} = \frac{1}{1 + e^{-z_i}} = \sigma(z_i)
    \label{eq:softplus-derivada}
\end{equacaodestaque}

\begin{figure}[htbp]
    \centering
    \begin{tikzpicture}
        \begin{axis}[
            xlabel={$z_i$},
            ylabel={$\text{Softplus}'(z_i)$},
            xmin=-5.5, xmax=5.5,
            ymin=-0.3, ymax=1.3,
            axis lines=middle,
            grid=major,
            legend pos=outer north east,
            legend style={font=\scriptsize}
        ]
        
        % Plota a derivada da função Softplus (função Sigmoide)
        \addplot[teal, thick, domain=-5.5:5.5, samples=101, smooth] 
            { 1 / (1 + exp(-x)) };
        \addlegendentry{$\text{Softplus}'(z_i) = \sigma(z_i)$}
        
        \end{axis}
    \end{tikzpicture}
    \caption{Gráfico da derivada da função de ativação Softplus, que é a função Sigmoide.}
    \label{fig:softplus-derivada}
    \fonte{O autor (2025).}
\end{figure}

\subsection{Exponencial (exp)}

\begin{equacaodestaque}{Exponencial}
    \mathcal{A}_{\text{Exp}}(z_i) = e^{z_i}
    \label{eq:exp}
\end{equacaodestaque}

\begin{figure}[htbp]
    \centering
    \begin{tikzpicture}
        \begin{axis}[
            xlabel={$z_i$},
            ylabel={$\exp(z_i)$},
            xmin=-4.5, xmax=2.5, % Eixos ajustados
            ymin=-0.5, ymax=12.5, % Eixos ajustados
            axis lines=middle,
            grid=major,
            legend pos=outer north east,
            legend style={font=\scriptsize}
        ]
        % Plota a função Exponencial: exp(x)
        \addplot[red, thick, domain=-4.5:2.5, samples=100, smooth] 
            {exp(x)};
        \addlegendentry{Exp($z_i$) = $e^{z_i}$}
        
        \end{axis}
    \end{tikzpicture}
    \caption{Gráfico da função de ativação Exponencial (eixos ajustados).}
    \label{fig:exp}
    \fonte{O autor (2025).}
\end{figure}

\begin{equacaodestaque}{Derivada Exponencial}
    \frac{d}{dz_i} [\mathcal{A}_{\text{Exp}}](z_i) = e^{z_i}
    \label{eq:exp-derivada}
\end{equacaodestaque}

\begin{figure}[htbp]
    \centering
    \begin{tikzpicture}
        \begin{axis}[
            xlabel={$z_i$},
            ylabel={$\exp'(z_i)$},
            xmin=-4.5, xmax=2.5, % Eixos ajustados
            ymin=-0.5, ymax=12.5, % Eixos ajustados
            axis lines=middle,
            grid=major,
            legend pos=outer north east,
            legend style={font=\scriptsize}
        ]
        
        % Plota a derivada da função Exponencial: exp(x)
        \addplot[orange, thick, domain=-4.5:2.5, samples=101, smooth] 
            {exp(x)};
        \addlegendentry{$\text{Exp}'(z_i) = e^{z_i}$}
        
        \end{axis}
    \end{tikzpicture}
    \caption{Gráfico da derivada da função de ativação Exponencial (eixos ajustados).}
    \label{fig:exp-derivada}
    \fonte{O autor (2025).}
\end{figure}




% ===================================================================
% Arquivo: capitulos/parte-III-pilares/cap-10-perda-binaria.tex
% ===================================================================

\chapter{Funções de Perda para Regressão}
\label{cap:perda-regressao}

Até agora foi visto o funcionamento da retropropagação, e como ela faz uso dos otimizadores, os quais funcionam como um barco, percorrendo a função de perda em busca de pontos de mínimos. Além disso, em seguida foram vistas diversas funções de ativação, começando pelas sigmoidais, depois pelas retificadoras, e por fim uma coletânea de diferentes funções. Contudo, está na hora de entender o outro lado da retropropagação: as funções de perda.

Para isso, esse capítulo busca explicar diversas funções de perda e suas aplicações, começando pelas funções para problemas de regressão, conhecendo as clássicas erro quadrático médio e erro absoluto médio, além da \textit{hubber loss}, uma função que busca unir o melhor dessas duas funções de perda. Seguindo adiante, são introduzidas as funções de perda para classificação binária, como a \textit{BCE}. Visto os problemas de classificação binária, é possível também conhecer os problemas de classifação multi com a \textit{categorical cross entropy}.

Mais adiante, está apresentado não funções, mas esquemas de como a perda pode ser medida para problemas como o de redes adversárias. Mas as perdas não são a única forma de medir como um modelo está performando, para isso, o final do capítulo é dedicado para explicar outros diferentes métodos de medir o desempenho do modelo que está sendo construído.

\section{A Intuição da Perda: Medindo o Erro do Modelo}

\section{Exemplo Ilustrativo: Jogando Dados}

\section{Funções de Perda para Regressão para Propósitos Gerais}

\subsection{Erro Quadrático Médio (Mean Squared Error - MSE)}


\begin{equacaodestaque}{Erro Quadrático Médio (MSE)}
    L_{\text{MSE}} = \frac{1}{N} \sum_{i=1}^{N} (y_i - \hat{y}_i)^2
    \label{eq:mse}
\end{equacaodestaque}

\begin{tikzpicture}
    \begin{axis}[
        title={Erro Quadrático Médio (MSE)},
        xlabel={Erro ($y - \hat{y}$)},
        ylabel={Perda Calculada},
        axis lines=middle,          % Eixos centrados em (0,0)
        grid=major,                 % Adiciona uma grade principal
        grid style={dashed, gray!40}, % Estilo da grade
        xmin=-3.5, xmax=3.5,        % Limites do eixo x
        ymin=-0.5, ymax=9.5,         % Limites do eixo y
        legend pos=north west,      % Posição da legenda
        width=12cm,                 % Largura do gráfico
        height=9cm,                 % Altura do gráfico
        title style={font=\bfseries},
        label style={font=\small},
        tick label style={font=\scriptsize}
    ]
        % Adiciona o gráfico da função x^2
        \addplot[
            domain=-3:3, 
            samples=100, 
            color=blue, 
            very thick
        ] {x^2};
        
        % Adiciona uma entrada na legenda
        \addlegendentry{$L = \text{erro}^2$}
    \end{axis}
\end{tikzpicture}

\begin{equacaodestaque}{Derivada do MSE}
    \frac{\partial L_{\text{MSE}}}{\partial \hat{y}_i} = \frac{2}{N}(\hat{y}_i - y_i)
    \label{eq:mse-derivada}
\end{equacaodestaque}


\subsection{Erro Absoluto Médio (Mean Absolute Error - MAE)}


\begin{equacaodestaque}{Erro Absoluto Médio (MAE)}
    L_{\text{MAE}} = \frac{1}{N} \sum_{i=1}^{N} |y_i - \hat{y}_i|
    \label{eq:mae}
\end{equacaodestaque}

\begin{tikzpicture}
    \begin{axis}[
        title={Função de Perda: Erro Absoluto Médio (MAE)},
        xlabel={Erro ($y - \hat{y}$)},
        ylabel={Perda Calculada},
        axis lines=middle,          % Eixos centrados em (0,0)
        grid=major,                 % Adiciona uma grade principal
        grid style={dashed, gray!40}, % Estilo da grade
        xmin=-4.5, xmax=4.5,        % Limites do eixo x
        ymin=-0.5, ymax=4.5,         % Limites do eixo y
        legend pos=north west,      % Posição da legenda
        width=12cm,                 % Largura do gráfico
        height=9cm,                 % Altura do gráfico
        title style={font=\bfseries},
        label style={font=\small},
        tick label style={font=\scriptsize}
    ]
        % Adiciona o gráfico da função abs(x)
        \addplot[
            domain=-4:4, 
            samples=100, 
            color=red, 
            very thick
        ] {abs(x)};
        
        % Adiciona uma entrada na legenda
        \addlegendentry{$L = |\text{erro}|$}
    \end{axis}
\end{tikzpicture}

\begin{equacaodestaque}{Derivada do MAE}
    \frac{\partial L_{\text{MAE}}}{\partial \hat{y}_i} = \frac{1}{N} \cdot \text{sgn}(\hat{y}_i - y_i) = 
    \begin{cases} 
      +\frac{1}{N} & \text{se } \hat{y}_i > y_i \\
      -\frac{1}{N} & \text{se } \hat{y}_i < y_i \\
      0 & \text{se } \hat{y}_i = y_i
    \end{cases}
    \label{eq:mae-derivada}
\end{equacaodestaque}


\subsection{Huber Loss: O Melhor de Dois Mundos}


\begin{equacaodestaque}{Huber Loss}
    L_{\delta}(y, \hat{y}) = 
    \begin{cases} 
      \frac{1}{2}(y - \hat{y})^2 & \text{para } |y - \hat{y}| \le \delta \\
      \delta (|y - \hat{y}| - \frac{1}{2}\delta) & \text{caso contrário}
    \end{cases}
    \label{eq:huber-loss}
\end{equacaodestaque}

\begin{tikzpicture}
    \begin{axis}[
        title={Função de Perda: Huber Loss ($\delta=1$)},
        xlabel={Erro ($y - \hat{y}$)},
        ylabel={Perda Calculada},
        axis lines=middle,          % Eixos centrados em (0,0)
        grid=major,                 % Adiciona uma grade principal
        grid style={dashed, gray!40}, % Estilo da grade
        xmin=-4.5, xmax=4.5,        % Limites do eixo x
        ymin=-0.5, ymax=4.5,         % Limites do eixo y
        legend pos=north west,      % Posição da legenda
        width=12cm,                 % Largura do gráfico
        height=9cm,                 % Altura do gráfico
        title style={font=\bfseries},
        label style={font=\small},
        tick label style={font=\scriptsize}
    ]
        % Define o valor de delta
        \def\delta{1.0}

        % Adiciona o gráfico da função Huber usando uma expressão condicional
        % Se |x| <= delta, usa 0.5*x^2. Senão, usa delta*(|x| - 0.5*delta).
        \addplot[
            domain=-4:4, 
            samples=201, % Samples ímpares para incluir o ponto x=0
            color=orange, 
            very thick
        ] { abs(x) <= \delta ? 0.5*x^2 : \delta*(abs(x) - 0.5*\delta) };
        
        % Adiciona uma entrada na legenda
        \addlegendentry{$L_{\delta=1}(\text{erro})$}

        % Opcional: Adiciona linhas para mostrar a transição em delta
        \draw[dashed, gray] (axis cs:-\delta, 0) -- (axis cs:-\delta, {\delta*(\delta-0.5*\delta)});
        \draw[dashed, gray] (axis cs:\delta, 0) -- (axis cs:\delta, {\delta*(\delta-0.5*\delta)});

    \end{axis}
\end{tikzpicture}

\begin{equacaodestaque}{Derivada da Huber Loss}
    \frac{\partial L_{\delta}}{\partial \hat{y}} = 
    \begin{cases} 
      \hat{y} - y & \text{para } |\hat{y} - y| \le \delta \\
      \delta \cdot \text{sgn}(\hat{y} - y) & \text{caso contrário}
    \end{cases}
    \label{eq:huber-loss-derivada}
\end{equacaodestaque}

\subsection{Perda Log-Cosh}

\section{Lidando com a Escala: Foco no Erro Relativo}

\subsection{Erro Quadrático Médio Logarítmico (MSLE)}

\section{Mudando o Objetivo da Previsão: Além da Média}

\subsection{Perda Quantílica}

\subsection{Perda Epsilon-Insensível}

\section{Perdas Baseadas em Distribuições de Dados}

\subsection{Perda de Poisson}

\subsection{Perda de Tweedie}

\subsection{Divergência Kullback-Leibler}

\section{Comparativo: Funções de Perda para Regressão}

\section{Fluxograma: Escolhendo a Função de Perda Ideal}
% ===================================================================
% Arquivo: capitulos/parte-III-pilares/cap-10-perda-binaria.tex
% ===================================================================

\chapter{Funções de Perda para Classificação}
\label{cap:perda-classificacao}

\section{Exemplo Ilustrativo:}

\section{Funções de Perda para Classificação Binária}

\subsection{Entropia Cruzada Binária (Binary Cross-Entropy - BCE): A função de perda padrão}
\label{sec:binary-cross-entropy}

Para entender a função de perda \textit{Binary Cross Entropy} (\textit{BCE}) é importante antes conhecer o conceito de Entropia, o qual é fundamental para o cálculo dessa função. Em \textit{A Matematical Theory of Comunication}, \textcite{EntropyShannon} estava estudando sobre formas eficientes de comunicação, para isso, em um dos momentos do artigo ele define o conceito de Entropia, sendo uma medida de incerteza, ou da "escolha", associada a um conjunto de eventos com determinada probabilidades. Essa fórmula pode ser vista na Equação \ref{eq:entropia-de-shannon}.

\begin{equacaodestaque}{Entropia de Shannon}
    H(p) = - k \sum_{i = 1}^{n} p_i \log pi
    \label{eq:entropia-de-shannon}
\end{equacaodestaque}

Passado alguns anos, outros autores já estavam trabalhando com esse conceito introduzido por Shannon. Um desses casos é o de \textcite{KullbackLeiblerDivergence}, que em \textit{On Information and Sufficiency} expandem o conceito de Entropia para lidar também com casos contínuos, mas, mais que isso, introduzem uma media para comparar duas distribuições de probabilidades $p$ e $q$, chamando-a de informação para discriminação. Essa medida futuramente passa a ser conhecida com divergência de Kullback-Leibler (\textit{KL Divergence}), ela está representada na Equação \ref{eq:kl-divergence}

\begin{equacaodestaque}{Divergência de Kullback-Leibler}
    I(1:2) = I_{1:2}(X) = \int f_1 (x) \log \frac{f_1(x)}{f_2 (x)} d \lambda (x)
    \label{eq:kl-divergence}
\end{equacaodestaque}

Essa função tem como principal objetivo medir a "perda" ou o "excesso" de informação quando é utilizada uma distribuição $q$ para aproximar a distribuição real $q$. Além disso, é a partir dessa fórmula que é possível chegar na definição de Entropia-Cruzada (\textit{Cross-Entropy}), para isso, o primeiro passo, é utilizar as propriedades do logarítmo para reescrever da divergência KL, assim, tem-se:

\[
    I(1:2) = I_{1:2}(X) = \int f_1 (x) (\log f_1 (x) - \log f_2 (x)) d\lambda (x)
\]

Em seguida, deve-se expandir a equação, utilizando a propriedade distributiva, encontrando então:

\[
    I(1:2) = I_{1:2}(X) = \int f_1 (x) \log f_1 (x) - f_1 (x) \log f_2 (x) d \lambda (x)
\]

A partir dessa nova equação, o próximo passo é separar a integral em duas diferentes, chegando na expressão:

\[
    I(1:2) = \int f_1 (x) \log f_1 (x) d\lambda (x) - \int f_1 (x) \log f_2 (x) d\lambda(x)
\]

Note que o primeiro termo é quase igual a fórmula proposta por Shannon para a Entropia, exceto por um sinal de menos, assim, o primeiro termo pode ser reescrito como $-H(f_1)$. Já o segundo termo é a própria definição de Entropia-Cruzada entre $f_1$ e $f_2$, portanto $H(f_1, f_2)$, o qual pode ser visto separadamente na Equação \ref{eq:cross-entropy}.

\begin{equacaodestaque}{Entropia-Cruzada (\textit{Cross-Entropy})}
    H(f_1, f_2) = \int f_1 (x) \log f_2 (x) d\lambda(x)
    \label{eq:cross-entropy}
\end{equacaodestaque}

Com base nesses dois termos, o da Entropia-Cruzada, e o da Entropia de Shannon, é possível mais uma vez reescrever a definição de xx agora utilizando os termos resumidos:

\[
    I(1:2) = H(f_1, f_2) - H(f_2)
\]

Ou também pode ser escrita como:

\[
    D_{KL} (f_1 || f_2) = H(f_1, f_2) - H(f_1)
\]

Portanto, é possível dizer que a Divergência de Kullback-Leibler é a diferença entre a Entropia-Cruzada e a Entropia de Shannon. Reescrevendo a equação mais uma vez é possível chegar em:

\[
    H(f_1, f_2) = H(f_1) + I(1:2)
\]

Essa equação nos diz que o custo real de codificar os dados usando um modelo imperfeito $H(f_1, f_2)$ é igual ao csuto de codificar usando um modelo perfeito $H(f_1)$ mais uma penalidade extra pela diferença entre o modelo perfeito e imperfeito. Assim, é possível concluir que ao minimizar a Divergência KL, é o mesmo que minimizar a Entropia-Cruzada em cenários de aprendizado de máquina. Isso se dá pois como a Entropia dos dados reais é uma constante, ao reduzir a Entropia-Cruzada, é ao mesmo tempo forçar a redução da Divergência KL, isso aproxima mais o modelo dos dados da realidade.

A partir desses dois conceitos é possível conhecer melhor a função de perda \textit{Binary-Cross-Entropy}. Ela é dada pela Equação \ref{eq:binary-cross-entropy}. Note, que a definição da \textit{BCE} apresente grandes similaridades com Entropia-Cruzada, principalmente pelos primeiros termos $y \log (\hat{y})$, que é justamente a definição de Entropia-Cruzada.

\begin{equacaodestaque}{Entropia Cruzada Binária (\textit{BCE}) para um par de Amostras}
    \Loss(y_j, \hat{y}_j) = -[y_j \log(\hat{y}_j) + (1 - y_j) \log(1 - \hat{y_j})]
    \label{eq:binary-cross-entropy}
\end{equacaodestaque}

Em que:

\begin{itemize}
    \item $y_j$ representa o valor real para a saída;
    \item $\hat{y}_j$ representa o valor predito pelo modelo;
\end{itemize}

Perceba que a fórmula da entropia cruzada binária faz um uso muito inteligente para calcular a perda, ela aplica o cálculo de duas entropias cruzadas. Para isso, a primeira entropia cruzada $y_j \log(\hat{y}_j)$ calcula a distância de uma classe 0 com o retorno do modelo $\hat{y}_j$. A segunda entropia cruzada $(1 - y_j) \log(1 - \hat{y_j})$ faz o cálculo da distância para os casos em que está sendo analisada a classe 1. Dessa forma, em um cenário em que o resultado é a classe 0, a primeira entropia cruzada é toda multiplicado por zero, sendo eliminado da fórmula, o mesmo vale para quando o resultado é a classe 1, em que a segunda entropia cruzada passa a ser zero.

\begin{equacaodestaque}{Entropia Cruzada Binária (\textit{BCE}) para $N$ Amostras}
    \Loss_{BCE} = \frac{1}{N} \sum_{j=1}^{N} \Loss_{BCE} (y_j, \hat{y}_j)
    \label{eq:binary-cross-entropy-para-n-amostras}
\end{equacaodestaque}

Além disso, a \textit{BCE} já vem sendo utilizada no contexto de aprendizado de máquina a um bom tempo. Um dos trabalhos que cita o uso dessa função para ser utilizada em cenários de classificação binária é o \textit{Connectionist Learning Procedures} de \textcite{HintonConnectionist}, em que o autor cita que ao minimizar a Entropia-Cruzada-Binária para as distribuições do resultado desejado e o atual resultado era semelhante a maximivizar a a versossimilhança do modelo gerar as saídas corretas.

Dito isso, o próximo passo para entender a \textit{BCE} é conhecer a sua representação gráfica, a qual está presente na Figura \ref{fig:binary-cross-entropy}. Note que o gráfico apresenta duas curvas, uma para a distribuição para a classe real, e outra para a distribuição para a classe de saídas do modelo. Perceba que o ponto de mínimo do gráfico é aquele em que as duas curvas se encontram, e com isso a distância entre as duas é mínima, e consequentemente a perda também será.

\begin{figure}
    \begin{tikzpicture}
        \begin{axis}[
            xlabel={Probabilidade Prevista ($\hat{y}$)},
            ylabel={Perda Calculada},
            axis lines=left,              % Eixos no canto inferior esquerdo
            grid=major,                   % Adiciona uma grade principal
            grid style={dashed, gray!40},   % Estilo da grade
            xmin=0, xmax=1,               % Limites do eixo x
            ymin=0, ymax=5,               % Limites do eixo y
            legend pos=north west,      % Posição da legenda
            width=12cm,                   % Largura do gráfico
            height=9cm,                   % Altura do gráfico
            title style={font=\bfseries},
            label style={font=\small},
            tick label style={font=\scriptsize}
        ]
            % Curva para a classe real y=1
            \addplot[
                domain=0.01:0.999, % Domínio para evitar log(0)
                samples=100,
                color=blue,
                very thick
            ] {-ln(x)};
            \addlegendentry{Classe Real = 1 ($L = -\log(\hat{y})$)}

            % Curva para a classe real y=0
            \addplot[
                domain=0.001:0.99, % Domínio para evitar log(0)
                samples=100,
                color=red,
                very thick
            ] {-ln(1-x)};
            \addlegendentry{Classe Real = 0 ($L = -\log(1-\hat{y})$)}
            
        \end{axis}
    \end{tikzpicture}
    \caption{Representação gráfica da função de perda Entropia-Cruzada-Binária (\textit{Binary-Cross-Entropy}).}
    \label{fig:binary-cross-entropy}
    \fonte{O autor (2025).}
\end{figure}

Conhecendo como funcionam os princípios por trás da entropia-cruzada binária, além das suas fórmulas e gráficos, o próximo passo é discutir propriedades interessantes dessa função de perda. 

\begin{itemize}
    \item \textbf{Origem na teoria da informação:} Como foi visto no início da seção a \textit{BCE} está estritamente ligada com o conceito de Entropia de Shannon. Além disso, como explicam \textcite{LossesArticle}, essa função é responsável por medir a distância entre duas distribuições de Bernoulli, a distribuição real $P(y)$ e a distribuição de predições $Q(y)$.
    \item \textbf{Diferenciabilidade:} A \textit{BCE} é diferenciável em relação à $\hat{y}_j \in (0,1)$ \parencite{LossesArticle}. Assim, faz-se necessário utilizar na saída do modelo uma função que retorne valores apenas nesse intervalo. Aí entra a sigmoide logística como uma função ideal para resolver esse tipo de problema, pois sua saída está justamente nesse intervalo de diferenciabilidade da \textit{BCE}. 
    \item \textbf{Pune erros confiantes:} \textcite{LossesArticle} explicam que quando $\hat{y}_j$ é perto de 1, mas o valor real é o oposto, neste caso, $y_j = 0$, o termo logarítimico $\log(\hat{y}j)$ ou $\log(1 - \hat{y}_j)$ fica grande em magnitudade, penalizando muito os erros confiantes. Essa relação também vale para o contrário, em que $\hat{y}_j$ é perto de 0, mas $y_j = 1$.
    \item \textbf{Desbalanceamento de classes:} Um último ponto que vale ser destacado sobre a função de perda \textit{binary cross-entropy} é que em cenários em que uma classe é significativamente mais presente que outra, a \textit{BCE} pode gerar modelos enviesados \footnote{Como uma possível solução para esse problema, é possível utilizar a função de perda \textit{binary weighted cross-entropy} (a qual está explica na Seção \ref{sec:binary-weighted-cross-entropy}), essa função busca resolver o problema do desbalanceamento de classes aplicando pesos para as diferentes classes do problema estudado.} \parencite{LossesArticle}.
\end{itemize}

Sabendo dessas propriedades e caraterísticas da \textit{BCE}, é possível agora discutir a sua derivada. As derivadas das funções de perda são muito úteis para aqueles modelos que aprendem por meio da descida do gradiente. Pois, o primeiro gradiente a ser calculado, é o gradiente da perda para a camada de saída do modelo. Esse gradiente é então propagado para trás, no chamado \textit{backward-pass}, atualizando os pesos e vieses do modelo.

A derivada da entropia cruzada binária pode ser vista na Equação \ref{eq:binary-cross-entropy-derivada}. Note que nessa expressão está sendo calculada a derivada com relação os valores preditos pelo modelo $\hat{y}_j$, mas também é calculada a derivada para os valores reais $y_j$, de forma que juntas, essas duas derivadas compõem o vetor gradiente.

\begin{equacaodestaque}{Entropia Cruzada Binária (\textit{BCE}) Derivada}
    \frac{\partial \Loss}{\partial \hat{y}_j} = \frac{\hat{y}_j - y_j}{\hat{y}_j(1 - \hat{y})_j}
    \label{eq:binary-cross-entropy-derivada}
\end{equacaodestaque}

Também vale a pena disctuir o gráfico da derivada da entropia cruzada binária, para isso, ele está representado na Figura \ref{fig:binary-cross-entropy-derivada}.

\begin{figure}[h!]
    \centering
    \begin{tikzpicture}
        \begin{axis}[
            xlabel={Probabilidade Prevista ($\hat{y}$)},
            ylabel={Gradiente da Perda ($\frac{\partial L}{\partial \hat{y}}$)},
            axis lines=middle,
            grid=major,
            grid style={dashed, gray!40},
            xmin=-0.1, xmax=1.1,
            ymin=-15, ymax=15, % Aumentar o range do y para ver a assíntota
            legend pos=outer north east,
            width=12cm,
            height=9cm,
            title style={font=\bfseries},
            label style={font=\small},
            tick label style={font=\scriptsize}
        ]
            % Derivada para a classe real y=1
            % Fórmula: (y_hat - 1) / (y_hat * (1 - y_hat)) = -1 / y_hat
            \addplot[
                domain=0.05:1, % Domínio para evitar divisão por zero
                samples=100,
                color=blue,
                very thick
            ] {-1/x};
            \addlegendentry{Classe Real = 1 ($\frac{\hat{y}-1}{\hat{y}(1-\hat{y})}$)}

            % Derivada para a classe real y=0
            % Fórmula: (y_hat - 0) / (y_hat * (1 - y_hat)) = 1 / (1 - y_hat)
            \addplot[
                domain=0:0.95, % Domínio para evitar divisão por zero
                samples=100,
                color=red,
                very thick
            ] {1/(1-x)};
            \addlegendentry{Classe Real = 0 ($\frac{\hat{y}}{\hat{y}(1-\hat{y})}$)}
            
        \end{axis}
    \end{tikzpicture}
    \caption{Representação gráfica da derivada da função de perda Entropia-Cruzada-Binária.}
    \label{fig:binary-cross-entropy-derivada}
    \fonte{O autor (2025).}
\end{figure}

\medskip
\begin{center}
 * * *
\end{center}
\medskip

\textbf{Algumas Aplicações da Entropia-Cruzada Binária em Problemas de Classificação Binária}

\begin{itemize}
    \item \textbf{Aplicação 1 (Área):}
    \item \textbf{Aplicação 2 (Área):}
    \item \textbf{Aplicação 3 (Área):}
    \item \textbf{Aplicação 4 (Área):}
\end{itemize}

Visto a entropia cruzada binária, agora pode ser discutido uma variante dessa função, a \textit{binary weighted cross-entropy} (\textit(WCE)), que busca corrigir um dos pontos fracos da \textit{BCE} original: o desbalanceamento de classes.

\subsection{Entropia Cruzada Ponderada Binária (Binary Weighted Cross-Entropy -WCE)}
\label{sec:binary-weighted-cross-entropy}

Uma variante da entropia-cruzada binária é a entropia cruzada ponderada binária (\textit{WCE}), ela tem como principal objetivo ser utilizada em casos em que uma classe é mais presente que outra \parencite{LossesArticle}. Para isso, essa função atribui pesos para as diferentes classes, assim, a classe que é menos presente é possui um peso maior, dessa forma, o modelo que está sendo treinado consegue "prestar mais atenção" nas classes menos frequêntes. \textcite{LossesArticle} explicam que essa é uma função utilizada em cenários em que os erros são caros e críticos.

Dessa forma, é possível escrever a \textit{binary weighted cross-entropy} como a Equação \ref{eq:binary-weighted-cross-entropy}.

\begin{equacaodestaque}{Entropia Cruzada Ponderada Binária (\textit{WCE})}
    \Loss_{WCE} (y_j, \hat{y}_j) = - [\alpha_1 y_j \log (\hat{y}_j) + \alpha_0 (1 - y_j) \log (1 - \hat{y}_j)]
    \label{eq:binary-weighted-cross-entropy}
\end{equacaodestaque}

Neste caso, os valores de $\alpha_1$ e $\alpha_0$ representam os pesos atribuídos para cada uma das classes, dessa forma, a perda geral é obtida ao calcular a média com base de $N$ amostras, assim, é possível escrever como na Equação \ref{eq:binary-weighted-cross-entropy-para-n-amostras}.

\begin{equacaodestaque}{Entropia Cruzada Ponderada Binária  (\textit{WCE}) para $N$ Amostras}
    \Loss_{WCE} = \frac{1}{N} \sum_{j = 1}^{N} \Loss_{WCE} (y_j, \hat{y}_j)
    \label{eq:binary-weighted-cross-entropy-para-n-amostras}
\end{equacaodestaque}

É possível ver o gráfico da \textit{binary weighted cross-entropy} na Figura \ref{fig:comparativo-entropia-cruzada-ponderada-binaria}. Na Figura \ref{fig:comparativo-entropia-cruzada-ponderada-binaria-com-alto-peso-para-classe-1} é mostrada uma sitação em que o valor de $\alpha_0$ é consideravelmente maior que o de $\alpha_1$, resultando em um gráfico parecido com o da \textit{BCE} original, mas neste caso com o ponto de mínimo deslocado para a esquerda. Já na Figura \ref{fig:comparativo-entropia-cruzada-ponderada-binaria-com-alto-peso-para-classe-0}, é possível ver a sitação inversa $\alpha_0 \gg \alpha_1$, nesse caso, o ponto de mínimo está deslocado para a direta do gráfico, indicando que provavelmente está sendo trabalhado com mais elementos da classe 0.

\begin{figure}[h!]
    \centering
    % Figura da Esquerda (Peso maior para a Classe 0)
    \begin{subfigure}[b]{0.48\textwidth}
        \centering
        \begin{tikzpicture}
            \def\alphaZero{5.0} % Peso alto para a classe 0
            \def\alphaUm{1.0}   % Peso normal para a classe 1
            \begin{axis}[
                title={WCE com $\alpha_0=5.0, \alpha_1=1.0$},
                xlabel={Probabilidade Prevista ($\hat{y}$)},
                ylabel={Perda Calculada},
                axis lines=left,
                grid=major,
                grid style={dashed, gray!40},
                xmin=0, xmax=1,
                ymin=0, ymax=25, % Ajustar ymax para a curva mais íngreme
                legend pos=north east,
                width=\textwidth,
                label style={font=\small},
                tick label style={font=\scriptsize},
                title style={font=\bfseries, yshift=-5pt},
            ]
                % Curva para y=1
                \addplot[
                    domain=0.01:0.999, samples=100, color=blue, thick
                ] {-\alphaUm*ln(x)};
                \addlegendentry{$y=1$ ($L = -1.0 \cdot \log(\hat{y})$)}

                % Curva para y=0
                \addplot[
                    domain=0.001:0.99, samples=100, color=red, very thick
                ] {-\alphaZero*ln(1-x)};
                \addlegendentry{$y=0$ ($L = -5.0 \cdot \log(1-\hat{y})$)}
            \end{axis}
        \end{tikzpicture}
        \caption{Alto peso para a classe 0.}
        \label{fig:comparativo-entropia-cruzada-ponderada-binaria-com-alto-peso-para-classe-0}
    \end{subfigure}
    \hfill % Espaço entre as figuras
    % Figura da Direita (Peso maior para a Classe 1)
    \begin{subfigure}[b]{0.48\textwidth}
        \centering
        \begin{tikzpicture}
            \def\alphaZero{1.0} % Peso normal para a classe 0
            \def\alphaUm{5.0}   % Peso alto para a classe 1
            \begin{axis}[
                title={WCE com $\alpha_0=1.0, \alpha_1=5.0$},
                xlabel={Probabilidade Prevista ($\hat{y}$)},
                ylabel={Perda Calculada},
                axis lines=left,
                grid=major,
                grid style={dashed, gray!40},
                xmin=0, xmax=1,
                ymin=0, ymax=25, % Manter a mesma escala de y
                legend pos=north west,
                width=\textwidth,
                label style={font=\small},
                tick label style={font=\scriptsize},
                title style={font=\bfseries, yshift=-5pt},
            ]
                % Curva para y=1
                \addplot[
                    domain=0.01:0.999, samples=100, color=blue, very thick
                ] {-\alphaUm*ln(x)};
                \addlegendentry{$y=1$ ($L = -5.0 \cdot \log(\hat{y})$)}

                % Curva para y=0
                \addplot[
                    domain=0.001:0.99, samples=100, color=red, thick
                ] {-\alphaZero*ln(1-x)};
                \addlegendentry{$y=0$ ($L = -1.0 \cdot \log(1-\hat{y})$)}
            \end{axis}
        \end{tikzpicture}
        \caption{Alto peso para a classe 1.}
        \label{fig:comparativo-entropia-cruzada-ponderada-binaria-com-alto-peso-para-classe-1}
    \end{subfigure}
    
    \caption{Comparação da Entropia Cruzada Ponderada (\textit{WCE}) alterando os pesos $\alpha_0$ e $\alpha_1$.}
    \label{fig:comparativo-entropia-cruzada-ponderada-binaria}
\end{figure}

Também cabe analisar a derivada da entropia cruzada ponderada binária, a qual é dada pela Equação \ref{eq:binary-cross-entropy-derivada}. Perceba que o resultado é parecido com o da entropia cruzada binária, mas neste caso, adicionando os pesos $\alpha_0$ e $\alpha_1$.

\begin{equacaodestaque}{Entropia Cruzada Ponderada Binária (\textit{WCE}) Derivada}
    \frac{\partial \Loss_{WCE}}{\partial \hat{y}_j} = \frac{\alpha_0(1-y_j)\hat{y}_j - \alpha_1 y_j(1-\hat{y}_j)}{\hat{y}_j(1-\hat{y}_j)}
    \label{eq:binary-weighted-cross-entropy-derivada}
\end{equacaodestaque}

Tendo a derivada da (\textit{WCE}) o próximo passo é conhecer o seu gráfico, dado pela Figura \ref{fig:binary-weighted-cross-entropy-derivada-comparacao}, neste caso, é a partir do gráfico da derivada que é possível ter uma ideia de qual é o tamanho da correção que deve ser feita para que o modelo atinja métricas melhores. Na Figura \ref{fig:binary-weighted-cross-entropy-derivada-comparacao}, é possível ver dois gráficos diferentes, começando pelo da esquerda, a Figura \ref{fig:wce-derivada-alpha0}, é mostrado o gráfico de uma função \textit{WCE} em que há uma maior presença de itens da classe 1, e como consequência $\alpha_0 > \alpha_1$, isso gera um deslocamento da curva de erros da classe 1 mais próxima do eixo das abssissas. Já na Figura \ref{fig:wce-derivada-alpha1}, ocorre o contrário, há mais itens da classe 0, e consequentemente $\alpha_1 > \alpha_0$, assim, a curva de erros da classe 0 fica mais próxima do eixo $x$.

\begin{figure}[h!]
    \centering
    % Figura da Esquerda (Peso maior para a Classe 0)
    \begin{subfigure}[b]{0.48\textwidth}
        \centering
        \begin{tikzpicture}
            \def\alphaZero{5.0} % Peso alto para a classe 0
            \def\alphaUm{1.0}   % Peso normal para a classe 1
            \begin{axis}[
                title={Derivada da WCE com $\alpha_0=5.0, \alpha_1=1.0$},
                xlabel={Probabilidade Prevista ($\hat{y}$)},
                ylabel={Gradiente da Perda},
                axis lines=middle,
                grid=major,
                grid style={dashed, gray!40},
                xmin=-0.1, xmax=1.1,
                ymin=-55, ymax=55, % Aumentar range para ver o efeito
                legend pos=outer north east,
                width=\textwidth,
                label style={font=\small},
                tick label style={font=\scriptsize},
                title style={font=\bfseries, yshift=-5pt},
            ]
                % Derivada para y=1
                \addplot[
                    domain=0.02:1, samples=100, color=blue, thick
                ] {-\alphaUm/x};
                \addlegendentry{$y=1$ ($-\frac{1.0}{\hat{y}}$)}

                % Derivada para y=0
                \addplot[
                    domain=0:0.98, samples=100, color=red, very thick
                ] {\alphaZero/(1-x)};
                \addlegendentry{$y=0$ ($\frac{5.0}{1-\hat{y}}$)}
            \end{axis}
        \end{tikzpicture}
        \caption{Gradiente amplificado para erros na classe 0.}
        \label{fig:wce-derivada-alpha0}
    \end{subfigure}
    \hfill % Espaço entre as figuras
    % Figura da Direita (Peso maior para a Classe 1)
    \begin{subfigure}[b]{0.48\textwidth}
        \centering
        \begin{tikzpicture}
            \def\alphaZero{1.0} % Peso normal para a classe 0
            \def\alphaUm{5.0}   % Peso alto para a classe 1
            \begin{axis}[
                title={Derivada da WCE com $\alpha_0=1.0, \alpha_1=5.0$},
                xlabel={Probabilidade Prevista ($\hat{y}$)},
                ylabel={Gradiente da Perda},
                axis lines=middle,
                grid=major,
                grid style={dashed, gray!40},
                xmin=-0.1, xmax=1.1,
                ymin=-55, ymax=55, % Manter a mesma escala de y
                legend pos=outer north east,
                width=\textwidth,
                label style={font=\small},
                tick label style={font=\scriptsize},
                title style={font=\bfseries, yshift=-5pt},
            ]
                % Derivada para y=1
                \addplot[
                    domain=0.02:1, samples=100, color=blue, very thick
                ] {-\alphaUm/x};
                \addlegendentry{$y=1$ ($-\frac{5.0}{\hat{y}}$)}

                % Derivada para y=0
                \addplot[
                    domain=0:0.98, samples=100, color=red, thick
                ] {\alphaZero/(1-x)};
                \addlegendentry{$y=0$ ($\frac{1.0}{1-\hat{y}}$)}
            \end{axis}
        \end{tikzpicture}
        \caption{Gradiente amplificado para erros na classe 1.}
        \label{fig:wce-derivada-alpha1}
    \end{subfigure}
    
    \caption{Comparação da derivada da Entropia Cruzada Ponderada (\textit{WCE}).}
    \label{fig:binary-weighted-cross-entropy-derivada-comparacao}
\end{figure}

\medskip
\begin{center}
 * * *
\end{center}
\medskip

\textbf{Algumas Aplicações da Entropia-Cruzada Ponderada Binária em Problemas de Classificação Binária}

\begin{itemize}
    \item \textbf{Aplicação 1 (Área):}
    \item \textbf{Aplicação 2 (Área):}
    \item \textbf{Aplicação 3 (Área):}
    \item \textbf{Aplicação 4 (Área):}
\end{itemize}

\subsection{Perda Hinge (Hinge Loss)}

A Hinge Loss é uma função de perda que está relacionada com o uso de máquinas de vetores de suporte (\textit{SVMs}), ela aparece pela primeira vez no artigo \textit{Support-Vector Networks} dos autores \textcite{HingeLoss}. É possível fazer a sua dedução para entender melhor o problema que os pesquisadores estavam analisando ao desenvolver o trabalho. No artigo \textcite{HingeLoss} começam introduzindo um problema em que analisam dados que não podem ser separados sem erros, de forma que eles querem então separar esses dados gerando a menor quantidade possível de erros. Para isso, os autores introduzem um conjunto de variáveis não-negativas $\xi_i \ge 0, i = 1, 2, \cdots, l$. 

Assim, \textcite{HingeLoss} querem minimizar uma função da forma:

\[
    `\Phi ' = \sum_{i = 1}^l \xi_i^{\sigma}
\]

Para $\sigma > 0$, são sujeitas as restrições:

\[
    y_i (\textbf{w} \cdot \textbf{x}_i + b) \ge 1 - \xi_i, i = 1, 2, \cdots, l
\]

\[
    \xi \ge 0, i = 1, 2, \cdots, l
\]

Perceba então que para que $\xi$ siga as restrições propostas, ele deve ser maior ou igual as restrições, e como ele deve ser o menor possível para minimizar o erro, o menor que ele pode ser é o máximo entre 0 e $1 - y(\textbf{w} \cdot \textbf{x}_i + b)$. É possível então escrever algo da forma:

\[
    \xi = \max (0, 1 - y(\textbf{w} \cdot \textbf{x}_i + b))
\]

Essa função encontrada ao buscar o mínimo entre esses dois termos, é justamente a Hinge Loss, que agora pode ser expressa com as notações do livro na Equação \ref{eq:hinge-loss}.

\begin{equacaodestaque}{Hinge Loss}
    \Loss_{\text{Hinge}}(y, f(x)) = \max(0, 1 - y \cdot f(x))
    \label{eq:hinge-loss}
\end{equacaodestaque}

Em que:

\begin{itemize}
    \item $y$ representa os rótulos dos dados, e também segue o formato $y \in {-1, +1}$
    \item $f(x)$ representa a saída bruta do modelo ou a função de decisão (geralmente the signed distance da barreira de decisão)
\end{itemize}

\textcite{LossesArticle} explicam que para calcular a margem para uma amostra $x$, deve ser calculado o produto $yf(x)$, uma classificação correta com uma margem suficiente gera um resultado positivo para esse produto, reduzindo a perda para zero.

Cabe também disctutir o gráfico dessa função de perda, o qual é dado pela Figura \ref{fig:hinge-loss}. Perceba que diferente do gráfico da \textit{binary cross-entropy} que fazia uso de logarítimos e por isso apresentava curvas suaves, a \textit{hinge loss} utilizada a função $\max$ e equação da reta, como consequência, o seu gráfico é composto por duas retas que se cruzam no ponto de mínimo.

\begin{figure}
    \begin{tikzpicture}
        \begin{axis}[
            title={Função de Perda: Hinge Loss},
            xlabel={Saída Bruta do Modelo ($f(x)$)},
            ylabel={Perda Calculada},
            axis lines=middle,          % Eixos centrados em (0,0)
            grid=major,                 % Adiciona uma grade principal
            grid style={dashed, gray!40}, % Estilo da grade
            xmin=-3.5, xmax=3.5,        % Limites do eixo x
            ymin=-0.5, ymax=4.5,         % Limites do eixo y
            legend pos=north west,      % Posição da legenda
            width=12cm,                 % Largura do gráfico
            height=9cm,                 % Altura do gráfico
            title style={font=\bfseries},
            label style={font=\small},
            tick label style={font=\scriptsize}
        ]
            % Curva para a classe real y=+1
            \addplot[
                domain=-3:3, 
                samples=100, 
                color=blue, 
                very thick
            ] {max(0, 1-x)};
            \addlegendentry{Classe Real = 1 ($L=\max(0, 1-f(x))$)}

            % Curva para a classe real y=-1
            \addplot[
                domain=-3:3, 
                samples=100, 
                color=red, 
                very thick
            ] {max(0, 1+x)};
            \addlegendentry{Classe Real = -1 ($L=\max(0, 1+f(x))$)}
            
            % Opcional: Linhas tracejadas para marcar as margens
            \draw[dashed, gray!70] (axis cs:1, 0) -- (axis cs:1, 4.5);
            \draw[dashed, gray!70] (axis cs:-1, 0) -- (axis cs:-1, 4.5);
            \node[above, gray!80, font=\tiny, rotate=90] at (axis cs:1.1, 2) {Margem};
            \node[above, gray!80, font=\tiny, rotate=90] at (axis cs:-0.9, 2) {Margem};
            
        \end{axis}
    \end{tikzpicture}
    \caption{Representação gráfica da função de perda Hinge Loss.}
    \label{fig:hinge-loss}
    \fonte{O autor (2025).}
\end{figure}

Visto como surgiu a \textit{hinge loss}, sua equação e seu gráfico, é possível agora discutir algumas propriedades dessa função de perda:

\begin{itemize}
    \item \textbf{Maximização de margem:} Como explicam \textcite{LossesArticle}, a \textit{hinge loss} força que a predição esteja correta ($\text{sign} (f(x)) = y$) mas também que a margem $[f(x)]$ seja pelo menos 1, uma predição correta mas com uma margem menor que 1 continua gerando uma perda positiva.
    \item \textbf{Penalidade linear:} Perceba pelo gráfico da Figura \ref{fig:hinge-loss} que essa função é composta por duas retas. Quando o cálculo dos termos $yf(x)$ é menor que 1, a \textit{hinge loss} aumenta de forma linear com a distância $1-yf(x)$, essa penalidadde linear geralmente é responsável por gerar gradientes mais esparsos \parencite{LossesArticle}.
\end{itemize}

Também cabe destacar a derivada dessa função de perda, para calculá-la deve-se derivar as duas funções, as quais vão gerar uma função escrita através de chaves, como a mostrada na Equação \ref{eq:hinge-loss-derivada}.

\begin{equacaodestaque}{Derivada da Hinge Loss}
    \frac{\partial \Loss}{\partial f(x)} = 
    \begin{cases} 
      -y & \text{se } y \cdot f(x) < 1 \\
      0 & \text{se } y \cdot f(x) \ge 1
    \end{cases}
    \label{eq:hinge-loss-derivada}
\end{equacaodestaque}

Além da \textit{hinge loss} tradicional, existe uma variante que faz uso de um expoente ao quadrado para calcular a perda. Dessa forma, a perda conseque crescer de forma mais rápida e com isso aumentar a penalização dos erros. Essa é a \textit{squared hinge loss}, a qual está apresentada na seção seguinte.

\subsection{Squared Hinge Loss}

A fórmula da \textit{squared hinge loss} não difere muito da \textit{hinge loss} tradicional, neste caso, essa variante eleva ao quadrado todo o resultado da perda calculada pela \textit{hinge loss}. Ela pode ser vista na Equação \ref{eq:squared-hinge-loss}.

\begin{equacaodestaque}{Squared Hinge Loss}
    \Loss_{\text{Squared Hinge}}(y, f(x)) = (\max(0, 1 - y \cdot f(x)))^2
    \label{eq:squared-hinge-loss}
\end{equacaodestaque}

Por ultizar um termo ao quadrado para calcular a perda, o seu gráfico também muda. Como pode ser visto na Figura \ref{fig:squared-hinge-loss}, ele agora apresenta um comportamento mais suave, além disso, os seus valores agora crescem de forma mais rápida que a \textit{hinge loss} tradicional, fazendo com que ela lide com os erros de forma mais severa que sua versão tradicional.

\begin{figure}[h!]
    \centering
    \begin{tikzpicture}
        \begin{axis}[
            title={Função de Perda: Squared Hinge Loss},
            xlabel={Saída Bruta do Modelo ($f(x)$)},
            ylabel={Perda Calculada},
            axis lines=middle,
            grid=major,
            grid style={dashed, gray!40},
            xmin=-3.5, xmax=3.5,
            ymin=-0.5, ymax=9.5,         % Aumentei o ymax para acomodar as parábolas
            legend pos=north west,
            width=12cm,
            height=9cm,
            title style={font=\bfseries},
            label style={font=\small},
            tick label style={font=\scriptsize}
        ]
            % Curva para a classe real y=+1
            \addplot[
                domain=-3:3, 
                samples=201, % Aumentei os samples para uma curva mais suave
                color=blue, 
                very thick
            ] {(max(0, 1-x))^2}; % Adicionado o ^2
            \addlegendentry{Classe Real = 1 ($L=(\max(0, 1-f(x)))^2$)}

            % Curva para a classe real y=-1
            \addplot[
                domain=-3:3, 
                samples=201, 
                color=red, 
                very thick
            ] {(max(0, 1+x))^2}; % Adicionado o ^2
            \addlegendentry{Classe Real = -1 ($L=(\max(0, 1+f(x)))^2$)}
            
            % Linhas tracejadas para marcar as margens
            \draw[dashed, gray!70] (axis cs:1, 0) -- (axis cs:1, 9.5);
            \draw[dashed, gray!70] (axis cs:-1, 0) -- (axis cs:-1, 9.5);
            \node[above, gray!80, font=\tiny, rotate=90] at (axis cs:1.1, 4) {Margem};
            \node[above, gray!80, font=\tiny, rotate=90] at (axis cs:-0.9, 4) {Margem};
            
        \end{axis}
    \end{tikzpicture}
    \caption{Representação gráfica da função de perda Squared Hinge Loss.}
    \label{fig:squared-hinge-loss}
    \fonte{O autor (2025).}
\end{figure}

Para calcular a sua derivada deve-se aplicar a regra da cadeia, separando então termo ao quadrado da função perda hinge tradicional. De forma que ao fazer isso é possível encontrar a Equação \ref{eq:squared-hinge-loss-derivada}.

\begin{equacaodestaque}{Derivada da Squared Hinge Loss}
    \frac{\partial \Loss{\text{Squared Hinge}}}{\partial f(x)} = 
    \begin{cases} 
        -2y(1 - y \cdot f(x)) & \text{se } y \cdot f(x) < 1 \\
        0 & \text{se } y \cdot f(x) \ge 1
    \end{cases}
    \label{eq:squared-hinge-loss-derivada}
\end{equacaodestaque}

De forma semelhante ao que foi feito com as outras funções até agora, é possível plotar o gráfico da derivada da \textit{hinge loss}. Ele está representado na Figura \ref{fig:squared-hinge-loss-derivada}. Note, que diferente do gráfico da \textit{hinge loss} em que a correção a ser feita é dada de forma constante, a derivada da \textit{squared hinge loss} é calculada de forma linear.

\begin{figure}[h!]
    \centering
    \begin{tikzpicture}
        \begin{axis}[
            title={Derivada da Squared Hinge Loss},
            xlabel={Saída Bruta do Modelo ($f(x)$)},
            ylabel={Gradiente da Perda ($\frac{\partial L}{\partial f(x)}$)},
            axis lines=middle,
            grid=major,
            grid style={dashed, gray!40},
            xmin=-3.5, xmax=3.5,
            ymin=-8.5, ymax=8.5,         % Ajustar o y para a escala linear da derivada
            legend pos=north west,
            width=12cm,
            height=9cm,
            title style={font=\bfseries},
            label style={font=\small},
            tick label style={font=\scriptsize}
        ]
            % Derivada para a classe real y=+1
            % Fórmula: (x < 1) ? (2*(x-1)) : 0
            \addplot[
                domain=-3:3, 
                samples=10, 
                color=blue, 
                very thick
            ] {(x < 1) ? (2*(x-1)) : 0};
            \addlegendentry{Classe Real = 1}

            % Derivada para a classe real y=-1
            % Fórmula: (x > -1) ? (2*(x+1)) : 0
            \addplot[
                domain=-3:3, 
                samples=10, 
                color=red, 
                very thick
            ] {(x > -1) ? (2*(x+1)) : 0};
            \addlegendentry{Classe Real = -1}
            
            % Linhas tracejadas para marcar as margens
            \draw[dashed, gray!70] (axis cs:1, -8.5) -- (axis cs:1, 8.5);
            \draw[dashed, gray!70] (axis cs:-1, -8.5) -- (axis cs:-1, 8.5);
            \node[above, gray!80, font=\tiny, rotate=90] at (axis cs:1.1, 2) {Margem};
            \node[above, gray!80, font=\tiny, rotate=90] at (axis cs:-0.9, 2) {Margem};
            
        \end{axis}
    \end{tikzpicture}
    \caption{Representação gráfica da derivada da função de perda Squared Hinge Loss. O gradiente é proporcional à distância da margem.}
    \label{fig:squared-hinge-loss-derivada}
    \fonte{O autor (2025).}
\end{figure}

\medskip
\begin{center}
 * * *
\end{center}
\medskip

\textbf{Algumas Aplicações da Perda Hinge Problemas de Classificação Binária}

\begin{itemize}
    \item \textbf{Aplicação 1 (Área):}
    \item \textbf{Aplicação 2 (Área):}
    \item \textbf{Aplicação 3 (Área):}
    \item \textbf{Aplicação 4 (Área):}
\end{itemize}

\section{Funções de Perda para Classificação Multi-Classe}

\subsection{Exemplo Ilustrativo:}

\subsection{Entropia Cruzada Categórica (Categorical Cross-Entropy - CCE)} 

Entendido os conceitos de entropia cruzada binária, o próximo passo é conhecer a entropia cruzada categórica, uma das funções de perda que se é utilizada para resolver problemas de classificação multi-classe. Para isso, a \textit{CCE} extende o conceito da \textit{BCE} para lidar com $y_i \in {0,1}^C$ em uma representação \textit{one-hot} entre $C$ diferentes classes \parencite{LossesArticle}. Dessa forma, diferente da entropia cruzada binária em que haviam dois cálculos de entropia-cruzada que eram então subtraídos para chegar na perda para um conjunto $(y_j, \hat{y}_j)$, a \textit{CCE} faz uso de múltiplas entropias-cruzadas, uma para cada classe, as quais são somadas chegando então na perda.

Considerando um conjunto $\hat{y}_j = [\hat{y}_{j,1}, \dots, \hat{y}_{j,C}]$ que se refere a dsitribuição de probabilidade para uma amostra $j$, a entropia cruzada categórica por amostra é dada então pela Equação \ref{eq:categorical-cross-entropy-per-sample}.

\begin{equacaodestaque}{Entropia Cruzada Categórica (\textit{CCE}) para uma Amostra $j$}
    \Loss_{CCE}(y_j, \hat{y}_j) = - \sum_{c=1}^{C} y_{j,c} \log(\hat{y}_{j,c})
    \label{eq:categorical-cross-entropy-per-sample}
\end{equacaodestaque}

De forma que ao calcular a média sobre $n$ amostras é possível chegar na Equação \ref{eq:categorical-cross-entropy-per-n-samples}, que representa a perda entropia cruzada categórica para $n$ amostras.

\begin{equacaodestaque}{Entropia Cruzada Categórica (\textit{CCE}) para $n$ Amostras}
    \Loss_{CCE}(\theta) = - \frac{1}{n} \sum_{i = 1}^n \sum_{c=1}^C y_{j, c} \log(\hat{y}_{j,c})
    \label{eq:categorical-cross-entropy-per-n-samples}
\end{equacaodestaque}

Considerando essas duas fórmulas, é possível também analisar o gráfico da entropia cruzada categórica, o qual está representado na Figura \ref{fig:categorical-cross-entropy}. Neste caso, está sendo representado o gráfico de apenas uma das funções $-\log(\hat{y}_k)$, mas o gráfico real da \textit{CCE} é mais complexo, pois seria o resultado da soma de um conjunto de funções $-\log(\hat{y}_k)$, impendido a plotagem do gráfico real.

\begin{figure}

    \begin{tikzpicture}
        \begin{axis}[
            title={Função de Perda: Entropia Cruzada Categórica},
            xlabel={Probabilidade Prevista para a Classe Correta ($\hat{y}_k$)},
            ylabel={Perda Calculada},
            axis lines=left,              % Eixos no canto inferior esquerdo
            grid=major,                   % Adiciona uma grade principal
            grid style={dashed, gray!40},   % Estilo da grade
            xmin=0, xmax=1.05,            % Limites do eixo x
            ymin=0, ymax=5,               % Limites do eixo y
            legend pos=north east,        % Posição da legenda
            width=12cm,                   % Largura do gráfico
            height=9cm,                   % Altura do gráfico
            title style={font=\bfseries},
            label style={font=\small},
            tick label style={font=\scriptsize}
        ]
            % Plota a função -log(y_k_hat)
            \addplot[
                domain=0.01:1, % Domínio para evitar log(0)
                samples=100,
                color=purple,
                very thick
            ] {-ln(x)};
            
            \addlegendentry{$L = -\log(\hat{y}_k)$}
            
        \end{axis}
    \end{tikzpicture}

    \caption{Representação gráfica da função de perda Entropia Cruzada Categórica (\textit{Categorical Cross Entropy}).}
    \label{fig:categorical-cross-entropy}
    \fonte{O autor (2025).}

\end{figure}

Considerando esses cenários, agora cabe destacar algumas características dessa função de perda:

\begin{itemize}
    \item \textbf{Saída softmax:} Semelhante a função de entropia cruzada binária, a qual aceitava valores entre 0 e 1 e por isso precisava de ser utilizada junto com funções como a sigmoide logística, a entropia cruzada categórica também tem seus requisitos. Como um modelo de rede neural geralmente retorna como saída \textit{logits} $z_i \in \mathbb{R}^C$, ao utilizar uma função \textit{softmax} é possível garantir que a $\sum_{j} \hat{p}_{i,j} = 1$, ou seja, que xxxx \parencite{LossesArticle}.
    \item \textbf{Penalização alta para erros confiantes:} Como \textcite{LossesArticle} explicam, a entropia cruzada categórica segue a mesma ideia da entropia cruzada binária, de forma que essa função também tem como propriedade punir os erros confiantes de forma mais elevada.
    \item \textbf{Requisitos \textit{one-hot}}: Como dito anteriormente a \textit{CCE} aceita um vetor em formato \textit{one-hot}, isso significa que os valores tanto para $y_j$ quanto para $\hat{y}_j$ devem estar em formato de probabilidade, estando em valores entre zero e um \footnote{Para trabalhar com valores em que os rôtulos estão organizados com valores inteiros é possível utilizar uma variante da \textit{categorical cross-entropy}: a \textit{sparse cross-entropy}. Essa função está melhor explicada na Seção \ref{sec:sparse-cross-entropy}}.
\end{itemize}

É possível também discutir a derivada da entropia cruzada categórica, pois ela trás uma discussão interessante. Como dito anteriormente, na maioria das vezes a saída da última camada de um modelo de classificação multi-classe é adicionada a função de ativação \textit{softmax}, permitindo que os valores da saída se enquandrem no intervalo ${0, 1}$, como consequência, é possível simplificar o cálculo do gradiente para essa função. 

Antes é lembrar da fórmula da \textit{softmax}, a qual é dada pela Equação \ref{eq:perdas-softmax}.

\begin{equation}
    \text{Softmax}(z_i) = \frac{e^{z_i}}{\sum_{j=1}^{K} e^{z_j}}
    \label{eq:perdas-softmax}
\end{equation}

\textbf{Passo 1: Aplicar a regra da cadeia}

Considerando a equação da entropia cruzada categórica e a da função de ativação \textit{softmax}, o primeiro passo é calcular a regra da cadeia, pois a perda $\Loss_{CCE}$ depende das previsões $\hat{y}_j$ e as previsões por sua vez dependem dos \textit{logits}. Além disso, vale notar que um único \textit{logit} $z_i$ pode afetar todas as saídas da \textit{softmax}, por conta do somatório no denomizador. 

Assim, é possível escrever a regra da cadeia como sendo:

\[
    \frac{\partial \Loss}{\partial z_i} = \sum_{j=1}^{C} \frac{\partial \Loss}{\partial \hat{y}_j} \cdot \frac{\partial \hat{y}_j}{\partial z_i}
\]

\textbf{Passo 2: Calcular as derivadas parciais}

\textbf{Passo 2A: Derivada da Perda em Relação à Previsão ($\frac{\partial \Loss}{\partial \hat{y}_j}$) }

Para fazer esse cálculo é possível derivar a perda em relação a uma única saída $\hat{y}_j$ encontrando então:

\[
    \frac{\partial \Loss}{\partial \hat{y}_j} = \frac{\partial}{\partial \hat{y}_j} \left( - \sum_{k=1}^{C} y_k \log(\hat{y}_k) \right) = -y_j \cdot \frac{1}{\hat{y}_j} = -\frac{y_j}{\hat{y}_j}
\]

\textbf{Passo 2B: Derivada da Softmax}

Para fazer esse cálculo é preciso considerar dois diferentes casos, o primeiro é quando $i = j$, ou seja derivada da saída de uma classe em relação à sua própria entrada de logit, assim, utilizando a regra do quociente é possível chegar na expressão:

\[
    \frac{\partial \hat{y}_i}{\partial z_i} = \frac{e^{z_i}(\sum_k e^{z_k}) - e^{z_i} \cdot e^{z_i}}{(\sum_k e^{z_k})^2}
\]

Simplificando a expressão uma primeira vez:

\[
    \frac{e^{z_i}}{\sum_k e^{z_k}} - \left( \frac{e^{z_i}}{\sum_k e^{z_k}} \right)^2
\]

É possível chegar então nos termos:

\[
    \hat{y}_i - \hat{y}_i^2 = \hat{y}_i (1 - \hat{y}_i)
\]

Já para o caso em que que é calculada a derivada da saída de uma classe em relação à entrada de \textit{logit} de outra classe, ou seja $i \neq j$ a equação é dada por:

\[
    \frac{\partial \hat{y}_j}{\partial z_i} = \frac{0 \cdot (\sum_k e^{z_k}) - e^{z_j} \cdot e^{z_i}}{(\sum_k e^{z_k})}
\]

Simplificando ela é possível chegar em:

\[
    - \left( \frac{e^{z_j}}{\sum_k e^{z_k}} \right) \left( \frac{e^{z_i}}{\sum_k e^{e_z}} \right) = -\hat{y}_j \hat{y}_i
\].

\textbf{Passo 3: Juntando os termos e simplificando as expressões}

Agora, o próximo passo é substituir as derivadas parciais calculadas na regra da cadeia, para isso será separado o somatório nos casos em que $i = j$ e nos casos que $i \neq = j$.

\[
    \frac{\partial \Loss}{\partial z_i} = \left( \frac{\partial \Loss}{\partial \hat{y}_i} \cdot \frac{\partial \hat{y}_i}{\partial z_i} \right) + \sum_{j \neq i} \left( \frac{\partial \Loss}{\partial \hat{y}_j} \cdot \frac{\partial \hat{y}_j}{\partial z_i} \right)
\]

Subtituindo os resultados encontrados:

\[
    \left( - \frac{y_i}{\hat{y}_i} \cdot \hat{y}_i (1 - \hat{y}_i) \right) + \sum_{j \neq i} \left( - \frac{y_j}{\hat{y}_j} \cdot (-\hat{y}_j\hat{y}_i) \right)
\]

Simplificando os termos:

\[
    - y_i (1 - \hat{y}_i) + \sum_{j \neq i} (y_j \hat{y}_i)
\]

Aplicando a distributiva:

\[
    - y_i + y_i \hat{y}_i + \hat{y}_i \sum_{j \neq i} y_j
 \]

 Perceba um detalhe interessante, $y$ representa um vetor \textit{one-hot encoded}, o qual contém todas as probabilidades para cada uma das classes que o modelo está analisando, isso sigfica que a soma de todos esses elementos vai ser 1. Com isso é possível escrever que $\sum_{j = 1}^C y_j = 1$. Contudo, na expressão que está sendo desenvolvida nos temos todos os elementos exceto pelo o i-ésimo, portanto: $\sum_{j\neq i} y_j = 1 - y_i$.

 Substituindo essa informação na equação é possível chegar em:

 \[
    y_i + y_i \hat{y}_i + \hat{y}_i (1 - y_i)
 \]

 Aplicando mais uma vez a distributiva para expandir o termo $\hat{y}_i (1 - y_i)$:

 \[
    -y_i + y_i \hat{y}_i + \hat{y}_i - y_i \hat{y}_i
 \]

 Perceba que os termos $y_i \hat{y}_i $ e $y_i \hat{y}_i$ se cancelam, então é possível chegar na expressão \ref{eq:category-cross-entropy-derivada}. A qual representa a derivada da \textit{categorical cross-entropy} para um cenário em que o último componente do modelo é a função de ativação softmax. Note que ao invés de ser uma derivada complexa como nos outros casos vistos até agora, a derivada da \textit{CCE} passa a ser apenas o cálculo da diferença entre os valores preditos e os valores reais.

\begin{equacaodestaque}{Derivada da Entropia Cruzada Categórica para a Softmax}
    \frac{\partial \Loss}{\partial z_i} = \hat{y}_i - y_i
    \label{eq:category-cross-entropy-derivada}
\end{equacaodestaque}

Note também que a que essa simplificação dos cálculos para a derivada na entropia cruzada utilizando a \textit{softmax} também reflete no gráfico, que é composto apenas de , como é possível ver na Figura \ref{fig:categorical-cross-entropy-derivada-com-softmax}.

\begin{figure}[h!]
    \centering
    \begin{tikzpicture}
        \begin{axis}[
            title={Derivada da Entropia Cruzada Categórica com Softmax},
            xlabel={Probabilidade Prevista para a Classe $i$ ($\hat{y}_i$)},
            ylabel={Gradiente da Perda ($\frac{\partial L}{\partial z_i}$)},
            axis lines=middle,             % Eixos centrados em (0,0)
            grid=major,                  % Adiciona uma grade principal
            grid style={dashed, gray!40},  % Estilo da grade
            xmin=-0.1, xmax=1.1,           % Limites do eixo x
            ymin=-1.1, ymax=1.1,           % Limites do eixo y
            legend pos=north west,         % Posição da legenda
            width=12cm,                    % Largura do gráfico
            height=9cm,                    % Altura do gráfico
            title style={font=\bfseries},
            label style={font=\small},
            tick label style={font=\scriptsize}
        ]
            % Curva para a classe correta (y_i = 1)
            % A derivada é y_hat - 1
            \addplot[
                domain=0:1,
                samples=10,
                color=blue,
                very thick
            ] {x - 1};
            \addlegendentry{Classe Correta ($y_i=1 \implies \hat{y}_i - 1$)}

            % Curva para uma classe incorreta (y_i = 0)
            % A derivada é y_hat - 0
            \addplot[
                domain=0:1,
                samples=10,
                color=red,
                very thick
            ] {x};
            \addlegendentry{Classe Incorreta ($y_i=0 \implies \hat{y}_i - 0$)}
            
        \end{axis}
    \end{tikzpicture}
    \caption{Representação gráfica da derivada da \textit{categorical cross entropy} com Softmax.}
    \label{fig:categorical-cross-entropy-derivada-com-softmax}
    \fonte{O autor (2025).}
\end{figure}

Caso a \textit{cetegorical cross entropy} não seja utilizada em conjunto com a função de ativação \textit{softmax} na saída do modelo, a sua derivada passa a ser um pouco mais complexa, sendo representada pela Equação \ref{eq:categorical-cross-entropy-derivada}.

\begin{equacaodestaque}{Derivada da Entropia Cruzada Categórica (em relação à previsão)}
    \frac{\partial \Loss}{\partial \hat{y}_i} = -\frac{y_i}{\hat{y}_i}
    \label{eq:categorical-cross-entropy-derivada}
\end{equacaodestaque}

\medskip
\begin{center}
 * * *
\end{center}
\medskip

\textbf{Algumas Aplicações da Entropia-Cruzada Categórica em Problemas de Classificação Multi-Classe}

\begin{itemize}
    \item \textbf{Aplicação 1 (Área):}
    \item \textbf{Aplicação 2 (Área):}
    \item \textbf{Aplicação 3 (Área):}
    \item \textbf{Aplicação 4 (Área):}
\end{itemize}

Conhecida a \textit{CCE}, uma das principais funcões de perda para ser utilizada para problemas de classificação multi-classe, é possível se perguntar: O que acontece se os dados não estiverem codificados em formato \textit{one-hot}? Para contornar esse problema, uma solução é utilizar a \textit{sparse categorical cross-entropy}, a qual será explicada em sequência.

\subsection{Entropia Cruzada Categórica Esparsa (Sparse Categorical Cross-Entropy)}
\label{sec:sparse-cross-entropy}

A entropia cruzada categórica esparsa é utilizada para os cenários em que os rótulos das classes são dados em inteiros $y_j \in {1, 2, \cdots C}$ \parencite{LossesArticle}. A \textit{sparese CCE} para um conjunto de individual de predição é dada pela Equação \ref{eq:sparse-categorical-cross-entropy-per-sample}

\begin{equacaodestaque}{Entropia Cruzada Categórica Esparsa (\textit{Esparse CCE}) para uma Amostra $j$}
    \Loss_{\text{sparse CCE}}(y_j, \hat{y}_j) = - \sum_{c=1}^{C} y_{j,c} \log(\hat{y}_{j,c})
    \label{eq:sparse-categorical-cross-entropy-per-sample}
\end{equacaodestaque}

Já para um conjunto maior de amostras, o cálculo da perda é dado de forma semelhante à \textit{CCE}, calculando as médias das perdas, como na Equação \ref{eq:sparse-categorical-cross-entropy-per-n-samples}

\begin{equacaodestaque}{Entropia Cruzada Categórica Esparsa (Sparse \textit{CCE}) para $n$ Amostras}
    \Loss_{\text{sparse CCE}}(\theta) = - \frac{1}{n} \sum_{i = 1}^n \sum_{c=1}^C y_{j, c} \log(\hat{y}_{j,c})
    \label{eq:sparse-categorical-cross-entropy-per-n-samples}
\end{equacaodestaque}

Perceba que as fórmulas da \textit{sparse categorical cross entropy} são iguais as da sua versão para rótulos codificados para formato \textit{one-hot}, por isso os seus gráficos também serão iguais, assim, o próximo passo é discutir algumas das características dessa função de perda.

\begin{itemize}
    \item \textbf{Eficiência:} \textcite{LossesArticle} explicam que em problemas de classificação com um grande número de classes, a codificação \textit{one-hot} pode ser memória-intensiva, ao utilizar a \textit{sparse CCE}, os seus indexes que já apontam diretamente para a probabilidade conseguem escapar de trabalhar com dados em formato \textit{one-hot}.
    \item \textbf{Similadirades com a \textit{CCE}:} A \textit{sparse CCE} possui grandes similaridades com a \textit{CCE} original, isso significa que vantagens como a diferenciabilidade e a penalização de erros muito confiantes que são características da \textit{categorical cross entropy}, também estão presentes na sua versão esparsa.
\end{itemize}

Além dessa variante, a \textit{categorical cross-entropy} possui uma versão que faz uso de pesos para as classes, buscando trabalhar com casos em que existe uma presença maior de algumas classes do que de outras. Essa é a \textit{weighted categorical cross-entropy}, que será vista em seguida.

\medskip
\begin{center}
 * * *
\end{center}
\medskip

\textbf{Algumas Aplicações da Entropia-Cruzada Categórica Esparsa em Problemas de Classificação Multi-Classe}

\begin{itemize}
    \item \textbf{Aplicação 1 (Área):}
    \item \textbf{Aplicação 2 (Área):}
    \item \textbf{Aplicação 3 (Área):}
    \item \textbf{Aplicação 4 (Área):}
\end{itemize}

\subsection{Weighted Categorical Cross-Entropy (WCCE)}

A fórmula da \textit{weighted categorical cross-entropy} lembra bastante a fórmula da \textit{weighted binary cross-entropy}, pois sua única diferença com a variante original é a adição de um peso multiplicando o erro para aquela classe. Essa função de perda para uma amostra $j$ é dada pela Equação \ref{eq:weighted-categorical-cross-entropy-per-sample}.

\begin{equacaodestaque}{Entropia Cruzada Categórica Ponderada (\textit{WCCE}) para uma Amostra $j$}
    \Loss_{\text{WCCE}}(y_j, \hat{y}_j) = - \sum_{c=1}^{C} \alpha y_{j,c} \log(\hat{y}_{j,c})
    \label{eq:weighted-categorical-cross-entropy-per-sample}
\end{equacaodestaque}

De forma semelhante as outras funções, é possível expandir essa fórmula para calcular a perda para um conjunto $n$ de amostras, para isso, é feito o cálculo da média das perdas individuais, como é mostrado na Equação \ref{eq:weighted-categorical-cross-entropy}.

\begin{equacaodestaque}{Entropia Cruzada Categórica Ponderada (\textit{CWCE}) para $n$ Amostras}
    \Loss_{\text{WCCE}}(\theta) = - \frac{1}{n} \sum_{i = 1}^n \sum_{c=1}^C \alpha_n y_{j, c} \log(\hat{y}_{j,c})
    \label{eq:weighted-categorical-cross-entropy}
\end{equacaodestaque}

Assim, de forma semelhante à \textit{weighted binary cross-entropy} é possível utilizar a sua variante multi-classe para ser trabalhada em cenários em que uma classe, ou um grupo de classes aparece de forma mais frequente que outro.

\medskip
\begin{center}
 * * *
\end{center}
\medskip

\textbf{Algumas Aplicações da Entropia-Cruzada Categórica Ponderada em Problemas de Classificação Multi-Classe}

\begin{itemize}
    \item \textbf{Aplicação 1 (Área):}
    \item \textbf{Aplicação 2 (Área):}
    \item \textbf{Aplicação 3 (Área):}
    \item \textbf{Aplicação 4 (Área):}
\end{itemize}

\section{Multilabel Loss}

\begin{equacaodestaque}{Perda para Classificação Multirrótulo}
    \Loss_{\text{Multilabel}} = - \sum_{j=1}^{q} [y_j \log(\hat{y}_j) + (1 - y_j) \log(1 - \hat{y}_j)]
    \label{eq:multilabel-loss}
\end{equacaodestaque}

\begin{figure}[h!]
    \centering
    \begin{tikzpicture}
        \begin{axis}[
            title={Contribuição de um Rótulo para a Perda Multirrótulo},
            xlabel={Probabilidade Prevista para o Rótulo $j$ ($\hat{y}_j$)},
            ylabel={Contribuição Individual para a Perda},
            axis lines=left,
            grid=major,
            grid style={dashed, gray!40},
            xmin=0, xmax=1,
            ymin=0, ymax=5,
            legend pos=north west,
            width=12cm,
            height=9cm,
            title style={font=\bfseries},
            label style={font=\small},
            tick label style={font=\scriptsize}
        ]
            % Curva para quando o rótulo está presente (y_j = 1)
            \addplot[
                domain=0.01:0.999, samples=100, color=blue, very thick
            ] {-ln(x)};
            \addlegendentry{Rótulo Presente ($y_j=1$)}

            % Curva para quando o rótulo está ausente (y_j = 0)
            \addplot[
                domain=0.001:0.99, samples=100, color=red, very thick
            ] {-ln(1-x)};
            \addlegendentry{Rótulo Ausente ($y_j=0$)}
            
        \end{axis}
    \end{tikzpicture}
    \caption{A perda multirrótulo é a soma das contribuições de cada rótulo individual, que se comportam como a BCE.}
    \label{fig:multilabel-loss}
    \fonte{O autor (2025).}
\end{figure}

\begin{equacaodestaque}{Derivada da Perda Multirrótulo}
    \frac{\partial \Loss_{\text{Multilabel}}}{\partial z_i} = \hat{y}_i - y_i
    \label{eq:multilabel-loss-derivada}
\end{equacaodestaque}

\begin{figure}[h!]
    \centering
    \begin{tikzpicture}
        \begin{axis}[
            title={Derivada da Perda Multirrótulo (para um Rótulo)},
            xlabel={Probabilidade Prevista para o Rótulo $i$ ($\hat{y}_i$)},
            ylabel={Gradiente da Perda ($\frac{\partial L}{\partial z_i}$)},
            axis lines=middle,
            grid=major,
            grid style={dashed, gray!40},
            xmin=-0.1, xmax=1.1,
            ymin=-1.1, ymax=1.1,
            legend pos=north west,
            width=12cm,
            height=9cm,
            title style={font=\bfseries},
            label style={font=\small},
            tick label style={font=\scriptsize}
        ]
            % Curva para rótulo presente (y_i = 1)
            \addplot[domain=0:1, samples=10, color=blue, very thick] {x - 1};
            \addlegendentry{Rótulo Presente ($y_i=1$)}

            % Curva para rótulo ausente (y_i = 0)
            \addplot[domain=0:1, samples=10, color=red, very thick] {x};
            \addlegendentry{Rótulo Ausente ($y_i=0$)}
            
        \end{axis}
    \end{tikzpicture}
    \caption{O gradiente para cada rótulo é simplesmente a diferença entre a probabilidade prevista e o valor real (0 ou 1).}
    \label{fig:multilabel-loss-derivada}
    \fonte{O autor (2025).}
\end{figure}

\medskip
\begin{center}
 * * *
\end{center}
\medskip

\textbf{Algumas Aplicações da Perda Multirótulo em Problemas de Classificação Multi-Label}

\begin{itemize}
    \item \textbf{Aplicação 1 (Área):}
    \item \textbf{Aplicação 2 (Área):}
    \item \textbf{Aplicação 3 (Área):}
    \item \textbf{Aplicação 4 (Área):}
\end{itemize}

\section{Comparativo: Funções de Perda para Classificação}

\section{Fluxograma: Escolhendo a Função de Perda Ideal}

\chapter{Funções de Perda para Usos Específicos}
\label{cap:perdas-especificas}

\section{Focal Loss}

\begin{equacaodestaque}{Focal Loss}
    \Loss_{\text{FL}}(p_t) = -(1 - p_t)^\gamma \log(p_t)
    \label{eq:focal-loss}
\end{equacaodestaque}

\begin{figure}[h!]
    \centering
    \begin{tikzpicture}
        \begin{axis}[
            title={Comparação: Cross-Entropy vs. Focal Loss},
            xlabel={Probabilidade Prevista para a Classe Correta ($p_t$)},
            ylabel={Perda Calculada},
            axis lines=left,
            grid=major,
            grid style={dashed, gray!40},
            xmin=0, xmax=1.05,
            ymin=0, ymax=5,
            legend pos=north east,
            width=12cm,
            height=9cm,
            title style={font=\bfseries},
            label style={font=\small},
            tick label style={font=\scriptsize}
        ]
            % Cross-Entropy Padrão
            \addplot[
                domain=0.01:1, samples=100, color=gray, dashed, thick
            ] {-ln(x)};
            \addlegendentry{Cross-Entropy ($L = -\log(p_t)$)}

            % Focal Loss com gamma=2
            \addplot[
                domain=0.01:1, samples=100, color=purple, very thick
            ] {-(1-x)^2 * ln(x)};
            \addlegendentry{Focal Loss ($\gamma=2$)}
            
        \end{axis}
    \end{tikzpicture}
    \caption{Gráfico da Focal Loss em comparação com a Entropia Cruzada padrão. A perda para exemplos fáceis ($p_t \to 1$) é drasticamente reduzida.}
    \label{fig:focal-loss}
    \fonte{O autor (2025).}
\end{figure}

\begin{equacaodestaque}{Derivada da Focal Loss}
    \frac{\partial \Loss_{\text{FL}}}{\partial z} = 
    \begin{cases} 
        \hat{y}(\gamma(1-\hat{y})\log(\hat{y}) + \hat{y} - 1) & \text{se } y=1 \\
        (1-\hat{y})(\gamma\hat{y}\log(1-\hat{y}) + \hat{y}) & \text{se } y=0
    \end{cases}
    \label{eq:focal-loss-derivada}
\end{equacaodestaque}

\begin{figure}[h!]
    \centering
    \begin{tikzpicture}
        \begin{axis}[
            title={Derivada da Focal Loss ($\gamma=2$)},
            xlabel={Probabilidade Prevista ($\hat{y}$)},
            ylabel={Gradiente da Perda ($\frac{\partial L}{\partial z}$)},
            axis lines=middle,
            grid=major,
            grid style={dashed, gray!40},
            xmin=-0.1, xmax=1.1,
            ymin=-1.1, ymax=1.1,
            legend pos=south east,
            width=12cm,
            height=9cm,
            title style={font=\bfseries},
            label style={font=\small},
            tick label style={font=\scriptsize}
        ]
            % Derivada da Cross-Entropy padrão (y_hat - y)
            \addplot[domain=0:1, samples=10, color=gray, dashed, thick] {x-1};
            \addplot[domain=0:1, samples=10, color=gray, dashed, thick] {x};
            \addlegendentry{Derivada CE}
            
            % Derivada da Focal Loss para y=1
            \addplot[
                domain=0:1, samples=101, color=purple, very thick
            ] {x*(2*(1-x)*ln(x) + x - 1)};
            \addlegendentry{Derivada FL ($y=1$)}

            % Derivada da Focal Loss para y=0
            \addplot[
                domain=0:1, samples=101, color=orange, very thick
            ] {(1-x)*(2*x*ln(1-x) + x)};
            \addlegendentry{Derivada FL ($y=0$)}
            
        \end{axis}
    \end{tikzpicture}
    \caption{Gráfico da derivada da Focal Loss. O gradiente para exemplos fáceis (próximo das bordas 0 e 1) é suprimido em comparação com a derivada da Entropia Cruzada padrão.}
    \label{fig:focal-loss-derivada}
    \fonte{O autor (2025).}
\end{figure}

\section{Fluxograma:}

% ===================================================================
% Arquivo: capitulos/parte-III-pilares/cap-10-perda-binaria.tex
% ===================================================================

\chapter{Métricas de Avaliação}
\label{cap:metricas-de-avaliacao}

\section{Métricas de Avaliação}

\subsection{Acurácia}

\begin{equacaodestaque}{Acurácia}
    \text{Acurácia} = \frac{VP + VN}{VP + VN + FP + FN}
    \label{eq:acuracia}
\end{equacaodestaque}

\subsection{Precisão}

\begin{equacaodestaque}{Precisão (Precision)}
    \text{Precisão} = \frac{VP}{VP + FP}
    \label{eq:precisao}
\end{equacaodestaque}

\subsection{Revocação ou Sensibilidade}

\begin{equacaodestaque}{Revocação (Recall) ou Sensibilidade}
    \text{Revocação} = \frac{VP}{VP + FN}
    \label{eq:revocacao}
\end{equacaodestaque}

\subsection{F1-Score}

\begin{equacaodestaque}{F1-Score}
    \text{F1-Score} = 2 \times \frac{\text{Precisão} \times \text{Revocação}}{\text{Precisão} + \text{Revocação}}
    \label{eq:f1_score}
\end{equacaodestaque}

\subsection{Curva ROC e AUC}

\begin{equacaodestaque}{Taxa de Falsos Positivos (FPR)}
    \text{FPR} = \frac{FP}{FP + VN}
    \label{eq:fpr}
\end{equacaodestaque}

\subsection{Métricas Para Regressão ($R^2$)}

\begin{equacaodestaque}{Raiz do Erro Quadrático Médio (RMSE)}
    L_{\text{RMSE}} = \sqrt{\frac{1}{N} \sum_{i=1}^{N} (y_i - \hat{y}_i)^2}
    \label{eq:rmse}
\end{equacaodestaque}

\begin{equacaodestaque}{Coeficiente de Determinação (R²)}
    R^2 = 1 - \frac{\sum_{i=1}^{N}(y_i - \hat{y}_i)^2}{\sum_{i=1}^{N}(y_i - \bar{y})^2}
    \label{eq:r_quadrado}
\end{equacaodestaque}



\chapter{Autodiff}
\label{cap:autodiff}

\section{Técnicas de Diferenciação}

\section{Autodiff Direta}

\subsection{Autodiff Direta}

\subsection{Autodiff Direta Com Números Duais}

\section{Autodiff Reversa}
% ===================================================================
% Arquivo: capitulos/parte-III-pilares/cap-12-metaheuristicas.tex
% ===================================================================

\chapter{Metaheurísticas: Otimizando Redes Neurais Sem o Gradiente}
\label{cap:otimizacao-metaheuristicas}

O texto do seu capítulo começa aqui...

\section{Algoritmos Evolutivos}

\section{Inteligência de Enxame}

% =======================================================
% PARTE IV: APRENDIZADO DE MÁQUINA CLÁSSICO
% =======================================================
\part{Aprendizado de Máquina Clássico}

% O comando '\include' inicia uma nova página para cada capítulo e
% carrega o conteúdo do arquivo .tex especificado.
% ===================================================================
% Arquivo: capitulos/parte_IV_ml_classico/cap_13_regressao.tex
% ===================================================================

\chapter{Regressão Linear e Logística}
\label{cap:regressao}

O texto do seu capítulo começa aqui...
% ===================================================================
% Arquivo: capitulos/parte_IV_ml_classico/cap_14_arvores.tex
% ===================================================================

\chapter{Árvores de Decisão e Florestas Aleatórias}
\label{cap:arvores}

O texto do seu capítulo começa aqui...
% ===================================================================
% Arquivo: capitulos/parte_IV_ml_classico/cap_15_svm.tex
% ===================================================================

\chapter{Máquinas de Vetores de Suporte}
\label{cap:svm}

O texto do seu capítulo começa aqui...
% ===================================================================
% Arquivo: capitulos/parte_IV_ml_classico/cap_16_emsamble.tex
% ===================================================================

\chapter{Ensamble}
\label{cap:ensamble}

O texto do seu capítulo começa aqui...
% ===================================================================
% Arquivo: capitulos/parte_IV_ml_classico/cap_17_dimensionalidade.tex
% ===================================================================

\chapter{Dimensionalidade}
\label{cap:dimensionalidade}

\section{Exemplo Ilustrativo}

\section{A Maldição da Dimensionalidade}

\section{Seleção de Características (Feature Selection)}

\section{Extração de Características (Feature Extraction)}

\subsection{Análise de Componentes Principais (PCA)} 

\subsection{t-SNE (t-Distributed Stochastic Neighbor Embedding) e UMAP}
% ===================================================================
% Arquivo: capitulos/parte_IV_ml_classico/cap_18_clusterizacao.tex
% ===================================================================

\chapter{Clusterização}
\label{cap:clusterizacao}

O texto do seu capítulo começa aqui...

% =======================================================
% PARTE V: REDES NEURAIS PROFUNDAS (DNNs)
% =======================================================
\part{Redes Neurais Profundas (DNNs)}

% ===================================================================
% Arquivo: capitulos/parte_V_deep_learning/cap_19_mlp.tex
% ===================================================================

\chapter{Perceptrons MLP - Redes Neurais Artificiais}
\label{cap:mlp}

O texto do seu capítulo começa aqui...
% ===================================================================
% Arquivo: capitulos/parte_V_deep_learning/cap_20_ffn.tex
% ===================================================================

\chapter{Redes FeedForward (FFNs)}
\label{cap:ffn}

O texto do seu capítulo começa aqui...
% ===================================================================
% Arquivo: capitulos/parte_V_deep_learning/cap_21_dbn.tex
% ===================================================================

\chapter{Redes de Crença Profunda (DBNs) e Máquinas de Boltzmann Restritas}
\label{cap:dbn}

O texto do seu capítulo começa aqui...
% ===================================================================
% Arquivo: capitulos/parte_V_deep_learning/cap_22_cnn.tex
% ===================================================================

\chapter{Redes Neurais Convolucionais (CNN)}
\label{cap:cnn}

% ===================================================================
% Resumo do capítulo
% ===================================================================

\section{Exemplo Ilustrativo}

\section{Camadas Convolucionais: O Bloco Fundamental para as CNNs}

\subsection{Implementação em Python}

\section{Camadas de Poooling: Reduzindo a Dimensionalidade}

\subsection{Max Pooling}

\subsection{Avg Pooling}

\subsection{Global Abg Pooling}

\subsection{Implementação em Python}

\section{Camada Flatten: Achatando os Dados}

\subsection{Implementação em Python}

\section{Criando uma CNN}

\section{Detecção de Objetos}

\section{Redes Totalmente Convolucionais (FCNs)}

\section{You Only Look Once (YOLO)}

\section{Algumas Arquiteturas de CNNs}

\subsection{LeNet-5}

\subsection{AlexNet}

\subsection{GoogLeNet}

\subsection{VGGNet}

\subsection{ResNet}

\subsection{Xception}

\subsection{SENet}
% ===================================================================
% Arquivo: capitulos/parte_V_deep_learning/cap_23_resnet.tex
% ===================================================================

\chapter{Redes Residuais (ResNets)}
\label{cap:resnet}

O texto do seu capítulo começa aqui...
% ===================================================================
% Arquivo: capitulos/parte_V_deep_learning/cap_24_rnn.tex
% ===================================================================

\chapter{Redes Neurais Recorrentes (RNN)}
\label{cap:rnn}

O texto do seu capítulo começa aqui...

% ===================================================================
% resumo do capítulo
% ===================================================================

\section{Exemplo Ilustrativo}

\section{Neurônios e Células Recorrentes}

\subsection{Implementação em Python}

\section{Células de Memória}

\subsection{Implementação em Python}

\section{Criando uma RNN}

\section{O Problema da Memória de Curto Prazo}

\subsection{Células LSTM}

\subsection{Conexões Peephole}

\subsection{Células GRU}
% ===================================================================
% Arquivo: capitulos/parte_V_deep_learning/cap_25_regularizacao.tex
% ===================================================================

\chapter{Técnicas para Melhorar o Desempenho de Redes Neurais}
\label{cap:regularizacao}

O texto do seu capítulo começa aqui...
% ===================================================================
% Arquivo: capitulos/parte_V_deep_learning/cap_26_transformers.tex
% ===================================================================

\chapter{Transformers}
\label{cap:transformers}

O texto do seu capítulo começa aqui...
% ===================================================================
% Arquivo: capitulos/parte_V_deep_learning/cap_30_gans.tex
% ===================================================================

\chapter{Redes Adversárias Generativas (GANs)}
\label{cap:gans}

O texto do seu capítulo começa aqui...
% ===================================================================
% Arquivo: capitulos/parte_V_deep_learning/cap_28_moe.tex
% ===================================================================

\chapter{Mixture of Experts (MoE)}
\label{cap:moe}

O texto do seu capítulo começa aqui...
% ===================================================================
% Arquivo: capitulos/parte_V_deep_learning/cap_29_diffusion.tex
% ===================================================================

\chapter{Modelos de Difusão}
\label{cap:diffusion}

O texto do seu capítulo começa aqui...
% ===================================================================
% Arquivo: capitulos/parte_V_deep_learning/cap_31_gnn.tex
% ===================================================================

\chapter{Redes Neurais de Grafos (GNNs)}
\label{cap:gnn}

O texto do seu capítulo começa aqui...
% Adicione o capítulo de otimizadores se ele estiver aqui


% --- APÊNDICES ---
% O comando \appendix muda a formatação dos capítulos para "Apêndice A", "Apêndice B", etc.
\appendix
\part{Apêndices}

\chapter{Comparativo dos Otimizadores}
\label{cap:comparativo-otimizadores}

A escolha do otimizador é um passo crucial no treinamento de redes neurais. Este capítulo apresenta os principais algoritmos de otimização baseados em gradiente, desde os métodos clássicos até as variantes adaptativas modernas, detalhando suas equações, ideias centrais e características.

% ===================================================================
% Otimizadores Clássicos
% ===================================================================
\section{Otimizadores Clássicos}

\subsection{Gradiente Descendente (GD)}

\textit{Atualiza os parâmetros na direção oposta ao gradiente, calculado sobre \textbf{todo} o conjunto de dados.}

\begin{equacaodestaque}{Atualização do Gradiente Descendente}
    \theta_{t+1} = \theta_t - \eta \nabla f(\theta_t)
\end{equacaodestaque}

\subsubsection*{Vantagens}
\begin{itemize}
    \item Garante uma convergência estável para um mínimo local (ou global, em funções convexas).
\end{itemize}

\subsubsection*{Desvantagens}
\begin{itemize}
    \item É computacionalmente caro e lento para datasets grandes, pois exige que todos os dados estejam na memória para cada atualização.
\end{itemize}

\subsection{Gradiente Descendente Estocástico (SGD)}

\textit{Atualiza os parâmetros usando o gradiente de \textbf{uma única amostra} aleatória por vez.}

\begin{equacaodestaque}{Atualização do Gradiente Descendente Estocástico}
    \theta_{t+1} = \theta_t - \eta \nabla f(\theta_t; x^{(i)})
\end{equacaodestaque}

\subsubsection*{Vantagens}
\begin{itemize}
    \item Muito mais rápido por iteração em comparação com o GD em lote.
    \item A natureza ruidosa das atualizações pode ajudar a escapar de mínimos locais rasos.
\end{itemize}

\subsubsection*{Desvantagens}
\begin{itemize}
    \item Apresenta uma trajetória de convergência ruidosa e com alta variância, podendo nunca se estabilizar no mínimo exato.
\end{itemize}

\subsection{Gradiente Descendente com Momento}

\textit{Adiciona "inércia" à atualização, acumulando uma média móvel dos gradientes passados para acelerar a descida.}

\begin{equacaodestaque}{Atualização com Momento}
    v_t = \beta v_{t-1} + \eta \nabla f(\theta_t) \\
    \theta_{t+1} = \theta_t - v_t
\end{equacaodestaque}

\subsubsection*{Vantagens}
\begin{itemize}
    \item Acelera a convergência, especialmente em direções onde o gradiente é consistente.
    \item Ajuda a amortecer oscilações em direções de alta curvatura.
\end{itemize}

\subsubsection*{Desvantagens}
\begin{itemize}
    \item Adiciona o hiperparâmetro de momento $\beta$, que precisa ser ajustado.
\end{itemize}

\subsection{Gradiente Acelerado de Nesterov (NAG)}

\textit{Um momento "mais inteligente" que calcula o gradiente em um ponto futuro estimado ("lookahead") para corrigir a direção da atualização.}

\begin{equacaodestaque}{Atualização com Momento de Nesterov}
    g_t = \nabla f(\theta_t - \beta v_{t-1}) \\
    v_t = \beta v_{t-1} + \eta g_t \\
    \theta_{t+1} = \theta_t - v_t
\end{equacaodestaque}

\subsubsection*{Vantagens}
\begin{itemize}
    \item Frequentemente converge mais rápido que o momento padrão.
    \item É mais eficaz em "antecipar" a curvatura, evitando ultrapassar o ponto de mínimo.
\end{itemize}

% ===================================================================
% Otimizadores Adaptativos Modernos
% ===================================================================
\section{Otimizadores Adaptativos Modernos}

\subsection{AdaGrad (Adaptive Gradient Algorithm)}

\textit{Adapta a taxa de aprendizado para cada parâmetro individualmente, diminuindo-a para parâmetros com gradientes grandes e frequentes.}

\begin{equacaodestaque}{Atualização do AdaGrad}
    \theta_{t+1} = \theta_t - \frac{\eta}{\sqrt{N_t + \epsilon}} g_t \\
    \text{(onde } N_t \text{ acumula os quadrados dos gradientes } g_t^2\text{)}
\end{equacaodestaque}

\subsubsection*{Vantagens}
\begin{itemize}
    \item É muito eficaz para lidar com dados esparsos, como em processamento de linguagem natural.
\end{itemize}

\subsubsection*{Desvantagens}
\begin{itemize}
    \item A taxa de aprendizado pode decair de forma muito agressiva e parar o treinamento prematuramente, pois o acumulador de gradientes no denominador só cresce.
\end{itemize}

\subsection{RMSProp (Root Mean Square Propagation)}

\textit{Resolve o problema do AdaGrad usando uma média móvel exponencial dos quadrados dos gradientes, o que evita que a taxa de aprendizado decaia para zero.}

\begin{equacaodestaque}{Atualização do RMSProp}
    \theta_{t+1} = \theta_t - \frac{\eta}{\sqrt{E[g^2]_t + \epsilon}} g_t \\
    \text{(onde } E[g^2]_t \text{ é uma média móvel de } g_t^2\text{)}
\end{equacaodestaque}

\subsubsection*{Vantagens}
\begin{itemize}
    \item Apresenta bom desempenho em problemas não-estacionários (onde a distribuição dos dados muda).
    \item É uma melhoria direta sobre o AdaGrad.
\end{itemize}

\subsection{Adam (Adaptive Moment Estimation)}

\textit{Calcula taxas de aprendizado adaptativas para cada parâmetro usando estimativas de primeiro (momento) e segundo (RMSProp) momentos dos gradientes.}

\begin{equacaodestaque}{Conceito do Adam}
    \text{Combina a inércia do \textbf{Momento} (1º momento, $m_t$)} \\
    \text{com a escala adaptativa do \textbf{RMSProp} (2º momento, $v_t$)} \\
    \text{e adiciona uma etapa de correção de viés.}
\end{equacaodestaque}

\subsubsection*{Vantagens}
\begin{itemize}
    \item Geralmente considerado o otimizador padrão para a maioria dos problemas.
    \item Combina os benefícios dos métodos de momento e de taxa de aprendizado adaptativa, sendo robusto e eficiente.
\end{itemize}

\subsection{AdaMax}

\textit{Generaliza o segundo momento do Adam, substituindo a média dos quadrados ($L_2$) pelo máximo ($L_\infty$) dos gradientes recentes, tornando a atualização mais estável.}

\begin{equacaodestaque}{Conceito do AdaMax}
    \text{Variante do Adam que usa a norma infinita ($L_\infty$) para o} \\
    \text{segundo momento, em vez da norma $L_2$.}
\end{equacaodestaque}

\subsubsection*{Vantagens}
\begin{itemize}
    \item Pode ser mais estável que o Adam, especialmente em cenários com gradientes ruidosos ou esparsos.
\end{itemize}

\subsection{Nadam (Nesterov-accelerated Adam)}

\textit{Incorpora o conceito de "lookahead" do Gradiente Acelerado de Nesterov (NAG) na estimativa do primeiro momento do Adam para uma atualização mais precisa.}

\begin{equacaodestaque}{Conceito do Nadam}
    \text{Combina o otimizador \textbf{Adam} com o momento de \textbf{Nesterov (NAG)}.}
\end{equacaodestaque}

\subsubsection*{Vantagens}
\begin{itemize}
    \item Frequentemente converge mais rápido que o Adam, especialmente em problemas com gradientes complexos e ruidosos.
\end{itemize}

\subsection{AdamW (Adam with Decoupled Weight Decay)}

\textit{Corrige a implementação da regularização L2 (decaimento de peso) no Adam, desacoplando-a da atualização do gradiente e aplicando-a diretamente aos pesos.}

\begin{equacaodestaque}{Atualização do AdamW}
    \theta_{t+1} = \theta_t - \eta \cdot (\text{update}_{Adam} + \lambda\theta_t)
\end{equacaodestaque}

\subsubsection*{Vantagens}
\begin{itemize}
    \item Melhora a generalização do modelo em comparação com o Adam padrão com regularização L2.
    \item Torna o ajuste da taxa de aprendizado e do decaimento de peso mais independente um do outro.
\end{itemize}
\chapter{Tabela das Funções de Ativação}

\begin{longtable}{@{} l p{0.25\linewidth} p{0.3\linewidth} p{0.3\linewidth} @{}}
    
    % --- TÍTULO (CAPTION) ---
    \caption{Comparativo das famílias de funções de ativação, suas propriedades, vantagens e desvantagens.}
    \label{tab:funcoes_comparativo_completo} \\

    % --- CABEÇALHO DA PRIMEIRA PÁGINA ---
    \toprule
    \textbf{Função} & \textbf{Equação e Derivada} & \textbf{Vantagens} & \textbf{Desvantagens} \\
    \midrule
    \endfirsthead

    % --- CABEÇALHO DAS PÁGINAS SEGUINTES ---
    \multicolumn{4}{l}{\small\textbf{Tabela \thetable{} – Continuação}} \\
    \toprule
    \textbf{Função} & \textbf{Equação e Derivada} & \textbf{Vantagens} & \textbf{Desvantagens} \\
    \midrule
    \endhead

    % --- RODAPÉ DE CONTINUAÇÃO ---
    \multicolumn{4}{r}{\small\textit{(Continua na próxima página)}} \\
    \endfoot

    % --- RODAPÉ FINAL (NA ÚLTIMA PÁGINA) ---
    \bottomrule
    \multicolumn{4}{l}{\parbox{\linewidth}{\small\textit{Fonte: O autor (2025).}}} \\
    \endlastfoot

    % --- CONTEÚDO DA TABELA (UNIFICADO) ---

    % --- Família Sigmoidal ---
    \textbf{Sigmoide} & 
    $\displaystyle \frac{1}{1 + e^{-z_i}}$ \newline\vspace{0.2cm}
    $\displaystyle \sigma(z_i)(1 - \sigma(z_i))$ 
    & 
    \begin{itemize}[noitemsep, topsep=0pt, partopsep=0pt, leftmargin=*]
        \item Saída no intervalo (0, 1), interpretável como probabilidade.
        \item Função suave e diferenciável.
    \end{itemize}
    &
    \begin{itemize}[noitemsep, topsep=0pt, partopsep=0pt, leftmargin=*]
        \item Não é centrada em zero.
        \item Sofre com o desvanecimento do gradiente.
    \end{itemize}
    \\ \addlinespace
    
    \textbf{Tangente hip.} & 
    $\displaystyle \frac{e^{z_i} - e^{-z_i}}{e^{z_i} + e^{-z_i}}$ \newline\vspace{0.2cm}
    $\displaystyle 1 - \tanh^2(z_i)$
    &
    \begin{itemize}[noitemsep, topsep=0pt, partopsep=0pt, leftmargin=*]
        \item Centrada em zero, acelera a convergência.
        \item Gradiente mais forte que a sigmoide.
    \end{itemize}
    &
    \begin{itemize}[noitemsep, topsep=0pt, partopsep=0pt, leftmargin=*]
        \item Ainda sofre com o desvanecimento do gradiente.
    \end{itemize}
    \\ \addlinespace
    
    \textbf{Softsign} &
    $\displaystyle \frac{z_i}{1 + |z_i|}$ \newline\vspace{0.2cm}
    $\displaystyle \frac{1}{(1 + |z_i|)^2}$
    &
    \begin{itemize}[noitemsep, topsep=0pt, partopsep=0pt, leftmargin=*]
        \item Computacionalmente eficiente.
        \item Satura mais lentamente que a Tanh.
    \end{itemize}
    &
    \begin{itemize}[noitemsep, topsep=0pt, partopsep=0pt, leftmargin=*]
        \item Derivada não pode ser expressa em termos da própria função.
    \end{itemize}
    \\ \addlinespace

    \textbf{Hard Sigmoid} &
    $\displaystyle \begin{cases} 0 & \text{se } z_i < -3 \\ z_i/6 + 0.5 & \text{se } -3 \le z_i \le 3 \\ 1 & \text{se } z_i > 3 \end{cases}$ \newline\vspace{0.2cm}
    $\displaystyle \begin{cases} 0 & \text{se } z_i < -3 \\ 1/6 & \text{se } -3 < z_i < 3 \\ 0 & \text{se } z_i > 3 \end{cases}$
    &
    \begin{itemize}[noitemsep, topsep=0pt, partopsep=0pt, leftmargin=*]
        \item Extremamente rápida e eficiente.
        \item Ideal para hardware com poucos recursos.
    \end{itemize}
    &
    \begin{itemize}[noitemsep, topsep=0pt, partopsep=0pt, leftmargin=*]
        \item Não é suave; pode "matar" gradientes.
        \item É uma aproximação.
    \end{itemize}
    \\ \addlinespace
    
    \textbf{Hard Tanh} &
    $\displaystyle \begin{cases} -1 & \text{se } z_i < -1 \\ z_i & \text{se } -1 \le z_i \le 1 \\ 1 & \text{se } z_i > 1 \end{cases}$ \newline\vspace{0.2cm}
    $\displaystyle \begin{cases} 0 & \text{se } z_i < -1 \\ 1 & \text{se } -1 < z_i < 1 \\ 0 & \text{se } z_i > 1 \end{cases}$
    &
    \begin{itemize}[noitemsep, topsep=0pt, partopsep=0pt, leftmargin=*]
        \item Extremamente rápida e centrada em zero.
        \item Ótima para hardware de baixo consumo.
    \end{itemize}
    &
    \begin{itemize}[noitemsep, topsep=0pt, partopsep=0pt, leftmargin=*]
        \item Não é suave; derivada nula em grande parte do domínio.
    \end{itemize}
    \\ \addlinespace

    % --- Família Retificadora ---
    \textbf{ReLU} & 
    $ \begin{cases}z_i, & \text{se } z_i > 0 \\0, & \text{se } z_i \leq 0\end{cases} $ \newline\vspace{0.2cm}
    $ \begin{cases}1, & \text{se } z_i > 0 \\0, & \text{se } z_i < 0 \\ \nexists & \text{se } z_i = 0\end{cases}$ 
    & 
    \begin{itemize}[noitemsep, topsep=0pt, partopsep=0pt, leftmargin=*]
        \item Computacionalmente eficiente.
        \item Evita o desvanecimento do gradiente.
        \item Promove esparsidade na rede.
    \end{itemize}
    &
    \begin{itemize}[noitemsep, topsep=0pt, partopsep=0pt, leftmargin=*]
        \item Não é centrada em zero.
        \item Pode "morrer" (Dying ReLU).
        \item Pode sofrer com a explosão de gradientes.
    \end{itemize}
    \\ \addlinespace

    \textbf{LReLU} & 
    $ \begin{cases}z_i, & \text{se } z_i \ge 0 \\ \alpha \cdot z_i, & \text{se } z_i < 0\end{cases} $ \newline\vspace{0.2cm}
    $\begin{cases}1, & \text{se } z_i > 0 \\ \alpha, & \text{se } z_i < 0 \\ \nexists, & \text{se } z_i = 0\end{cases}$
    &
    \begin{itemize}[noitemsep, topsep=0pt, partopsep=0pt, leftmargin=*]
        \item Resolve o problema da "Dying ReLU".
    \end{itemize}
    &
    \begin{itemize}[noitemsep, topsep=0pt, partopsep=0pt, leftmargin=*]
        \item O valor de $\alpha$ não é aprendido.
        \item Resultados podem ser inconsistentes.
    \end{itemize}
    \\ \addlinespace

    \textbf{PReLU} &
    $ \begin{cases}z_i, & \text{se } z_i \ge 0 \\ \alpha_i \cdot z_i, & \text{se } z_i < 0\end{cases} $ \newline\vspace{0.2cm}
    $\begin{cases}1, & \text{se } z_i > 0 \\ \alpha_i, & \text{se } z_i < 0 \\ \nexists, & \text{se } z_i = 0\end{cases}$
    &
    \begin{itemize}[noitemsep, topsep=0pt, partopsep=0pt, leftmargin=*]
        \item Variação da LReLU onde $\alpha$ é um parâmetro aprendido.
        \item Pode melhorar a performance.
    \end{itemize}
    &
    \begin{itemize}[noitemsep, topsep=0pt, partopsep=0pt, leftmargin=*]
        \item Risco de sobreajuste (overfitting) se os dados forem poucos.
    \end{itemize}
    \\ \addlinespace

    \textbf{ELU} &
    $\begin{cases}z_i, & \text{se } z_i \ge 0 \\ \alpha (e^{z_i} - 1) , & \text{se } z_i < 0\end{cases}$ \newline\vspace{0.2cm}
    $ \begin{cases}1, & \text{se } z_i > 0 \\ \alpha e^{z_i}, & \text{se } z_i < 0 \end{cases}$
    &
    \begin{itemize}[noitemsep, topsep=0pt, partopsep=0pt, leftmargin=*]
        \item Saídas com média próxima de zero.
        \item Mais robusta a ruído que LReLU/PReLU.
    \end{itemize}
    &
    \begin{itemize}[noitemsep, topsep=0pt, partopsep=0pt, leftmargin=*]
        \item Computacionalmente mais custosa (exponencial).
    \end{itemize}
    \\ \addlinespace

    \textbf{SELU} &
    $\lambda \begin{cases}z_i, & \text{se } z_i > 0 \\ \alpha (e^{z_i} - 1) , & \text{se } z_i \le 0\end{cases}$ \newline\vspace{0.2cm}
    $ \lambda \begin{cases}1, & \text{se } z_i > 0 \\ \alpha e^{z_i}, & \text{se } z_i \le 0\end{cases}$
    &
    \begin{itemize}[noitemsep, topsep=0pt, partopsep=0pt, leftmargin=*]
        \item Propriedades de autonormalização.
        \item Evita gradientes explosivos/desvanecentes em redes muito profundas.
    \end{itemize}
    &
    \begin{itemize}[noitemsep, topsep=0pt, partopsep=0pt, leftmargin=*]
        \item Requer inicialização de pesos específica (LeCun normal).
        \item Computacionalmente mais custosa.
    \end{itemize}
    \\ \addlinespace

    \textbf{GELU} &
    $z_i \cdot \Phi(z_i)$ \newline\vspace{0.2cm}
    $ \Phi(z_i) + z_i\phi(z_i)$
    &
    \begin{itemize}[noitemsep, topsep=0pt, partopsep=0pt, leftmargin=*]
        \item Suave e diferenciável em todos os pontos.
        \item Performance estado da arte em Transformers (BERT, GPT).
    \end{itemize}
    &
    \begin{itemize}[noitemsep, topsep=0pt, partopsep=0pt, leftmargin=*]
        \item Computacionalmente mais custosa que ReLU.
    \end{itemize}
    \\
    
\end{longtable}
\chapter{Comparativo das Funções de Perda}
\label{cap:comparativo-perda}

Este capítulo detalha as principais funções de perda utilizadas em tarefas de regressão e classificação, apresentando suas formulações matemáticas, principais vantagens e considerações de uso.

% ===================================================================
% Funções de Perda para Regressão
% ===================================================================
\section{Funções de Perda para Regressão}

\subsection{Erro Quadrático Médio (MSE)}

\textit{Uma das perdas mais comuns para regressão, que mede a média dos erros quadrados, penalizando fortemente previsões distantes do valor real.}

\begin{equacaodestaque}{Erro Quadrático Médio (MSE) e sua Derivada}
    \Loss = \frac{1}{N} \sum_{j=1}^{N} (y_j - \hat{y}_j)^2 \\
    \frac{\partial \Loss}{\partial \hat{y}_j} = \frac{2}{N}(\hat{y}_j - y_j)
\end{equacaodestaque}

\subsubsection*{Vantagens / Quando Usar}
\begin{itemize}
    \item Penaliza erros grandes de forma quadrática, sendo ideal para cenários onde grandes desvios são indesejáveis.
    \item É uma função convexa, o que garante um único mínimo global e facilita a otimização.
\end{itemize}

\subsubsection*{Desvantagens / Considerações}
\begin{itemize}
    \item É muito sensível a \textit{outliers}, que podem dominar o gradiente e prejudicar o treinamento.
    \item A unidade da perda (ex: metros quadrados) é diferente da unidade original dos dados (ex: metros), o que dificulta a interpretação direta do erro.
\end{itemize}

\subsection{Erro Absoluto Médio (MAE)}

\textit{Mede a média dos erros absolutos, sendo menos sensível a outliers e mais intuitiva que o MSE, pois mantém a unidade original dos dados.}

\begin{equacaodestaque}{Erro Absoluto Médio (MAE) e sua Derivada}
    \Loss = \frac{1}{N} \sum_{j=1}^{N} |y_j - \hat{y}_j| \\
    \frac{\partial \Loss}{\partial \hat{y}_j} = \text{sgn}(\hat{y}_j - y_j)
\end{equacaodestaque}

\subsubsection*{Vantagens / Quando Usar}
\begin{itemize}
    \item É robusta a \textit{outliers} devido à sua penalidade linear para os erros.
    \item A perda é intuitiva, pois está na mesma escala da variável alvo.
    \item Recomendada quando \textit{outliers} são esperados e não devem influenciar excessivamente o modelo.
\end{itemize}

\subsubsection*{Desvantagens / Considerações}
\begin{itemize}
    \item Não é diferenciável no ponto zero, embora isso seja contornável na prática com subgradientes.
    \item O gradiente é constante, o que pode dificultar a convergência para o mínimo exato, exigindo taxas de aprendizado menores no final do treino.
\end{itemize}

\subsection{Huber Loss}

\textit{Uma perda híbrida que combina o melhor do MSE para erros pequenos e do MAE para erros grandes, oferecendo robustez a outliers sem sacrificar a estabilidade perto do mínimo.}

\begin{equacaodestaque}{Huber Loss e sua Derivada}
    \Loss_{\delta}(y, \hat{y}) = \begin{cases} \frac{1}{2}(y - \hat{y})^2 & \text{se } |y - \hat{y}| \le \delta \\ \delta |y - \hat{y}| - \frac{1}{2}\delta^2 & \text{caso contrário} \end{cases} \\
    \frac{\partial \Loss_{\delta}}{\partial \hat{y}} = \begin{cases} \hat{y} - y & \text{se } |y - \hat{y}| \le \delta \\ \delta \cdot \text{sgn}(\hat{y} - y) & \text{caso contrário} \end{cases}
\end{equacaodestaque}

\subsubsection*{Vantagens / Quando Usar}
\begin{itemize}
    \item Combina a boa convergência do MSE perto do mínimo com a robustez do MAE para erros grandes.
    \item É diferenciável em todos os pontos, exceto em $\pm\delta$.
\end{itemize}

\subsubsection*{Desvantagens / Considerações}
\begin{itemize}
    \item Requer o ajuste do hiperparâmetro $\delta$, que define o limiar entre o comportamento quadrático e linear.
\end{itemize}

\subsection{Log-Cosh Loss}

\textit{Uma função de perda suave que se comporta como o MSE para erros pequenos e como o MAE para erros grandes, sendo uma alternativa à Huber Loss que não requer hiperparâmetros.}

\begin{equacaodestaque}{Log-Cosh Loss e sua Derivada}
    \Loss = \sum_{j=1}^{N} \log(\cosh(y_j - \hat{y}_j)) \\
    \frac{\partial \Loss}{\partial \hat{y}_j} = \tanh(\hat{y}_j - y_j)
\end{equacaodestaque}

\subsubsection*{Vantagens / Quando Usar}
\begin{itemize}
    \item É duplamente diferenciável em todos os pontos, o que a torna suave e bem-comportada para otimizadores baseados em gradiente.
    \item Não requer o ajuste de hiperparâmetros como a Huber Loss.
\end{itemize}

\subsubsection*{Desvantagens / Considerações}
\begin{itemize}
    \item É computacionalmente mais custosa que o MSE e o MAE devido às funções $\log$ e $\cosh$.
\end{itemize}

\subsection{Quantile Loss (Pinball Loss)}

\textit{Utilizada para prever um quantil específico (como a mediana ou o 90º percentil) em vez da média, sendo útil para estimar intervalos de incerteza.}

\begin{equacaodestaque}{Quantile Loss e sua Derivada}
    \Loss_{\tau}(y, \hat{y}) = \begin{cases} \tau (y - \hat{y}) & \text{se } y \ge \hat{y} \\ (1-\tau)(\hat{y}-y) & \text{caso contrário} \end{cases} \\
    \frac{\partial \Loss_{\tau}}{\partial \hat{y}} = \begin{cases} -(1-\tau) & \text{se } \hat{y}>y \\ -\tau & \text{se } \hat{y}<y \end{cases}
\end{equacaodestaque}

\subsubsection*{Vantagens / Quando Usar}
\begin{itemize}
    \item Permite a criação de modelos que preveem diferentes quantis, fornecendo uma visão mais completa da distribuição da variável alvo.
    \item Muito útil em finanças, meteorologia e análise de risco para estimar intervalos de confiança.
\end{itemize}

\subsubsection*{Desvantagens / Considerações}
\begin{itemize}
    \item Requer a definição do quantil $\tau$ como um hiperparâmetro.
    \item Não é diferenciável em zero, assim como o MAE.
\end{itemize}

% ===================================================================
% Funções de Perda para Classificação
% ===================================================================
\section{Funções de Perda para Classificação}

\subsection{Entropia Cruzada Binária (BCE)}

\textit{A função de perda padrão para problemas de classificação binária, que mede a "distância" entre a distribuição de probabilidade prevista e a distribuição real (0 ou 1).}

\begin{equacaodestaque}{Entropia Cruzada Binária (BCE) e sua Derivada}
    \Loss = -[y \log(\hat{y}) + (1-y)\log(1-\hat{y})] \\
    \frac{\partial \Loss}{\partial z} = \hat{y} - y \quad \text{(com Sigmoide na saída)}
\end{equacaodestaque}

\subsubsection*{Vantagens / Quando Usar}
\begin{itemize}
    \item É a escolha padrão e mais eficaz para tarefas de classificação binária.
    \item Otimiza diretamente a log-verossimilhança do modelo, resultando em previsões probabilísticas.
    \item Penaliza fortemente previsões que estão confiantes e erradas.
\end{itemize}

\subsubsection*{Desvantagens / Considerações}
\begin{itemize}
    \item Pode levar a modelos enviesados em cenários com classes muito desbalanceadas.
    \item Exige que a saída do modelo seja uma probabilidade no intervalo (0, 1), geralmente obtida com uma função Sigmoide.
\end{itemize}

\subsection{Hinge Loss}

\textit{Projetada para treinamento de classificadores de máxima margem, como as Support Vector Machines (SVMs), penalizando apenas previsões incorretas ou corretas mas com pouca confiança.}

\begin{equacaodestaque}{Hinge Loss e sua Derivada}
    \Loss = \max(0, 1 - y \cdot \hat{y}) \quad (y \in \{-1, 1\}) \\
    \frac{\partial \Loss}{\partial \hat{y}} = \begin{cases} -y & \text{se } y \cdot \hat{y} < 1 \\ 0 & \text{caso contrário} \end{cases}
\end{equacaodestaque}

\subsubsection*{Vantagens / Quando Usar}
\begin{itemize}
    \item Otimiza explicitamente para maximizar a margem de separação entre as classes.
    \item Não penaliza exemplos que já estão classificados corretamente e fora da margem, focando o aprendizado nos pontos difíceis.
\end{itemize}

\subsubsection*{Desvantagens / Considerações}
\begin{itemize}
    \item A saída do modelo não é uma probabilidade, mas sim uma pontuação de decisão.
    \item Pode ser mais sensível a \textit{outliers} do que perdas baseadas em entropia.
\end{itemize}

\subsection{Entropia Cruzada Categórica (CCE)}

\textit{A extensão da BCE para problemas de classificação multi-classe, comparando a distribuição de probabilidade prevista com o rótulo real em formato one-hot.}

\begin{equacaodestaque}{Entropia Cruzada Categórica (CCE) e sua Derivada}
    \Loss = - \sum_{c=1}^{C} y_{c} \log(\hat{y}_{c}) \\
    \frac{\partial \Loss}{\partial z_i} = \hat{y}_i - y_i \quad \text{(com Softmax na saída)}
\end{equacaodestaque}

\subsubsection*{Vantagens / Quando Usar}
\begin{itemize}
    \item É a função de perda padrão e mais eficaz para classificação multi-classe.
    \item Requer que os rótulos de destino estejam em formato \textit{one-hot}.
\end{itemize}

\subsubsection*{Desvantagens / Considerações}
\begin{itemize}
    \item Assume que as classes são mutuamente exclusivas (cada amostra pertence a apenas uma classe).
    \item É sensível ao desbalanceamento de classes, podendo favorecer a classe majoritária.
\end{itemize}

\subsection{Focal Loss}

\textit{Uma modificação da Entropia Cruzada que reduz a perda atribuída a exemplos fáceis e bem classificados, forçando o modelo a focar em exemplos difíceis e mal classificados.}

\begin{equacaodestaque}{Focal Loss}
    \Loss = -(1 - p_t)^\gamma \log(p_t) \\
    \text{onde } p_t \text{ é a probabilidade da classe correta}
\end{equacaodestaque}

\subsubsection*{Vantagens / Quando Usar}
\begin{itemize}
    \item Reduz o peso de exemplos fáceis, permitindo que o modelo se concentre nos erros mais significativos.
    \item É extremamente eficaz para treinar modelos em datasets com grande desbalanceamento de classes, como na detecção de objetos.
\end{itemize}

\subsubsection*{Desvantagens / Considerações}
\begin{itemize}
    \item Adiciona o hiperparâmetro de foco $\gamma$, que precisa ser ajustado para balancear o peso entre exemplos fáceis e difíceis.
\end{itemize}
\chapter{Comparativo das Métricas de Avaliação}
\label{cap:comparativo-metricas}

Após o treinamento de um modelo, é essencial avaliar seu desempenho de forma quantitativa. Este capítulo apresenta as métricas de avaliação mais comuns para tarefas de classificação e regressão, detalhando suas fórmulas, interpretações e contextos de aplicação.

% ===================================================================
% Métricas para Classificação
% ===================================================================
\section{Métricas para Classificação}

\subsection{Acurácia (Accuracy)}

\textit{Uma métrica direta que mede a proporção de previsões corretas sobre o total de previsões, oferecendo uma visão geral do desempenho do modelo.}

\begin{equacaodestaque}{Acurácia}
    \text{Acurácia} = \frac{VP + VN}{VP + VN + FP + FN}
\end{equacaodestaque}

\subsubsection*{Interpretação}
\begin{itemize}
    \item Representa o percentual de predições corretas (Verdadeiros Positivos + Verdadeiros Negativos) em relação ao total de amostras.
    \item Fornece uma medida geral e intuitiva da performance do classificador.
\end{itemize}

\subsubsection*{Quando Usar / Considerações}
\begin{itemize}
    \item É mais útil em cenários com classes bem balanceadas.
    \item Pode ser uma métrica enganosa em datasets com classes desbalanceadas. Por exemplo, se 95\% das amostras são da classe A, um modelo que sempre prevê A terá 95\% de acurácia, mas será inútil.
\end{itemize}

\subsection{Precisão (Precision)}

\textit{Avalia a exatidão das previsões positivas, respondendo à pergunta: "De todas as vezes que o modelo previu a classe positiva, quantas estavam corretas?".}

\begin{equacaodestaque}{Precisão}
    \text{Precisão} = \frac{VP}{VP + FP}
\end{equacaodestaque}

\subsubsection*{Interpretação}
\begin{itemize}
    \item Mede a proporção de Verdadeiros Positivos entre todas as predições que o modelo classificou como positivas.
    \item Indica a "qualidade" ou "confiabilidade" das predições positivas.
\end{itemize}

\subsubsection*{Quando Usar / Considerações}
\begin{itemize}
    \item É crucial quando o custo de um Falso Positivo (FP) é alto. Por exemplo, em um filtro de spam (marcar um e-mail importante como spam) ou em um diagnóstico médico (diagnosticar uma pessoa saudável com uma doença).
\end{itemize}

\subsection{Revocação (Recall / Sensibilidade)}

\textit{Mede a capacidade do modelo de encontrar todas as amostras positivas relevantes, respondendo à pergunta: "De todos os exemplos realmente positivos, quantos o modelo conseguiu identificar?".}

\begin{equacaodestaque}{Revocação}
    \text{Revocação} = \frac{VP}{VP + FN}
\end{equacaodestaque}

\subsubsection*{Interpretação}
\begin{itemize}
    \item Mede a proporção de Verdadeiros Positivos que foram corretamente identificados pelo modelo.
    \item Indica a "completude" ou a "abrangência" do classificador em relação à classe positiva.
\end{itemize}

\subsubsection*{Quando Usar / Considerações}
\begin{itemize}
    \item É crucial quando o custo de um Falso Negativo (FN) é alto. Por exemplo, na detecção de fraudes (não identificar uma transação fraudulenta) ou no diagnóstico de uma doença grave (não diagnosticar um paciente doente).
\end{itemize}

\subsection{F1-Score}

\textit{A média harmônica entre Precisão e Revocação, fornecendo uma única pontuação que equilibra o trade-off entre as duas métricas.}

\begin{equacaodestaque}{F1-Score}
    \text{F1-Score} = 2 \times \frac{\text{Precisão} \times \text{Revocação}}{\text{Precisão} + \text{Revocação}}
\end{equacaodestaque}

\subsubsection*{Interpretação}
\begin{itemize}
    \item É a média harmônica de Precisão e Revocação, o que significa que penaliza valores extremos de uma das métricas.
    \item Fornece uma única métrica que busca um balanço entre a qualidade (Precisão) e a completude (Revocação) das predições positivas.
\end{itemize}

\subsubsection*{Quando Usar / Considerações}
\begin{itemize}
    \item É especialmente útil em cenários com classes desbalanceadas, onde a Acurácia pode ser enganosa e é necessário um bom equilíbrio entre Precisão e Revocação.
\end{itemize}

\subsection{AUC-ROC}

\textit{Mede a capacidade geral de um modelo de distinguir entre as classes positiva e negativa, independentemente do limiar de classificação escolhido.}

\begin{equacaodestaque}{Curva ROC}
    \text{A curva ROC é plotada com a Taxa de Verdadeiros Positivos} \\
    \text{(Revocação) no eixo Y e a Taxa de Falsos Positivos no eixo X.} \\
    \text{FPR} = \frac{FP}{FP+VN}
\end{equacaodestaque}

\subsubsection*{Interpretação}
\begin{itemize}
    \item A AUC (Área Sob a Curva) representa a probabilidade de que o modelo classifique uma amostra positiva aleatória com uma pontuação maior do que uma amostra negativa aleatória.
    \item AUC = 1.0 indica um classificador perfeito.
    \item AUC = 0.5 indica um desempenho equivalente a um classificador aleatório.
\end{itemize}

\subsubsection*{Quando Usar / Considerações}
\begin{itemize}
    \item Para avaliar e comparar o desempenho geral de modelos de forma agnóstica ao limiar de decisão.
    \item É uma boa métrica agregada para problemas de classificação binária.
\end{itemize}

% ===================================================================
% Métricas para Regressão
% ===================================================================
\section{Métricas para Regressão}

\subsection{RMSE (Root Mean Square Error)}

\textit{Representa o desvio padrão dos erros de predição (resíduos), medindo a magnitude média dos erros na mesma unidade da variável alvo.}

\begin{equacaodestaque}{Raiz do Erro Quadrático Médio (RMSE)}
    \text{RMSE} = \sqrt{\frac{1}{N} \sum_{i=1}^{N} (y_i - \hat{y}_i)^2}
\end{equacaodestaque}

\subsubsection*{Interpretação}
\begin{itemize}
    \item É a raiz quadrada do MSE. Um RMSE de 10, por exemplo, significa que, em média, as previsões do modelo estão a 10 unidades de distância dos valores reais.
\end{itemize}

\subsubsection*{Quando Usar / Considerações}
\begin{itemize}
    \item Quando se deseja que o erro seja expresso na mesma unidade da variável alvo para facilitar a interpretação.
    \item Assim como o MSE, penaliza erros maiores mais fortemente que o MAE devido ao termo quadrático.
\end{itemize}

\subsection{R² (Coeficiente de Determinação)}

\textit{Indica a proporção da variância na variável alvo que é explicada pelo modelo, fornecendo uma medida da qualidade do ajuste.}

\begin{equacaodestaque}{Coeficiente de Determinação (R²)}
    R^2 = 1 - \frac{\sum_{i=1}^{N}(y_i - \hat{y}_i)^2}{\sum_{i=1}^{N}(y_i - \bar{y})^2}
\end{equacaodestaque}

\subsubsection*{Interpretação}
\begin{itemize}
    \item Um R² de 0.85 significa que 85\% da variabilidade da variável alvo é explicada pelas variáveis preditoras do modelo.
    \item Seus valores variam de $-\infty$ a 1. Quanto mais próximo de 1, melhor o modelo se ajusta aos dados. Um valor negativo indica que o modelo é pior que um modelo ingênuo que sempre prevê a média.
\end{itemize}

\subsubsection*{Quando Usar / Considerações}
\begin{itemize}
    \item Para entender o quão bem as variáveis de entrada explicam a variação da variável de saída.
    \item Cuidado: o valor do R² tende a aumentar à medida que mais variáveis são adicionadas ao modelo, mesmo que elas não sejam úteis. Nesses casos, o R² ajustado é uma métrica mais apropriada.
\end{itemize}


% \include{apendices/ap_B_guia_setup}


% --- ELEMENTOS PÓS-TEXTUAIS ---
% O comando \backmatter é usado para as seções finais do livro.
\backmatter

% Gera a lista de Referências a partir do arquivo 'bibliografia.bib',
% formatada no estilo ABNT pelo biblatex.
\printbibliography[title={Referências}]

% --- IMPRIMIR O GLOSSÁRIO ---
\cleardoublepage % Começar em página nova
\phantomsection % Ajuda links
\printglossaries % Este comando imprime TODAS as listas de glossário definidas
% Não precisa de \addcontentsline se usou a opção [toc] no \usepackage

\printindex


\end{document}
% ===================================================================
% FIM DO DOCUMENTO
% ===================================================================