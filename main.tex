% ===================================================================
% ARQUIVO MESTRE DO LIVRO
% Autor: Seu Nome
% Projeto: Um Mergulho Profundo no Aprendizado de Máquina
% ===================================================================

% --- CLASSE DO DOCUMENTO ---
% Usamos a classe 'abntex2' para seguir as normas da ABNT.
% As opções configuram o formato para um livro acadêmico padrão.
\documentclass[
    12pt,            % Tamanho da fonte do corpo do texto
    a4paper,         % Tamanho do papel
    book,            % Formato de livro
    openright,       % Força capítulos a começarem em páginas ímpares (da direita)
    twoside,         % Layout para impressão frente e verso (margens diferentes)
    brazil           % Configurações para o idioma português do Brasil
]{abntex2}


% --- PREÂMBULO ---
% Importa todas as configurações, pacotes e comandos customizados
% do arquivo 'preambulo.tex'. Isso mantém este arquivo principal limpo
% e focado apenas na estrutura do conteúdo do livro.
% ===================================================================
% ARQUIVO DE PREÂMBULO
% Contém todas as configurações de pacotes e layout para o livro.
% ===================================================================


% --- CONFIGURAÇÕES ESSENCIAIS DO DOCUMENTO ---

% Codificacao da fonte. Essencial para a correta exibição de caracteres acentuados no PDF.
\usepackage[T1]{fontenc}

% Codificacao de entrada do arquivo .tex. Permite que você escreva com acentos diretamente.
\usepackage[utf8]{inputenc}

% Suporte ao idioma Português do Brasil.
% A classe abntex2 já cuida disso, mas é uma boa prática garantir.
% A opção 'brazil' ajusta hifenizacao, nomes de seções ("Sumário"), etc.
\usepackage[brazil]{babel}

% Para usar a fonte Palatino
\usepackage{mathpazo}

% --- PACOTES PARA CONTEÚDO TÉCNICO E MATEMÁTICO ---

% Pacotes padrao e essenciais para matemática avançada.
\usepackage{amsmath, amsfonts, amssymb}

% Pacote para inclusão de imagens.
\usepackage{graphicx}
% Define um ou mais diretórios padrão para suas imagens.
% Isso evita que você precise digitar "imagens/" toda vez.
\graphicspath{ {./imagens/} }

% Pacote para formatar e incluir snippets de código-fonte.
\usepackage{listings}
\usepackage{xcolor} % Necessário para definir cores no listings

% Define um estilo customizado para código Python, para ficar mais bonito.
\lstdefinestyle{PythonStyle}{
    language=Python,
    backgroundcolor=\color{gray!10},   % Cor de fundo leve
    basicstyle=\ttfamily\small,       % Fonte monoespaçada e pequena
    keywordstyle=\color{blue},        % Palavras-chave em azul
    commentstyle=\color{green!50!black},% Comentários em verde escuro
    stringstyle=\color{purple},       % Strings em roxo
    breaklines=true,                  % Quebra de linhas longas
    showstringspaces=false,           % Não mostra espaços em strings com um símbolo
    frame=single,                     % Adiciona uma moldura
    rulecolor=\color{black!30},       % Cor da moldura
    numbers=left,                     % Numeração de linhas à esquerda
    numberstyle=\tiny\color{gray},    % Estilo dos números de linha
}


% --- REFERÊNCIAS, CITAÇÕES E LINKS (PADRÃO ABNT) ---

% O sistema moderno e mais poderoso para bibliografia em LaTeX, já configurado para ABNT.
\usepackage[
    backend=biber,     % Ferramenta que processa a bibliografia (mais moderna que bibtex)
    style=abnt,        % Estilo de citação e referência da ABNT
    ittitles,          % Títulos de artigos, livros, etc., em itálico
    backref=true,      % Nas referências, mostra em quais páginas a citação aparece
]{biblatex}

\renewcommand*{\mkbibnamefamily}[1]{\MakeUppercase{#1}}
\renewcommand*{\mkbibnamelast}[1]{\MakeUppercase{#1}}

% Aponta para o arquivo que contém seu banco de dados de referências.
\addbibresource{bibliografia.bib}

% Melhora a formatação de URLs nas referências.
\usepackage{url}
\urlstyle{same} % Usa a mesma fonte do texto para as URLs.

% --- LAYOUT E ESTRUTURA ---

% Pacote para ajustar as margens e geometria da página.
% A classe abntex2 já define margens padrão ABNT, use para sobrescrever se necessário.
\usepackage{geometry}
\geometry{
    a4paper,
    left=3cm,
    right=2cm,
    top=3cm,
    bottom=2cm
}

% --- CONFIGURAÇÃO DO HYPERREF ---
% A classe abntex2 já carrega o pacote hyperref.
% Apenas usamos \hypersetup para definir as opções que queremos.

\hypersetup{
    pdftitle={O Título do Seu Livro},
    pdfauthor={Seu Nome},
    pdfsubject={Assunto do Livro},
    pdfkeywords={Palavra-chave1, Palavra-chave2},
    colorlinks=true,                % Habilita links coloridos
    linkcolor=blue,                 % Cor dos links internos (sumário, etc.)
    citecolor=red,                  % Cor das citações
    urlcolor=cyan,                  % Cor das URLs
    hidelinks=false,                % Se 'true', esconde as caixas e cores (bom para impressão)
    pdfstartview={FitH},            % Abre o PDF ajustado à largura da página
    bookmarksopen=true              % Abre o painel de bookmarks (índices)
}

% Pacote para customização avançada de cabeçalhos e rodapés.
% Descomente as linhas abaixo apenas se precisar de um layout diferente
% do padrão oferecido pelo abntex2.
%
% \usepackage{fancyhdr}
% \pagestyle{fancy}
% \fancyhf{} % Limpa todos os campos
% \fancyhead[LE,RO]{\nouppercase{\leftmark}} % Nome do capítulo no cabeçalho
% \fancyfoot[LE,RO]{\thepage} % Número da página no rodapé


% --- COMANDOS CUSTOMIZADOS (OPCIONAL) ---
% Aqui você pode definir seus próprios comandos para agilizar a escrita.
% Exemplo:
% \newcommand{\keras}{\texttt{Keras}}

\usepackage{tikz}
\usetikzlibrary{patterns, positioning, fit, shapes.geometric, arrows.meta, fpu, decorations.pathmorphing, shadows, shapes.symbols}
\usepackage{pgfplots}
\pgfplotsset{compat=1.18}

% --- Definindo as cores (hues) ---
\definecolor{darkblue}{rgb}{0.16, 0.32, 0.75}
\definecolor{darkorange}{rgb}{0.9, 0.35, 0.0}
\definecolor{pointred}{rgb}{0.85, 0.2, 0.2}

\usepackage[most]{tcolorbox}
\tcbuselibrary{skins} % Necessário para o estilo enhanced

\newtcolorbox{definicaomoderna}{
    enhanced,
    frame hidden, % Remove a moldura da caixa
    borderline west={2pt}{0pt}{gray!80}, % Cria uma linha de 2pt à esquerda (west)
    colback=gray!5, % Um fundo cinza muito sutil para diferenciar
    fonttitle=\bfseries,
    coltitle=black,
    sharp corners, % Cantos retos
    boxsep=5pt,
    left=10pt, % Espaço interno à esquerda
    attach boxed title to top left={yshift=-2mm, xshift=5mm}, % Posição do título
    boxed title style={
        frame hidden,
        colback=gray!5 % Fundo do título igual ao da caixa
    },
}

% --- Comando especial para equação destacada COM NOME ---
\newtcolorbox{equacaodestaque}[1]{ % O [1] indica que o comando aceita um argumento
    enhanced, ams equation,
    frame hidden,
    borderline west={2pt}{0pt}{gray!80},
    colback=gray!5,
    sharp corners,
    % --- Opções para o título ---
    fonttitle=\bfseries,
    coltitle=black,
    attach boxed title to top left={yshift=-2mm, xshift=5mm}, % Posiciona o título
    boxed title style={
        frame hidden,
        colback=gray!5 % Faz o fundo do título ser igual ao da caixa
    },
    title={#1} % Usa o argumento como título da caixa
}

\usepackage{xcolor} % Para definir cores customizadas
\usepackage{listings} % Para formatar o código
\usepackage[most]{tcolorbox} % Para criar a caixa estilizada
\usepackage{witharrows}

% --- 1. Definição das Cores para o Código ---
\definecolor{codegray}{rgb}{0.5,0.5,0.5}
\definecolor{codepurple}{rgb}{0.58,0,0.82}
\definecolor{codeblue}{rgb}{0,0,0.8}
\definecolor{codered}{rgb}{0.8,0,0}

% --- 2. Configuração Global do Pacote listings ---
\lstset{
    language=Python,
    backgroundcolor=\color{white},   
    commentstyle=\color{codegray}\itshape,
    keywordstyle=\color{codeblue}\bfseries,
    numberstyle=\tiny\color{codegray},
    stringstyle=\color{codered},
    basicstyle=\ttfamily\small,
    breakatwhitespace=false,         
    breaklines=true,                 
    captionpos=b,                    
    keepspaces=true,                 
    numbers=left,                    
    numbersep=5pt,                  
    showspaces=false,                
    showstringspaces=false,
    showtabs=false,                  
    tabsize=4
}

% --- Criação do Ambiente de Listagem de Código Estilizado (Fundo Branco) ---
\newtcblisting{codelisting}[2]{
  listing only,
  enhanced,
  colback=white,  % Fundo branco, igual ao da página
  colframe=gray!80,
  frame hidden,
  borderline west={2pt}{0pt}{gray!80},
  sharp corners,
  top=5pt,
  bottom=5pt,
  % --- Título e numeração (Listagem X: Título) ---
  fonttitle=\bfseries,
  coltitle=black,
  title={Bloco de Código~\thetcbcounter: #1},
  label={lst:#2},
  % --- Opções passadas para o pacote listings ---
  listing options={
    language=Python,
    numbers=left,
  }
}

\usepackage{algorithm} % Para o ambiente algorithm
\usepackage{algpseudocode} % Para os comandos do pseudocódigo

\algrenewcommand\algorithmicrequire{\textbf{Requer:}}
\algrenewcommand\algorithmicreturn{\textbf{Retorne:}}

\usepackage{threeparttable}

\usepackage{longtable}     % Para tabelas que ocupam várias páginas
\usepackage{booktabs}      % Para as linhas \toprule, \midrule, \bottomrule
\usepackage{amsmath}       % Para os ambientes matemáticos como 'cases'
\usepackage[labelfont=bf, singlelinecheck=false, labelsep=endash]{caption} % Para formatar o título

\usepackage{makecell}

\numberwithin{figure}{chapter}
\numberwithin{table}{chapter}
\numberwithin{equation}{chapter}

\usepackage{nomencl}
\usepackage{etoolbox} % Para customizações avançadas

% --- Comandos Customizados para Funções ---
\newcommand{\fativ}[1]{\ensuremath{\mathcal{A}_{#1}}}
\newcommand{\fperd}[1]{\ensuremath{\mathcal{L}_{#1}}}

\usepackage{ragged2e}   % Para um melhor alinhamento do texto justificado

% ===================================================================
% FIM DO PREÂMBULO
% ===================================================================


% --- INFORMAÇÕES DO DOCUMENTO (para a folha de rosto) ---
\title{APRENDIZADO DE MÁQUINA}
\author{Luiz Guilherme Morais da Costa Faria}
\date{\today} % Usa a data atual no momento da compilação

% Informações adicionais que o abntex2 usa (opcional)
\instituicao{Universidade de Brasília}
\local{Brasília, DF}
\orientador{Nome do Orientador/Revisor (se aplicável)}


% ===================================================================
% INÍCIO DO DOCUMENTO
% ===================================================================
\begin{document}

% --- ELEMENTOS PRÉ-TEXTUAIS ---
% O comando \frontmatter inicia a contagem de páginas em algarismos romanos (i, ii, ...)
% e desativa a numeração de capítulos para os elementos iniciais.
\frontmatter

% Gera a capa e a folha de rosto com base nas informações acima,
% seguindo o padrão ABNT.
\imprimircapa
\imprimirfolhaderosto

% Páginas opcionais como Dedicatória, Agradecimentos, Epígrafe...
% \begin{dedicatoria}
%    \vspace*{\fill} % Centraliza verticalmente
%    \noindent
%    \textit{Para ...}
%    \vspace*{\fill}
% \end{dedicatoria}

% Gera o Sumário automaticamente com base nos comandos \part, \chapter, \section, etc.
\tableofcontents


% --- ELEMENTOS TEXTUAIS (O CONTEÚDO PRINCIPAL DO LIVRO) ---
% O comando \mainmatter reinicia a contagem de páginas em algarismos arábicos (1, 2, ...)
% e reativa a numeração de capítulos.
\mainmatter

% =======================================================
% PARTE I: HISTÓRIA DA IA E DO COMPUTADOR
% =======================================================
\part{História da IA e do Computador}

% ===================================================================
% Arquivo: capitulos/parte-01-historia/cap-01-historia-do-computador.tex
% ===================================================================

\chapter{Uma Breve História do Computador}
\label{cap:historia-computador}

\section{A Necessidade de Contar ao Longo das Eras}

\subsection{Ábaco}

\subsection{Régua de Cálculo}

\subsection{Bastões de Napier}

\subsection{Pascalina}
% ===================================================================
% Arquivo: capitulos/parte-01-historia/cap-01-historia-da-ia.tex
% ===================================================================

\chapter{Uma Breve História da Inteligência Artificial}
\label{cap:historia-ia}

O texto do seu capítulo começa aqui...

% =======================================================
% PARTE II: Conceitos Matemáticos
% =======================================================
\part{Conceitos Matemáticos}

% ===================================================================
% Arquivo: capitulos/parte-II-matematica/cap-03-calculo.tex
% ===================================================================

\chapter{Cálculo para Aprendizado de Máquina}
\label{cap:calculo-ia}

O texto do seu capítulo começa aqui...
% ===================================================================
% Arquivo: capitulos/parte-II-matematica/cap-04-algebra-linear.tex
% ===================================================================

\chapter{Álgebra Linear para Aprendizado de Máquina}
\label{cap:algebra-linear-ia}

O texto do seu capítulo começa aqui...
% ===================================================================
% Arquivo: capitulos/parte-II-matematica/cap-05-probabilidade-e-estatistica.tex
% ===================================================================

\chapter{Probabilidade e Estatística para Aprendizado de Máquina}
\label{cap:probabilidade-e-estatistica-ia}

O texto do seu capítulo começa aqui...

% =======================================================
% PARTE III: Pilares das Redes Neurais
% =======================================================
\part{Pilares das Redes Neurais}

% ===================================================================
% Arquivo: capitulos/parte-III-pilares/cap-06-sigmoidais.tex
% ===================================================================

\chapter{Funções de Ativação Sigmoidais}
\label{cap:ativacao-sigmoidais}

O texto do seu capítulo começa aqui...
% ===================================================================
% Arquivo: capitulos/parte-III-pilares/cap-07-retificadoras.tex
% ===================================================================

\chapter{Funções de Ativação Retificadoras}
\label{cap:ativacao-retificadoras}

O texto do seu capítulo começa aqui...
% ===================================================================
% Arquivo: capitulos/parte-III-pilares/cap-08-perda-binaria.tex
% ===================================================================

\chapter{Funções de Perda para Classificação Binária}
\label{cap:perda-binaria}

O texto do seu capítulo começa aqui...
% ===================================================================
% Arquivo: capitulos/parte-III-pilares/cap-09-perda-multi.tex
% ===================================================================

\chapter{Funções de Perda para Classificação Multilabel}
\label{cap:perda-multi}

O texto do seu capítulo começa aqui...
% ===================================================================
% Arquivo: capitulos/parte-III-pilares/cap-10-retropropagacao-e-gradiente.tex
% ===================================================================

\chapter{O Algoritmo da Repropropagação e Os Otimizadores Baseados em Gradiente}
\label{cap:retropropagacao-gradiente}

O texto do seu capítulo começa aqui...
% ===================================================================
% Arquivo: capitulos/parte-III-pilares/cap-11-metaheuristicas.tex
% ===================================================================

\chapter{Metaheurísticas: Otimizando Redes Neurais Sem o Gradiente}
\label{cap:otimizacao-metaheuristicas}

O texto do seu capítulo começa aqui...

% =======================================================
% PARTE IV: APRENDIZADO DE MÁQUINA CLÁSSICO
% =======================================================
\part{Aprendizado de Máquina Clássico}

% O comando '\include' inicia uma nova página para cada capítulo e
% carrega o conteúdo do arquivo .tex especificado.
% ===================================================================
% Arquivo: capitulos/parte_IV_ml_classico/cap_12_regressao.tex
% ===================================================================

\chapter{Regressão Linear e Logística}
\label{cap:regressao}

O texto do seu capítulo começa aqui...
% ===================================================================
% Arquivo: capitulos/parte_IV_ml_classico/cap_13_arvores.tex
% ===================================================================

\chapter{Árvores de Decisão e Florestas Aleatórias}
\label{cap:arvores}

O texto do seu capítulo começa aqui...
% ===================================================================
% Arquivo: capitulos/parte_IV_ml_classico/cap_14_svm.tex
% ===================================================================

\chapter{Máquinas de Vetores de Suporte}
\label{cap:svm}

O texto do seu capítulo começa aqui...
% ===================================================================
% Arquivo: capitulos/parte_IV_ml_classico/cap_15_dimensionalidade.tex
% ===================================================================

\chapter{Dimensionalidade}
\label{cap:dimensionalidade}

O texto do seu capítulo começa aqui...
% ===================================================================
% Arquivo: capitulos/parte_IV_ml_classico/cap_16_clusterizacao.tex
% ===================================================================

\chapter{Clusterização}
\label{cap:clusterizacao}

O texto do seu capítulo começa aqui...

% =======================================================
% PARTE V: REDES NEURAIS PROFUNDAS (DNNs)
% =======================================================
\part{Redes Neurais Profundas (DNNs)}

% ===================================================================
% Arquivo: capitulos/parte_V_deep_learning/cap_17_mlp.tex
% ===================================================================

\chapter{Perceptrons MLP - Redes Neurais Artificiais}
\label{cap:mlp}

O texto do seu capítulo começa aqui...
% ===================================================================
% Arquivo: capitulos/parte_V_deep_learning/cap_18_ffn.tex
% ===================================================================

\chapter{Redes FeedForward (FFNs)}
\label{cap:ffn}

O texto do seu capítulo começa aqui...
% ===================================================================
% Arquivo: capitulos/parte_V_deep_learning/cap_19_dbn.tex
% ===================================================================

\chapter{Redes de Crença Profunda (DBNs) e Máquinas de Boltzmann Restritas}
\label{cap:dbn}

O texto do seu capítulo começa aqui...
% ===================================================================
% Arquivo: capitulos/parte_V_deep_learning/cap_20_cnn.tex
% ===================================================================

\chapter{Redes Neurais Convolucionais (CNN)}
\label{cap:cnn}

O texto do seu capítulo começa aqui...
% ===================================================================
% Arquivo: capitulos/parte_V_deep_learning/cap_21_resnet.tex
% ===================================================================

\chapter{Redes Residuais (ResNets)}
\label{cap:resnet}

O texto do seu capítulo começa aqui...
% ===================================================================
% Arquivo: capitulos/parte_V_deep_learning/cap_22_rnn.tex
% ===================================================================

\chapter{Redes Neurais Recorrentes (RNN)}
\label{cap:rnn}

O texto do seu capítulo começa aqui...
% ===================================================================
% Arquivo: capitulos/parte_V_deep_learning/cap_23_transformers.tex
% ===================================================================

\chapter{Transformers}
\label{cap:transformers}

O texto do seu capítulo começa aqui...
% ===================================================================
% Arquivo: capitulos/parte_V_deep_learning/cap_24_gans.tex
% ===================================================================

\chapter{Redes Adversárias Generativas (GANs)}
\label{cap:gans}

O texto do seu capítulo começa aqui...
% ===================================================================
% Arquivo: capitulos/parte_V_deep_learning/cap_25_moe.tex
% ===================================================================

\chapter{Mixture of Experts (MoE)}
\label{cap:moe}

O texto do seu capítulo começa aqui...
% ===================================================================
% Arquivo: capitulos/parte_V_deep_learning/cap_26_diffusion.tex
% ===================================================================

\chapter{Modelos de Difusão}
\label{cap:diffusion}

O texto do seu capítulo começa aqui...
% ===================================================================
% Arquivo: capitulos/parte_V_deep_learning/cap_27_gnn.tex
% ===================================================================

\chapter{Redes Neurais de Grafos (GNNs)}
\label{cap:gnn}

O texto do seu capítulo começa aqui...
% Adicione o capítulo de otimizadores se ele estiver aqui


% --- APÊNDICES ---
% O comando \appendix muda a formatação dos capítulos para "Apêndice A", "Apêndice B", etc.
\appendix
\part{Apêndices}
% \include{apendices/ap_B_guia_setup}


% --- ELEMENTOS PÓS-TEXTUAIS ---
% O comando \backmatter é usado para as seções finais do livro.
\backmatter

% Gera a lista de Referências a partir do arquivo 'bibliografia.bib',
% formatada no estilo ABNT pelo biblatex.
\printbibliography[title={Referências}]


\end{document}
% ===================================================================
% FIM DO DOCUMENTO
% ===================================================================